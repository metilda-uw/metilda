%% Generated by Sphinx.
\def\sphinxdocclass{report}
\documentclass[letterpaper,10pt,english]{sphinxmanual}
\ifdefined\pdfpxdimen
   \let\sphinxpxdimen\pdfpxdimen\else\newdimen\sphinxpxdimen
\fi \sphinxpxdimen=.75bp\relax
\ifdefined\pdfimageresolution
    \pdfimageresolution= \numexpr \dimexpr1in\relax/\sphinxpxdimen\relax
\fi
%% let collapsible pdf bookmarks panel have high depth per default
\PassOptionsToPackage{bookmarksdepth=5}{hyperref}

\PassOptionsToPackage{booktabs}{sphinx}
\PassOptionsToPackage{colorrows}{sphinx}

\PassOptionsToPackage{warn}{textcomp}
\usepackage[utf8]{inputenc}
\ifdefined\DeclareUnicodeCharacter
% support both utf8 and utf8x syntaxes
  \ifdefined\DeclareUnicodeCharacterAsOptional
    \def\sphinxDUC#1{\DeclareUnicodeCharacter{"#1}}
  \else
    \let\sphinxDUC\DeclareUnicodeCharacter
  \fi
  \sphinxDUC{00A0}{\nobreakspace}
  \sphinxDUC{2500}{\sphinxunichar{2500}}
  \sphinxDUC{2502}{\sphinxunichar{2502}}
  \sphinxDUC{2514}{\sphinxunichar{2514}}
  \sphinxDUC{251C}{\sphinxunichar{251C}}
  \sphinxDUC{2572}{\textbackslash}
\fi
\usepackage{cmap}
\usepackage[T1]{fontenc}
\usepackage{amsmath,amssymb,amstext}
\usepackage{babel}



\usepackage{tgtermes}
\usepackage{tgheros}
\renewcommand{\ttdefault}{txtt}



\usepackage[Bjarne]{fncychap}
\usepackage{sphinx}

\fvset{fontsize=auto}
\usepackage{geometry}


% Include hyperref last.
\usepackage{hyperref}
% Fix anchor placement for figures with captions.
\usepackage{hypcap}% it must be loaded after hyperref.
% Set up styles of URL: it should be placed after hyperref.
\urlstyle{same}


\usepackage{sphinxmessages}
\setcounter{tocdepth}{3}
\setcounter{secnumdepth}{3}


\title{MeTILDA}
\date{Feb 11, 2024}
\release{1.0.0v}
\author{Dr Chen}
\newcommand{\sphinxlogo}{\vbox{}}
\renewcommand{\releasename}{Release}
\makeindex
\begin{document}

\ifdefined\shorthandoff
  \ifnum\catcode`\=\string=\active\shorthandoff{=}\fi
  \ifnum\catcode`\"=\active\shorthandoff{"}\fi
\fi

\pagestyle{empty}
\sphinxmaketitle
\pagestyle{plain}
\sphinxtableofcontents
\pagestyle{normal}
\phantomsection\label{\detokenize{index::doc}}



\chapter{metilda package}
\label{\detokenize{index:metilda-package}}

\section{Subpackages}
\label{\detokenize{index:subpackages}}
\sphinxstepscope


\subsection{metilda.controllers package}
\label{\detokenize{metilda.controllers:metilda-controllers-package}}\label{\detokenize{metilda.controllers::doc}}

\subsubsection{Submodules}
\label{\detokenize{metilda.controllers:submodules}}

\subsubsection{metilda.controllers.Postgres module}
\label{\detokenize{metilda.controllers:module-metilda.controllers.Postgres}}\label{\detokenize{metilda.controllers:metilda-controllers-postgres-module}}\index{module@\spxentry{module}!metilda.controllers.Postgres@\spxentry{metilda.controllers.Postgres}}\index{metilda.controllers.Postgres@\spxentry{metilda.controllers.Postgres}!module@\spxentry{module}}\index{Postgres (class in metilda.controllers.Postgres)@\spxentry{Postgres}\spxextra{class in metilda.controllers.Postgres}}

\begin{fulllineitems}
\phantomsection\label{\detokenize{metilda.controllers:metilda.controllers.Postgres.Postgres}}
\pysigstartsignatures
\pysigline{\sphinxbfcode{\sphinxupquote{class\DUrole{w,w}{  }}}\sphinxcode{\sphinxupquote{metilda.controllers.Postgres.}}\sphinxbfcode{\sphinxupquote{Postgres}}}
\pysigstopsignatures
\sphinxAtStartPar
Bases: \sphinxcode{\sphinxupquote{object}}

\sphinxAtStartPar
Class that contains the details of Postgres database
\index{execute\_insert\_query() (metilda.controllers.Postgres.Postgres method)@\spxentry{execute\_insert\_query()}\spxextra{metilda.controllers.Postgres.Postgres method}}

\begin{fulllineitems}
\phantomsection\label{\detokenize{metilda.controllers:metilda.controllers.Postgres.Postgres.execute_insert_query}}
\pysigstartsignatures
\pysiglinewithargsret{\sphinxbfcode{\sphinxupquote{execute\_insert\_query}}}{\sphinxparam{\DUrole{n,n}{query}}\sphinxparamcomma \sphinxparam{\DUrole{n,n}{record}}\sphinxparamcomma \sphinxparam{\DUrole{n,n}{need\_last\_row\_id}\DUrole{o,o}{=}\DUrole{default_value}{True}}}{}
\pysigstopsignatures
\end{fulllineitems}

\index{execute\_select\_query() (metilda.controllers.Postgres.Postgres method)@\spxentry{execute\_select\_query()}\spxextra{metilda.controllers.Postgres.Postgres method}}

\begin{fulllineitems}
\phantomsection\label{\detokenize{metilda.controllers:metilda.controllers.Postgres.Postgres.execute_select_query}}
\pysigstartsignatures
\pysiglinewithargsret{\sphinxbfcode{\sphinxupquote{execute\_select\_query}}}{\sphinxparam{\DUrole{n,n}{query}}\sphinxparamcomma \sphinxparam{\DUrole{n,n}{record}\DUrole{o,o}{=}\DUrole{default_value}{None}}}{}
\pysigstopsignatures
\end{fulllineitems}

\index{execute\_update\_query() (metilda.controllers.Postgres.Postgres method)@\spxentry{execute\_update\_query()}\spxextra{metilda.controllers.Postgres.Postgres method}}

\begin{fulllineitems}
\phantomsection\label{\detokenize{metilda.controllers:metilda.controllers.Postgres.Postgres.execute_update_query}}
\pysigstartsignatures
\pysiglinewithargsret{\sphinxbfcode{\sphinxupquote{execute\_update\_query}}}{\sphinxparam{\DUrole{n,n}{query}}\sphinxparamcomma \sphinxparam{\DUrole{n,n}{record}}}{}
\pysigstopsignatures
\end{fulllineitems}


\end{fulllineitems}



\subsubsection{metilda.controllers.controller\_firestore module}
\label{\detokenize{metilda.controllers:module-metilda.controllers.controller_firestore}}\label{\detokenize{metilda.controllers:metilda-controllers-controller-firestore-module}}\index{module@\spxentry{module}!metilda.controllers.controller\_firestore@\spxentry{metilda.controllers.controller\_firestore}}\index{metilda.controllers.controller\_firestore@\spxentry{metilda.controllers.controller\_firestore}!module@\spxentry{module}}\index{getOrCreate\_Collections() (in module metilda.controllers.controller\_firestore)@\spxentry{getOrCreate\_Collections()}\spxextra{in module metilda.controllers.controller\_firestore}}

\begin{fulllineitems}
\phantomsection\label{\detokenize{metilda.controllers:metilda.controllers.controller_firestore.getOrCreate_Collections}}
\pysigstartsignatures
\pysiglinewithargsret{\sphinxcode{\sphinxupquote{metilda.controllers.controller\_firestore.}}\sphinxbfcode{\sphinxupquote{getOrCreate\_Collections}}}{\sphinxparam{\DUrole{n,n}{id}\DUrole{o,o}{=}\DUrole{default_value}{\textquotesingle{}0\textquotesingle{}}}}{}
\pysigstopsignatures
\end{fulllineitems}



\subsubsection{metilda.controllers.pitch\_art\_wizard module}
\label{\detokenize{metilda.controllers:module-metilda.controllers.pitch_art_wizard}}\label{\detokenize{metilda.controllers:metilda-controllers-pitch-art-wizard-module}}\index{module@\spxentry{module}!metilda.controllers.pitch\_art\_wizard@\spxentry{metilda.controllers.pitch\_art\_wizard}}\index{metilda.controllers.pitch\_art\_wizard@\spxentry{metilda.controllers.pitch\_art\_wizard}!module@\spxentry{module}}\index{add\_new\_user\_from\_admin() (in module metilda.controllers.pitch\_art\_wizard)@\spxentry{add\_new\_user\_from\_admin()}\spxextra{in module metilda.controllers.pitch\_art\_wizard}}

\begin{fulllineitems}
\phantomsection\label{\detokenize{metilda.controllers:metilda.controllers.pitch_art_wizard.add_new_user_from_admin}}
\pysigstartsignatures
\pysiglinewithargsret{\sphinxcode{\sphinxupquote{metilda.controllers.pitch\_art\_wizard.}}\sphinxbfcode{\sphinxupquote{add\_new\_user\_from\_admin}}}{}{}
\pysigstopsignatures
\end{fulllineitems}

\index{all\_audio\_pitches() (in module metilda.controllers.pitch\_art\_wizard)@\spxentry{all\_audio\_pitches()}\spxextra{in module metilda.controllers.pitch\_art\_wizard}}

\begin{fulllineitems}
\phantomsection\label{\detokenize{metilda.controllers:metilda.controllers.pitch_art_wizard.all_audio_pitches}}
\pysigstartsignatures
\pysiglinewithargsret{\sphinxcode{\sphinxupquote{metilda.controllers.pitch\_art\_wizard.}}\sphinxbfcode{\sphinxupquote{all\_audio\_pitches}}}{\sphinxparam{\DUrole{n,n}{upload\_id}}}{}
\pysigstopsignatures
\end{fulllineitems}

\index{all\_upload\_pitches() (in module metilda.controllers.pitch\_art\_wizard)@\spxentry{all\_upload\_pitches()}\spxextra{in module metilda.controllers.pitch\_art\_wizard}}

\begin{fulllineitems}
\phantomsection\label{\detokenize{metilda.controllers:metilda.controllers.pitch_art_wizard.all_upload_pitches}}
\pysigstartsignatures
\pysiglinewithargsret{\sphinxcode{\sphinxupquote{metilda.controllers.pitch\_art\_wizard.}}\sphinxbfcode{\sphinxupquote{all\_upload\_pitches}}}{}{}
\pysigstopsignatures
\end{fulllineitems}

\index{allowed\_file() (in module metilda.controllers.pitch\_art\_wizard)@\spxentry{allowed\_file()}\spxextra{in module metilda.controllers.pitch\_art\_wizard}}

\begin{fulllineitems}
\phantomsection\label{\detokenize{metilda.controllers:metilda.controllers.pitch_art_wizard.allowed_file}}
\pysigstartsignatures
\pysiglinewithargsret{\sphinxcode{\sphinxupquote{metilda.controllers.pitch\_art\_wizard.}}\sphinxbfcode{\sphinxupquote{allowed\_file}}}{\sphinxparam{\DUrole{n,n}{filename}}}{}
\pysigstopsignatures
\end{fulllineitems}

\index{annotationTimeSelection() (in module metilda.controllers.pitch\_art\_wizard)@\spxentry{annotationTimeSelection()}\spxextra{in module metilda.controllers.pitch\_art\_wizard}}

\begin{fulllineitems}
\phantomsection\label{\detokenize{metilda.controllers:metilda.controllers.pitch_art_wizard.annotationTimeSelection}}
\pysigstartsignatures
\pysiglinewithargsret{\sphinxcode{\sphinxupquote{metilda.controllers.pitch\_art\_wizard.}}\sphinxbfcode{\sphinxupquote{annotationTimeSelection}}}{\sphinxparam{\DUrole{n,n}{eaffilename}}\sphinxparamcomma \sphinxparam{\DUrole{n,n}{sound}}\sphinxparamcomma \sphinxparam{\DUrole{n,n}{start}}\sphinxparamcomma \sphinxparam{\DUrole{n,n}{end}}\sphinxparamcomma \sphinxparam{\DUrole{n,n}{text0}}\sphinxparamcomma \sphinxparam{\DUrole{n,n}{text1}}\sphinxparamcomma \sphinxparam{\DUrole{n,n}{text2}}\sphinxparamcomma \sphinxparam{\DUrole{n,n}{text3}}\sphinxparamcomma \sphinxparam{\DUrole{n,n}{text4}}\sphinxparamcomma \sphinxparam{\DUrole{n,n}{text5}}}{}
\pysigstopsignatures
\end{fulllineitems}

\index{api() (in module metilda.controllers.pitch\_art\_wizard)@\spxentry{api()}\spxextra{in module metilda.controllers.pitch\_art\_wizard}}

\begin{fulllineitems}
\phantomsection\label{\detokenize{metilda.controllers:metilda.controllers.pitch_art_wizard.api}}
\pysigstartsignatures
\pysiglinewithargsret{\sphinxcode{\sphinxupquote{metilda.controllers.pitch\_art\_wizard.}}\sphinxbfcode{\sphinxupquote{api}}}{}{}
\pysigstopsignatures
\end{fulllineitems}

\index{audio() (in module metilda.controllers.pitch\_art\_wizard)@\spxentry{audio()}\spxextra{in module metilda.controllers.pitch\_art\_wizard}}

\begin{fulllineitems}
\phantomsection\label{\detokenize{metilda.controllers:metilda.controllers.pitch_art_wizard.audio}}
\pysigstartsignatures
\pysiglinewithargsret{\sphinxcode{\sphinxupquote{metilda.controllers.pitch\_art\_wizard.}}\sphinxbfcode{\sphinxupquote{audio}}}{\sphinxparam{\DUrole{n,n}{upload\_id}}}{}
\pysigstopsignatures
\end{fulllineitems}

\index{audio\_analysis\_image() (in module metilda.controllers.pitch\_art\_wizard)@\spxentry{audio\_analysis\_image()}\spxextra{in module metilda.controllers.pitch\_art\_wizard}}

\begin{fulllineitems}
\phantomsection\label{\detokenize{metilda.controllers:metilda.controllers.pitch_art_wizard.audio_analysis_image}}
\pysigstartsignatures
\pysiglinewithargsret{\sphinxcode{\sphinxupquote{metilda.controllers.pitch\_art\_wizard.}}\sphinxbfcode{\sphinxupquote{audio\_analysis\_image}}}{\sphinxparam{\DUrole{n,n}{upload\_id}}}{}
\pysigstopsignatures
\end{fulllineitems}

\index{authorize\_user() (in module metilda.controllers.pitch\_art\_wizard)@\spxentry{authorize\_user()}\spxextra{in module metilda.controllers.pitch\_art\_wizard}}

\begin{fulllineitems}
\phantomsection\label{\detokenize{metilda.controllers:metilda.controllers.pitch_art_wizard.authorize_user}}
\pysigstartsignatures
\pysiglinewithargsret{\sphinxcode{\sphinxupquote{metilda.controllers.pitch\_art\_wizard.}}\sphinxbfcode{\sphinxupquote{authorize\_user}}}{}{}
\pysigstopsignatures
\end{fulllineitems}

\index{available\_files() (in module metilda.controllers.pitch\_art\_wizard)@\spxentry{available\_files()}\spxextra{in module metilda.controllers.pitch\_art\_wizard}}

\begin{fulllineitems}
\phantomsection\label{\detokenize{metilda.controllers:metilda.controllers.pitch_art_wizard.available_files}}
\pysigstartsignatures
\pysiglinewithargsret{\sphinxcode{\sphinxupquote{metilda.controllers.pitch\_art\_wizard.}}\sphinxbfcode{\sphinxupquote{available\_files}}}{}{}
\pysigstopsignatures
\end{fulllineitems}

\index{avg\_pitch() (in module metilda.controllers.pitch\_art\_wizard)@\spxentry{avg\_pitch()}\spxextra{in module metilda.controllers.pitch\_art\_wizard}}

\begin{fulllineitems}
\phantomsection\label{\detokenize{metilda.controllers:metilda.controllers.pitch_art_wizard.avg_pitch}}
\pysigstartsignatures
\pysiglinewithargsret{\sphinxcode{\sphinxupquote{metilda.controllers.pitch\_art\_wizard.}}\sphinxbfcode{\sphinxupquote{avg\_pitch}}}{\sphinxparam{\DUrole{n,n}{upload\_id}}}{}
\pysigstopsignatures
\end{fulllineitems}

\index{beta() (in module metilda.controllers.pitch\_art\_wizard)@\spxentry{beta()}\spxextra{in module metilda.controllers.pitch\_art\_wizard}}

\begin{fulllineitems}
\phantomsection\label{\detokenize{metilda.controllers:metilda.controllers.pitch_art_wizard.beta}}
\pysigstartsignatures
\pysiglinewithargsret{\sphinxcode{\sphinxupquote{metilda.controllers.pitch\_art\_wizard.}}\sphinxbfcode{\sphinxupquote{beta}}}{\sphinxparam{\DUrole{n,n}{a}}\sphinxparamcomma \sphinxparam{\DUrole{n,n}{b}}\sphinxparamcomma \sphinxparam{\DUrole{n,n}{size}\DUrole{o,o}{=}\DUrole{default_value}{None}}}{}
\pysigstopsignatures
\sphinxAtStartPar
Draw samples from a Beta distribution.

\sphinxAtStartPar
The Beta distribution is a special case of the Dirichlet distribution,
and is related to the Gamma distribution.  It has the probability
distribution function
\begin{equation*}
\begin{split}f(x; a,b) = \frac{1}{B(\alpha, \beta)} x^{\alpha - 1}
(1 - x)^{\beta - 1},\end{split}
\end{equation*}
\sphinxAtStartPar
where the normalization, B, is the beta function,
\begin{equation*}
\begin{split}B(\alpha, \beta) = \int_0^1 t^{\alpha - 1}
(1 - t)^{\beta - 1} dt.\end{split}
\end{equation*}
\sphinxAtStartPar
It is often seen in Bayesian inference and order statistics.

\begin{sphinxadmonition}{note}{Note:}
\sphinxAtStartPar
New code should use the \sphinxtitleref{\textasciitilde{}numpy.random.Generator.beta}
method of a \sphinxtitleref{\textasciitilde{}numpy.random.Generator} instance instead;
please see the \DUrole{xref,std,std-ref}{random\sphinxhyphen{}quick\sphinxhyphen{}start}.
\end{sphinxadmonition}
\begin{quote}\begin{description}
\sphinxlineitem{Parameters}\begin{itemize}
\item {} 
\sphinxAtStartPar
\sphinxstyleliteralstrong{\sphinxupquote{a}} (\sphinxstyleliteralemphasis{\sphinxupquote{float}}\sphinxstyleliteralemphasis{\sphinxupquote{ or }}\sphinxstyleliteralemphasis{\sphinxupquote{array\_like}}\sphinxstyleliteralemphasis{\sphinxupquote{ of }}\sphinxstyleliteralemphasis{\sphinxupquote{floats}}) \textendash{} Alpha, positive (\textgreater{}0).

\item {} 
\sphinxAtStartPar
\sphinxstyleliteralstrong{\sphinxupquote{b}} (\sphinxstyleliteralemphasis{\sphinxupquote{float}}\sphinxstyleliteralemphasis{\sphinxupquote{ or }}\sphinxstyleliteralemphasis{\sphinxupquote{array\_like}}\sphinxstyleliteralemphasis{\sphinxupquote{ of }}\sphinxstyleliteralemphasis{\sphinxupquote{floats}}) \textendash{} Beta, positive (\textgreater{}0).

\item {} 
\sphinxAtStartPar
\sphinxstyleliteralstrong{\sphinxupquote{size}} (\sphinxstyleliteralemphasis{\sphinxupquote{int}}\sphinxstyleliteralemphasis{\sphinxupquote{ or }}\sphinxstyleliteralemphasis{\sphinxupquote{tuple}}\sphinxstyleliteralemphasis{\sphinxupquote{ of }}\sphinxstyleliteralemphasis{\sphinxupquote{ints}}\sphinxstyleliteralemphasis{\sphinxupquote{, }}\sphinxstyleliteralemphasis{\sphinxupquote{optional}}) \textendash{} Output shape.  If the given shape is, e.g., \sphinxcode{\sphinxupquote{(m, n, k)}}, then
\sphinxcode{\sphinxupquote{m * n * k}} samples are drawn.  If size is \sphinxcode{\sphinxupquote{None}} (default),
a single value is returned if \sphinxcode{\sphinxupquote{a}} and \sphinxcode{\sphinxupquote{b}} are both scalars.
Otherwise, \sphinxcode{\sphinxupquote{np.broadcast(a, b).size}} samples are drawn.

\end{itemize}

\sphinxlineitem{Returns}
\sphinxAtStartPar
\sphinxstylestrong{out} \textendash{} Drawn samples from the parameterized beta distribution.

\sphinxlineitem{Return type}
\sphinxAtStartPar
ndarray or scalar

\end{description}\end{quote}


\begin{sphinxseealso}{See also:}
\begin{description}
\sphinxlineitem{\sphinxcode{\sphinxupquote{random.Generator.beta}}}
\sphinxAtStartPar
which should be used for new code.

\end{description}


\end{sphinxseealso}


\end{fulllineitems}

\index{binomial() (in module metilda.controllers.pitch\_art\_wizard)@\spxentry{binomial()}\spxextra{in module metilda.controllers.pitch\_art\_wizard}}

\begin{fulllineitems}
\phantomsection\label{\detokenize{metilda.controllers:metilda.controllers.pitch_art_wizard.binomial}}
\pysigstartsignatures
\pysiglinewithargsret{\sphinxcode{\sphinxupquote{metilda.controllers.pitch\_art\_wizard.}}\sphinxbfcode{\sphinxupquote{binomial}}}{\sphinxparam{\DUrole{n,n}{n}}\sphinxparamcomma \sphinxparam{\DUrole{n,n}{p}}\sphinxparamcomma \sphinxparam{\DUrole{n,n}{size}\DUrole{o,o}{=}\DUrole{default_value}{None}}}{}
\pysigstopsignatures
\sphinxAtStartPar
Draw samples from a binomial distribution.

\sphinxAtStartPar
Samples are drawn from a binomial distribution with specified
parameters, n trials and p probability of success where
n an integer \textgreater{}= 0 and p is in the interval {[}0,1{]}. (n may be
input as a float, but it is truncated to an integer in use)

\begin{sphinxadmonition}{note}{Note:}
\sphinxAtStartPar
New code should use the \sphinxtitleref{\textasciitilde{}numpy.random.Generator.binomial}
method of a \sphinxtitleref{\textasciitilde{}numpy.random.Generator} instance instead;
please see the \DUrole{xref,std,std-ref}{random\sphinxhyphen{}quick\sphinxhyphen{}start}.
\end{sphinxadmonition}
\begin{quote}\begin{description}
\sphinxlineitem{Parameters}\begin{itemize}
\item {} 
\sphinxAtStartPar
\sphinxstyleliteralstrong{\sphinxupquote{n}} (\sphinxstyleliteralemphasis{\sphinxupquote{int}}\sphinxstyleliteralemphasis{\sphinxupquote{ or }}\sphinxstyleliteralemphasis{\sphinxupquote{array\_like}}\sphinxstyleliteralemphasis{\sphinxupquote{ of }}\sphinxstyleliteralemphasis{\sphinxupquote{ints}}) \textendash{} Parameter of the distribution, \textgreater{}= 0. Floats are also accepted,
but they will be truncated to integers.

\item {} 
\sphinxAtStartPar
\sphinxstyleliteralstrong{\sphinxupquote{p}} (\sphinxstyleliteralemphasis{\sphinxupquote{float}}\sphinxstyleliteralemphasis{\sphinxupquote{ or }}\sphinxstyleliteralemphasis{\sphinxupquote{array\_like}}\sphinxstyleliteralemphasis{\sphinxupquote{ of }}\sphinxstyleliteralemphasis{\sphinxupquote{floats}}) \textendash{} Parameter of the distribution, \textgreater{}= 0 and \textless{}=1.

\item {} 
\sphinxAtStartPar
\sphinxstyleliteralstrong{\sphinxupquote{size}} (\sphinxstyleliteralemphasis{\sphinxupquote{int}}\sphinxstyleliteralemphasis{\sphinxupquote{ or }}\sphinxstyleliteralemphasis{\sphinxupquote{tuple}}\sphinxstyleliteralemphasis{\sphinxupquote{ of }}\sphinxstyleliteralemphasis{\sphinxupquote{ints}}\sphinxstyleliteralemphasis{\sphinxupquote{, }}\sphinxstyleliteralemphasis{\sphinxupquote{optional}}) \textendash{} Output shape.  If the given shape is, e.g., \sphinxcode{\sphinxupquote{(m, n, k)}}, then
\sphinxcode{\sphinxupquote{m * n * k}} samples are drawn.  If size is \sphinxcode{\sphinxupquote{None}} (default),
a single value is returned if \sphinxcode{\sphinxupquote{n}} and \sphinxcode{\sphinxupquote{p}} are both scalars.
Otherwise, \sphinxcode{\sphinxupquote{np.broadcast(n, p).size}} samples are drawn.

\end{itemize}

\sphinxlineitem{Returns}
\sphinxAtStartPar
\sphinxstylestrong{out} \textendash{} Drawn samples from the parameterized binomial distribution, where
each sample is equal to the number of successes over the n trials.

\sphinxlineitem{Return type}
\sphinxAtStartPar
ndarray or scalar

\end{description}\end{quote}


\begin{sphinxseealso}{See also:}
\begin{description}
\sphinxlineitem{\sphinxcode{\sphinxupquote{scipy.stats.binom}}}
\sphinxAtStartPar
probability density function, distribution or cumulative density function, etc.

\sphinxlineitem{\sphinxcode{\sphinxupquote{random.Generator.binomial}}}
\sphinxAtStartPar
which should be used for new code.

\end{description}


\end{sphinxseealso}

\subsubsection*{Notes}

\sphinxAtStartPar
The probability density for the binomial distribution is
\begin{equation*}
\begin{split}P(N) = \binom{n}{N}p^N(1-p)^{n-N},\end{split}
\end{equation*}
\sphinxAtStartPar
where \(n\) is the number of trials, \(p\) is the probability
of success, and \(N\) is the number of successes.

\sphinxAtStartPar
When estimating the standard error of a proportion in a population by
using a random sample, the normal distribution works well unless the
product p*n \textless{}=5, where p = population proportion estimate, and n =
number of samples, in which case the binomial distribution is used
instead. For example, a sample of 15 people shows 4 who are left
handed, and 11 who are right handed. Then p = 4/15 = 27\%. 0.27*15 = 4,
so the binomial distribution should be used in this case.
\subsubsection*{References}
\subsubsection*{Examples}

\sphinxAtStartPar
Draw samples from the distribution:

\begin{sphinxVerbatim}[commandchars=\\\{\}]
\PYG{g+gp}{\PYGZgt{}\PYGZgt{}\PYGZgt{} }\PYG{n}{n}\PYG{p}{,} \PYG{n}{p} \PYG{o}{=} \PYG{l+m+mi}{10}\PYG{p}{,} \PYG{l+m+mf}{.5}  \PYG{c+c1}{\PYGZsh{} number of trials, probability of each trial}
\PYG{g+gp}{\PYGZgt{}\PYGZgt{}\PYGZgt{} }\PYG{n}{s} \PYG{o}{=} \PYG{n}{np}\PYG{o}{.}\PYG{n}{random}\PYG{o}{.}\PYG{n}{binomial}\PYG{p}{(}\PYG{n}{n}\PYG{p}{,} \PYG{n}{p}\PYG{p}{,} \PYG{l+m+mi}{1000}\PYG{p}{)}
\PYG{g+go}{\PYGZsh{} result of flipping a coin 10 times, tested 1000 times.}
\end{sphinxVerbatim}

\sphinxAtStartPar
A real world example. A company drills 9 wild\sphinxhyphen{}cat oil exploration
wells, each with an estimated probability of success of 0.1. All nine
wells fail. What is the probability of that happening?

\sphinxAtStartPar
Let’s do 20,000 trials of the model, and count the number that
generate zero positive results.

\begin{sphinxVerbatim}[commandchars=\\\{\}]
\PYG{g+gp}{\PYGZgt{}\PYGZgt{}\PYGZgt{} }\PYG{n+nb}{sum}\PYG{p}{(}\PYG{n}{np}\PYG{o}{.}\PYG{n}{random}\PYG{o}{.}\PYG{n}{binomial}\PYG{p}{(}\PYG{l+m+mi}{9}\PYG{p}{,} \PYG{l+m+mf}{0.1}\PYG{p}{,} \PYG{l+m+mi}{20000}\PYG{p}{)} \PYG{o}{==} \PYG{l+m+mi}{0}\PYG{p}{)}\PYG{o}{/}\PYG{l+m+mf}{20000.}
\PYG{g+go}{\PYGZsh{} answer = 0.38885, or 38\PYGZpc{}.}
\end{sphinxVerbatim}

\end{fulllineitems}

\index{chisquare() (in module metilda.controllers.pitch\_art\_wizard)@\spxentry{chisquare()}\spxextra{in module metilda.controllers.pitch\_art\_wizard}}

\begin{fulllineitems}
\phantomsection\label{\detokenize{metilda.controllers:metilda.controllers.pitch_art_wizard.chisquare}}
\pysigstartsignatures
\pysiglinewithargsret{\sphinxcode{\sphinxupquote{metilda.controllers.pitch\_art\_wizard.}}\sphinxbfcode{\sphinxupquote{chisquare}}}{\sphinxparam{\DUrole{n,n}{df}}\sphinxparamcomma \sphinxparam{\DUrole{n,n}{size}\DUrole{o,o}{=}\DUrole{default_value}{None}}}{}
\pysigstopsignatures
\sphinxAtStartPar
Draw samples from a chi\sphinxhyphen{}square distribution.

\sphinxAtStartPar
When \sphinxtitleref{df} independent random variables, each with standard normal
distributions (mean 0, variance 1), are squared and summed, the
resulting distribution is chi\sphinxhyphen{}square (see Notes).  This distribution
is often used in hypothesis testing.

\begin{sphinxadmonition}{note}{Note:}
\sphinxAtStartPar
New code should use the \sphinxtitleref{\textasciitilde{}numpy.random.Generator.chisquare}
method of a \sphinxtitleref{\textasciitilde{}numpy.random.Generator} instance instead;
please see the \DUrole{xref,std,std-ref}{random\sphinxhyphen{}quick\sphinxhyphen{}start}.
\end{sphinxadmonition}
\begin{quote}\begin{description}
\sphinxlineitem{Parameters}\begin{itemize}
\item {} 
\sphinxAtStartPar
\sphinxstyleliteralstrong{\sphinxupquote{df}} (\sphinxstyleliteralemphasis{\sphinxupquote{float}}\sphinxstyleliteralemphasis{\sphinxupquote{ or }}\sphinxstyleliteralemphasis{\sphinxupquote{array\_like}}\sphinxstyleliteralemphasis{\sphinxupquote{ of }}\sphinxstyleliteralemphasis{\sphinxupquote{floats}}) \textendash{} Number of degrees of freedom, must be \textgreater{} 0.

\item {} 
\sphinxAtStartPar
\sphinxstyleliteralstrong{\sphinxupquote{size}} (\sphinxstyleliteralemphasis{\sphinxupquote{int}}\sphinxstyleliteralemphasis{\sphinxupquote{ or }}\sphinxstyleliteralemphasis{\sphinxupquote{tuple}}\sphinxstyleliteralemphasis{\sphinxupquote{ of }}\sphinxstyleliteralemphasis{\sphinxupquote{ints}}\sphinxstyleliteralemphasis{\sphinxupquote{, }}\sphinxstyleliteralemphasis{\sphinxupquote{optional}}) \textendash{} Output shape.  If the given shape is, e.g., \sphinxcode{\sphinxupquote{(m, n, k)}}, then
\sphinxcode{\sphinxupquote{m * n * k}} samples are drawn.  If size is \sphinxcode{\sphinxupquote{None}} (default),
a single value is returned if \sphinxcode{\sphinxupquote{df}} is a scalar.  Otherwise,
\sphinxcode{\sphinxupquote{np.array(df).size}} samples are drawn.

\end{itemize}

\sphinxlineitem{Returns}
\sphinxAtStartPar
\sphinxstylestrong{out} \textendash{} Drawn samples from the parameterized chi\sphinxhyphen{}square distribution.

\sphinxlineitem{Return type}
\sphinxAtStartPar
ndarray or scalar

\sphinxlineitem{Raises}
\sphinxAtStartPar
\sphinxstyleliteralstrong{\sphinxupquote{ValueError}} \textendash{} When \sphinxtitleref{df} \textless{}= 0 or when an inappropriate \sphinxtitleref{size} (e.g. \sphinxcode{\sphinxupquote{size=\sphinxhyphen{}1}})
    is given.

\end{description}\end{quote}


\begin{sphinxseealso}{See also:}
\begin{description}
\sphinxlineitem{\sphinxcode{\sphinxupquote{random.Generator.chisquare}}}
\sphinxAtStartPar
which should be used for new code.

\end{description}


\end{sphinxseealso}

\subsubsection*{Notes}

\sphinxAtStartPar
The variable obtained by summing the squares of \sphinxtitleref{df} independent,
standard normally distributed random variables:
\begin{equation*}
\begin{split}Q = \sum_{i=0}^{\mathtt{df}} X^2_i\end{split}
\end{equation*}
\sphinxAtStartPar
is chi\sphinxhyphen{}square distributed, denoted
\begin{equation*}
\begin{split}Q \sim \chi^2_k.\end{split}
\end{equation*}
\sphinxAtStartPar
The probability density function of the chi\sphinxhyphen{}squared distribution is
\begin{equation*}
\begin{split}p(x) = \frac{(1/2)^{k/2}}{\Gamma(k/2)}
x^{k/2 - 1} e^{-x/2},\end{split}
\end{equation*}
\sphinxAtStartPar
where \(\Gamma\) is the gamma function,
\begin{equation*}
\begin{split}\Gamma(x) = \int_0^{-\infty} t^{x - 1} e^{-t} dt.\end{split}
\end{equation*}\subsubsection*{References}
\subsubsection*{Examples}

\begin{sphinxVerbatim}[commandchars=\\\{\}]
\PYG{g+gp}{\PYGZgt{}\PYGZgt{}\PYGZgt{} }\PYG{n}{np}\PYG{o}{.}\PYG{n}{random}\PYG{o}{.}\PYG{n}{chisquare}\PYG{p}{(}\PYG{l+m+mi}{2}\PYG{p}{,}\PYG{l+m+mi}{4}\PYG{p}{)}
\PYG{g+go}{array([ 1.89920014,  9.00867716,  3.13710533,  5.62318272]) \PYGZsh{} random}
\end{sphinxVerbatim}

\end{fulllineitems}

\index{choice() (in module metilda.controllers.pitch\_art\_wizard)@\spxentry{choice()}\spxextra{in module metilda.controllers.pitch\_art\_wizard}}

\begin{fulllineitems}
\phantomsection\label{\detokenize{metilda.controllers:metilda.controllers.pitch_art_wizard.choice}}
\pysigstartsignatures
\pysiglinewithargsret{\sphinxcode{\sphinxupquote{metilda.controllers.pitch\_art\_wizard.}}\sphinxbfcode{\sphinxupquote{choice}}}{\sphinxparam{\DUrole{n,n}{a}}\sphinxparamcomma \sphinxparam{\DUrole{n,n}{size}\DUrole{o,o}{=}\DUrole{default_value}{None}}\sphinxparamcomma \sphinxparam{\DUrole{n,n}{replace}\DUrole{o,o}{=}\DUrole{default_value}{True}}\sphinxparamcomma \sphinxparam{\DUrole{n,n}{p}\DUrole{o,o}{=}\DUrole{default_value}{None}}}{}
\pysigstopsignatures
\sphinxAtStartPar
Generates a random sample from a given 1\sphinxhyphen{}D array

\sphinxAtStartPar
\DUrole{versionmodified,added}{New in version 1.7.0.}

\begin{sphinxadmonition}{note}{Note:}
\sphinxAtStartPar
New code should use the \sphinxtitleref{\textasciitilde{}numpy.random.Generator.choice}
method of a \sphinxtitleref{\textasciitilde{}numpy.random.Generator} instance instead;
please see the \DUrole{xref,std,std-ref}{random\sphinxhyphen{}quick\sphinxhyphen{}start}.
\end{sphinxadmonition}
\begin{quote}\begin{description}
\sphinxlineitem{Parameters}\begin{itemize}
\item {} 
\sphinxAtStartPar
\sphinxstyleliteralstrong{\sphinxupquote{a}} (\sphinxstyleliteralemphasis{\sphinxupquote{1\sphinxhyphen{}D array\sphinxhyphen{}like}}\sphinxstyleliteralemphasis{\sphinxupquote{ or }}\sphinxstyleliteralemphasis{\sphinxupquote{int}}) \textendash{} If an ndarray, a random sample is generated from its elements.
If an int, the random sample is generated as if it were \sphinxcode{\sphinxupquote{np.arange(a)}}

\item {} 
\sphinxAtStartPar
\sphinxstyleliteralstrong{\sphinxupquote{size}} (\sphinxstyleliteralemphasis{\sphinxupquote{int}}\sphinxstyleliteralemphasis{\sphinxupquote{ or }}\sphinxstyleliteralemphasis{\sphinxupquote{tuple}}\sphinxstyleliteralemphasis{\sphinxupquote{ of }}\sphinxstyleliteralemphasis{\sphinxupquote{ints}}\sphinxstyleliteralemphasis{\sphinxupquote{, }}\sphinxstyleliteralemphasis{\sphinxupquote{optional}}) \textendash{} Output shape.  If the given shape is, e.g., \sphinxcode{\sphinxupquote{(m, n, k)}}, then
\sphinxcode{\sphinxupquote{m * n * k}} samples are drawn.  Default is None, in which case a
single value is returned.

\item {} 
\sphinxAtStartPar
\sphinxstyleliteralstrong{\sphinxupquote{replace}} (\sphinxstyleliteralemphasis{\sphinxupquote{boolean}}\sphinxstyleliteralemphasis{\sphinxupquote{, }}\sphinxstyleliteralemphasis{\sphinxupquote{optional}}) \textendash{} Whether the sample is with or without replacement. Default is True,
meaning that a value of \sphinxcode{\sphinxupquote{a}} can be selected multiple times.

\item {} 
\sphinxAtStartPar
\sphinxstyleliteralstrong{\sphinxupquote{p}} (\sphinxstyleliteralemphasis{\sphinxupquote{1\sphinxhyphen{}D array\sphinxhyphen{}like}}\sphinxstyleliteralemphasis{\sphinxupquote{, }}\sphinxstyleliteralemphasis{\sphinxupquote{optional}}) \textendash{} The probabilities associated with each entry in a.
If not given, the sample assumes a uniform distribution over all
entries in \sphinxcode{\sphinxupquote{a}}.

\end{itemize}

\sphinxlineitem{Returns}
\sphinxAtStartPar
\sphinxstylestrong{samples} \textendash{} The generated random samples

\sphinxlineitem{Return type}
\sphinxAtStartPar
single item or ndarray

\sphinxlineitem{Raises}
\sphinxAtStartPar
\sphinxstyleliteralstrong{\sphinxupquote{ValueError}} \textendash{} If a is an int and less than zero, if a or p are not 1\sphinxhyphen{}dimensional,
    if a is an array\sphinxhyphen{}like of size 0, if p is not a vector of
    probabilities, if a and p have different lengths, or if
    replace=False and the sample size is greater than the population
    size

\end{description}\end{quote}


\begin{sphinxseealso}{See also:}

\sphinxAtStartPar
{\hyperref[\detokenize{metilda.controllers:metilda.controllers.pitch_art_wizard.randint}]{\sphinxcrossref{\sphinxcode{\sphinxupquote{randint}}}}}, {\hyperref[\detokenize{metilda.controllers:metilda.controllers.pitch_art_wizard.shuffle}]{\sphinxcrossref{\sphinxcode{\sphinxupquote{shuffle}}}}}, {\hyperref[\detokenize{metilda.controllers:metilda.controllers.pitch_art_wizard.permutation}]{\sphinxcrossref{\sphinxcode{\sphinxupquote{permutation}}}}}
\begin{description}
\sphinxlineitem{\sphinxcode{\sphinxupquote{random.Generator.choice}}}
\sphinxAtStartPar
which should be used in new code

\end{description}


\end{sphinxseealso}

\subsubsection*{Notes}

\sphinxAtStartPar
Setting user\sphinxhyphen{}specified probabilities through \sphinxcode{\sphinxupquote{p}} uses a more general but less
efficient sampler than the default. The general sampler produces a different sample
than the optimized sampler even if each element of \sphinxcode{\sphinxupquote{p}} is 1 / len(a).

\sphinxAtStartPar
Sampling random rows from a 2\sphinxhyphen{}D array is not possible with this function,
but is possible with \sphinxtitleref{Generator.choice} through its \sphinxcode{\sphinxupquote{axis}} keyword.
\subsubsection*{Examples}

\sphinxAtStartPar
Generate a uniform random sample from np.arange(5) of size 3:

\begin{sphinxVerbatim}[commandchars=\\\{\}]
\PYG{g+gp}{\PYGZgt{}\PYGZgt{}\PYGZgt{} }\PYG{n}{np}\PYG{o}{.}\PYG{n}{random}\PYG{o}{.}\PYG{n}{choice}\PYG{p}{(}\PYG{l+m+mi}{5}\PYG{p}{,} \PYG{l+m+mi}{3}\PYG{p}{)}
\PYG{g+go}{array([0, 3, 4]) \PYGZsh{} random}
\PYG{g+gp}{\PYGZgt{}\PYGZgt{}\PYGZgt{} }\PYG{c+c1}{\PYGZsh{}This is equivalent to np.random.randint(0,5,3)}
\end{sphinxVerbatim}

\sphinxAtStartPar
Generate a non\sphinxhyphen{}uniform random sample from np.arange(5) of size 3:

\begin{sphinxVerbatim}[commandchars=\\\{\}]
\PYG{g+gp}{\PYGZgt{}\PYGZgt{}\PYGZgt{} }\PYG{n}{np}\PYG{o}{.}\PYG{n}{random}\PYG{o}{.}\PYG{n}{choice}\PYG{p}{(}\PYG{l+m+mi}{5}\PYG{p}{,} \PYG{l+m+mi}{3}\PYG{p}{,} \PYG{n}{p}\PYG{o}{=}\PYG{p}{[}\PYG{l+m+mf}{0.1}\PYG{p}{,} \PYG{l+m+mi}{0}\PYG{p}{,} \PYG{l+m+mf}{0.3}\PYG{p}{,} \PYG{l+m+mf}{0.6}\PYG{p}{,} \PYG{l+m+mi}{0}\PYG{p}{]}\PYG{p}{)}
\PYG{g+go}{array([3, 3, 0]) \PYGZsh{} random}
\end{sphinxVerbatim}

\sphinxAtStartPar
Generate a uniform random sample from np.arange(5) of size 3 without
replacement:

\begin{sphinxVerbatim}[commandchars=\\\{\}]
\PYG{g+gp}{\PYGZgt{}\PYGZgt{}\PYGZgt{} }\PYG{n}{np}\PYG{o}{.}\PYG{n}{random}\PYG{o}{.}\PYG{n}{choice}\PYG{p}{(}\PYG{l+m+mi}{5}\PYG{p}{,} \PYG{l+m+mi}{3}\PYG{p}{,} \PYG{n}{replace}\PYG{o}{=}\PYG{k+kc}{False}\PYG{p}{)}
\PYG{g+go}{array([3,1,0]) \PYGZsh{} random}
\PYG{g+gp}{\PYGZgt{}\PYGZgt{}\PYGZgt{} }\PYG{c+c1}{\PYGZsh{}This is equivalent to np.random.permutation(np.arange(5))[:3]}
\end{sphinxVerbatim}

\sphinxAtStartPar
Generate a non\sphinxhyphen{}uniform random sample from np.arange(5) of size
3 without replacement:

\begin{sphinxVerbatim}[commandchars=\\\{\}]
\PYG{g+gp}{\PYGZgt{}\PYGZgt{}\PYGZgt{} }\PYG{n}{np}\PYG{o}{.}\PYG{n}{random}\PYG{o}{.}\PYG{n}{choice}\PYG{p}{(}\PYG{l+m+mi}{5}\PYG{p}{,} \PYG{l+m+mi}{3}\PYG{p}{,} \PYG{n}{replace}\PYG{o}{=}\PYG{k+kc}{False}\PYG{p}{,} \PYG{n}{p}\PYG{o}{=}\PYG{p}{[}\PYG{l+m+mf}{0.1}\PYG{p}{,} \PYG{l+m+mi}{0}\PYG{p}{,} \PYG{l+m+mf}{0.3}\PYG{p}{,} \PYG{l+m+mf}{0.6}\PYG{p}{,} \PYG{l+m+mi}{0}\PYG{p}{]}\PYG{p}{)}
\PYG{g+go}{array([2, 3, 0]) \PYGZsh{} random}
\end{sphinxVerbatim}

\sphinxAtStartPar
Any of the above can be repeated with an arbitrary array\sphinxhyphen{}like
instead of just integers. For instance:

\begin{sphinxVerbatim}[commandchars=\\\{\}]
\PYG{g+gp}{\PYGZgt{}\PYGZgt{}\PYGZgt{} }\PYG{n}{aa\PYGZus{}milne\PYGZus{}arr} \PYG{o}{=} \PYG{p}{[}\PYG{l+s+s1}{\PYGZsq{}}\PYG{l+s+s1}{pooh}\PYG{l+s+s1}{\PYGZsq{}}\PYG{p}{,} \PYG{l+s+s1}{\PYGZsq{}}\PYG{l+s+s1}{rabbit}\PYG{l+s+s1}{\PYGZsq{}}\PYG{p}{,} \PYG{l+s+s1}{\PYGZsq{}}\PYG{l+s+s1}{piglet}\PYG{l+s+s1}{\PYGZsq{}}\PYG{p}{,} \PYG{l+s+s1}{\PYGZsq{}}\PYG{l+s+s1}{Christopher}\PYG{l+s+s1}{\PYGZsq{}}\PYG{p}{]}
\PYG{g+gp}{\PYGZgt{}\PYGZgt{}\PYGZgt{} }\PYG{n}{np}\PYG{o}{.}\PYG{n}{random}\PYG{o}{.}\PYG{n}{choice}\PYG{p}{(}\PYG{n}{aa\PYGZus{}milne\PYGZus{}arr}\PYG{p}{,} \PYG{l+m+mi}{5}\PYG{p}{,} \PYG{n}{p}\PYG{o}{=}\PYG{p}{[}\PYG{l+m+mf}{0.5}\PYG{p}{,} \PYG{l+m+mf}{0.1}\PYG{p}{,} \PYG{l+m+mf}{0.1}\PYG{p}{,} \PYG{l+m+mf}{0.3}\PYG{p}{]}\PYG{p}{)}
\PYG{g+go}{array([\PYGZsq{}pooh\PYGZsq{}, \PYGZsq{}pooh\PYGZsq{}, \PYGZsq{}pooh\PYGZsq{}, \PYGZsq{}Christopher\PYGZsq{}, \PYGZsq{}piglet\PYGZsq{}], \PYGZsh{} random}
\PYG{g+go}{      dtype=\PYGZsq{}\PYGZlt{}U11\PYGZsq{})}
\end{sphinxVerbatim}

\end{fulllineitems}

\index{countVoicedFrames() (in module metilda.controllers.pitch\_art\_wizard)@\spxentry{countVoicedFrames()}\spxextra{in module metilda.controllers.pitch\_art\_wizard}}

\begin{fulllineitems}
\phantomsection\label{\detokenize{metilda.controllers:metilda.controllers.pitch_art_wizard.countVoicedFrames}}
\pysigstartsignatures
\pysiglinewithargsret{\sphinxcode{\sphinxupquote{metilda.controllers.pitch\_art\_wizard.}}\sphinxbfcode{\sphinxupquote{countVoicedFrames}}}{\sphinxparam{\DUrole{n,n}{sound}}}{}
\pysigstopsignatures
\end{fulllineitems}

\index{create\_analysis() (in module metilda.controllers.pitch\_art\_wizard)@\spxentry{create\_analysis()}\spxextra{in module metilda.controllers.pitch\_art\_wizard}}

\begin{fulllineitems}
\phantomsection\label{\detokenize{metilda.controllers:metilda.controllers.pitch_art_wizard.create_analysis}}
\pysigstartsignatures
\pysiglinewithargsret{\sphinxcode{\sphinxupquote{metilda.controllers.pitch\_art\_wizard.}}\sphinxbfcode{\sphinxupquote{create\_analysis}}}{}{}
\pysigstopsignatures
\end{fulllineitems}

\index{create\_db\_user() (in module metilda.controllers.pitch\_art\_wizard)@\spxentry{create\_db\_user()}\spxextra{in module metilda.controllers.pitch\_art\_wizard}}

\begin{fulllineitems}
\phantomsection\label{\detokenize{metilda.controllers:metilda.controllers.pitch_art_wizard.create_db_user}}
\pysigstartsignatures
\pysiglinewithargsret{\sphinxcode{\sphinxupquote{metilda.controllers.pitch\_art\_wizard.}}\sphinxbfcode{\sphinxupquote{create\_db\_user}}}{}{}
\pysigstopsignatures
\end{fulllineitems}

\index{create\_eaf() (in module metilda.controllers.pitch\_art\_wizard)@\spxentry{create\_eaf()}\spxextra{in module metilda.controllers.pitch\_art\_wizard}}

\begin{fulllineitems}
\phantomsection\label{\detokenize{metilda.controllers:metilda.controllers.pitch_art_wizard.create_eaf}}
\pysigstartsignatures
\pysiglinewithargsret{\sphinxcode{\sphinxupquote{metilda.controllers.pitch\_art\_wizard.}}\sphinxbfcode{\sphinxupquote{create\_eaf}}}{}{}
\pysigstopsignatures
\end{fulllineitems}

\index{create\_file() (in module metilda.controllers.pitch\_art\_wizard)@\spxentry{create\_file()}\spxextra{in module metilda.controllers.pitch\_art\_wizard}}

\begin{fulllineitems}
\phantomsection\label{\detokenize{metilda.controllers:metilda.controllers.pitch_art_wizard.create_file}}
\pysigstartsignatures
\pysiglinewithargsret{\sphinxcode{\sphinxupquote{metilda.controllers.pitch\_art\_wizard.}}\sphinxbfcode{\sphinxupquote{create\_file}}}{}{}
\pysigstopsignatures
\end{fulllineitems}

\index{create\_folder() (in module metilda.controllers.pitch\_art\_wizard)@\spxentry{create\_folder()}\spxextra{in module metilda.controllers.pitch\_art\_wizard}}

\begin{fulllineitems}
\phantomsection\label{\detokenize{metilda.controllers:metilda.controllers.pitch_art_wizard.create_folder}}
\pysigstartsignatures
\pysiglinewithargsret{\sphinxcode{\sphinxupquote{metilda.controllers.pitch\_art\_wizard.}}\sphinxbfcode{\sphinxupquote{create\_folder}}}{}{}
\pysigstopsignatures
\end{fulllineitems}

\index{create\_image() (in module metilda.controllers.pitch\_art\_wizard)@\spxentry{create\_image()}\spxextra{in module metilda.controllers.pitch\_art\_wizard}}

\begin{fulllineitems}
\phantomsection\label{\detokenize{metilda.controllers:metilda.controllers.pitch_art_wizard.create_image}}
\pysigstartsignatures
\pysiglinewithargsret{\sphinxcode{\sphinxupquote{metilda.controllers.pitch\_art\_wizard.}}\sphinxbfcode{\sphinxupquote{create\_image}}}{}{}
\pysigstopsignatures
\end{fulllineitems}

\index{create\_user\_research\_language() (in module metilda.controllers.pitch\_art\_wizard)@\spxentry{create\_user\_research\_language()}\spxextra{in module metilda.controllers.pitch\_art\_wizard}}

\begin{fulllineitems}
\phantomsection\label{\detokenize{metilda.controllers:metilda.controllers.pitch_art_wizard.create_user_research_language}}
\pysigstartsignatures
\pysiglinewithargsret{\sphinxcode{\sphinxupquote{metilda.controllers.pitch\_art\_wizard.}}\sphinxbfcode{\sphinxupquote{create\_user\_research\_language}}}{}{}
\pysigstopsignatures
\end{fulllineitems}

\index{create\_user\_research\_role() (in module metilda.controllers.pitch\_art\_wizard)@\spxentry{create\_user\_research\_role()}\spxextra{in module metilda.controllers.pitch\_art\_wizard}}

\begin{fulllineitems}
\phantomsection\label{\detokenize{metilda.controllers:metilda.controllers.pitch_art_wizard.create_user_research_role}}
\pysigstartsignatures
\pysiglinewithargsret{\sphinxcode{\sphinxupquote{metilda.controllers.pitch\_art\_wizard.}}\sphinxbfcode{\sphinxupquote{create\_user\_research\_role}}}{}{}
\pysigstopsignatures
\end{fulllineitems}

\index{delete\_eaf\_file() (in module metilda.controllers.pitch\_art\_wizard)@\spxentry{delete\_eaf\_file()}\spxextra{in module metilda.controllers.pitch\_art\_wizard}}

\begin{fulllineitems}
\phantomsection\label{\detokenize{metilda.controllers:metilda.controllers.pitch_art_wizard.delete_eaf_file}}
\pysigstartsignatures
\pysiglinewithargsret{\sphinxcode{\sphinxupquote{metilda.controllers.pitch\_art\_wizard.}}\sphinxbfcode{\sphinxupquote{delete\_eaf\_file}}}{}{}
\pysigstopsignatures
\end{fulllineitems}

\index{delete\_file() (in module metilda.controllers.pitch\_art\_wizard)@\spxentry{delete\_file()}\spxextra{in module metilda.controllers.pitch\_art\_wizard}}

\begin{fulllineitems}
\phantomsection\label{\detokenize{metilda.controllers:metilda.controllers.pitch_art_wizard.delete_file}}
\pysigstartsignatures
\pysiglinewithargsret{\sphinxcode{\sphinxupquote{metilda.controllers.pitch\_art\_wizard.}}\sphinxbfcode{\sphinxupquote{delete\_file}}}{}{}
\pysigstopsignatures
\end{fulllineitems}

\index{delete\_folder() (in module metilda.controllers.pitch\_art\_wizard)@\spxentry{delete\_folder()}\spxextra{in module metilda.controllers.pitch\_art\_wizard}}

\begin{fulllineitems}
\phantomsection\label{\detokenize{metilda.controllers:metilda.controllers.pitch_art_wizard.delete_folder}}
\pysigstartsignatures
\pysiglinewithargsret{\sphinxcode{\sphinxupquote{metilda.controllers.pitch\_art\_wizard.}}\sphinxbfcode{\sphinxupquote{delete\_folder}}}{}{}
\pysigstopsignatures
\end{fulllineitems}

\index{delete\_image() (in module metilda.controllers.pitch\_art\_wizard)@\spxentry{delete\_image()}\spxextra{in module metilda.controllers.pitch\_art\_wizard}}

\begin{fulllineitems}
\phantomsection\label{\detokenize{metilda.controllers:metilda.controllers.pitch_art_wizard.delete_image}}
\pysigstartsignatures
\pysiglinewithargsret{\sphinxcode{\sphinxupquote{metilda.controllers.pitch\_art\_wizard.}}\sphinxbfcode{\sphinxupquote{delete\_image}}}{}{}
\pysigstopsignatures
\end{fulllineitems}

\index{delete\_previous\_user\_research\_language() (in module metilda.controllers.pitch\_art\_wizard)@\spxentry{delete\_previous\_user\_research\_language()}\spxextra{in module metilda.controllers.pitch\_art\_wizard}}

\begin{fulllineitems}
\phantomsection\label{\detokenize{metilda.controllers:metilda.controllers.pitch_art_wizard.delete_previous_user_research_language}}
\pysigstartsignatures
\pysiglinewithargsret{\sphinxcode{\sphinxupquote{metilda.controllers.pitch\_art\_wizard.}}\sphinxbfcode{\sphinxupquote{delete\_previous\_user\_research\_language}}}{}{}
\pysigstopsignatures
\end{fulllineitems}

\index{delete\_previous\_user\_roles() (in module metilda.controllers.pitch\_art\_wizard)@\spxentry{delete\_previous\_user\_roles()}\spxextra{in module metilda.controllers.pitch\_art\_wizard}}

\begin{fulllineitems}
\phantomsection\label{\detokenize{metilda.controllers:metilda.controllers.pitch_art_wizard.delete_previous_user_roles}}
\pysigstartsignatures
\pysiglinewithargsret{\sphinxcode{\sphinxupquote{metilda.controllers.pitch\_art\_wizard.}}\sphinxbfcode{\sphinxupquote{delete\_previous\_user\_roles}}}{}{}
\pysigstopsignatures
\end{fulllineitems}

\index{delete\_recording() (in module metilda.controllers.pitch\_art\_wizard)@\spxentry{delete\_recording()}\spxextra{in module metilda.controllers.pitch\_art\_wizard}}

\begin{fulllineitems}
\phantomsection\label{\detokenize{metilda.controllers:metilda.controllers.pitch_art_wizard.delete_recording}}
\pysigstartsignatures
\pysiglinewithargsret{\sphinxcode{\sphinxupquote{metilda.controllers.pitch\_art\_wizard.}}\sphinxbfcode{\sphinxupquote{delete\_recording}}}{}{}
\pysigstopsignatures
\end{fulllineitems}

\index{delete\_shared\_user() (in module metilda.controllers.pitch\_art\_wizard)@\spxentry{delete\_shared\_user()}\spxextra{in module metilda.controllers.pitch\_art\_wizard}}

\begin{fulllineitems}
\phantomsection\label{\detokenize{metilda.controllers:metilda.controllers.pitch_art_wizard.delete_shared_user}}
\pysigstartsignatures
\pysiglinewithargsret{\sphinxcode{\sphinxupquote{metilda.controllers.pitch\_art\_wizard.}}\sphinxbfcode{\sphinxupquote{delete\_shared\_user}}}{}{}
\pysigstopsignatures
\end{fulllineitems}

\index{delete\_user() (in module metilda.controllers.pitch\_art\_wizard)@\spxentry{delete\_user()}\spxextra{in module metilda.controllers.pitch\_art\_wizard}}

\begin{fulllineitems}
\phantomsection\label{\detokenize{metilda.controllers:metilda.controllers.pitch_art_wizard.delete_user}}
\pysigstartsignatures
\pysiglinewithargsret{\sphinxcode{\sphinxupquote{metilda.controllers.pitch\_art\_wizard.}}\sphinxbfcode{\sphinxupquote{delete\_user}}}{}{}
\pysigstopsignatures
\end{fulllineitems}

\index{dirichlet() (in module metilda.controllers.pitch\_art\_wizard)@\spxentry{dirichlet()}\spxextra{in module metilda.controllers.pitch\_art\_wizard}}

\begin{fulllineitems}
\phantomsection\label{\detokenize{metilda.controllers:metilda.controllers.pitch_art_wizard.dirichlet}}
\pysigstartsignatures
\pysiglinewithargsret{\sphinxcode{\sphinxupquote{metilda.controllers.pitch\_art\_wizard.}}\sphinxbfcode{\sphinxupquote{dirichlet}}}{\sphinxparam{\DUrole{n,n}{alpha}}\sphinxparamcomma \sphinxparam{\DUrole{n,n}{size}\DUrole{o,o}{=}\DUrole{default_value}{None}}}{}
\pysigstopsignatures
\sphinxAtStartPar
Draw samples from the Dirichlet distribution.

\sphinxAtStartPar
Draw \sphinxtitleref{size} samples of dimension k from a Dirichlet distribution. A
Dirichlet\sphinxhyphen{}distributed random variable can be seen as a multivariate
generalization of a Beta distribution. The Dirichlet distribution
is a conjugate prior of a multinomial distribution in Bayesian
inference.

\begin{sphinxadmonition}{note}{Note:}
\sphinxAtStartPar
New code should use the \sphinxtitleref{\textasciitilde{}numpy.random.Generator.dirichlet}
method of a \sphinxtitleref{\textasciitilde{}numpy.random.Generator} instance instead;
please see the \DUrole{xref,std,std-ref}{random\sphinxhyphen{}quick\sphinxhyphen{}start}.
\end{sphinxadmonition}
\begin{quote}\begin{description}
\sphinxlineitem{Parameters}\begin{itemize}
\item {} 
\sphinxAtStartPar
\sphinxstyleliteralstrong{\sphinxupquote{alpha}} (\sphinxstyleliteralemphasis{\sphinxupquote{sequence}}\sphinxstyleliteralemphasis{\sphinxupquote{ of }}\sphinxstyleliteralemphasis{\sphinxupquote{floats}}\sphinxstyleliteralemphasis{\sphinxupquote{, }}\sphinxstyleliteralemphasis{\sphinxupquote{length k}}) \textendash{} Parameter of the distribution (length \sphinxcode{\sphinxupquote{k}} for sample of
length \sphinxcode{\sphinxupquote{k}}).

\item {} 
\sphinxAtStartPar
\sphinxstyleliteralstrong{\sphinxupquote{size}} (\sphinxstyleliteralemphasis{\sphinxupquote{int}}\sphinxstyleliteralemphasis{\sphinxupquote{ or }}\sphinxstyleliteralemphasis{\sphinxupquote{tuple}}\sphinxstyleliteralemphasis{\sphinxupquote{ of }}\sphinxstyleliteralemphasis{\sphinxupquote{ints}}\sphinxstyleliteralemphasis{\sphinxupquote{, }}\sphinxstyleliteralemphasis{\sphinxupquote{optional}}) \textendash{} Output shape.  If the given shape is, e.g., \sphinxcode{\sphinxupquote{(m, n)}}, then
\sphinxcode{\sphinxupquote{m * n * k}} samples are drawn.  Default is None, in which case a
vector of length \sphinxcode{\sphinxupquote{k}} is returned.

\end{itemize}

\sphinxlineitem{Returns}
\sphinxAtStartPar
\sphinxstylestrong{samples} \textendash{} The drawn samples, of shape \sphinxcode{\sphinxupquote{(size, k)}}.

\sphinxlineitem{Return type}
\sphinxAtStartPar
ndarray,

\sphinxlineitem{Raises}
\sphinxAtStartPar
\sphinxstyleliteralstrong{\sphinxupquote{ValueError}} \textendash{} If any value in \sphinxcode{\sphinxupquote{alpha}} is less than or equal to zero

\end{description}\end{quote}


\begin{sphinxseealso}{See also:}
\begin{description}
\sphinxlineitem{\sphinxcode{\sphinxupquote{random.Generator.dirichlet}}}
\sphinxAtStartPar
which should be used for new code.

\end{description}


\end{sphinxseealso}

\subsubsection*{Notes}

\sphinxAtStartPar
The Dirichlet distribution is a distribution over vectors
\(x\) that fulfil the conditions \(x_i>0\) and
\(\sum_{i=1}^k x_i = 1\).

\sphinxAtStartPar
The probability density function \(p\) of a
Dirichlet\sphinxhyphen{}distributed random vector \(X\) is
proportional to
\begin{equation*}
\begin{split}p(x) \propto \prod_{i=1}^{k}{x^{\alpha_i-1}_i},\end{split}
\end{equation*}
\sphinxAtStartPar
where \(\alpha\) is a vector containing the positive
concentration parameters.

\sphinxAtStartPar
The method uses the following property for computation: let \(Y\)
be a random vector which has components that follow a standard gamma
distribution, then \(X = \frac{1}{\sum_{i=1}^k{Y_i}} Y\)
is Dirichlet\sphinxhyphen{}distributed
\subsubsection*{References}
\subsubsection*{Examples}

\sphinxAtStartPar
Taking an example cited in Wikipedia, this distribution can be used if
one wanted to cut strings (each of initial length 1.0) into K pieces
with different lengths, where each piece had, on average, a designated
average length, but allowing some variation in the relative sizes of
the pieces.

\begin{sphinxVerbatim}[commandchars=\\\{\}]
\PYG{g+gp}{\PYGZgt{}\PYGZgt{}\PYGZgt{} }\PYG{n}{s} \PYG{o}{=} \PYG{n}{np}\PYG{o}{.}\PYG{n}{random}\PYG{o}{.}\PYG{n}{dirichlet}\PYG{p}{(}\PYG{p}{(}\PYG{l+m+mi}{10}\PYG{p}{,} \PYG{l+m+mi}{5}\PYG{p}{,} \PYG{l+m+mi}{3}\PYG{p}{)}\PYG{p}{,} \PYG{l+m+mi}{20}\PYG{p}{)}\PYG{o}{.}\PYG{n}{transpose}\PYG{p}{(}\PYG{p}{)}
\end{sphinxVerbatim}

\begin{sphinxVerbatim}[commandchars=\\\{\}]
\PYG{g+gp}{\PYGZgt{}\PYGZgt{}\PYGZgt{} }\PYG{k+kn}{import} \PYG{n+nn}{matplotlib}\PYG{n+nn}{.}\PYG{n+nn}{pyplot} \PYG{k}{as} \PYG{n+nn}{plt}
\PYG{g+gp}{\PYGZgt{}\PYGZgt{}\PYGZgt{} }\PYG{n}{plt}\PYG{o}{.}\PYG{n}{barh}\PYG{p}{(}\PYG{n+nb}{range}\PYG{p}{(}\PYG{l+m+mi}{20}\PYG{p}{)}\PYG{p}{,} \PYG{n}{s}\PYG{p}{[}\PYG{l+m+mi}{0}\PYG{p}{]}\PYG{p}{)}
\PYG{g+gp}{\PYGZgt{}\PYGZgt{}\PYGZgt{} }\PYG{n}{plt}\PYG{o}{.}\PYG{n}{barh}\PYG{p}{(}\PYG{n+nb}{range}\PYG{p}{(}\PYG{l+m+mi}{20}\PYG{p}{)}\PYG{p}{,} \PYG{n}{s}\PYG{p}{[}\PYG{l+m+mi}{1}\PYG{p}{]}\PYG{p}{,} \PYG{n}{left}\PYG{o}{=}\PYG{n}{s}\PYG{p}{[}\PYG{l+m+mi}{0}\PYG{p}{]}\PYG{p}{,} \PYG{n}{color}\PYG{o}{=}\PYG{l+s+s1}{\PYGZsq{}}\PYG{l+s+s1}{g}\PYG{l+s+s1}{\PYGZsq{}}\PYG{p}{)}
\PYG{g+gp}{\PYGZgt{}\PYGZgt{}\PYGZgt{} }\PYG{n}{plt}\PYG{o}{.}\PYG{n}{barh}\PYG{p}{(}\PYG{n+nb}{range}\PYG{p}{(}\PYG{l+m+mi}{20}\PYG{p}{)}\PYG{p}{,} \PYG{n}{s}\PYG{p}{[}\PYG{l+m+mi}{2}\PYG{p}{]}\PYG{p}{,} \PYG{n}{left}\PYG{o}{=}\PYG{n}{s}\PYG{p}{[}\PYG{l+m+mi}{0}\PYG{p}{]}\PYG{o}{+}\PYG{n}{s}\PYG{p}{[}\PYG{l+m+mi}{1}\PYG{p}{]}\PYG{p}{,} \PYG{n}{color}\PYG{o}{=}\PYG{l+s+s1}{\PYGZsq{}}\PYG{l+s+s1}{r}\PYG{l+s+s1}{\PYGZsq{}}\PYG{p}{)}
\PYG{g+gp}{\PYGZgt{}\PYGZgt{}\PYGZgt{} }\PYG{n}{plt}\PYG{o}{.}\PYG{n}{title}\PYG{p}{(}\PYG{l+s+s2}{\PYGZdq{}}\PYG{l+s+s2}{Lengths of Strings}\PYG{l+s+s2}{\PYGZdq{}}\PYG{p}{)}
\end{sphinxVerbatim}

\end{fulllineitems}

\index{download\_file() (in module metilda.controllers.pitch\_art\_wizard)@\spxentry{download\_file()}\spxextra{in module metilda.controllers.pitch\_art\_wizard}}

\begin{fulllineitems}
\phantomsection\label{\detokenize{metilda.controllers:metilda.controllers.pitch_art_wizard.download_file}}
\pysigstartsignatures
\pysiglinewithargsret{\sphinxcode{\sphinxupquote{metilda.controllers.pitch\_art\_wizard.}}\sphinxbfcode{\sphinxupquote{download\_file}}}{}{}
\pysigstopsignatures
\end{fulllineitems}

\index{drawSound() (in module metilda.controllers.pitch\_art\_wizard)@\spxentry{drawSound()}\spxextra{in module metilda.controllers.pitch\_art\_wizard}}

\begin{fulllineitems}
\phantomsection\label{\detokenize{metilda.controllers:metilda.controllers.pitch_art_wizard.drawSound}}
\pysigstartsignatures
\pysiglinewithargsret{\sphinxcode{\sphinxupquote{metilda.controllers.pitch\_art\_wizard.}}\sphinxbfcode{\sphinxupquote{drawSound}}}{\sphinxparam{\DUrole{n,n}{upload\_id}}}{}
\pysigstopsignatures
\end{fulllineitems}

\index{drawSoundWithTime() (in module metilda.controllers.pitch\_art\_wizard)@\spxentry{drawSoundWithTime()}\spxextra{in module metilda.controllers.pitch\_art\_wizard}}

\begin{fulllineitems}
\phantomsection\label{\detokenize{metilda.controllers:metilda.controllers.pitch_art_wizard.drawSoundWithTime}}
\pysigstartsignatures
\pysiglinewithargsret{\sphinxcode{\sphinxupquote{metilda.controllers.pitch\_art\_wizard.}}\sphinxbfcode{\sphinxupquote{drawSoundWithTime}}}{\sphinxparam{\DUrole{n,n}{sound}}\sphinxparamcomma \sphinxparam{\DUrole{n,n}{startTime}}\sphinxparamcomma \sphinxparam{\DUrole{n,n}{endTime}}}{}
\pysigstopsignatures
\end{fulllineitems}

\index{exponential() (in module metilda.controllers.pitch\_art\_wizard)@\spxentry{exponential()}\spxextra{in module metilda.controllers.pitch\_art\_wizard}}

\begin{fulllineitems}
\phantomsection\label{\detokenize{metilda.controllers:metilda.controllers.pitch_art_wizard.exponential}}
\pysigstartsignatures
\pysiglinewithargsret{\sphinxcode{\sphinxupquote{metilda.controllers.pitch\_art\_wizard.}}\sphinxbfcode{\sphinxupquote{exponential}}}{\sphinxparam{\DUrole{n,n}{scale}\DUrole{o,o}{=}\DUrole{default_value}{1.0}}\sphinxparamcomma \sphinxparam{\DUrole{n,n}{size}\DUrole{o,o}{=}\DUrole{default_value}{None}}}{}
\pysigstopsignatures
\sphinxAtStartPar
Draw samples from an exponential distribution.

\sphinxAtStartPar
Its probability density function is
\begin{equation*}
\begin{split}f(x; \frac{1}{\beta}) = \frac{1}{\beta} \exp(-\frac{x}{\beta}),\end{split}
\end{equation*}
\sphinxAtStartPar
for \sphinxcode{\sphinxupquote{x \textgreater{} 0}} and 0 elsewhere. \(\beta\) is the scale parameter,
which is the inverse of the rate parameter \(\lambda = 1/\beta\).
The rate parameter is an alternative, widely used parameterization
of the exponential distribution {\color{red}\bfseries{}{[}3{]}\_}.

\sphinxAtStartPar
The exponential distribution is a continuous analogue of the
geometric distribution.  It describes many common situations, such as
the size of raindrops measured over many rainstorms {\color{red}\bfseries{}{[}1{]}\_}, or the time
between page requests to Wikipedia {\color{red}\bfseries{}{[}2{]}\_}.

\begin{sphinxadmonition}{note}{Note:}
\sphinxAtStartPar
New code should use the \sphinxtitleref{\textasciitilde{}numpy.random.Generator.exponential}
method of a \sphinxtitleref{\textasciitilde{}numpy.random.Generator} instance instead;
please see the \DUrole{xref,std,std-ref}{random\sphinxhyphen{}quick\sphinxhyphen{}start}.
\end{sphinxadmonition}
\begin{quote}\begin{description}
\sphinxlineitem{Parameters}\begin{itemize}
\item {} 
\sphinxAtStartPar
\sphinxstyleliteralstrong{\sphinxupquote{scale}} (\sphinxstyleliteralemphasis{\sphinxupquote{float}}\sphinxstyleliteralemphasis{\sphinxupquote{ or }}\sphinxstyleliteralemphasis{\sphinxupquote{array\_like}}\sphinxstyleliteralemphasis{\sphinxupquote{ of }}\sphinxstyleliteralemphasis{\sphinxupquote{floats}}) \textendash{} The scale parameter, \(\beta = 1/\lambda\). Must be
non\sphinxhyphen{}negative.

\item {} 
\sphinxAtStartPar
\sphinxstyleliteralstrong{\sphinxupquote{size}} (\sphinxstyleliteralemphasis{\sphinxupquote{int}}\sphinxstyleliteralemphasis{\sphinxupquote{ or }}\sphinxstyleliteralemphasis{\sphinxupquote{tuple}}\sphinxstyleliteralemphasis{\sphinxupquote{ of }}\sphinxstyleliteralemphasis{\sphinxupquote{ints}}\sphinxstyleliteralemphasis{\sphinxupquote{, }}\sphinxstyleliteralemphasis{\sphinxupquote{optional}}) \textendash{} Output shape.  If the given shape is, e.g., \sphinxcode{\sphinxupquote{(m, n, k)}}, then
\sphinxcode{\sphinxupquote{m * n * k}} samples are drawn.  If size is \sphinxcode{\sphinxupquote{None}} (default),
a single value is returned if \sphinxcode{\sphinxupquote{scale}} is a scalar.  Otherwise,
\sphinxcode{\sphinxupquote{np.array(scale).size}} samples are drawn.

\end{itemize}

\sphinxlineitem{Returns}
\sphinxAtStartPar
\sphinxstylestrong{out} \textendash{} Drawn samples from the parameterized exponential distribution.

\sphinxlineitem{Return type}
\sphinxAtStartPar
ndarray or scalar

\end{description}\end{quote}


\begin{sphinxseealso}{See also:}
\begin{description}
\sphinxlineitem{\sphinxcode{\sphinxupquote{random.Generator.exponential}}}
\sphinxAtStartPar
which should be used for new code.

\end{description}


\end{sphinxseealso}

\subsubsection*{References}

\end{fulllineitems}

\index{f() (in module metilda.controllers.pitch\_art\_wizard)@\spxentry{f()}\spxextra{in module metilda.controllers.pitch\_art\_wizard}}

\begin{fulllineitems}
\phantomsection\label{\detokenize{metilda.controllers:metilda.controllers.pitch_art_wizard.f}}
\pysigstartsignatures
\pysiglinewithargsret{\sphinxcode{\sphinxupquote{metilda.controllers.pitch\_art\_wizard.}}\sphinxbfcode{\sphinxupquote{f}}}{\sphinxparam{\DUrole{n,n}{dfnum}}\sphinxparamcomma \sphinxparam{\DUrole{n,n}{dfden}}\sphinxparamcomma \sphinxparam{\DUrole{n,n}{size}\DUrole{o,o}{=}\DUrole{default_value}{None}}}{}
\pysigstopsignatures
\sphinxAtStartPar
Draw samples from an F distribution.

\sphinxAtStartPar
Samples are drawn from an F distribution with specified parameters,
\sphinxtitleref{dfnum} (degrees of freedom in numerator) and \sphinxtitleref{dfden} (degrees of
freedom in denominator), where both parameters must be greater than
zero.

\sphinxAtStartPar
The random variate of the F distribution (also known as the
Fisher distribution) is a continuous probability distribution
that arises in ANOVA tests, and is the ratio of two chi\sphinxhyphen{}square
variates.

\begin{sphinxadmonition}{note}{Note:}
\sphinxAtStartPar
New code should use the \sphinxtitleref{\textasciitilde{}numpy.random.Generator.f}
method of a \sphinxtitleref{\textasciitilde{}numpy.random.Generator} instance instead;
please see the \DUrole{xref,std,std-ref}{random\sphinxhyphen{}quick\sphinxhyphen{}start}.
\end{sphinxadmonition}
\begin{quote}\begin{description}
\sphinxlineitem{Parameters}\begin{itemize}
\item {} 
\sphinxAtStartPar
\sphinxstyleliteralstrong{\sphinxupquote{dfnum}} (\sphinxstyleliteralemphasis{\sphinxupquote{float}}\sphinxstyleliteralemphasis{\sphinxupquote{ or }}\sphinxstyleliteralemphasis{\sphinxupquote{array\_like}}\sphinxstyleliteralemphasis{\sphinxupquote{ of }}\sphinxstyleliteralemphasis{\sphinxupquote{floats}}) \textendash{} Degrees of freedom in numerator, must be \textgreater{} 0.

\item {} 
\sphinxAtStartPar
\sphinxstyleliteralstrong{\sphinxupquote{dfden}} (\sphinxstyleliteralemphasis{\sphinxupquote{float}}\sphinxstyleliteralemphasis{\sphinxupquote{ or }}\sphinxstyleliteralemphasis{\sphinxupquote{array\_like}}\sphinxstyleliteralemphasis{\sphinxupquote{ of }}\sphinxstyleliteralemphasis{\sphinxupquote{float}}) \textendash{} Degrees of freedom in denominator, must be \textgreater{} 0.

\item {} 
\sphinxAtStartPar
\sphinxstyleliteralstrong{\sphinxupquote{size}} (\sphinxstyleliteralemphasis{\sphinxupquote{int}}\sphinxstyleliteralemphasis{\sphinxupquote{ or }}\sphinxstyleliteralemphasis{\sphinxupquote{tuple}}\sphinxstyleliteralemphasis{\sphinxupquote{ of }}\sphinxstyleliteralemphasis{\sphinxupquote{ints}}\sphinxstyleliteralemphasis{\sphinxupquote{, }}\sphinxstyleliteralemphasis{\sphinxupquote{optional}}) \textendash{} Output shape.  If the given shape is, e.g., \sphinxcode{\sphinxupquote{(m, n, k)}}, then
\sphinxcode{\sphinxupquote{m * n * k}} samples are drawn.  If size is \sphinxcode{\sphinxupquote{None}} (default),
a single value is returned if \sphinxcode{\sphinxupquote{dfnum}} and \sphinxcode{\sphinxupquote{dfden}} are both scalars.
Otherwise, \sphinxcode{\sphinxupquote{np.broadcast(dfnum, dfden).size}} samples are drawn.

\end{itemize}

\sphinxlineitem{Returns}
\sphinxAtStartPar
\sphinxstylestrong{out} \textendash{} Drawn samples from the parameterized Fisher distribution.

\sphinxlineitem{Return type}
\sphinxAtStartPar
ndarray or scalar

\end{description}\end{quote}


\begin{sphinxseealso}{See also:}
\begin{description}
\sphinxlineitem{\sphinxcode{\sphinxupquote{scipy.stats.f}}}
\sphinxAtStartPar
probability density function, distribution or cumulative density function, etc.

\sphinxlineitem{\sphinxcode{\sphinxupquote{random.Generator.f}}}
\sphinxAtStartPar
which should be used for new code.

\end{description}


\end{sphinxseealso}

\subsubsection*{Notes}

\sphinxAtStartPar
The F statistic is used to compare in\sphinxhyphen{}group variances to between\sphinxhyphen{}group
variances. Calculating the distribution depends on the sampling, and
so it is a function of the respective degrees of freedom in the
problem.  The variable \sphinxtitleref{dfnum} is the number of samples minus one, the
between\sphinxhyphen{}groups degrees of freedom, while \sphinxtitleref{dfden} is the within\sphinxhyphen{}groups
degrees of freedom, the sum of the number of samples in each group
minus the number of groups.
\subsubsection*{References}
\subsubsection*{Examples}

\sphinxAtStartPar
An example from Glantz{[}1{]}, pp 47\sphinxhyphen{}40:

\sphinxAtStartPar
Two groups, children of diabetics (25 people) and children from people
without diabetes (25 controls). Fasting blood glucose was measured,
case group had a mean value of 86.1, controls had a mean value of
82.2. Standard deviations were 2.09 and 2.49 respectively. Are these
data consistent with the null hypothesis that the parents diabetic
status does not affect their children’s blood glucose levels?
Calculating the F statistic from the data gives a value of 36.01.

\sphinxAtStartPar
Draw samples from the distribution:

\begin{sphinxVerbatim}[commandchars=\\\{\}]
\PYG{g+gp}{\PYGZgt{}\PYGZgt{}\PYGZgt{} }\PYG{n}{dfnum} \PYG{o}{=} \PYG{l+m+mf}{1.} \PYG{c+c1}{\PYGZsh{} between group degrees of freedom}
\PYG{g+gp}{\PYGZgt{}\PYGZgt{}\PYGZgt{} }\PYG{n}{dfden} \PYG{o}{=} \PYG{l+m+mf}{48.} \PYG{c+c1}{\PYGZsh{} within groups degrees of freedom}
\PYG{g+gp}{\PYGZgt{}\PYGZgt{}\PYGZgt{} }\PYG{n}{s} \PYG{o}{=} \PYG{n}{np}\PYG{o}{.}\PYG{n}{random}\PYG{o}{.}\PYG{n}{f}\PYG{p}{(}\PYG{n}{dfnum}\PYG{p}{,} \PYG{n}{dfden}\PYG{p}{,} \PYG{l+m+mi}{1000}\PYG{p}{)}
\end{sphinxVerbatim}

\sphinxAtStartPar
The lower bound for the top 1\% of the samples is :

\begin{sphinxVerbatim}[commandchars=\\\{\}]
\PYG{g+gp}{\PYGZgt{}\PYGZgt{}\PYGZgt{} }\PYG{n}{np}\PYG{o}{.}\PYG{n}{sort}\PYG{p}{(}\PYG{n}{s}\PYG{p}{)}\PYG{p}{[}\PYG{o}{\PYGZhy{}}\PYG{l+m+mi}{10}\PYG{p}{]}
\PYG{g+go}{7.61988120985 \PYGZsh{} random}
\end{sphinxVerbatim}

\sphinxAtStartPar
So there is about a 1\% chance that the F statistic will exceed 7.62,
the measured value is 36, so the null hypothesis is rejected at the 1\%
level.

\end{fulllineitems}

\index{formantCountAtFrame() (in module metilda.controllers.pitch\_art\_wizard)@\spxentry{formantCountAtFrame()}\spxextra{in module metilda.controllers.pitch\_art\_wizard}}

\begin{fulllineitems}
\phantomsection\label{\detokenize{metilda.controllers:metilda.controllers.pitch_art_wizard.formantCountAtFrame}}
\pysigstartsignatures
\pysiglinewithargsret{\sphinxcode{\sphinxupquote{metilda.controllers.pitch\_art\_wizard.}}\sphinxbfcode{\sphinxupquote{formantCountAtFrame}}}{\sphinxparam{\DUrole{n,n}{sound}}\sphinxparamcomma \sphinxparam{\DUrole{n,n}{frame}}}{}
\pysigstopsignatures
\end{fulllineitems}

\index{formantFrameCount() (in module metilda.controllers.pitch\_art\_wizard)@\spxentry{formantFrameCount()}\spxextra{in module metilda.controllers.pitch\_art\_wizard}}

\begin{fulllineitems}
\phantomsection\label{\detokenize{metilda.controllers:metilda.controllers.pitch_art_wizard.formantFrameCount}}
\pysigstartsignatures
\pysiglinewithargsret{\sphinxcode{\sphinxupquote{metilda.controllers.pitch\_art\_wizard.}}\sphinxbfcode{\sphinxupquote{formantFrameCount}}}{\sphinxparam{\DUrole{n,n}{sound}}}{}
\pysigstopsignatures
\end{fulllineitems}

\index{formantValueAtTime() (in module metilda.controllers.pitch\_art\_wizard)@\spxentry{formantValueAtTime()}\spxextra{in module metilda.controllers.pitch\_art\_wizard}}

\begin{fulllineitems}
\phantomsection\label{\detokenize{metilda.controllers:metilda.controllers.pitch_art_wizard.formantValueAtTime}}
\pysigstartsignatures
\pysiglinewithargsret{\sphinxcode{\sphinxupquote{metilda.controllers.pitch\_art\_wizard.}}\sphinxbfcode{\sphinxupquote{formantValueAtTime}}}{\sphinxparam{\DUrole{n,n}{sound}}\sphinxparamcomma \sphinxparam{\DUrole{n,n}{formantNumber}}\sphinxparamcomma \sphinxparam{\DUrole{n,n}{time}}}{}
\pysigstopsignatures
\end{fulllineitems}

\index{gamma() (in module metilda.controllers.pitch\_art\_wizard)@\spxentry{gamma()}\spxextra{in module metilda.controllers.pitch\_art\_wizard}}

\begin{fulllineitems}
\phantomsection\label{\detokenize{metilda.controllers:metilda.controllers.pitch_art_wizard.gamma}}
\pysigstartsignatures
\pysiglinewithargsret{\sphinxcode{\sphinxupquote{metilda.controllers.pitch\_art\_wizard.}}\sphinxbfcode{\sphinxupquote{gamma}}}{\sphinxparam{\DUrole{n,n}{shape}}\sphinxparamcomma \sphinxparam{\DUrole{n,n}{scale}\DUrole{o,o}{=}\DUrole{default_value}{1.0}}\sphinxparamcomma \sphinxparam{\DUrole{n,n}{size}\DUrole{o,o}{=}\DUrole{default_value}{None}}}{}
\pysigstopsignatures
\sphinxAtStartPar
Draw samples from a Gamma distribution.

\sphinxAtStartPar
Samples are drawn from a Gamma distribution with specified parameters,
\sphinxtitleref{shape} (sometimes designated “k”) and \sphinxtitleref{scale} (sometimes designated
“theta”), where both parameters are \textgreater{} 0.

\begin{sphinxadmonition}{note}{Note:}
\sphinxAtStartPar
New code should use the \sphinxtitleref{\textasciitilde{}numpy.random.Generator.gamma}
method of a \sphinxtitleref{\textasciitilde{}numpy.random.Generator} instance instead;
please see the \DUrole{xref,std,std-ref}{random\sphinxhyphen{}quick\sphinxhyphen{}start}.
\end{sphinxadmonition}
\begin{quote}\begin{description}
\sphinxlineitem{Parameters}\begin{itemize}
\item {} 
\sphinxAtStartPar
\sphinxstyleliteralstrong{\sphinxupquote{shape}} (\sphinxstyleliteralemphasis{\sphinxupquote{float}}\sphinxstyleliteralemphasis{\sphinxupquote{ or }}\sphinxstyleliteralemphasis{\sphinxupquote{array\_like}}\sphinxstyleliteralemphasis{\sphinxupquote{ of }}\sphinxstyleliteralemphasis{\sphinxupquote{floats}}) \textendash{} The shape of the gamma distribution. Must be non\sphinxhyphen{}negative.

\item {} 
\sphinxAtStartPar
\sphinxstyleliteralstrong{\sphinxupquote{scale}} (\sphinxstyleliteralemphasis{\sphinxupquote{float}}\sphinxstyleliteralemphasis{\sphinxupquote{ or }}\sphinxstyleliteralemphasis{\sphinxupquote{array\_like}}\sphinxstyleliteralemphasis{\sphinxupquote{ of }}\sphinxstyleliteralemphasis{\sphinxupquote{floats}}\sphinxstyleliteralemphasis{\sphinxupquote{, }}\sphinxstyleliteralemphasis{\sphinxupquote{optional}}) \textendash{} The scale of the gamma distribution. Must be non\sphinxhyphen{}negative.
Default is equal to 1.

\item {} 
\sphinxAtStartPar
\sphinxstyleliteralstrong{\sphinxupquote{size}} (\sphinxstyleliteralemphasis{\sphinxupquote{int}}\sphinxstyleliteralemphasis{\sphinxupquote{ or }}\sphinxstyleliteralemphasis{\sphinxupquote{tuple}}\sphinxstyleliteralemphasis{\sphinxupquote{ of }}\sphinxstyleliteralemphasis{\sphinxupquote{ints}}\sphinxstyleliteralemphasis{\sphinxupquote{, }}\sphinxstyleliteralemphasis{\sphinxupquote{optional}}) \textendash{} Output shape.  If the given shape is, e.g., \sphinxcode{\sphinxupquote{(m, n, k)}}, then
\sphinxcode{\sphinxupquote{m * n * k}} samples are drawn.  If size is \sphinxcode{\sphinxupquote{None}} (default),
a single value is returned if \sphinxcode{\sphinxupquote{shape}} and \sphinxcode{\sphinxupquote{scale}} are both scalars.
Otherwise, \sphinxcode{\sphinxupquote{np.broadcast(shape, scale).size}} samples are drawn.

\end{itemize}

\sphinxlineitem{Returns}
\sphinxAtStartPar
\sphinxstylestrong{out} \textendash{} Drawn samples from the parameterized gamma distribution.

\sphinxlineitem{Return type}
\sphinxAtStartPar
ndarray or scalar

\end{description}\end{quote}


\begin{sphinxseealso}{See also:}
\begin{description}
\sphinxlineitem{\sphinxcode{\sphinxupquote{scipy.stats.gamma}}}
\sphinxAtStartPar
probability density function, distribution or cumulative density function, etc.

\sphinxlineitem{\sphinxcode{\sphinxupquote{random.Generator.gamma}}}
\sphinxAtStartPar
which should be used for new code.

\end{description}


\end{sphinxseealso}

\subsubsection*{Notes}

\sphinxAtStartPar
The probability density for the Gamma distribution is
\begin{equation*}
\begin{split}p(x) = x^{k-1}\frac{e^{-x/\theta}}{\theta^k\Gamma(k)},\end{split}
\end{equation*}
\sphinxAtStartPar
where \(k\) is the shape and \(\theta\) the scale,
and \(\Gamma\) is the Gamma function.

\sphinxAtStartPar
The Gamma distribution is often used to model the times to failure of
electronic components, and arises naturally in processes for which the
waiting times between Poisson distributed events are relevant.
\subsubsection*{References}
\subsubsection*{Examples}

\sphinxAtStartPar
Draw samples from the distribution:

\begin{sphinxVerbatim}[commandchars=\\\{\}]
\PYG{g+gp}{\PYGZgt{}\PYGZgt{}\PYGZgt{} }\PYG{n}{shape}\PYG{p}{,} \PYG{n}{scale} \PYG{o}{=} \PYG{l+m+mf}{2.}\PYG{p}{,} \PYG{l+m+mf}{2.}  \PYG{c+c1}{\PYGZsh{} mean=4, std=2*sqrt(2)}
\PYG{g+gp}{\PYGZgt{}\PYGZgt{}\PYGZgt{} }\PYG{n}{s} \PYG{o}{=} \PYG{n}{np}\PYG{o}{.}\PYG{n}{random}\PYG{o}{.}\PYG{n}{gamma}\PYG{p}{(}\PYG{n}{shape}\PYG{p}{,} \PYG{n}{scale}\PYG{p}{,} \PYG{l+m+mi}{1000}\PYG{p}{)}
\end{sphinxVerbatim}

\sphinxAtStartPar
Display the histogram of the samples, along with
the probability density function:

\begin{sphinxVerbatim}[commandchars=\\\{\}]
\PYG{g+gp}{\PYGZgt{}\PYGZgt{}\PYGZgt{} }\PYG{k+kn}{import} \PYG{n+nn}{matplotlib}\PYG{n+nn}{.}\PYG{n+nn}{pyplot} \PYG{k}{as} \PYG{n+nn}{plt}
\PYG{g+gp}{\PYGZgt{}\PYGZgt{}\PYGZgt{} }\PYG{k+kn}{import} \PYG{n+nn}{scipy}\PYG{n+nn}{.}\PYG{n+nn}{special} \PYG{k}{as} \PYG{n+nn}{sps}  
\PYG{g+gp}{\PYGZgt{}\PYGZgt{}\PYGZgt{} }\PYG{n}{count}\PYG{p}{,} \PYG{n}{bins}\PYG{p}{,} \PYG{n}{ignored} \PYG{o}{=} \PYG{n}{plt}\PYG{o}{.}\PYG{n}{hist}\PYG{p}{(}\PYG{n}{s}\PYG{p}{,} \PYG{l+m+mi}{50}\PYG{p}{,} \PYG{n}{density}\PYG{o}{=}\PYG{k+kc}{True}\PYG{p}{)}
\PYG{g+gp}{\PYGZgt{}\PYGZgt{}\PYGZgt{} }\PYG{n}{y} \PYG{o}{=} \PYG{n}{bins}\PYG{o}{*}\PYG{o}{*}\PYG{p}{(}\PYG{n}{shape}\PYG{o}{\PYGZhy{}}\PYG{l+m+mi}{1}\PYG{p}{)}\PYG{o}{*}\PYG{p}{(}\PYG{n}{np}\PYG{o}{.}\PYG{n}{exp}\PYG{p}{(}\PYG{o}{\PYGZhy{}}\PYG{n}{bins}\PYG{o}{/}\PYG{n}{scale}\PYG{p}{)} \PYG{o}{/}  
\PYG{g+gp}{... }                     \PYG{p}{(}\PYG{n}{sps}\PYG{o}{.}\PYG{n}{gamma}\PYG{p}{(}\PYG{n}{shape}\PYG{p}{)}\PYG{o}{*}\PYG{n}{scale}\PYG{o}{*}\PYG{o}{*}\PYG{n}{shape}\PYG{p}{)}\PYG{p}{)}
\PYG{g+gp}{\PYGZgt{}\PYGZgt{}\PYGZgt{} }\PYG{n}{plt}\PYG{o}{.}\PYG{n}{plot}\PYG{p}{(}\PYG{n}{bins}\PYG{p}{,} \PYG{n}{y}\PYG{p}{,} \PYG{n}{linewidth}\PYG{o}{=}\PYG{l+m+mi}{2}\PYG{p}{,} \PYG{n}{color}\PYG{o}{=}\PYG{l+s+s1}{\PYGZsq{}}\PYG{l+s+s1}{r}\PYG{l+s+s1}{\PYGZsq{}}\PYG{p}{)}  
\PYG{g+gp}{\PYGZgt{}\PYGZgt{}\PYGZgt{} }\PYG{n}{plt}\PYG{o}{.}\PYG{n}{show}\PYG{p}{(}\PYG{p}{)}
\end{sphinxVerbatim}

\end{fulllineitems}

\index{geometric() (in module metilda.controllers.pitch\_art\_wizard)@\spxentry{geometric()}\spxextra{in module metilda.controllers.pitch\_art\_wizard}}

\begin{fulllineitems}
\phantomsection\label{\detokenize{metilda.controllers:metilda.controllers.pitch_art_wizard.geometric}}
\pysigstartsignatures
\pysiglinewithargsret{\sphinxcode{\sphinxupquote{metilda.controllers.pitch\_art\_wizard.}}\sphinxbfcode{\sphinxupquote{geometric}}}{\sphinxparam{\DUrole{n,n}{p}}\sphinxparamcomma \sphinxparam{\DUrole{n,n}{size}\DUrole{o,o}{=}\DUrole{default_value}{None}}}{}
\pysigstopsignatures
\sphinxAtStartPar
Draw samples from the geometric distribution.

\sphinxAtStartPar
Bernoulli trials are experiments with one of two outcomes:
success or failure (an example of such an experiment is flipping
a coin).  The geometric distribution models the number of trials
that must be run in order to achieve success.  It is therefore
supported on the positive integers, \sphinxcode{\sphinxupquote{k = 1, 2, ...}}.

\sphinxAtStartPar
The probability mass function of the geometric distribution is
\begin{equation*}
\begin{split}f(k) = (1 - p)^{k - 1} p\end{split}
\end{equation*}
\sphinxAtStartPar
where \sphinxtitleref{p} is the probability of success of an individual trial.

\begin{sphinxadmonition}{note}{Note:}
\sphinxAtStartPar
New code should use the \sphinxtitleref{\textasciitilde{}numpy.random.Generator.geometric}
method of a \sphinxtitleref{\textasciitilde{}numpy.random.Generator} instance instead;
please see the \DUrole{xref,std,std-ref}{random\sphinxhyphen{}quick\sphinxhyphen{}start}.
\end{sphinxadmonition}
\begin{quote}\begin{description}
\sphinxlineitem{Parameters}\begin{itemize}
\item {} 
\sphinxAtStartPar
\sphinxstyleliteralstrong{\sphinxupquote{p}} (\sphinxstyleliteralemphasis{\sphinxupquote{float}}\sphinxstyleliteralemphasis{\sphinxupquote{ or }}\sphinxstyleliteralemphasis{\sphinxupquote{array\_like}}\sphinxstyleliteralemphasis{\sphinxupquote{ of }}\sphinxstyleliteralemphasis{\sphinxupquote{floats}}) \textendash{} The probability of success of an individual trial.

\item {} 
\sphinxAtStartPar
\sphinxstyleliteralstrong{\sphinxupquote{size}} (\sphinxstyleliteralemphasis{\sphinxupquote{int}}\sphinxstyleliteralemphasis{\sphinxupquote{ or }}\sphinxstyleliteralemphasis{\sphinxupquote{tuple}}\sphinxstyleliteralemphasis{\sphinxupquote{ of }}\sphinxstyleliteralemphasis{\sphinxupquote{ints}}\sphinxstyleliteralemphasis{\sphinxupquote{, }}\sphinxstyleliteralemphasis{\sphinxupquote{optional}}) \textendash{} Output shape.  If the given shape is, e.g., \sphinxcode{\sphinxupquote{(m, n, k)}}, then
\sphinxcode{\sphinxupquote{m * n * k}} samples are drawn.  If size is \sphinxcode{\sphinxupquote{None}} (default),
a single value is returned if \sphinxcode{\sphinxupquote{p}} is a scalar.  Otherwise,
\sphinxcode{\sphinxupquote{np.array(p).size}} samples are drawn.

\end{itemize}

\sphinxlineitem{Returns}
\sphinxAtStartPar
\sphinxstylestrong{out} \textendash{} Drawn samples from the parameterized geometric distribution.

\sphinxlineitem{Return type}
\sphinxAtStartPar
ndarray or scalar

\end{description}\end{quote}


\begin{sphinxseealso}{See also:}
\begin{description}
\sphinxlineitem{\sphinxcode{\sphinxupquote{random.Generator.geometric}}}
\sphinxAtStartPar
which should be used for new code.

\end{description}


\end{sphinxseealso}

\subsubsection*{Examples}

\sphinxAtStartPar
Draw ten thousand values from the geometric distribution,
with the probability of an individual success equal to 0.35:

\begin{sphinxVerbatim}[commandchars=\\\{\}]
\PYG{g+gp}{\PYGZgt{}\PYGZgt{}\PYGZgt{} }\PYG{n}{z} \PYG{o}{=} \PYG{n}{np}\PYG{o}{.}\PYG{n}{random}\PYG{o}{.}\PYG{n}{geometric}\PYG{p}{(}\PYG{n}{p}\PYG{o}{=}\PYG{l+m+mf}{0.35}\PYG{p}{,} \PYG{n}{size}\PYG{o}{=}\PYG{l+m+mi}{10000}\PYG{p}{)}
\end{sphinxVerbatim}

\sphinxAtStartPar
How many trials succeeded after a single run?

\begin{sphinxVerbatim}[commandchars=\\\{\}]
\PYG{g+gp}{\PYGZgt{}\PYGZgt{}\PYGZgt{} }\PYG{p}{(}\PYG{n}{z} \PYG{o}{==} \PYG{l+m+mi}{1}\PYG{p}{)}\PYG{o}{.}\PYG{n}{sum}\PYG{p}{(}\PYG{p}{)} \PYG{o}{/} \PYG{l+m+mf}{10000.}
\PYG{g+go}{0.34889999999999999 \PYGZsh{}random}
\end{sphinxVerbatim}

\end{fulllineitems}

\index{getBounds() (in module metilda.controllers.pitch\_art\_wizard)@\spxentry{getBounds()}\spxextra{in module metilda.controllers.pitch\_art\_wizard}}

\begin{fulllineitems}
\phantomsection\label{\detokenize{metilda.controllers:metilda.controllers.pitch_art_wizard.getBounds}}
\pysigstartsignatures
\pysiglinewithargsret{\sphinxcode{\sphinxupquote{metilda.controllers.pitch\_art\_wizard.}}\sphinxbfcode{\sphinxupquote{getBounds}}}{\sphinxparam{\DUrole{n,n}{sound}}}{}
\pysigstopsignatures
\end{fulllineitems}

\index{getEnergy() (in module metilda.controllers.pitch\_art\_wizard)@\spxentry{getEnergy()}\spxextra{in module metilda.controllers.pitch\_art\_wizard}}

\begin{fulllineitems}
\phantomsection\label{\detokenize{metilda.controllers:metilda.controllers.pitch_art_wizard.getEnergy}}
\pysigstartsignatures
\pysiglinewithargsret{\sphinxcode{\sphinxupquote{metilda.controllers.pitch\_art\_wizard.}}\sphinxbfcode{\sphinxupquote{getEnergy}}}{\sphinxparam{\DUrole{n,n}{sound}}}{}
\pysigstopsignatures
\end{fulllineitems}

\index{getOrCreateWords() (in module metilda.controllers.pitch\_art\_wizard)@\spxentry{getOrCreateWords()}\spxextra{in module metilda.controllers.pitch\_art\_wizard}}

\begin{fulllineitems}
\phantomsection\label{\detokenize{metilda.controllers:metilda.controllers.pitch_art_wizard.getOrCreateWords}}
\pysigstartsignatures
\pysiglinewithargsret{\sphinxcode{\sphinxupquote{metilda.controllers.pitch\_art\_wizard.}}\sphinxbfcode{\sphinxupquote{getOrCreateWords}}}{}{}
\pysigstopsignatures
\end{fulllineitems}

\index{get\_admin() (in module metilda.controllers.pitch\_art\_wizard)@\spxentry{get\_admin()}\spxextra{in module metilda.controllers.pitch\_art\_wizard}}

\begin{fulllineitems}
\phantomsection\label{\detokenize{metilda.controllers:metilda.controllers.pitch_art_wizard.get_admin}}
\pysigstartsignatures
\pysiglinewithargsret{\sphinxcode{\sphinxupquote{metilda.controllers.pitch\_art\_wizard.}}\sphinxbfcode{\sphinxupquote{get\_admin}}}{\sphinxparam{\DUrole{n,n}{user\_id}}}{}
\pysigstopsignatures
\end{fulllineitems}

\index{get\_all\_images() (in module metilda.controllers.pitch\_art\_wizard)@\spxentry{get\_all\_images()}\spxextra{in module metilda.controllers.pitch\_art\_wizard}}

\begin{fulllineitems}
\phantomsection\label{\detokenize{metilda.controllers:metilda.controllers.pitch_art_wizard.get_all_images}}
\pysigstartsignatures
\pysiglinewithargsret{\sphinxcode{\sphinxupquote{metilda.controllers.pitch\_art\_wizard.}}\sphinxbfcode{\sphinxupquote{get\_all\_images}}}{\sphinxparam{\DUrole{n,n}{user\_id}}}{}
\pysigstopsignatures
\end{fulllineitems}

\index{get\_analyses\_for\_file() (in module metilda.controllers.pitch\_art\_wizard)@\spxentry{get\_analyses\_for\_file()}\spxextra{in module metilda.controllers.pitch\_art\_wizard}}

\begin{fulllineitems}
\phantomsection\label{\detokenize{metilda.controllers:metilda.controllers.pitch_art_wizard.get_analyses_for_file}}
\pysigstartsignatures
\pysiglinewithargsret{\sphinxcode{\sphinxupquote{metilda.controllers.pitch\_art\_wizard.}}\sphinxbfcode{\sphinxupquote{get\_analyses\_for\_file}}}{\sphinxparam{\DUrole{n,n}{file\_id}}}{}
\pysigstopsignatures
\end{fulllineitems}

\index{get\_analyses\_for\_image() (in module metilda.controllers.pitch\_art\_wizard)@\spxentry{get\_analyses\_for\_image()}\spxextra{in module metilda.controllers.pitch\_art\_wizard}}

\begin{fulllineitems}
\phantomsection\label{\detokenize{metilda.controllers:metilda.controllers.pitch_art_wizard.get_analyses_for_image}}
\pysigstartsignatures
\pysiglinewithargsret{\sphinxcode{\sphinxupquote{metilda.controllers.pitch\_art\_wizard.}}\sphinxbfcode{\sphinxupquote{get\_analyses\_for\_image}}}{\sphinxparam{\DUrole{n,n}{image\_id}}}{}
\pysigstopsignatures
\end{fulllineitems}

\index{get\_analysis\_file\_path() (in module metilda.controllers.pitch\_art\_wizard)@\spxentry{get\_analysis\_file\_path()}\spxextra{in module metilda.controllers.pitch\_art\_wizard}}

\begin{fulllineitems}
\phantomsection\label{\detokenize{metilda.controllers:metilda.controllers.pitch_art_wizard.get_analysis_file_path}}
\pysigstartsignatures
\pysiglinewithargsret{\sphinxcode{\sphinxupquote{metilda.controllers.pitch\_art\_wizard.}}\sphinxbfcode{\sphinxupquote{get\_analysis\_file\_path}}}{\sphinxparam{\DUrole{n,n}{analysis\_id}}}{}
\pysigstopsignatures
\end{fulllineitems}

\index{get\_eaf\_file() (in module metilda.controllers.pitch\_art\_wizard)@\spxentry{get\_eaf\_file()}\spxextra{in module metilda.controllers.pitch\_art\_wizard}}

\begin{fulllineitems}
\phantomsection\label{\detokenize{metilda.controllers:metilda.controllers.pitch_art_wizard.get_eaf_file}}
\pysigstartsignatures
\pysiglinewithargsret{\sphinxcode{\sphinxupquote{metilda.controllers.pitch\_art\_wizard.}}\sphinxbfcode{\sphinxupquote{get\_eaf\_file}}}{\sphinxparam{\DUrole{n,n}{audio\_id}}}{}
\pysigstopsignatures
\end{fulllineitems}

\index{get\_eaf\_file\_path() (in module metilda.controllers.pitch\_art\_wizard)@\spxentry{get\_eaf\_file\_path()}\spxextra{in module metilda.controllers.pitch\_art\_wizard}}

\begin{fulllineitems}
\phantomsection\label{\detokenize{metilda.controllers:metilda.controllers.pitch_art_wizard.get_eaf_file_path}}
\pysigstartsignatures
\pysiglinewithargsret{\sphinxcode{\sphinxupquote{metilda.controllers.pitch\_art\_wizard.}}\sphinxbfcode{\sphinxupquote{get\_eaf\_file\_path}}}{\sphinxparam{\DUrole{n,n}{eaf\_id}}}{}
\pysigstopsignatures
\end{fulllineitems}

\index{get\_eafs\_for\_files() (in module metilda.controllers.pitch\_art\_wizard)@\spxentry{get\_eafs\_for\_files()}\spxextra{in module metilda.controllers.pitch\_art\_wizard}}

\begin{fulllineitems}
\phantomsection\label{\detokenize{metilda.controllers:metilda.controllers.pitch_art_wizard.get_eafs_for_files}}
\pysigstartsignatures
\pysiglinewithargsret{\sphinxcode{\sphinxupquote{metilda.controllers.pitch\_art\_wizard.}}\sphinxbfcode{\sphinxupquote{get\_eafs\_for\_files}}}{\sphinxparam{\DUrole{n,n}{audio\_id}}}{}
\pysigstopsignatures
\end{fulllineitems}

\index{get\_file() (in module metilda.controllers.pitch\_art\_wizard)@\spxentry{get\_file()}\spxextra{in module metilda.controllers.pitch\_art\_wizard}}

\begin{fulllineitems}
\phantomsection\label{\detokenize{metilda.controllers:metilda.controllers.pitch_art_wizard.get_file}}
\pysigstartsignatures
\pysiglinewithargsret{\sphinxcode{\sphinxupquote{metilda.controllers.pitch\_art\_wizard.}}\sphinxbfcode{\sphinxupquote{get\_file}}}{\sphinxparam{\DUrole{n,n}{user\_id}}}{}
\pysigstopsignatures
\end{fulllineitems}

\index{get\_files\_and\_folders() (in module metilda.controllers.pitch\_art\_wizard)@\spxentry{get\_files\_and\_folders()}\spxextra{in module metilda.controllers.pitch\_art\_wizard}}

\begin{fulllineitems}
\phantomsection\label{\detokenize{metilda.controllers:metilda.controllers.pitch_art_wizard.get_files_and_folders}}
\pysigstartsignatures
\pysiglinewithargsret{\sphinxcode{\sphinxupquote{metilda.controllers.pitch\_art\_wizard.}}\sphinxbfcode{\sphinxupquote{get\_files\_and\_folders}}}{\sphinxparam{\DUrole{n,n}{user\_id}}\sphinxparamcomma \sphinxparam{\DUrole{n,n}{folder\_name}}}{}
\pysigstopsignatures
\end{fulllineitems}

\index{get\_image\_for\_analysis() (in module metilda.controllers.pitch\_art\_wizard)@\spxentry{get\_image\_for\_analysis()}\spxextra{in module metilda.controllers.pitch\_art\_wizard}}

\begin{fulllineitems}
\phantomsection\label{\detokenize{metilda.controllers:metilda.controllers.pitch_art_wizard.get_image_for_analysis}}
\pysigstartsignatures
\pysiglinewithargsret{\sphinxcode{\sphinxupquote{metilda.controllers.pitch\_art\_wizard.}}\sphinxbfcode{\sphinxupquote{get\_image\_for\_analysis}}}{\sphinxparam{\DUrole{n,n}{analysis\_id}}}{}
\pysigstopsignatures
\end{fulllineitems}

\index{get\_shared\_users() (in module metilda.controllers.pitch\_art\_wizard)@\spxentry{get\_shared\_users()}\spxextra{in module metilda.controllers.pitch\_art\_wizard}}

\begin{fulllineitems}
\phantomsection\label{\detokenize{metilda.controllers:metilda.controllers.pitch_art_wizard.get_shared_users}}
\pysigstartsignatures
\pysiglinewithargsret{\sphinxcode{\sphinxupquote{metilda.controllers.pitch\_art\_wizard.}}\sphinxbfcode{\sphinxupquote{get\_shared\_users}}}{\sphinxparam{\DUrole{n,n}{audio\_id}}}{}
\pysigstopsignatures
\end{fulllineitems}

\index{get\_state() (in module metilda.controllers.pitch\_art\_wizard)@\spxentry{get\_state()}\spxextra{in module metilda.controllers.pitch\_art\_wizard}}

\begin{fulllineitems}
\phantomsection\label{\detokenize{metilda.controllers:metilda.controllers.pitch_art_wizard.get_state}}
\pysigstartsignatures
\pysiglinewithargsret{\sphinxcode{\sphinxupquote{metilda.controllers.pitch\_art\_wizard.}}\sphinxbfcode{\sphinxupquote{get\_state}}}{\sphinxparam{\DUrole{n,n}{legacy}\DUrole{o,o}{=}\DUrole{default_value}{True}}}{}
\pysigstopsignatures
\sphinxAtStartPar
Return a tuple representing the internal state of the generator.

\sphinxAtStartPar
For more details, see \sphinxtitleref{set\_state}.
\begin{quote}\begin{description}
\sphinxlineitem{Parameters}
\sphinxAtStartPar
\sphinxstyleliteralstrong{\sphinxupquote{legacy}} (\sphinxstyleliteralemphasis{\sphinxupquote{bool}}\sphinxstyleliteralemphasis{\sphinxupquote{, }}\sphinxstyleliteralemphasis{\sphinxupquote{optional}}) \textendash{} Flag indicating to return a legacy tuple state when the BitGenerator
is MT19937, instead of a dict. Raises ValueError if the underlying
bit generator is not an instance of MT19937.

\sphinxlineitem{Returns}
\sphinxAtStartPar

\sphinxAtStartPar
\sphinxstylestrong{out} \textendash{} If legacy is True, the returned tuple has the following items:
\begin{enumerate}
\sphinxsetlistlabels{\arabic}{enumi}{enumii}{}{.}%
\item {} 
\sphinxAtStartPar
the string ‘MT19937’.

\item {} 
\sphinxAtStartPar
a 1\sphinxhyphen{}D array of 624 unsigned integer keys.

\item {} 
\sphinxAtStartPar
an integer \sphinxcode{\sphinxupquote{pos}}.

\item {} 
\sphinxAtStartPar
an integer \sphinxcode{\sphinxupquote{has\_gauss}}.

\item {} 
\sphinxAtStartPar
a float \sphinxcode{\sphinxupquote{cached\_gaussian}}.

\end{enumerate}

\sphinxAtStartPar
If \sphinxtitleref{legacy} is False, or the BitGenerator is not MT19937, then
state is returned as a dictionary.


\sphinxlineitem{Return type}
\sphinxAtStartPar
\{tuple(str, ndarray of 624 uints, int, int, float), dict\}

\end{description}\end{quote}


\begin{sphinxseealso}{See also:}

\sphinxAtStartPar
{\hyperref[\detokenize{metilda.controllers:metilda.controllers.pitch_art_wizard.set_state}]{\sphinxcrossref{\sphinxcode{\sphinxupquote{set\_state}}}}}


\end{sphinxseealso}

\subsubsection*{Notes}

\sphinxAtStartPar
\sphinxtitleref{set\_state} and \sphinxtitleref{get\_state} are not needed to work with any of the
random distributions in NumPy. If the internal state is manually altered,
the user should know exactly what he/she is doing.

\end{fulllineitems}

\index{get\_student\_recordings() (in module metilda.controllers.pitch\_art\_wizard)@\spxentry{get\_student\_recordings()}\spxextra{in module metilda.controllers.pitch\_art\_wizard}}

\begin{fulllineitems}
\phantomsection\label{\detokenize{metilda.controllers:metilda.controllers.pitch_art_wizard.get_student_recordings}}
\pysigstartsignatures
\pysiglinewithargsret{\sphinxcode{\sphinxupquote{metilda.controllers.pitch\_art\_wizard.}}\sphinxbfcode{\sphinxupquote{get\_student\_recordings}}}{}{}
\pysigstopsignatures
\end{fulllineitems}

\index{get\_user\_research\_language() (in module metilda.controllers.pitch\_art\_wizard)@\spxentry{get\_user\_research\_language()}\spxextra{in module metilda.controllers.pitch\_art\_wizard}}

\begin{fulllineitems}
\phantomsection\label{\detokenize{metilda.controllers:metilda.controllers.pitch_art_wizard.get_user_research_language}}
\pysigstartsignatures
\pysiglinewithargsret{\sphinxcode{\sphinxupquote{metilda.controllers.pitch\_art\_wizard.}}\sphinxbfcode{\sphinxupquote{get\_user\_research\_language}}}{\sphinxparam{\DUrole{n,n}{user\_id}}}{}
\pysigstopsignatures
\end{fulllineitems}

\index{get\_user\_roles() (in module metilda.controllers.pitch\_art\_wizard)@\spxentry{get\_user\_roles()}\spxextra{in module metilda.controllers.pitch\_art\_wizard}}

\begin{fulllineitems}
\phantomsection\label{\detokenize{metilda.controllers:metilda.controllers.pitch_art_wizard.get_user_roles}}
\pysigstartsignatures
\pysiglinewithargsret{\sphinxcode{\sphinxupquote{metilda.controllers.pitch\_art\_wizard.}}\sphinxbfcode{\sphinxupquote{get\_user\_roles}}}{\sphinxparam{\DUrole{n,n}{user\_id}}}{}
\pysigstopsignatures
\end{fulllineitems}

\index{get\_users() (in module metilda.controllers.pitch\_art\_wizard)@\spxentry{get\_users()}\spxextra{in module metilda.controllers.pitch\_art\_wizard}}

\begin{fulllineitems}
\phantomsection\label{\detokenize{metilda.controllers:metilda.controllers.pitch_art_wizard.get_users}}
\pysigstartsignatures
\pysiglinewithargsret{\sphinxcode{\sphinxupquote{metilda.controllers.pitch\_art\_wizard.}}\sphinxbfcode{\sphinxupquote{get\_users}}}{}{}
\pysigstopsignatures
\end{fulllineitems}

\index{gumbel() (in module metilda.controllers.pitch\_art\_wizard)@\spxentry{gumbel()}\spxextra{in module metilda.controllers.pitch\_art\_wizard}}

\begin{fulllineitems}
\phantomsection\label{\detokenize{metilda.controllers:metilda.controllers.pitch_art_wizard.gumbel}}
\pysigstartsignatures
\pysiglinewithargsret{\sphinxcode{\sphinxupquote{metilda.controllers.pitch\_art\_wizard.}}\sphinxbfcode{\sphinxupquote{gumbel}}}{\sphinxparam{\DUrole{n,n}{loc}\DUrole{o,o}{=}\DUrole{default_value}{0.0}}\sphinxparamcomma \sphinxparam{\DUrole{n,n}{scale}\DUrole{o,o}{=}\DUrole{default_value}{1.0}}\sphinxparamcomma \sphinxparam{\DUrole{n,n}{size}\DUrole{o,o}{=}\DUrole{default_value}{None}}}{}
\pysigstopsignatures
\sphinxAtStartPar
Draw samples from a Gumbel distribution.

\sphinxAtStartPar
Draw samples from a Gumbel distribution with specified location and
scale.  For more information on the Gumbel distribution, see
Notes and References below.

\begin{sphinxadmonition}{note}{Note:}
\sphinxAtStartPar
New code should use the \sphinxtitleref{\textasciitilde{}numpy.random.Generator.gumbel}
method of a \sphinxtitleref{\textasciitilde{}numpy.random.Generator} instance instead;
please see the \DUrole{xref,std,std-ref}{random\sphinxhyphen{}quick\sphinxhyphen{}start}.
\end{sphinxadmonition}
\begin{quote}\begin{description}
\sphinxlineitem{Parameters}\begin{itemize}
\item {} 
\sphinxAtStartPar
\sphinxstyleliteralstrong{\sphinxupquote{loc}} (\sphinxstyleliteralemphasis{\sphinxupquote{float}}\sphinxstyleliteralemphasis{\sphinxupquote{ or }}\sphinxstyleliteralemphasis{\sphinxupquote{array\_like}}\sphinxstyleliteralemphasis{\sphinxupquote{ of }}\sphinxstyleliteralemphasis{\sphinxupquote{floats}}\sphinxstyleliteralemphasis{\sphinxupquote{, }}\sphinxstyleliteralemphasis{\sphinxupquote{optional}}) \textendash{} The location of the mode of the distribution. Default is 0.

\item {} 
\sphinxAtStartPar
\sphinxstyleliteralstrong{\sphinxupquote{scale}} (\sphinxstyleliteralemphasis{\sphinxupquote{float}}\sphinxstyleliteralemphasis{\sphinxupquote{ or }}\sphinxstyleliteralemphasis{\sphinxupquote{array\_like}}\sphinxstyleliteralemphasis{\sphinxupquote{ of }}\sphinxstyleliteralemphasis{\sphinxupquote{floats}}\sphinxstyleliteralemphasis{\sphinxupquote{, }}\sphinxstyleliteralemphasis{\sphinxupquote{optional}}) \textendash{} The scale parameter of the distribution. Default is 1. Must be non\sphinxhyphen{}
negative.

\item {} 
\sphinxAtStartPar
\sphinxstyleliteralstrong{\sphinxupquote{size}} (\sphinxstyleliteralemphasis{\sphinxupquote{int}}\sphinxstyleliteralemphasis{\sphinxupquote{ or }}\sphinxstyleliteralemphasis{\sphinxupquote{tuple}}\sphinxstyleliteralemphasis{\sphinxupquote{ of }}\sphinxstyleliteralemphasis{\sphinxupquote{ints}}\sphinxstyleliteralemphasis{\sphinxupquote{, }}\sphinxstyleliteralemphasis{\sphinxupquote{optional}}) \textendash{} Output shape.  If the given shape is, e.g., \sphinxcode{\sphinxupquote{(m, n, k)}}, then
\sphinxcode{\sphinxupquote{m * n * k}} samples are drawn.  If size is \sphinxcode{\sphinxupquote{None}} (default),
a single value is returned if \sphinxcode{\sphinxupquote{loc}} and \sphinxcode{\sphinxupquote{scale}} are both scalars.
Otherwise, \sphinxcode{\sphinxupquote{np.broadcast(loc, scale).size}} samples are drawn.

\end{itemize}

\sphinxlineitem{Returns}
\sphinxAtStartPar
\sphinxstylestrong{out} \textendash{} Drawn samples from the parameterized Gumbel distribution.

\sphinxlineitem{Return type}
\sphinxAtStartPar
ndarray or scalar

\end{description}\end{quote}


\begin{sphinxseealso}{See also:}

\sphinxAtStartPar
\sphinxcode{\sphinxupquote{scipy.stats.gumbel\_l}}, \sphinxcode{\sphinxupquote{scipy.stats.gumbel\_r}}, \sphinxcode{\sphinxupquote{scipy.stats.genextreme}}, {\hyperref[\detokenize{metilda.controllers:metilda.controllers.pitch_art_wizard.weibull}]{\sphinxcrossref{\sphinxcode{\sphinxupquote{weibull}}}}}
\begin{description}
\sphinxlineitem{\sphinxcode{\sphinxupquote{random.Generator.gumbel}}}
\sphinxAtStartPar
which should be used for new code.

\end{description}


\end{sphinxseealso}

\subsubsection*{Notes}

\sphinxAtStartPar
The Gumbel (or Smallest Extreme Value (SEV) or the Smallest Extreme
Value Type I) distribution is one of a class of Generalized Extreme
Value (GEV) distributions used in modeling extreme value problems.
The Gumbel is a special case of the Extreme Value Type I distribution
for maximums from distributions with “exponential\sphinxhyphen{}like” tails.

\sphinxAtStartPar
The probability density for the Gumbel distribution is
\begin{equation*}
\begin{split}p(x) = \frac{e^{-(x - \mu)/ \beta}}{\beta} e^{ -e^{-(x - \mu)/
\beta}},\end{split}
\end{equation*}
\sphinxAtStartPar
where \(\mu\) is the mode, a location parameter, and
\(\beta\) is the scale parameter.

\sphinxAtStartPar
The Gumbel (named for German mathematician Emil Julius Gumbel) was used
very early in the hydrology literature, for modeling the occurrence of
flood events. It is also used for modeling maximum wind speed and
rainfall rates.  It is a “fat\sphinxhyphen{}tailed” distribution \sphinxhyphen{} the probability of
an event in the tail of the distribution is larger than if one used a
Gaussian, hence the surprisingly frequent occurrence of 100\sphinxhyphen{}year
floods. Floods were initially modeled as a Gaussian process, which
underestimated the frequency of extreme events.

\sphinxAtStartPar
It is one of a class of extreme value distributions, the Generalized
Extreme Value (GEV) distributions, which also includes the Weibull and
Frechet.

\sphinxAtStartPar
The function has a mean of \(\mu + 0.57721\beta\) and a variance
of \(\frac{\pi^2}{6}\beta^2\).
\subsubsection*{References}
\subsubsection*{Examples}

\sphinxAtStartPar
Draw samples from the distribution:

\begin{sphinxVerbatim}[commandchars=\\\{\}]
\PYG{g+gp}{\PYGZgt{}\PYGZgt{}\PYGZgt{} }\PYG{n}{mu}\PYG{p}{,} \PYG{n}{beta} \PYG{o}{=} \PYG{l+m+mi}{0}\PYG{p}{,} \PYG{l+m+mf}{0.1} \PYG{c+c1}{\PYGZsh{} location and scale}
\PYG{g+gp}{\PYGZgt{}\PYGZgt{}\PYGZgt{} }\PYG{n}{s} \PYG{o}{=} \PYG{n}{np}\PYG{o}{.}\PYG{n}{random}\PYG{o}{.}\PYG{n}{gumbel}\PYG{p}{(}\PYG{n}{mu}\PYG{p}{,} \PYG{n}{beta}\PYG{p}{,} \PYG{l+m+mi}{1000}\PYG{p}{)}
\end{sphinxVerbatim}

\sphinxAtStartPar
Display the histogram of the samples, along with
the probability density function:

\begin{sphinxVerbatim}[commandchars=\\\{\}]
\PYG{g+gp}{\PYGZgt{}\PYGZgt{}\PYGZgt{} }\PYG{k+kn}{import} \PYG{n+nn}{matplotlib}\PYG{n+nn}{.}\PYG{n+nn}{pyplot} \PYG{k}{as} \PYG{n+nn}{plt}
\PYG{g+gp}{\PYGZgt{}\PYGZgt{}\PYGZgt{} }\PYG{n}{count}\PYG{p}{,} \PYG{n}{bins}\PYG{p}{,} \PYG{n}{ignored} \PYG{o}{=} \PYG{n}{plt}\PYG{o}{.}\PYG{n}{hist}\PYG{p}{(}\PYG{n}{s}\PYG{p}{,} \PYG{l+m+mi}{30}\PYG{p}{,} \PYG{n}{density}\PYG{o}{=}\PYG{k+kc}{True}\PYG{p}{)}
\PYG{g+gp}{\PYGZgt{}\PYGZgt{}\PYGZgt{} }\PYG{n}{plt}\PYG{o}{.}\PYG{n}{plot}\PYG{p}{(}\PYG{n}{bins}\PYG{p}{,} \PYG{p}{(}\PYG{l+m+mi}{1}\PYG{o}{/}\PYG{n}{beta}\PYG{p}{)}\PYG{o}{*}\PYG{n}{np}\PYG{o}{.}\PYG{n}{exp}\PYG{p}{(}\PYG{o}{\PYGZhy{}}\PYG{p}{(}\PYG{n}{bins} \PYG{o}{\PYGZhy{}} \PYG{n}{mu}\PYG{p}{)}\PYG{o}{/}\PYG{n}{beta}\PYG{p}{)}
\PYG{g+gp}{... }         \PYG{o}{*} \PYG{n}{np}\PYG{o}{.}\PYG{n}{exp}\PYG{p}{(} \PYG{o}{\PYGZhy{}}\PYG{n}{np}\PYG{o}{.}\PYG{n}{exp}\PYG{p}{(} \PYG{o}{\PYGZhy{}}\PYG{p}{(}\PYG{n}{bins} \PYG{o}{\PYGZhy{}} \PYG{n}{mu}\PYG{p}{)} \PYG{o}{/}\PYG{n}{beta}\PYG{p}{)} \PYG{p}{)}\PYG{p}{,}
\PYG{g+gp}{... }         \PYG{n}{linewidth}\PYG{o}{=}\PYG{l+m+mi}{2}\PYG{p}{,} \PYG{n}{color}\PYG{o}{=}\PYG{l+s+s1}{\PYGZsq{}}\PYG{l+s+s1}{r}\PYG{l+s+s1}{\PYGZsq{}}\PYG{p}{)}
\PYG{g+gp}{\PYGZgt{}\PYGZgt{}\PYGZgt{} }\PYG{n}{plt}\PYG{o}{.}\PYG{n}{show}\PYG{p}{(}\PYG{p}{)}
\end{sphinxVerbatim}

\sphinxAtStartPar
Show how an extreme value distribution can arise from a Gaussian process
and compare to a Gaussian:

\begin{sphinxVerbatim}[commandchars=\\\{\}]
\PYG{g+gp}{\PYGZgt{}\PYGZgt{}\PYGZgt{} }\PYG{n}{means} \PYG{o}{=} \PYG{p}{[}\PYG{p}{]}
\PYG{g+gp}{\PYGZgt{}\PYGZgt{}\PYGZgt{} }\PYG{n}{maxima} \PYG{o}{=} \PYG{p}{[}\PYG{p}{]}
\PYG{g+gp}{\PYGZgt{}\PYGZgt{}\PYGZgt{} }\PYG{k}{for} \PYG{n}{i} \PYG{o+ow}{in} \PYG{n+nb}{range}\PYG{p}{(}\PYG{l+m+mi}{0}\PYG{p}{,}\PYG{l+m+mi}{1000}\PYG{p}{)} \PYG{p}{:}
\PYG{g+gp}{... }   \PYG{n}{a} \PYG{o}{=} \PYG{n}{np}\PYG{o}{.}\PYG{n}{random}\PYG{o}{.}\PYG{n}{normal}\PYG{p}{(}\PYG{n}{mu}\PYG{p}{,} \PYG{n}{beta}\PYG{p}{,} \PYG{l+m+mi}{1000}\PYG{p}{)}
\PYG{g+gp}{... }   \PYG{n}{means}\PYG{o}{.}\PYG{n}{append}\PYG{p}{(}\PYG{n}{a}\PYG{o}{.}\PYG{n}{mean}\PYG{p}{(}\PYG{p}{)}\PYG{p}{)}
\PYG{g+gp}{... }   \PYG{n}{maxima}\PYG{o}{.}\PYG{n}{append}\PYG{p}{(}\PYG{n}{a}\PYG{o}{.}\PYG{n}{max}\PYG{p}{(}\PYG{p}{)}\PYG{p}{)}
\PYG{g+gp}{\PYGZgt{}\PYGZgt{}\PYGZgt{} }\PYG{n}{count}\PYG{p}{,} \PYG{n}{bins}\PYG{p}{,} \PYG{n}{ignored} \PYG{o}{=} \PYG{n}{plt}\PYG{o}{.}\PYG{n}{hist}\PYG{p}{(}\PYG{n}{maxima}\PYG{p}{,} \PYG{l+m+mi}{30}\PYG{p}{,} \PYG{n}{density}\PYG{o}{=}\PYG{k+kc}{True}\PYG{p}{)}
\PYG{g+gp}{\PYGZgt{}\PYGZgt{}\PYGZgt{} }\PYG{n}{beta} \PYG{o}{=} \PYG{n}{np}\PYG{o}{.}\PYG{n}{std}\PYG{p}{(}\PYG{n}{maxima}\PYG{p}{)} \PYG{o}{*} \PYG{n}{np}\PYG{o}{.}\PYG{n}{sqrt}\PYG{p}{(}\PYG{l+m+mi}{6}\PYG{p}{)} \PYG{o}{/} \PYG{n}{np}\PYG{o}{.}\PYG{n}{pi}
\PYG{g+gp}{\PYGZgt{}\PYGZgt{}\PYGZgt{} }\PYG{n}{mu} \PYG{o}{=} \PYG{n}{np}\PYG{o}{.}\PYG{n}{mean}\PYG{p}{(}\PYG{n}{maxima}\PYG{p}{)} \PYG{o}{\PYGZhy{}} \PYG{l+m+mf}{0.57721}\PYG{o}{*}\PYG{n}{beta}
\PYG{g+gp}{\PYGZgt{}\PYGZgt{}\PYGZgt{} }\PYG{n}{plt}\PYG{o}{.}\PYG{n}{plot}\PYG{p}{(}\PYG{n}{bins}\PYG{p}{,} \PYG{p}{(}\PYG{l+m+mi}{1}\PYG{o}{/}\PYG{n}{beta}\PYG{p}{)}\PYG{o}{*}\PYG{n}{np}\PYG{o}{.}\PYG{n}{exp}\PYG{p}{(}\PYG{o}{\PYGZhy{}}\PYG{p}{(}\PYG{n}{bins} \PYG{o}{\PYGZhy{}} \PYG{n}{mu}\PYG{p}{)}\PYG{o}{/}\PYG{n}{beta}\PYG{p}{)}
\PYG{g+gp}{... }         \PYG{o}{*} \PYG{n}{np}\PYG{o}{.}\PYG{n}{exp}\PYG{p}{(}\PYG{o}{\PYGZhy{}}\PYG{n}{np}\PYG{o}{.}\PYG{n}{exp}\PYG{p}{(}\PYG{o}{\PYGZhy{}}\PYG{p}{(}\PYG{n}{bins} \PYG{o}{\PYGZhy{}} \PYG{n}{mu}\PYG{p}{)}\PYG{o}{/}\PYG{n}{beta}\PYG{p}{)}\PYG{p}{)}\PYG{p}{,}
\PYG{g+gp}{... }         \PYG{n}{linewidth}\PYG{o}{=}\PYG{l+m+mi}{2}\PYG{p}{,} \PYG{n}{color}\PYG{o}{=}\PYG{l+s+s1}{\PYGZsq{}}\PYG{l+s+s1}{r}\PYG{l+s+s1}{\PYGZsq{}}\PYG{p}{)}
\PYG{g+gp}{\PYGZgt{}\PYGZgt{}\PYGZgt{} }\PYG{n}{plt}\PYG{o}{.}\PYG{n}{plot}\PYG{p}{(}\PYG{n}{bins}\PYG{p}{,} \PYG{l+m+mi}{1}\PYG{o}{/}\PYG{p}{(}\PYG{n}{beta} \PYG{o}{*} \PYG{n}{np}\PYG{o}{.}\PYG{n}{sqrt}\PYG{p}{(}\PYG{l+m+mi}{2} \PYG{o}{*} \PYG{n}{np}\PYG{o}{.}\PYG{n}{pi}\PYG{p}{)}\PYG{p}{)}
\PYG{g+gp}{... }         \PYG{o}{*} \PYG{n}{np}\PYG{o}{.}\PYG{n}{exp}\PYG{p}{(}\PYG{o}{\PYGZhy{}}\PYG{p}{(}\PYG{n}{bins} \PYG{o}{\PYGZhy{}} \PYG{n}{mu}\PYG{p}{)}\PYG{o}{*}\PYG{o}{*}\PYG{l+m+mi}{2} \PYG{o}{/} \PYG{p}{(}\PYG{l+m+mi}{2} \PYG{o}{*} \PYG{n}{beta}\PYG{o}{*}\PYG{o}{*}\PYG{l+m+mi}{2}\PYG{p}{)}\PYG{p}{)}\PYG{p}{,}
\PYG{g+gp}{... }         \PYG{n}{linewidth}\PYG{o}{=}\PYG{l+m+mi}{2}\PYG{p}{,} \PYG{n}{color}\PYG{o}{=}\PYG{l+s+s1}{\PYGZsq{}}\PYG{l+s+s1}{g}\PYG{l+s+s1}{\PYGZsq{}}\PYG{p}{)}
\PYG{g+gp}{\PYGZgt{}\PYGZgt{}\PYGZgt{} }\PYG{n}{plt}\PYG{o}{.}\PYG{n}{show}\PYG{p}{(}\PYG{p}{)}
\end{sphinxVerbatim}

\end{fulllineitems}

\index{harmonicityGetMax() (in module metilda.controllers.pitch\_art\_wizard)@\spxentry{harmonicityGetMax()}\spxextra{in module metilda.controllers.pitch\_art\_wizard}}

\begin{fulllineitems}
\phantomsection\label{\detokenize{metilda.controllers:metilda.controllers.pitch_art_wizard.harmonicityGetMax}}
\pysigstartsignatures
\pysiglinewithargsret{\sphinxcode{\sphinxupquote{metilda.controllers.pitch\_art\_wizard.}}\sphinxbfcode{\sphinxupquote{harmonicityGetMax}}}{\sphinxparam{\DUrole{n,n}{sound}}\sphinxparamcomma \sphinxparam{\DUrole{n,n}{start}}\sphinxparamcomma \sphinxparam{\DUrole{n,n}{end}}}{}
\pysigstopsignatures
\end{fulllineitems}

\index{harmonicityGetMin() (in module metilda.controllers.pitch\_art\_wizard)@\spxentry{harmonicityGetMin()}\spxextra{in module metilda.controllers.pitch\_art\_wizard}}

\begin{fulllineitems}
\phantomsection\label{\detokenize{metilda.controllers:metilda.controllers.pitch_art_wizard.harmonicityGetMin}}
\pysigstartsignatures
\pysiglinewithargsret{\sphinxcode{\sphinxupquote{metilda.controllers.pitch\_art\_wizard.}}\sphinxbfcode{\sphinxupquote{harmonicityGetMin}}}{\sphinxparam{\DUrole{n,n}{sound}}\sphinxparamcomma \sphinxparam{\DUrole{n,n}{start}}\sphinxparamcomma \sphinxparam{\DUrole{n,n}{end}}}{}
\pysigstopsignatures
\end{fulllineitems}

\index{harmonicityValueAtTime() (in module metilda.controllers.pitch\_art\_wizard)@\spxentry{harmonicityValueAtTime()}\spxextra{in module metilda.controllers.pitch\_art\_wizard}}

\begin{fulllineitems}
\phantomsection\label{\detokenize{metilda.controllers:metilda.controllers.pitch_art_wizard.harmonicityValueAtTime}}
\pysigstartsignatures
\pysiglinewithargsret{\sphinxcode{\sphinxupquote{metilda.controllers.pitch\_art\_wizard.}}\sphinxbfcode{\sphinxupquote{harmonicityValueAtTime}}}{\sphinxparam{\DUrole{n,n}{sound}}\sphinxparamcomma \sphinxparam{\DUrole{n,n}{time}}}{}
\pysigstopsignatures
\end{fulllineitems}

\index{hypergeometric() (in module metilda.controllers.pitch\_art\_wizard)@\spxentry{hypergeometric()}\spxextra{in module metilda.controllers.pitch\_art\_wizard}}

\begin{fulllineitems}
\phantomsection\label{\detokenize{metilda.controllers:metilda.controllers.pitch_art_wizard.hypergeometric}}
\pysigstartsignatures
\pysiglinewithargsret{\sphinxcode{\sphinxupquote{metilda.controllers.pitch\_art\_wizard.}}\sphinxbfcode{\sphinxupquote{hypergeometric}}}{\sphinxparam{\DUrole{n,n}{ngood}}\sphinxparamcomma \sphinxparam{\DUrole{n,n}{nbad}}\sphinxparamcomma \sphinxparam{\DUrole{n,n}{nsample}}\sphinxparamcomma \sphinxparam{\DUrole{n,n}{size}\DUrole{o,o}{=}\DUrole{default_value}{None}}}{}
\pysigstopsignatures
\sphinxAtStartPar
Draw samples from a Hypergeometric distribution.

\sphinxAtStartPar
Samples are drawn from a hypergeometric distribution with specified
parameters, \sphinxtitleref{ngood} (ways to make a good selection), \sphinxtitleref{nbad} (ways to make
a bad selection), and \sphinxtitleref{nsample} (number of items sampled, which is less
than or equal to the sum \sphinxcode{\sphinxupquote{ngood + nbad}}).

\begin{sphinxadmonition}{note}{Note:}
\sphinxAtStartPar
New code should use the
\sphinxtitleref{\textasciitilde{}numpy.random.Generator.hypergeometric}
method of a \sphinxtitleref{\textasciitilde{}numpy.random.Generator} instance instead;
please see the \DUrole{xref,std,std-ref}{random\sphinxhyphen{}quick\sphinxhyphen{}start}.
\end{sphinxadmonition}
\begin{quote}\begin{description}
\sphinxlineitem{Parameters}\begin{itemize}
\item {} 
\sphinxAtStartPar
\sphinxstyleliteralstrong{\sphinxupquote{ngood}} (\sphinxstyleliteralemphasis{\sphinxupquote{int}}\sphinxstyleliteralemphasis{\sphinxupquote{ or }}\sphinxstyleliteralemphasis{\sphinxupquote{array\_like}}\sphinxstyleliteralemphasis{\sphinxupquote{ of }}\sphinxstyleliteralemphasis{\sphinxupquote{ints}}) \textendash{} Number of ways to make a good selection.  Must be nonnegative.

\item {} 
\sphinxAtStartPar
\sphinxstyleliteralstrong{\sphinxupquote{nbad}} (\sphinxstyleliteralemphasis{\sphinxupquote{int}}\sphinxstyleliteralemphasis{\sphinxupquote{ or }}\sphinxstyleliteralemphasis{\sphinxupquote{array\_like}}\sphinxstyleliteralemphasis{\sphinxupquote{ of }}\sphinxstyleliteralemphasis{\sphinxupquote{ints}}) \textendash{} Number of ways to make a bad selection.  Must be nonnegative.

\item {} 
\sphinxAtStartPar
\sphinxstyleliteralstrong{\sphinxupquote{nsample}} (\sphinxstyleliteralemphasis{\sphinxupquote{int}}\sphinxstyleliteralemphasis{\sphinxupquote{ or }}\sphinxstyleliteralemphasis{\sphinxupquote{array\_like}}\sphinxstyleliteralemphasis{\sphinxupquote{ of }}\sphinxstyleliteralemphasis{\sphinxupquote{ints}}) \textendash{} Number of items sampled.  Must be at least 1 and at most
\sphinxcode{\sphinxupquote{ngood + nbad}}.

\item {} 
\sphinxAtStartPar
\sphinxstyleliteralstrong{\sphinxupquote{size}} (\sphinxstyleliteralemphasis{\sphinxupquote{int}}\sphinxstyleliteralemphasis{\sphinxupquote{ or }}\sphinxstyleliteralemphasis{\sphinxupquote{tuple}}\sphinxstyleliteralemphasis{\sphinxupquote{ of }}\sphinxstyleliteralemphasis{\sphinxupquote{ints}}\sphinxstyleliteralemphasis{\sphinxupquote{, }}\sphinxstyleliteralemphasis{\sphinxupquote{optional}}) \textendash{} Output shape.  If the given shape is, e.g., \sphinxcode{\sphinxupquote{(m, n, k)}}, then
\sphinxcode{\sphinxupquote{m * n * k}} samples are drawn.  If size is \sphinxcode{\sphinxupquote{None}} (default),
a single value is returned if \sphinxtitleref{ngood}, \sphinxtitleref{nbad}, and \sphinxtitleref{nsample}
are all scalars.  Otherwise, \sphinxcode{\sphinxupquote{np.broadcast(ngood, nbad, nsample).size}}
samples are drawn.

\end{itemize}

\sphinxlineitem{Returns}
\sphinxAtStartPar
\sphinxstylestrong{out} \textendash{} Drawn samples from the parameterized hypergeometric distribution. Each
sample is the number of good items within a randomly selected subset of
size \sphinxtitleref{nsample} taken from a set of \sphinxtitleref{ngood} good items and \sphinxtitleref{nbad} bad items.

\sphinxlineitem{Return type}
\sphinxAtStartPar
ndarray or scalar

\end{description}\end{quote}


\begin{sphinxseealso}{See also:}
\begin{description}
\sphinxlineitem{\sphinxcode{\sphinxupquote{scipy.stats.hypergeom}}}
\sphinxAtStartPar
probability density function, distribution or cumulative density function, etc.

\sphinxlineitem{\sphinxcode{\sphinxupquote{random.Generator.hypergeometric}}}
\sphinxAtStartPar
which should be used for new code.

\end{description}


\end{sphinxseealso}

\subsubsection*{Notes}

\sphinxAtStartPar
The probability density for the Hypergeometric distribution is
\begin{equation*}
\begin{split}P(x) = \frac{\binom{g}{x}\binom{b}{n-x}}{\binom{g+b}{n}},\end{split}
\end{equation*}
\sphinxAtStartPar
where \(0 \le x \le n\) and \(n-b \le x \le g\)

\sphinxAtStartPar
for P(x) the probability of \sphinxcode{\sphinxupquote{x}} good results in the drawn sample,
g = \sphinxtitleref{ngood}, b = \sphinxtitleref{nbad}, and n = \sphinxtitleref{nsample}.

\sphinxAtStartPar
Consider an urn with black and white marbles in it, \sphinxtitleref{ngood} of them
are black and \sphinxtitleref{nbad} are white. If you draw \sphinxtitleref{nsample} balls without
replacement, then the hypergeometric distribution describes the
distribution of black balls in the drawn sample.

\sphinxAtStartPar
Note that this distribution is very similar to the binomial
distribution, except that in this case, samples are drawn without
replacement, whereas in the Binomial case samples are drawn with
replacement (or the sample space is infinite). As the sample space
becomes large, this distribution approaches the binomial.
\subsubsection*{References}
\subsubsection*{Examples}

\sphinxAtStartPar
Draw samples from the distribution:

\begin{sphinxVerbatim}[commandchars=\\\{\}]
\PYG{g+gp}{\PYGZgt{}\PYGZgt{}\PYGZgt{} }\PYG{n}{ngood}\PYG{p}{,} \PYG{n}{nbad}\PYG{p}{,} \PYG{n}{nsamp} \PYG{o}{=} \PYG{l+m+mi}{100}\PYG{p}{,} \PYG{l+m+mi}{2}\PYG{p}{,} \PYG{l+m+mi}{10}
\PYG{g+go}{\PYGZsh{} number of good, number of bad, and number of samples}
\PYG{g+gp}{\PYGZgt{}\PYGZgt{}\PYGZgt{} }\PYG{n}{s} \PYG{o}{=} \PYG{n}{np}\PYG{o}{.}\PYG{n}{random}\PYG{o}{.}\PYG{n}{hypergeometric}\PYG{p}{(}\PYG{n}{ngood}\PYG{p}{,} \PYG{n}{nbad}\PYG{p}{,} \PYG{n}{nsamp}\PYG{p}{,} \PYG{l+m+mi}{1000}\PYG{p}{)}
\PYG{g+gp}{\PYGZgt{}\PYGZgt{}\PYGZgt{} }\PYG{k+kn}{from} \PYG{n+nn}{matplotlib}\PYG{n+nn}{.}\PYG{n+nn}{pyplot} \PYG{k+kn}{import} \PYG{n}{hist}
\PYG{g+gp}{\PYGZgt{}\PYGZgt{}\PYGZgt{} }\PYG{n}{hist}\PYG{p}{(}\PYG{n}{s}\PYG{p}{)}
\PYG{g+go}{\PYGZsh{}   note that it is very unlikely to grab both bad items}
\end{sphinxVerbatim}

\sphinxAtStartPar
Suppose you have an urn with 15 white and 15 black marbles.
If you pull 15 marbles at random, how likely is it that
12 or more of them are one color?

\begin{sphinxVerbatim}[commandchars=\\\{\}]
\PYG{g+gp}{\PYGZgt{}\PYGZgt{}\PYGZgt{} }\PYG{n}{s} \PYG{o}{=} \PYG{n}{np}\PYG{o}{.}\PYG{n}{random}\PYG{o}{.}\PYG{n}{hypergeometric}\PYG{p}{(}\PYG{l+m+mi}{15}\PYG{p}{,} \PYG{l+m+mi}{15}\PYG{p}{,} \PYG{l+m+mi}{15}\PYG{p}{,} \PYG{l+m+mi}{100000}\PYG{p}{)}
\PYG{g+gp}{\PYGZgt{}\PYGZgt{}\PYGZgt{} }\PYG{n+nb}{sum}\PYG{p}{(}\PYG{n}{s}\PYG{o}{\PYGZgt{}}\PYG{o}{=}\PYG{l+m+mi}{12}\PYG{p}{)}\PYG{o}{/}\PYG{l+m+mf}{100000.} \PYG{o}{+} \PYG{n+nb}{sum}\PYG{p}{(}\PYG{n}{s}\PYG{o}{\PYGZlt{}}\PYG{o}{=}\PYG{l+m+mi}{3}\PYG{p}{)}\PYG{o}{/}\PYG{l+m+mf}{100000.}
\PYG{g+go}{\PYGZsh{}   answer = 0.003 ... pretty unlikely!}
\end{sphinxVerbatim}

\end{fulllineitems}

\index{insert\_image\_analysis\_ids() (in module metilda.controllers.pitch\_art\_wizard)@\spxentry{insert\_image\_analysis\_ids()}\spxextra{in module metilda.controllers.pitch\_art\_wizard}}

\begin{fulllineitems}
\phantomsection\label{\detokenize{metilda.controllers:metilda.controllers.pitch_art_wizard.insert_image_analysis_ids}}
\pysigstartsignatures
\pysiglinewithargsret{\sphinxcode{\sphinxupquote{metilda.controllers.pitch\_art\_wizard.}}\sphinxbfcode{\sphinxupquote{insert\_image\_analysis\_ids}}}{}{}
\pysigstopsignatures
\end{fulllineitems}

\index{intensityBounds() (in module metilda.controllers.pitch\_art\_wizard)@\spxentry{intensityBounds()}\spxextra{in module metilda.controllers.pitch\_art\_wizard}}

\begin{fulllineitems}
\phantomsection\label{\detokenize{metilda.controllers:metilda.controllers.pitch_art_wizard.intensityBounds}}
\pysigstartsignatures
\pysiglinewithargsret{\sphinxcode{\sphinxupquote{metilda.controllers.pitch\_art\_wizard.}}\sphinxbfcode{\sphinxupquote{intensityBounds}}}{\sphinxparam{\DUrole{n,n}{sound}}}{}
\pysigstopsignatures
\end{fulllineitems}

\index{laplace() (in module metilda.controllers.pitch\_art\_wizard)@\spxentry{laplace()}\spxextra{in module metilda.controllers.pitch\_art\_wizard}}

\begin{fulllineitems}
\phantomsection\label{\detokenize{metilda.controllers:metilda.controllers.pitch_art_wizard.laplace}}
\pysigstartsignatures
\pysiglinewithargsret{\sphinxcode{\sphinxupquote{metilda.controllers.pitch\_art\_wizard.}}\sphinxbfcode{\sphinxupquote{laplace}}}{\sphinxparam{\DUrole{n,n}{loc}\DUrole{o,o}{=}\DUrole{default_value}{0.0}}\sphinxparamcomma \sphinxparam{\DUrole{n,n}{scale}\DUrole{o,o}{=}\DUrole{default_value}{1.0}}\sphinxparamcomma \sphinxparam{\DUrole{n,n}{size}\DUrole{o,o}{=}\DUrole{default_value}{None}}}{}
\pysigstopsignatures
\sphinxAtStartPar
Draw samples from the Laplace or double exponential distribution with
specified location (or mean) and scale (decay).

\sphinxAtStartPar
The Laplace distribution is similar to the Gaussian/normal distribution,
but is sharper at the peak and has fatter tails. It represents the
difference between two independent, identically distributed exponential
random variables.

\begin{sphinxadmonition}{note}{Note:}
\sphinxAtStartPar
New code should use the \sphinxtitleref{\textasciitilde{}numpy.random.Generator.laplace}
method of a \sphinxtitleref{\textasciitilde{}numpy.random.Generator} instance instead;
please see the \DUrole{xref,std,std-ref}{random\sphinxhyphen{}quick\sphinxhyphen{}start}.
\end{sphinxadmonition}
\begin{quote}\begin{description}
\sphinxlineitem{Parameters}\begin{itemize}
\item {} 
\sphinxAtStartPar
\sphinxstyleliteralstrong{\sphinxupquote{loc}} (\sphinxstyleliteralemphasis{\sphinxupquote{float}}\sphinxstyleliteralemphasis{\sphinxupquote{ or }}\sphinxstyleliteralemphasis{\sphinxupquote{array\_like}}\sphinxstyleliteralemphasis{\sphinxupquote{ of }}\sphinxstyleliteralemphasis{\sphinxupquote{floats}}\sphinxstyleliteralemphasis{\sphinxupquote{, }}\sphinxstyleliteralemphasis{\sphinxupquote{optional}}) \textendash{} The position, \(\mu\), of the distribution peak. Default is 0.

\item {} 
\sphinxAtStartPar
\sphinxstyleliteralstrong{\sphinxupquote{scale}} (\sphinxstyleliteralemphasis{\sphinxupquote{float}}\sphinxstyleliteralemphasis{\sphinxupquote{ or }}\sphinxstyleliteralemphasis{\sphinxupquote{array\_like}}\sphinxstyleliteralemphasis{\sphinxupquote{ of }}\sphinxstyleliteralemphasis{\sphinxupquote{floats}}\sphinxstyleliteralemphasis{\sphinxupquote{, }}\sphinxstyleliteralemphasis{\sphinxupquote{optional}}) \textendash{} \(\lambda\), the exponential decay. Default is 1. Must be non\sphinxhyphen{}
negative.

\item {} 
\sphinxAtStartPar
\sphinxstyleliteralstrong{\sphinxupquote{size}} (\sphinxstyleliteralemphasis{\sphinxupquote{int}}\sphinxstyleliteralemphasis{\sphinxupquote{ or }}\sphinxstyleliteralemphasis{\sphinxupquote{tuple}}\sphinxstyleliteralemphasis{\sphinxupquote{ of }}\sphinxstyleliteralemphasis{\sphinxupquote{ints}}\sphinxstyleliteralemphasis{\sphinxupquote{, }}\sphinxstyleliteralemphasis{\sphinxupquote{optional}}) \textendash{} Output shape.  If the given shape is, e.g., \sphinxcode{\sphinxupquote{(m, n, k)}}, then
\sphinxcode{\sphinxupquote{m * n * k}} samples are drawn.  If size is \sphinxcode{\sphinxupquote{None}} (default),
a single value is returned if \sphinxcode{\sphinxupquote{loc}} and \sphinxcode{\sphinxupquote{scale}} are both scalars.
Otherwise, \sphinxcode{\sphinxupquote{np.broadcast(loc, scale).size}} samples are drawn.

\end{itemize}

\sphinxlineitem{Returns}
\sphinxAtStartPar
\sphinxstylestrong{out} \textendash{} Drawn samples from the parameterized Laplace distribution.

\sphinxlineitem{Return type}
\sphinxAtStartPar
ndarray or scalar

\end{description}\end{quote}


\begin{sphinxseealso}{See also:}
\begin{description}
\sphinxlineitem{\sphinxcode{\sphinxupquote{random.Generator.laplace}}}
\sphinxAtStartPar
which should be used for new code.

\end{description}


\end{sphinxseealso}

\subsubsection*{Notes}

\sphinxAtStartPar
It has the probability density function
\begin{equation*}
\begin{split}f(x; \mu, \lambda) = \frac{1}{2\lambda}
\exp\left(-\frac{|x - \mu|}{\lambda}\right).\end{split}
\end{equation*}
\sphinxAtStartPar
The first law of Laplace, from 1774, states that the frequency
of an error can be expressed as an exponential function of the
absolute magnitude of the error, which leads to the Laplace
distribution. For many problems in economics and health
sciences, this distribution seems to model the data better
than the standard Gaussian distribution.
\subsubsection*{References}
\subsubsection*{Examples}

\sphinxAtStartPar
Draw samples from the distribution

\begin{sphinxVerbatim}[commandchars=\\\{\}]
\PYG{g+gp}{\PYGZgt{}\PYGZgt{}\PYGZgt{} }\PYG{n}{loc}\PYG{p}{,} \PYG{n}{scale} \PYG{o}{=} \PYG{l+m+mf}{0.}\PYG{p}{,} \PYG{l+m+mf}{1.}
\PYG{g+gp}{\PYGZgt{}\PYGZgt{}\PYGZgt{} }\PYG{n}{s} \PYG{o}{=} \PYG{n}{np}\PYG{o}{.}\PYG{n}{random}\PYG{o}{.}\PYG{n}{laplace}\PYG{p}{(}\PYG{n}{loc}\PYG{p}{,} \PYG{n}{scale}\PYG{p}{,} \PYG{l+m+mi}{1000}\PYG{p}{)}
\end{sphinxVerbatim}

\sphinxAtStartPar
Display the histogram of the samples, along with
the probability density function:

\begin{sphinxVerbatim}[commandchars=\\\{\}]
\PYG{g+gp}{\PYGZgt{}\PYGZgt{}\PYGZgt{} }\PYG{k+kn}{import} \PYG{n+nn}{matplotlib}\PYG{n+nn}{.}\PYG{n+nn}{pyplot} \PYG{k}{as} \PYG{n+nn}{plt}
\PYG{g+gp}{\PYGZgt{}\PYGZgt{}\PYGZgt{} }\PYG{n}{count}\PYG{p}{,} \PYG{n}{bins}\PYG{p}{,} \PYG{n}{ignored} \PYG{o}{=} \PYG{n}{plt}\PYG{o}{.}\PYG{n}{hist}\PYG{p}{(}\PYG{n}{s}\PYG{p}{,} \PYG{l+m+mi}{30}\PYG{p}{,} \PYG{n}{density}\PYG{o}{=}\PYG{k+kc}{True}\PYG{p}{)}
\PYG{g+gp}{\PYGZgt{}\PYGZgt{}\PYGZgt{} }\PYG{n}{x} \PYG{o}{=} \PYG{n}{np}\PYG{o}{.}\PYG{n}{arange}\PYG{p}{(}\PYG{o}{\PYGZhy{}}\PYG{l+m+mf}{8.}\PYG{p}{,} \PYG{l+m+mf}{8.}\PYG{p}{,} \PYG{l+m+mf}{.01}\PYG{p}{)}
\PYG{g+gp}{\PYGZgt{}\PYGZgt{}\PYGZgt{} }\PYG{n}{pdf} \PYG{o}{=} \PYG{n}{np}\PYG{o}{.}\PYG{n}{exp}\PYG{p}{(}\PYG{o}{\PYGZhy{}}\PYG{n+nb}{abs}\PYG{p}{(}\PYG{n}{x}\PYG{o}{\PYGZhy{}}\PYG{n}{loc}\PYG{p}{)}\PYG{o}{/}\PYG{n}{scale}\PYG{p}{)}\PYG{o}{/}\PYG{p}{(}\PYG{l+m+mf}{2.}\PYG{o}{*}\PYG{n}{scale}\PYG{p}{)}
\PYG{g+gp}{\PYGZgt{}\PYGZgt{}\PYGZgt{} }\PYG{n}{plt}\PYG{o}{.}\PYG{n}{plot}\PYG{p}{(}\PYG{n}{x}\PYG{p}{,} \PYG{n}{pdf}\PYG{p}{)}
\end{sphinxVerbatim}

\sphinxAtStartPar
Plot Gaussian for comparison:

\begin{sphinxVerbatim}[commandchars=\\\{\}]
\PYG{g+gp}{\PYGZgt{}\PYGZgt{}\PYGZgt{} }\PYG{n}{g} \PYG{o}{=} \PYG{p}{(}\PYG{l+m+mi}{1}\PYG{o}{/}\PYG{p}{(}\PYG{n}{scale} \PYG{o}{*} \PYG{n}{np}\PYG{o}{.}\PYG{n}{sqrt}\PYG{p}{(}\PYG{l+m+mi}{2} \PYG{o}{*} \PYG{n}{np}\PYG{o}{.}\PYG{n}{pi}\PYG{p}{)}\PYG{p}{)} \PYG{o}{*}
\PYG{g+gp}{... }     \PYG{n}{np}\PYG{o}{.}\PYG{n}{exp}\PYG{p}{(}\PYG{o}{\PYGZhy{}}\PYG{p}{(}\PYG{n}{x} \PYG{o}{\PYGZhy{}} \PYG{n}{loc}\PYG{p}{)}\PYG{o}{*}\PYG{o}{*}\PYG{l+m+mi}{2} \PYG{o}{/} \PYG{p}{(}\PYG{l+m+mi}{2} \PYG{o}{*} \PYG{n}{scale}\PYG{o}{*}\PYG{o}{*}\PYG{l+m+mi}{2}\PYG{p}{)}\PYG{p}{)}\PYG{p}{)}
\PYG{g+gp}{\PYGZgt{}\PYGZgt{}\PYGZgt{} }\PYG{n}{plt}\PYG{o}{.}\PYG{n}{plot}\PYG{p}{(}\PYG{n}{x}\PYG{p}{,}\PYG{n}{g}\PYG{p}{)}
\end{sphinxVerbatim}

\end{fulllineitems}

\index{logistic() (in module metilda.controllers.pitch\_art\_wizard)@\spxentry{logistic()}\spxextra{in module metilda.controllers.pitch\_art\_wizard}}

\begin{fulllineitems}
\phantomsection\label{\detokenize{metilda.controllers:metilda.controllers.pitch_art_wizard.logistic}}
\pysigstartsignatures
\pysiglinewithargsret{\sphinxcode{\sphinxupquote{metilda.controllers.pitch\_art\_wizard.}}\sphinxbfcode{\sphinxupquote{logistic}}}{\sphinxparam{\DUrole{n,n}{loc}\DUrole{o,o}{=}\DUrole{default_value}{0.0}}\sphinxparamcomma \sphinxparam{\DUrole{n,n}{scale}\DUrole{o,o}{=}\DUrole{default_value}{1.0}}\sphinxparamcomma \sphinxparam{\DUrole{n,n}{size}\DUrole{o,o}{=}\DUrole{default_value}{None}}}{}
\pysigstopsignatures
\sphinxAtStartPar
Draw samples from a logistic distribution.

\sphinxAtStartPar
Samples are drawn from a logistic distribution with specified
parameters, loc (location or mean, also median), and scale (\textgreater{}0).

\begin{sphinxadmonition}{note}{Note:}
\sphinxAtStartPar
New code should use the \sphinxtitleref{\textasciitilde{}numpy.random.Generator.logistic}
method of a \sphinxtitleref{\textasciitilde{}numpy.random.Generator} instance instead;
please see the \DUrole{xref,std,std-ref}{random\sphinxhyphen{}quick\sphinxhyphen{}start}.
\end{sphinxadmonition}
\begin{quote}\begin{description}
\sphinxlineitem{Parameters}\begin{itemize}
\item {} 
\sphinxAtStartPar
\sphinxstyleliteralstrong{\sphinxupquote{loc}} (\sphinxstyleliteralemphasis{\sphinxupquote{float}}\sphinxstyleliteralemphasis{\sphinxupquote{ or }}\sphinxstyleliteralemphasis{\sphinxupquote{array\_like}}\sphinxstyleliteralemphasis{\sphinxupquote{ of }}\sphinxstyleliteralemphasis{\sphinxupquote{floats}}\sphinxstyleliteralemphasis{\sphinxupquote{, }}\sphinxstyleliteralemphasis{\sphinxupquote{optional}}) \textendash{} Parameter of the distribution. Default is 0.

\item {} 
\sphinxAtStartPar
\sphinxstyleliteralstrong{\sphinxupquote{scale}} (\sphinxstyleliteralemphasis{\sphinxupquote{float}}\sphinxstyleliteralemphasis{\sphinxupquote{ or }}\sphinxstyleliteralemphasis{\sphinxupquote{array\_like}}\sphinxstyleliteralemphasis{\sphinxupquote{ of }}\sphinxstyleliteralemphasis{\sphinxupquote{floats}}\sphinxstyleliteralemphasis{\sphinxupquote{, }}\sphinxstyleliteralemphasis{\sphinxupquote{optional}}) \textendash{} Parameter of the distribution. Must be non\sphinxhyphen{}negative.
Default is 1.

\item {} 
\sphinxAtStartPar
\sphinxstyleliteralstrong{\sphinxupquote{size}} (\sphinxstyleliteralemphasis{\sphinxupquote{int}}\sphinxstyleliteralemphasis{\sphinxupquote{ or }}\sphinxstyleliteralemphasis{\sphinxupquote{tuple}}\sphinxstyleliteralemphasis{\sphinxupquote{ of }}\sphinxstyleliteralemphasis{\sphinxupquote{ints}}\sphinxstyleliteralemphasis{\sphinxupquote{, }}\sphinxstyleliteralemphasis{\sphinxupquote{optional}}) \textendash{} Output shape.  If the given shape is, e.g., \sphinxcode{\sphinxupquote{(m, n, k)}}, then
\sphinxcode{\sphinxupquote{m * n * k}} samples are drawn.  If size is \sphinxcode{\sphinxupquote{None}} (default),
a single value is returned if \sphinxcode{\sphinxupquote{loc}} and \sphinxcode{\sphinxupquote{scale}} are both scalars.
Otherwise, \sphinxcode{\sphinxupquote{np.broadcast(loc, scale).size}} samples are drawn.

\end{itemize}

\sphinxlineitem{Returns}
\sphinxAtStartPar
\sphinxstylestrong{out} \textendash{} Drawn samples from the parameterized logistic distribution.

\sphinxlineitem{Return type}
\sphinxAtStartPar
ndarray or scalar

\end{description}\end{quote}


\begin{sphinxseealso}{See also:}
\begin{description}
\sphinxlineitem{\sphinxcode{\sphinxupquote{scipy.stats.logistic}}}
\sphinxAtStartPar
probability density function, distribution or cumulative density function, etc.

\sphinxlineitem{\sphinxcode{\sphinxupquote{random.Generator.logistic}}}
\sphinxAtStartPar
which should be used for new code.

\end{description}


\end{sphinxseealso}

\subsubsection*{Notes}

\sphinxAtStartPar
The probability density for the Logistic distribution is
\begin{equation*}
\begin{split}P(x) = P(x) = \frac{e^{-(x-\mu)/s}}{s(1+e^{-(x-\mu)/s})^2},\end{split}
\end{equation*}
\sphinxAtStartPar
where \(\mu\) = location and \(s\) = scale.

\sphinxAtStartPar
The Logistic distribution is used in Extreme Value problems where it
can act as a mixture of Gumbel distributions, in Epidemiology, and by
the World Chess Federation (FIDE) where it is used in the Elo ranking
system, assuming the performance of each player is a logistically
distributed random variable.
\subsubsection*{References}
\subsubsection*{Examples}

\sphinxAtStartPar
Draw samples from the distribution:

\begin{sphinxVerbatim}[commandchars=\\\{\}]
\PYG{g+gp}{\PYGZgt{}\PYGZgt{}\PYGZgt{} }\PYG{n}{loc}\PYG{p}{,} \PYG{n}{scale} \PYG{o}{=} \PYG{l+m+mi}{10}\PYG{p}{,} \PYG{l+m+mi}{1}
\PYG{g+gp}{\PYGZgt{}\PYGZgt{}\PYGZgt{} }\PYG{n}{s} \PYG{o}{=} \PYG{n}{np}\PYG{o}{.}\PYG{n}{random}\PYG{o}{.}\PYG{n}{logistic}\PYG{p}{(}\PYG{n}{loc}\PYG{p}{,} \PYG{n}{scale}\PYG{p}{,} \PYG{l+m+mi}{10000}\PYG{p}{)}
\PYG{g+gp}{\PYGZgt{}\PYGZgt{}\PYGZgt{} }\PYG{k+kn}{import} \PYG{n+nn}{matplotlib}\PYG{n+nn}{.}\PYG{n+nn}{pyplot} \PYG{k}{as} \PYG{n+nn}{plt}
\PYG{g+gp}{\PYGZgt{}\PYGZgt{}\PYGZgt{} }\PYG{n}{count}\PYG{p}{,} \PYG{n}{bins}\PYG{p}{,} \PYG{n}{ignored} \PYG{o}{=} \PYG{n}{plt}\PYG{o}{.}\PYG{n}{hist}\PYG{p}{(}\PYG{n}{s}\PYG{p}{,} \PYG{n}{bins}\PYG{o}{=}\PYG{l+m+mi}{50}\PYG{p}{)}
\end{sphinxVerbatim}

\sphinxAtStartPar
\#   plot against distribution

\begin{sphinxVerbatim}[commandchars=\\\{\}]
\PYG{g+gp}{\PYGZgt{}\PYGZgt{}\PYGZgt{} }\PYG{k}{def} \PYG{n+nf}{logist}\PYG{p}{(}\PYG{n}{x}\PYG{p}{,} \PYG{n}{loc}\PYG{p}{,} \PYG{n}{scale}\PYG{p}{)}\PYG{p}{:}
\PYG{g+gp}{... }    \PYG{k}{return} \PYG{n}{np}\PYG{o}{.}\PYG{n}{exp}\PYG{p}{(}\PYG{p}{(}\PYG{n}{loc}\PYG{o}{\PYGZhy{}}\PYG{n}{x}\PYG{p}{)}\PYG{o}{/}\PYG{n}{scale}\PYG{p}{)}\PYG{o}{/}\PYG{p}{(}\PYG{n}{scale}\PYG{o}{*}\PYG{p}{(}\PYG{l+m+mi}{1}\PYG{o}{+}\PYG{n}{np}\PYG{o}{.}\PYG{n}{exp}\PYG{p}{(}\PYG{p}{(}\PYG{n}{loc}\PYG{o}{\PYGZhy{}}\PYG{n}{x}\PYG{p}{)}\PYG{o}{/}\PYG{n}{scale}\PYG{p}{)}\PYG{p}{)}\PYG{o}{*}\PYG{o}{*}\PYG{l+m+mi}{2}\PYG{p}{)}
\PYG{g+gp}{\PYGZgt{}\PYGZgt{}\PYGZgt{} }\PYG{n}{lgst\PYGZus{}val} \PYG{o}{=} \PYG{n}{logist}\PYG{p}{(}\PYG{n}{bins}\PYG{p}{,} \PYG{n}{loc}\PYG{p}{,} \PYG{n}{scale}\PYG{p}{)}
\PYG{g+gp}{\PYGZgt{}\PYGZgt{}\PYGZgt{} }\PYG{n}{plt}\PYG{o}{.}\PYG{n}{plot}\PYG{p}{(}\PYG{n}{bins}\PYG{p}{,} \PYG{n}{lgst\PYGZus{}val} \PYG{o}{*} \PYG{n}{count}\PYG{o}{.}\PYG{n}{max}\PYG{p}{(}\PYG{p}{)} \PYG{o}{/} \PYG{n}{lgst\PYGZus{}val}\PYG{o}{.}\PYG{n}{max}\PYG{p}{(}\PYG{p}{)}\PYG{p}{)}
\PYG{g+gp}{\PYGZgt{}\PYGZgt{}\PYGZgt{} }\PYG{n}{plt}\PYG{o}{.}\PYG{n}{show}\PYG{p}{(}\PYG{p}{)}
\end{sphinxVerbatim}

\end{fulllineitems}

\index{lognormal() (in module metilda.controllers.pitch\_art\_wizard)@\spxentry{lognormal()}\spxextra{in module metilda.controllers.pitch\_art\_wizard}}

\begin{fulllineitems}
\phantomsection\label{\detokenize{metilda.controllers:metilda.controllers.pitch_art_wizard.lognormal}}
\pysigstartsignatures
\pysiglinewithargsret{\sphinxcode{\sphinxupquote{metilda.controllers.pitch\_art\_wizard.}}\sphinxbfcode{\sphinxupquote{lognormal}}}{\sphinxparam{\DUrole{n,n}{mean}\DUrole{o,o}{=}\DUrole{default_value}{0.0}}\sphinxparamcomma \sphinxparam{\DUrole{n,n}{sigma}\DUrole{o,o}{=}\DUrole{default_value}{1.0}}\sphinxparamcomma \sphinxparam{\DUrole{n,n}{size}\DUrole{o,o}{=}\DUrole{default_value}{None}}}{}
\pysigstopsignatures
\sphinxAtStartPar
Draw samples from a log\sphinxhyphen{}normal distribution.

\sphinxAtStartPar
Draw samples from a log\sphinxhyphen{}normal distribution with specified mean,
standard deviation, and array shape.  Note that the mean and standard
deviation are not the values for the distribution itself, but of the
underlying normal distribution it is derived from.

\begin{sphinxadmonition}{note}{Note:}
\sphinxAtStartPar
New code should use the \sphinxtitleref{\textasciitilde{}numpy.random.Generator.lognormal}
method of a \sphinxtitleref{\textasciitilde{}numpy.random.Generator} instance instead;
please see the \DUrole{xref,std,std-ref}{random\sphinxhyphen{}quick\sphinxhyphen{}start}.
\end{sphinxadmonition}
\begin{quote}\begin{description}
\sphinxlineitem{Parameters}\begin{itemize}
\item {} 
\sphinxAtStartPar
\sphinxstyleliteralstrong{\sphinxupquote{mean}} (\sphinxstyleliteralemphasis{\sphinxupquote{float}}\sphinxstyleliteralemphasis{\sphinxupquote{ or }}\sphinxstyleliteralemphasis{\sphinxupquote{array\_like}}\sphinxstyleliteralemphasis{\sphinxupquote{ of }}\sphinxstyleliteralemphasis{\sphinxupquote{floats}}\sphinxstyleliteralemphasis{\sphinxupquote{, }}\sphinxstyleliteralemphasis{\sphinxupquote{optional}}) \textendash{} Mean value of the underlying normal distribution. Default is 0.

\item {} 
\sphinxAtStartPar
\sphinxstyleliteralstrong{\sphinxupquote{sigma}} (\sphinxstyleliteralemphasis{\sphinxupquote{float}}\sphinxstyleliteralemphasis{\sphinxupquote{ or }}\sphinxstyleliteralemphasis{\sphinxupquote{array\_like}}\sphinxstyleliteralemphasis{\sphinxupquote{ of }}\sphinxstyleliteralemphasis{\sphinxupquote{floats}}\sphinxstyleliteralemphasis{\sphinxupquote{, }}\sphinxstyleliteralemphasis{\sphinxupquote{optional}}) \textendash{} Standard deviation of the underlying normal distribution. Must be
non\sphinxhyphen{}negative. Default is 1.

\item {} 
\sphinxAtStartPar
\sphinxstyleliteralstrong{\sphinxupquote{size}} (\sphinxstyleliteralemphasis{\sphinxupquote{int}}\sphinxstyleliteralemphasis{\sphinxupquote{ or }}\sphinxstyleliteralemphasis{\sphinxupquote{tuple}}\sphinxstyleliteralemphasis{\sphinxupquote{ of }}\sphinxstyleliteralemphasis{\sphinxupquote{ints}}\sphinxstyleliteralemphasis{\sphinxupquote{, }}\sphinxstyleliteralemphasis{\sphinxupquote{optional}}) \textendash{} Output shape.  If the given shape is, e.g., \sphinxcode{\sphinxupquote{(m, n, k)}}, then
\sphinxcode{\sphinxupquote{m * n * k}} samples are drawn.  If size is \sphinxcode{\sphinxupquote{None}} (default),
a single value is returned if \sphinxcode{\sphinxupquote{mean}} and \sphinxcode{\sphinxupquote{sigma}} are both scalars.
Otherwise, \sphinxcode{\sphinxupquote{np.broadcast(mean, sigma).size}} samples are drawn.

\end{itemize}

\sphinxlineitem{Returns}
\sphinxAtStartPar
\sphinxstylestrong{out} \textendash{} Drawn samples from the parameterized log\sphinxhyphen{}normal distribution.

\sphinxlineitem{Return type}
\sphinxAtStartPar
ndarray or scalar

\end{description}\end{quote}


\begin{sphinxseealso}{See also:}
\begin{description}
\sphinxlineitem{\sphinxcode{\sphinxupquote{scipy.stats.lognorm}}}
\sphinxAtStartPar
probability density function, distribution, cumulative density function, etc.

\sphinxlineitem{\sphinxcode{\sphinxupquote{random.Generator.lognormal}}}
\sphinxAtStartPar
which should be used for new code.

\end{description}


\end{sphinxseealso}

\subsubsection*{Notes}

\sphinxAtStartPar
A variable \sphinxtitleref{x} has a log\sphinxhyphen{}normal distribution if \sphinxtitleref{log(x)} is normally
distributed.  The probability density function for the log\sphinxhyphen{}normal
distribution is:
\begin{equation*}
\begin{split}p(x) = \frac{1}{\sigma x \sqrt{2\pi}}
e^{(-\frac{(ln(x)-\mu)^2}{2\sigma^2})}\end{split}
\end{equation*}
\sphinxAtStartPar
where \(\mu\) is the mean and \(\sigma\) is the standard
deviation of the normally distributed logarithm of the variable.
A log\sphinxhyphen{}normal distribution results if a random variable is the \sphinxstyleemphasis{product}
of a large number of independent, identically\sphinxhyphen{}distributed variables in
the same way that a normal distribution results if the variable is the
\sphinxstyleemphasis{sum} of a large number of independent, identically\sphinxhyphen{}distributed
variables.
\subsubsection*{References}
\subsubsection*{Examples}

\sphinxAtStartPar
Draw samples from the distribution:

\begin{sphinxVerbatim}[commandchars=\\\{\}]
\PYG{g+gp}{\PYGZgt{}\PYGZgt{}\PYGZgt{} }\PYG{n}{mu}\PYG{p}{,} \PYG{n}{sigma} \PYG{o}{=} \PYG{l+m+mf}{3.}\PYG{p}{,} \PYG{l+m+mf}{1.} \PYG{c+c1}{\PYGZsh{} mean and standard deviation}
\PYG{g+gp}{\PYGZgt{}\PYGZgt{}\PYGZgt{} }\PYG{n}{s} \PYG{o}{=} \PYG{n}{np}\PYG{o}{.}\PYG{n}{random}\PYG{o}{.}\PYG{n}{lognormal}\PYG{p}{(}\PYG{n}{mu}\PYG{p}{,} \PYG{n}{sigma}\PYG{p}{,} \PYG{l+m+mi}{1000}\PYG{p}{)}
\end{sphinxVerbatim}

\sphinxAtStartPar
Display the histogram of the samples, along with
the probability density function:

\begin{sphinxVerbatim}[commandchars=\\\{\}]
\PYG{g+gp}{\PYGZgt{}\PYGZgt{}\PYGZgt{} }\PYG{k+kn}{import} \PYG{n+nn}{matplotlib}\PYG{n+nn}{.}\PYG{n+nn}{pyplot} \PYG{k}{as} \PYG{n+nn}{plt}
\PYG{g+gp}{\PYGZgt{}\PYGZgt{}\PYGZgt{} }\PYG{n}{count}\PYG{p}{,} \PYG{n}{bins}\PYG{p}{,} \PYG{n}{ignored} \PYG{o}{=} \PYG{n}{plt}\PYG{o}{.}\PYG{n}{hist}\PYG{p}{(}\PYG{n}{s}\PYG{p}{,} \PYG{l+m+mi}{100}\PYG{p}{,} \PYG{n}{density}\PYG{o}{=}\PYG{k+kc}{True}\PYG{p}{,} \PYG{n}{align}\PYG{o}{=}\PYG{l+s+s1}{\PYGZsq{}}\PYG{l+s+s1}{mid}\PYG{l+s+s1}{\PYGZsq{}}\PYG{p}{)}
\end{sphinxVerbatim}

\begin{sphinxVerbatim}[commandchars=\\\{\}]
\PYG{g+gp}{\PYGZgt{}\PYGZgt{}\PYGZgt{} }\PYG{n}{x} \PYG{o}{=} \PYG{n}{np}\PYG{o}{.}\PYG{n}{linspace}\PYG{p}{(}\PYG{n+nb}{min}\PYG{p}{(}\PYG{n}{bins}\PYG{p}{)}\PYG{p}{,} \PYG{n+nb}{max}\PYG{p}{(}\PYG{n}{bins}\PYG{p}{)}\PYG{p}{,} \PYG{l+m+mi}{10000}\PYG{p}{)}
\PYG{g+gp}{\PYGZgt{}\PYGZgt{}\PYGZgt{} }\PYG{n}{pdf} \PYG{o}{=} \PYG{p}{(}\PYG{n}{np}\PYG{o}{.}\PYG{n}{exp}\PYG{p}{(}\PYG{o}{\PYGZhy{}}\PYG{p}{(}\PYG{n}{np}\PYG{o}{.}\PYG{n}{log}\PYG{p}{(}\PYG{n}{x}\PYG{p}{)} \PYG{o}{\PYGZhy{}} \PYG{n}{mu}\PYG{p}{)}\PYG{o}{*}\PYG{o}{*}\PYG{l+m+mi}{2} \PYG{o}{/} \PYG{p}{(}\PYG{l+m+mi}{2} \PYG{o}{*} \PYG{n}{sigma}\PYG{o}{*}\PYG{o}{*}\PYG{l+m+mi}{2}\PYG{p}{)}\PYG{p}{)}
\PYG{g+gp}{... }       \PYG{o}{/} \PYG{p}{(}\PYG{n}{x} \PYG{o}{*} \PYG{n}{sigma} \PYG{o}{*} \PYG{n}{np}\PYG{o}{.}\PYG{n}{sqrt}\PYG{p}{(}\PYG{l+m+mi}{2} \PYG{o}{*} \PYG{n}{np}\PYG{o}{.}\PYG{n}{pi}\PYG{p}{)}\PYG{p}{)}\PYG{p}{)}
\end{sphinxVerbatim}

\begin{sphinxVerbatim}[commandchars=\\\{\}]
\PYG{g+gp}{\PYGZgt{}\PYGZgt{}\PYGZgt{} }\PYG{n}{plt}\PYG{o}{.}\PYG{n}{plot}\PYG{p}{(}\PYG{n}{x}\PYG{p}{,} \PYG{n}{pdf}\PYG{p}{,} \PYG{n}{linewidth}\PYG{o}{=}\PYG{l+m+mi}{2}\PYG{p}{,} \PYG{n}{color}\PYG{o}{=}\PYG{l+s+s1}{\PYGZsq{}}\PYG{l+s+s1}{r}\PYG{l+s+s1}{\PYGZsq{}}\PYG{p}{)}
\PYG{g+gp}{\PYGZgt{}\PYGZgt{}\PYGZgt{} }\PYG{n}{plt}\PYG{o}{.}\PYG{n}{axis}\PYG{p}{(}\PYG{l+s+s1}{\PYGZsq{}}\PYG{l+s+s1}{tight}\PYG{l+s+s1}{\PYGZsq{}}\PYG{p}{)}
\PYG{g+gp}{\PYGZgt{}\PYGZgt{}\PYGZgt{} }\PYG{n}{plt}\PYG{o}{.}\PYG{n}{show}\PYG{p}{(}\PYG{p}{)}
\end{sphinxVerbatim}

\sphinxAtStartPar
Demonstrate that taking the products of random samples from a uniform
distribution can be fit well by a log\sphinxhyphen{}normal probability density
function.

\begin{sphinxVerbatim}[commandchars=\\\{\}]
\PYG{g+gp}{\PYGZgt{}\PYGZgt{}\PYGZgt{} }\PYG{c+c1}{\PYGZsh{} Generate a thousand samples: each is the product of 100 random}
\PYG{g+gp}{\PYGZgt{}\PYGZgt{}\PYGZgt{} }\PYG{c+c1}{\PYGZsh{} values, drawn from a normal distribution.}
\PYG{g+gp}{\PYGZgt{}\PYGZgt{}\PYGZgt{} }\PYG{n}{b} \PYG{o}{=} \PYG{p}{[}\PYG{p}{]}
\PYG{g+gp}{\PYGZgt{}\PYGZgt{}\PYGZgt{} }\PYG{k}{for} \PYG{n}{i} \PYG{o+ow}{in} \PYG{n+nb}{range}\PYG{p}{(}\PYG{l+m+mi}{1000}\PYG{p}{)}\PYG{p}{:}
\PYG{g+gp}{... }   \PYG{n}{a} \PYG{o}{=} \PYG{l+m+mf}{10.} \PYG{o}{+} \PYG{n}{np}\PYG{o}{.}\PYG{n}{random}\PYG{o}{.}\PYG{n}{standard\PYGZus{}normal}\PYG{p}{(}\PYG{l+m+mi}{100}\PYG{p}{)}
\PYG{g+gp}{... }   \PYG{n}{b}\PYG{o}{.}\PYG{n}{append}\PYG{p}{(}\PYG{n}{np}\PYG{o}{.}\PYG{n}{product}\PYG{p}{(}\PYG{n}{a}\PYG{p}{)}\PYG{p}{)}
\end{sphinxVerbatim}

\begin{sphinxVerbatim}[commandchars=\\\{\}]
\PYG{g+gp}{\PYGZgt{}\PYGZgt{}\PYGZgt{} }\PYG{n}{b} \PYG{o}{=} \PYG{n}{np}\PYG{o}{.}\PYG{n}{array}\PYG{p}{(}\PYG{n}{b}\PYG{p}{)} \PYG{o}{/} \PYG{n}{np}\PYG{o}{.}\PYG{n}{min}\PYG{p}{(}\PYG{n}{b}\PYG{p}{)} \PYG{c+c1}{\PYGZsh{} scale values to be positive}
\PYG{g+gp}{\PYGZgt{}\PYGZgt{}\PYGZgt{} }\PYG{n}{count}\PYG{p}{,} \PYG{n}{bins}\PYG{p}{,} \PYG{n}{ignored} \PYG{o}{=} \PYG{n}{plt}\PYG{o}{.}\PYG{n}{hist}\PYG{p}{(}\PYG{n}{b}\PYG{p}{,} \PYG{l+m+mi}{100}\PYG{p}{,} \PYG{n}{density}\PYG{o}{=}\PYG{k+kc}{True}\PYG{p}{,} \PYG{n}{align}\PYG{o}{=}\PYG{l+s+s1}{\PYGZsq{}}\PYG{l+s+s1}{mid}\PYG{l+s+s1}{\PYGZsq{}}\PYG{p}{)}
\PYG{g+gp}{\PYGZgt{}\PYGZgt{}\PYGZgt{} }\PYG{n}{sigma} \PYG{o}{=} \PYG{n}{np}\PYG{o}{.}\PYG{n}{std}\PYG{p}{(}\PYG{n}{np}\PYG{o}{.}\PYG{n}{log}\PYG{p}{(}\PYG{n}{b}\PYG{p}{)}\PYG{p}{)}
\PYG{g+gp}{\PYGZgt{}\PYGZgt{}\PYGZgt{} }\PYG{n}{mu} \PYG{o}{=} \PYG{n}{np}\PYG{o}{.}\PYG{n}{mean}\PYG{p}{(}\PYG{n}{np}\PYG{o}{.}\PYG{n}{log}\PYG{p}{(}\PYG{n}{b}\PYG{p}{)}\PYG{p}{)}
\end{sphinxVerbatim}

\begin{sphinxVerbatim}[commandchars=\\\{\}]
\PYG{g+gp}{\PYGZgt{}\PYGZgt{}\PYGZgt{} }\PYG{n}{x} \PYG{o}{=} \PYG{n}{np}\PYG{o}{.}\PYG{n}{linspace}\PYG{p}{(}\PYG{n+nb}{min}\PYG{p}{(}\PYG{n}{bins}\PYG{p}{)}\PYG{p}{,} \PYG{n+nb}{max}\PYG{p}{(}\PYG{n}{bins}\PYG{p}{)}\PYG{p}{,} \PYG{l+m+mi}{10000}\PYG{p}{)}
\PYG{g+gp}{\PYGZgt{}\PYGZgt{}\PYGZgt{} }\PYG{n}{pdf} \PYG{o}{=} \PYG{p}{(}\PYG{n}{np}\PYG{o}{.}\PYG{n}{exp}\PYG{p}{(}\PYG{o}{\PYGZhy{}}\PYG{p}{(}\PYG{n}{np}\PYG{o}{.}\PYG{n}{log}\PYG{p}{(}\PYG{n}{x}\PYG{p}{)} \PYG{o}{\PYGZhy{}} \PYG{n}{mu}\PYG{p}{)}\PYG{o}{*}\PYG{o}{*}\PYG{l+m+mi}{2} \PYG{o}{/} \PYG{p}{(}\PYG{l+m+mi}{2} \PYG{o}{*} \PYG{n}{sigma}\PYG{o}{*}\PYG{o}{*}\PYG{l+m+mi}{2}\PYG{p}{)}\PYG{p}{)}
\PYG{g+gp}{... }       \PYG{o}{/} \PYG{p}{(}\PYG{n}{x} \PYG{o}{*} \PYG{n}{sigma} \PYG{o}{*} \PYG{n}{np}\PYG{o}{.}\PYG{n}{sqrt}\PYG{p}{(}\PYG{l+m+mi}{2} \PYG{o}{*} \PYG{n}{np}\PYG{o}{.}\PYG{n}{pi}\PYG{p}{)}\PYG{p}{)}\PYG{p}{)}
\end{sphinxVerbatim}

\begin{sphinxVerbatim}[commandchars=\\\{\}]
\PYG{g+gp}{\PYGZgt{}\PYGZgt{}\PYGZgt{} }\PYG{n}{plt}\PYG{o}{.}\PYG{n}{plot}\PYG{p}{(}\PYG{n}{x}\PYG{p}{,} \PYG{n}{pdf}\PYG{p}{,} \PYG{n}{color}\PYG{o}{=}\PYG{l+s+s1}{\PYGZsq{}}\PYG{l+s+s1}{r}\PYG{l+s+s1}{\PYGZsq{}}\PYG{p}{,} \PYG{n}{linewidth}\PYG{o}{=}\PYG{l+m+mi}{2}\PYG{p}{)}
\PYG{g+gp}{\PYGZgt{}\PYGZgt{}\PYGZgt{} }\PYG{n}{plt}\PYG{o}{.}\PYG{n}{show}\PYG{p}{(}\PYG{p}{)}
\end{sphinxVerbatim}

\end{fulllineitems}

\index{logseries() (in module metilda.controllers.pitch\_art\_wizard)@\spxentry{logseries()}\spxextra{in module metilda.controllers.pitch\_art\_wizard}}

\begin{fulllineitems}
\phantomsection\label{\detokenize{metilda.controllers:metilda.controllers.pitch_art_wizard.logseries}}
\pysigstartsignatures
\pysiglinewithargsret{\sphinxcode{\sphinxupquote{metilda.controllers.pitch\_art\_wizard.}}\sphinxbfcode{\sphinxupquote{logseries}}}{\sphinxparam{\DUrole{n,n}{p}}\sphinxparamcomma \sphinxparam{\DUrole{n,n}{size}\DUrole{o,o}{=}\DUrole{default_value}{None}}}{}
\pysigstopsignatures
\sphinxAtStartPar
Draw samples from a logarithmic series distribution.

\sphinxAtStartPar
Samples are drawn from a log series distribution with specified
shape parameter, 0 \textless{}= \sphinxcode{\sphinxupquote{p}} \textless{} 1.

\begin{sphinxadmonition}{note}{Note:}
\sphinxAtStartPar
New code should use the \sphinxtitleref{\textasciitilde{}numpy.random.Generator.logseries}
method of a \sphinxtitleref{\textasciitilde{}numpy.random.Generator} instance instead;
please see the \DUrole{xref,std,std-ref}{random\sphinxhyphen{}quick\sphinxhyphen{}start}.
\end{sphinxadmonition}
\begin{quote}\begin{description}
\sphinxlineitem{Parameters}\begin{itemize}
\item {} 
\sphinxAtStartPar
\sphinxstyleliteralstrong{\sphinxupquote{p}} (\sphinxstyleliteralemphasis{\sphinxupquote{float}}\sphinxstyleliteralemphasis{\sphinxupquote{ or }}\sphinxstyleliteralemphasis{\sphinxupquote{array\_like}}\sphinxstyleliteralemphasis{\sphinxupquote{ of }}\sphinxstyleliteralemphasis{\sphinxupquote{floats}}) \textendash{} Shape parameter for the distribution.  Must be in the range {[}0, 1).

\item {} 
\sphinxAtStartPar
\sphinxstyleliteralstrong{\sphinxupquote{size}} (\sphinxstyleliteralemphasis{\sphinxupquote{int}}\sphinxstyleliteralemphasis{\sphinxupquote{ or }}\sphinxstyleliteralemphasis{\sphinxupquote{tuple}}\sphinxstyleliteralemphasis{\sphinxupquote{ of }}\sphinxstyleliteralemphasis{\sphinxupquote{ints}}\sphinxstyleliteralemphasis{\sphinxupquote{, }}\sphinxstyleliteralemphasis{\sphinxupquote{optional}}) \textendash{} Output shape.  If the given shape is, e.g., \sphinxcode{\sphinxupquote{(m, n, k)}}, then
\sphinxcode{\sphinxupquote{m * n * k}} samples are drawn.  If size is \sphinxcode{\sphinxupquote{None}} (default),
a single value is returned if \sphinxcode{\sphinxupquote{p}} is a scalar.  Otherwise,
\sphinxcode{\sphinxupquote{np.array(p).size}} samples are drawn.

\end{itemize}

\sphinxlineitem{Returns}
\sphinxAtStartPar
\sphinxstylestrong{out} \textendash{} Drawn samples from the parameterized logarithmic series distribution.

\sphinxlineitem{Return type}
\sphinxAtStartPar
ndarray or scalar

\end{description}\end{quote}


\begin{sphinxseealso}{See also:}
\begin{description}
\sphinxlineitem{\sphinxcode{\sphinxupquote{scipy.stats.logser}}}
\sphinxAtStartPar
probability density function, distribution or cumulative density function, etc.

\sphinxlineitem{\sphinxcode{\sphinxupquote{random.Generator.logseries}}}
\sphinxAtStartPar
which should be used for new code.

\end{description}


\end{sphinxseealso}

\subsubsection*{Notes}

\sphinxAtStartPar
The probability density for the Log Series distribution is
\begin{equation*}
\begin{split}P(k) = \frac{-p^k}{k \ln(1-p)},\end{split}
\end{equation*}
\sphinxAtStartPar
where p = probability.

\sphinxAtStartPar
The log series distribution is frequently used to represent species
richness and occurrence, first proposed by Fisher, Corbet, and
Williams in 1943 {[}2{]}.  It may also be used to model the numbers of
occupants seen in cars {[}3{]}.
\subsubsection*{References}
\subsubsection*{Examples}

\sphinxAtStartPar
Draw samples from the distribution:

\begin{sphinxVerbatim}[commandchars=\\\{\}]
\PYG{g+gp}{\PYGZgt{}\PYGZgt{}\PYGZgt{} }\PYG{n}{a} \PYG{o}{=} \PYG{l+m+mf}{.6}
\PYG{g+gp}{\PYGZgt{}\PYGZgt{}\PYGZgt{} }\PYG{n}{s} \PYG{o}{=} \PYG{n}{np}\PYG{o}{.}\PYG{n}{random}\PYG{o}{.}\PYG{n}{logseries}\PYG{p}{(}\PYG{n}{a}\PYG{p}{,} \PYG{l+m+mi}{10000}\PYG{p}{)}
\PYG{g+gp}{\PYGZgt{}\PYGZgt{}\PYGZgt{} }\PYG{k+kn}{import} \PYG{n+nn}{matplotlib}\PYG{n+nn}{.}\PYG{n+nn}{pyplot} \PYG{k}{as} \PYG{n+nn}{plt}
\PYG{g+gp}{\PYGZgt{}\PYGZgt{}\PYGZgt{} }\PYG{n}{count}\PYG{p}{,} \PYG{n}{bins}\PYG{p}{,} \PYG{n}{ignored} \PYG{o}{=} \PYG{n}{plt}\PYG{o}{.}\PYG{n}{hist}\PYG{p}{(}\PYG{n}{s}\PYG{p}{)}
\end{sphinxVerbatim}

\sphinxAtStartPar
\#   plot against distribution

\begin{sphinxVerbatim}[commandchars=\\\{\}]
\PYG{g+gp}{\PYGZgt{}\PYGZgt{}\PYGZgt{} }\PYG{k}{def} \PYG{n+nf}{logseries}\PYG{p}{(}\PYG{n}{k}\PYG{p}{,} \PYG{n}{p}\PYG{p}{)}\PYG{p}{:}
\PYG{g+gp}{... }    \PYG{k}{return} \PYG{o}{\PYGZhy{}}\PYG{n}{p}\PYG{o}{*}\PYG{o}{*}\PYG{n}{k}\PYG{o}{/}\PYG{p}{(}\PYG{n}{k}\PYG{o}{*}\PYG{n}{np}\PYG{o}{.}\PYG{n}{log}\PYG{p}{(}\PYG{l+m+mi}{1}\PYG{o}{\PYGZhy{}}\PYG{n}{p}\PYG{p}{)}\PYG{p}{)}
\PYG{g+gp}{\PYGZgt{}\PYGZgt{}\PYGZgt{} }\PYG{n}{plt}\PYG{o}{.}\PYG{n}{plot}\PYG{p}{(}\PYG{n}{bins}\PYG{p}{,} \PYG{n}{logseries}\PYG{p}{(}\PYG{n}{bins}\PYG{p}{,} \PYG{n}{a}\PYG{p}{)}\PYG{o}{*}\PYG{n}{count}\PYG{o}{.}\PYG{n}{max}\PYG{p}{(}\PYG{p}{)}\PYG{o}{/}
\PYG{g+gp}{... }         \PYG{n}{logseries}\PYG{p}{(}\PYG{n}{bins}\PYG{p}{,} \PYG{n}{a}\PYG{p}{)}\PYG{o}{.}\PYG{n}{max}\PYG{p}{(}\PYG{p}{)}\PYG{p}{,} \PYG{l+s+s1}{\PYGZsq{}}\PYG{l+s+s1}{r}\PYG{l+s+s1}{\PYGZsq{}}\PYG{p}{)}
\PYG{g+gp}{\PYGZgt{}\PYGZgt{}\PYGZgt{} }\PYG{n}{plt}\PYG{o}{.}\PYG{n}{show}\PYG{p}{(}\PYG{p}{)}
\end{sphinxVerbatim}

\end{fulllineitems}

\index{modifyPitchOrImageDetails() (in module metilda.controllers.pitch\_art\_wizard)@\spxentry{modifyPitchOrImageDetails()}\spxextra{in module metilda.controllers.pitch\_art\_wizard}}

\begin{fulllineitems}
\phantomsection\label{\detokenize{metilda.controllers:metilda.controllers.pitch_art_wizard.modifyPitchOrImageDetails}}
\pysigstartsignatures
\pysiglinewithargsret{\sphinxcode{\sphinxupquote{metilda.controllers.pitch\_art\_wizard.}}\sphinxbfcode{\sphinxupquote{modifyPitchOrImageDetails}}}{\sphinxparam{\DUrole{n,n}{upload\_id}}}{}
\pysigstopsignatures
\end{fulllineitems}

\index{move\_to\_folder() (in module metilda.controllers.pitch\_art\_wizard)@\spxentry{move\_to\_folder()}\spxextra{in module metilda.controllers.pitch\_art\_wizard}}

\begin{fulllineitems}
\phantomsection\label{\detokenize{metilda.controllers:metilda.controllers.pitch_art_wizard.move_to_folder}}
\pysigstartsignatures
\pysiglinewithargsret{\sphinxcode{\sphinxupquote{metilda.controllers.pitch\_art\_wizard.}}\sphinxbfcode{\sphinxupquote{move\_to\_folder}}}{}{}
\pysigstopsignatures
\end{fulllineitems}

\index{multinomial() (in module metilda.controllers.pitch\_art\_wizard)@\spxentry{multinomial()}\spxextra{in module metilda.controllers.pitch\_art\_wizard}}

\begin{fulllineitems}
\phantomsection\label{\detokenize{metilda.controllers:metilda.controllers.pitch_art_wizard.multinomial}}
\pysigstartsignatures
\pysiglinewithargsret{\sphinxcode{\sphinxupquote{metilda.controllers.pitch\_art\_wizard.}}\sphinxbfcode{\sphinxupquote{multinomial}}}{\sphinxparam{\DUrole{n,n}{n}}\sphinxparamcomma \sphinxparam{\DUrole{n,n}{pvals}}\sphinxparamcomma \sphinxparam{\DUrole{n,n}{size}\DUrole{o,o}{=}\DUrole{default_value}{None}}}{}
\pysigstopsignatures
\sphinxAtStartPar
Draw samples from a multinomial distribution.

\sphinxAtStartPar
The multinomial distribution is a multivariate generalization of the
binomial distribution.  Take an experiment with one of \sphinxcode{\sphinxupquote{p}}
possible outcomes.  An example of such an experiment is throwing a dice,
where the outcome can be 1 through 6.  Each sample drawn from the
distribution represents \sphinxtitleref{n} such experiments.  Its values,
\sphinxcode{\sphinxupquote{X\_i = {[}X\_0, X\_1, ..., X\_p{]}}}, represent the number of times the
outcome was \sphinxcode{\sphinxupquote{i}}.

\begin{sphinxadmonition}{note}{Note:}
\sphinxAtStartPar
New code should use the \sphinxtitleref{\textasciitilde{}numpy.random.Generator.multinomial}
method of a \sphinxtitleref{\textasciitilde{}numpy.random.Generator} instance instead;
please see the \DUrole{xref,std,std-ref}{random\sphinxhyphen{}quick\sphinxhyphen{}start}.
\end{sphinxadmonition}
\begin{quote}\begin{description}
\sphinxlineitem{Parameters}\begin{itemize}
\item {} 
\sphinxAtStartPar
\sphinxstyleliteralstrong{\sphinxupquote{n}} (\sphinxstyleliteralemphasis{\sphinxupquote{int}}) \textendash{} Number of experiments.

\item {} 
\sphinxAtStartPar
\sphinxstyleliteralstrong{\sphinxupquote{pvals}} (\sphinxstyleliteralemphasis{\sphinxupquote{sequence}}\sphinxstyleliteralemphasis{\sphinxupquote{ of }}\sphinxstyleliteralemphasis{\sphinxupquote{floats}}\sphinxstyleliteralemphasis{\sphinxupquote{, }}\sphinxstyleliteralemphasis{\sphinxupquote{length p}}) \textendash{} Probabilities of each of the \sphinxcode{\sphinxupquote{p}} different outcomes.  These
must sum to 1 (however, the last element is always assumed to
account for the remaining probability, as long as
\sphinxcode{\sphinxupquote{sum(pvals{[}:\sphinxhyphen{}1{]}) \textless{}= 1)}}.

\item {} 
\sphinxAtStartPar
\sphinxstyleliteralstrong{\sphinxupquote{size}} (\sphinxstyleliteralemphasis{\sphinxupquote{int}}\sphinxstyleliteralemphasis{\sphinxupquote{ or }}\sphinxstyleliteralemphasis{\sphinxupquote{tuple}}\sphinxstyleliteralemphasis{\sphinxupquote{ of }}\sphinxstyleliteralemphasis{\sphinxupquote{ints}}\sphinxstyleliteralemphasis{\sphinxupquote{, }}\sphinxstyleliteralemphasis{\sphinxupquote{optional}}) \textendash{} Output shape.  If the given shape is, e.g., \sphinxcode{\sphinxupquote{(m, n, k)}}, then
\sphinxcode{\sphinxupquote{m * n * k}} samples are drawn.  Default is None, in which case a
single value is returned.

\end{itemize}

\sphinxlineitem{Returns}
\sphinxAtStartPar

\sphinxAtStartPar
\sphinxstylestrong{out} \textendash{} The drawn samples, of shape \sphinxstyleemphasis{size}, if that was provided.  If not,
the shape is \sphinxcode{\sphinxupquote{(N,)}}.

\sphinxAtStartPar
In other words, each entry \sphinxcode{\sphinxupquote{out{[}i,j,...,:{]}}} is an N\sphinxhyphen{}dimensional
value drawn from the distribution.


\sphinxlineitem{Return type}
\sphinxAtStartPar
ndarray

\end{description}\end{quote}


\begin{sphinxseealso}{See also:}
\begin{description}
\sphinxlineitem{\sphinxcode{\sphinxupquote{random.Generator.multinomial}}}
\sphinxAtStartPar
which should be used for new code.

\end{description}


\end{sphinxseealso}

\subsubsection*{Examples}

\sphinxAtStartPar
Throw a dice 20 times:

\begin{sphinxVerbatim}[commandchars=\\\{\}]
\PYG{g+gp}{\PYGZgt{}\PYGZgt{}\PYGZgt{} }\PYG{n}{np}\PYG{o}{.}\PYG{n}{random}\PYG{o}{.}\PYG{n}{multinomial}\PYG{p}{(}\PYG{l+m+mi}{20}\PYG{p}{,} \PYG{p}{[}\PYG{l+m+mi}{1}\PYG{o}{/}\PYG{l+m+mf}{6.}\PYG{p}{]}\PYG{o}{*}\PYG{l+m+mi}{6}\PYG{p}{,} \PYG{n}{size}\PYG{o}{=}\PYG{l+m+mi}{1}\PYG{p}{)}
\PYG{g+go}{array([[4, 1, 7, 5, 2, 1]]) \PYGZsh{} random}
\end{sphinxVerbatim}

\sphinxAtStartPar
It landed 4 times on 1, once on 2, etc.

\sphinxAtStartPar
Now, throw the dice 20 times, and 20 times again:

\begin{sphinxVerbatim}[commandchars=\\\{\}]
\PYG{g+gp}{\PYGZgt{}\PYGZgt{}\PYGZgt{} }\PYG{n}{np}\PYG{o}{.}\PYG{n}{random}\PYG{o}{.}\PYG{n}{multinomial}\PYG{p}{(}\PYG{l+m+mi}{20}\PYG{p}{,} \PYG{p}{[}\PYG{l+m+mi}{1}\PYG{o}{/}\PYG{l+m+mf}{6.}\PYG{p}{]}\PYG{o}{*}\PYG{l+m+mi}{6}\PYG{p}{,} \PYG{n}{size}\PYG{o}{=}\PYG{l+m+mi}{2}\PYG{p}{)}
\PYG{g+go}{array([[3, 4, 3, 3, 4, 3], \PYGZsh{} random}
\PYG{g+go}{       [2, 4, 3, 4, 0, 7]])}
\end{sphinxVerbatim}

\sphinxAtStartPar
For the first run, we threw 3 times 1, 4 times 2, etc.  For the second,
we threw 2 times 1, 4 times 2, etc.

\sphinxAtStartPar
A loaded die is more likely to land on number 6:

\begin{sphinxVerbatim}[commandchars=\\\{\}]
\PYG{g+gp}{\PYGZgt{}\PYGZgt{}\PYGZgt{} }\PYG{n}{np}\PYG{o}{.}\PYG{n}{random}\PYG{o}{.}\PYG{n}{multinomial}\PYG{p}{(}\PYG{l+m+mi}{100}\PYG{p}{,} \PYG{p}{[}\PYG{l+m+mi}{1}\PYG{o}{/}\PYG{l+m+mf}{7.}\PYG{p}{]}\PYG{o}{*}\PYG{l+m+mi}{5} \PYG{o}{+} \PYG{p}{[}\PYG{l+m+mi}{2}\PYG{o}{/}\PYG{l+m+mf}{7.}\PYG{p}{]}\PYG{p}{)}
\PYG{g+go}{array([11, 16, 14, 17, 16, 26]) \PYGZsh{} random}
\end{sphinxVerbatim}

\sphinxAtStartPar
The probability inputs should be normalized. As an implementation
detail, the value of the last entry is ignored and assumed to take
up any leftover probability mass, but this should not be relied on.
A biased coin which has twice as much weight on one side as on the
other should be sampled like so:

\begin{sphinxVerbatim}[commandchars=\\\{\}]
\PYG{g+gp}{\PYGZgt{}\PYGZgt{}\PYGZgt{} }\PYG{n}{np}\PYG{o}{.}\PYG{n}{random}\PYG{o}{.}\PYG{n}{multinomial}\PYG{p}{(}\PYG{l+m+mi}{100}\PYG{p}{,} \PYG{p}{[}\PYG{l+m+mf}{1.0} \PYG{o}{/} \PYG{l+m+mi}{3}\PYG{p}{,} \PYG{l+m+mf}{2.0} \PYG{o}{/} \PYG{l+m+mi}{3}\PYG{p}{]}\PYG{p}{)}  \PYG{c+c1}{\PYGZsh{} RIGHT}
\PYG{g+go}{array([38, 62]) \PYGZsh{} random}
\end{sphinxVerbatim}

\sphinxAtStartPar
not like:

\begin{sphinxVerbatim}[commandchars=\\\{\}]
\PYG{g+gp}{\PYGZgt{}\PYGZgt{}\PYGZgt{} }\PYG{n}{np}\PYG{o}{.}\PYG{n}{random}\PYG{o}{.}\PYG{n}{multinomial}\PYG{p}{(}\PYG{l+m+mi}{100}\PYG{p}{,} \PYG{p}{[}\PYG{l+m+mf}{1.0}\PYG{p}{,} \PYG{l+m+mf}{2.0}\PYG{p}{]}\PYG{p}{)}  \PYG{c+c1}{\PYGZsh{} WRONG}
\PYG{g+gt}{Traceback (most recent call last):}
\PYG{g+gr}{ValueError}: \PYG{n}{pvals \PYGZlt{} 0, pvals \PYGZgt{} 1 or pvals contains NaNs}
\end{sphinxVerbatim}

\end{fulllineitems}

\index{multivariate\_normal() (in module metilda.controllers.pitch\_art\_wizard)@\spxentry{multivariate\_normal()}\spxextra{in module metilda.controllers.pitch\_art\_wizard}}

\begin{fulllineitems}
\phantomsection\label{\detokenize{metilda.controllers:metilda.controllers.pitch_art_wizard.multivariate_normal}}
\pysigstartsignatures
\pysiglinewithargsret{\sphinxcode{\sphinxupquote{metilda.controllers.pitch\_art\_wizard.}}\sphinxbfcode{\sphinxupquote{multivariate\_normal}}}{\sphinxparam{\DUrole{n,n}{mean}}\sphinxparamcomma \sphinxparam{\DUrole{n,n}{cov}}\sphinxparamcomma \sphinxparam{\DUrole{n,n}{size}\DUrole{o,o}{=}\DUrole{default_value}{None}}\sphinxparamcomma \sphinxparam{\DUrole{n,n}{check\_valid}\DUrole{o,o}{=}\DUrole{default_value}{\textquotesingle{}warn\textquotesingle{}}}\sphinxparamcomma \sphinxparam{\DUrole{n,n}{tol}\DUrole{o,o}{=}\DUrole{default_value}{1e\sphinxhyphen{}8}}}{}
\pysigstopsignatures
\sphinxAtStartPar
Draw random samples from a multivariate normal distribution.

\sphinxAtStartPar
The multivariate normal, multinormal or Gaussian distribution is a
generalization of the one\sphinxhyphen{}dimensional normal distribution to higher
dimensions.  Such a distribution is specified by its mean and
covariance matrix.  These parameters are analogous to the mean
(average or “center”) and variance (standard deviation, or “width,”
squared) of the one\sphinxhyphen{}dimensional normal distribution.

\begin{sphinxadmonition}{note}{Note:}
\sphinxAtStartPar
New code should use the
\sphinxtitleref{\textasciitilde{}numpy.random.Generator.multivariate\_normal}
method of a \sphinxtitleref{\textasciitilde{}numpy.random.Generator} instance instead;
please see the \DUrole{xref,std,std-ref}{random\sphinxhyphen{}quick\sphinxhyphen{}start}.
\end{sphinxadmonition}
\begin{quote}\begin{description}
\sphinxlineitem{Parameters}\begin{itemize}
\item {} 
\sphinxAtStartPar
\sphinxstyleliteralstrong{\sphinxupquote{mean}} (\sphinxstyleliteralemphasis{\sphinxupquote{1\sphinxhyphen{}D array\_like}}\sphinxstyleliteralemphasis{\sphinxupquote{, of }}\sphinxstyleliteralemphasis{\sphinxupquote{length N}}) \textendash{} Mean of the N\sphinxhyphen{}dimensional distribution.

\item {} 
\sphinxAtStartPar
\sphinxstyleliteralstrong{\sphinxupquote{cov}} (\sphinxstyleliteralemphasis{\sphinxupquote{2\sphinxhyphen{}D array\_like}}\sphinxstyleliteralemphasis{\sphinxupquote{, of }}\sphinxstyleliteralemphasis{\sphinxupquote{shape}}\sphinxstyleliteralemphasis{\sphinxupquote{ (}}\sphinxstyleliteralemphasis{\sphinxupquote{N}}\sphinxstyleliteralemphasis{\sphinxupquote{, }}\sphinxstyleliteralemphasis{\sphinxupquote{N}}\sphinxstyleliteralemphasis{\sphinxupquote{)}}) \textendash{} Covariance matrix of the distribution. It must be symmetric and
positive\sphinxhyphen{}semidefinite for proper sampling.

\item {} 
\sphinxAtStartPar
\sphinxstyleliteralstrong{\sphinxupquote{size}} (\sphinxstyleliteralemphasis{\sphinxupquote{int}}\sphinxstyleliteralemphasis{\sphinxupquote{ or }}\sphinxstyleliteralemphasis{\sphinxupquote{tuple}}\sphinxstyleliteralemphasis{\sphinxupquote{ of }}\sphinxstyleliteralemphasis{\sphinxupquote{ints}}\sphinxstyleliteralemphasis{\sphinxupquote{, }}\sphinxstyleliteralemphasis{\sphinxupquote{optional}}) \textendash{} Given a shape of, for example, \sphinxcode{\sphinxupquote{(m,n,k)}}, \sphinxcode{\sphinxupquote{m*n*k}} samples are
generated, and packed in an \sphinxtitleref{m}\sphinxhyphen{}by\sphinxhyphen{}\sphinxtitleref{n}\sphinxhyphen{}by\sphinxhyphen{}\sphinxtitleref{k} arrangement.  Because
each sample is \sphinxtitleref{N}\sphinxhyphen{}dimensional, the output shape is \sphinxcode{\sphinxupquote{(m,n,k,N)}}.
If no shape is specified, a single (\sphinxtitleref{N}\sphinxhyphen{}D) sample is returned.

\item {} 
\sphinxAtStartPar
\sphinxstyleliteralstrong{\sphinxupquote{check\_valid}} (\sphinxstyleliteralemphasis{\sphinxupquote{\{ \textquotesingle{}warn\textquotesingle{}}}\sphinxstyleliteralemphasis{\sphinxupquote{, }}\sphinxstyleliteralemphasis{\sphinxupquote{\textquotesingle{}raise\textquotesingle{}}}\sphinxstyleliteralemphasis{\sphinxupquote{, }}\sphinxstyleliteralemphasis{\sphinxupquote{\textquotesingle{}ignore\textquotesingle{} \}}}\sphinxstyleliteralemphasis{\sphinxupquote{, }}\sphinxstyleliteralemphasis{\sphinxupquote{optional}}) \textendash{} Behavior when the covariance matrix is not positive semidefinite.

\item {} 
\sphinxAtStartPar
\sphinxstyleliteralstrong{\sphinxupquote{tol}} (\sphinxstyleliteralemphasis{\sphinxupquote{float}}\sphinxstyleliteralemphasis{\sphinxupquote{, }}\sphinxstyleliteralemphasis{\sphinxupquote{optional}}) \textendash{} Tolerance when checking the singular values in covariance matrix.
cov is cast to double before the check.

\end{itemize}

\sphinxlineitem{Returns}
\sphinxAtStartPar

\sphinxAtStartPar
\sphinxstylestrong{out} \textendash{} The drawn samples, of shape \sphinxstyleemphasis{size}, if that was provided.  If not,
the shape is \sphinxcode{\sphinxupquote{(N,)}}.

\sphinxAtStartPar
In other words, each entry \sphinxcode{\sphinxupquote{out{[}i,j,...,:{]}}} is an N\sphinxhyphen{}dimensional
value drawn from the distribution.


\sphinxlineitem{Return type}
\sphinxAtStartPar
ndarray

\end{description}\end{quote}


\begin{sphinxseealso}{See also:}
\begin{description}
\sphinxlineitem{\sphinxcode{\sphinxupquote{random.Generator.multivariate\_normal}}}
\sphinxAtStartPar
which should be used for new code.

\end{description}


\end{sphinxseealso}

\subsubsection*{Notes}

\sphinxAtStartPar
The mean is a coordinate in N\sphinxhyphen{}dimensional space, which represents the
location where samples are most likely to be generated.  This is
analogous to the peak of the bell curve for the one\sphinxhyphen{}dimensional or
univariate normal distribution.

\sphinxAtStartPar
Covariance indicates the level to which two variables vary together.
From the multivariate normal distribution, we draw N\sphinxhyphen{}dimensional
samples, \(X = [x_1, x_2, ... x_N]\).  The covariance matrix
element \(C_{ij}\) is the covariance of \(x_i\) and \(x_j\).
The element \(C_{ii}\) is the variance of \(x_i\) (i.e. its
“spread”).

\sphinxAtStartPar
Instead of specifying the full covariance matrix, popular
approximations include:
\begin{itemize}
\item {} 
\sphinxAtStartPar
Spherical covariance (\sphinxtitleref{cov} is a multiple of the identity matrix)

\item {} 
\sphinxAtStartPar
Diagonal covariance (\sphinxtitleref{cov} has non\sphinxhyphen{}negative elements, and only on
the diagonal)

\end{itemize}

\sphinxAtStartPar
This geometrical property can be seen in two dimensions by plotting
generated data\sphinxhyphen{}points:

\begin{sphinxVerbatim}[commandchars=\\\{\}]
\PYG{g+gp}{\PYGZgt{}\PYGZgt{}\PYGZgt{} }\PYG{n}{mean} \PYG{o}{=} \PYG{p}{[}\PYG{l+m+mi}{0}\PYG{p}{,} \PYG{l+m+mi}{0}\PYG{p}{]}
\PYG{g+gp}{\PYGZgt{}\PYGZgt{}\PYGZgt{} }\PYG{n}{cov} \PYG{o}{=} \PYG{p}{[}\PYG{p}{[}\PYG{l+m+mi}{1}\PYG{p}{,} \PYG{l+m+mi}{0}\PYG{p}{]}\PYG{p}{,} \PYG{p}{[}\PYG{l+m+mi}{0}\PYG{p}{,} \PYG{l+m+mi}{100}\PYG{p}{]}\PYG{p}{]}  \PYG{c+c1}{\PYGZsh{} diagonal covariance}
\end{sphinxVerbatim}

\sphinxAtStartPar
Diagonal covariance means that points are oriented along x or y\sphinxhyphen{}axis:

\begin{sphinxVerbatim}[commandchars=\\\{\}]
\PYG{g+gp}{\PYGZgt{}\PYGZgt{}\PYGZgt{} }\PYG{k+kn}{import} \PYG{n+nn}{matplotlib}\PYG{n+nn}{.}\PYG{n+nn}{pyplot} \PYG{k}{as} \PYG{n+nn}{plt}
\PYG{g+gp}{\PYGZgt{}\PYGZgt{}\PYGZgt{} }\PYG{n}{x}\PYG{p}{,} \PYG{n}{y} \PYG{o}{=} \PYG{n}{np}\PYG{o}{.}\PYG{n}{random}\PYG{o}{.}\PYG{n}{multivariate\PYGZus{}normal}\PYG{p}{(}\PYG{n}{mean}\PYG{p}{,} \PYG{n}{cov}\PYG{p}{,} \PYG{l+m+mi}{5000}\PYG{p}{)}\PYG{o}{.}\PYG{n}{T}
\PYG{g+gp}{\PYGZgt{}\PYGZgt{}\PYGZgt{} }\PYG{n}{plt}\PYG{o}{.}\PYG{n}{plot}\PYG{p}{(}\PYG{n}{x}\PYG{p}{,} \PYG{n}{y}\PYG{p}{,} \PYG{l+s+s1}{\PYGZsq{}}\PYG{l+s+s1}{x}\PYG{l+s+s1}{\PYGZsq{}}\PYG{p}{)}
\PYG{g+gp}{\PYGZgt{}\PYGZgt{}\PYGZgt{} }\PYG{n}{plt}\PYG{o}{.}\PYG{n}{axis}\PYG{p}{(}\PYG{l+s+s1}{\PYGZsq{}}\PYG{l+s+s1}{equal}\PYG{l+s+s1}{\PYGZsq{}}\PYG{p}{)}
\PYG{g+gp}{\PYGZgt{}\PYGZgt{}\PYGZgt{} }\PYG{n}{plt}\PYG{o}{.}\PYG{n}{show}\PYG{p}{(}\PYG{p}{)}
\end{sphinxVerbatim}

\sphinxAtStartPar
Note that the covariance matrix must be positive semidefinite (a.k.a.
nonnegative\sphinxhyphen{}definite). Otherwise, the behavior of this method is
undefined and backwards compatibility is not guaranteed.
\subsubsection*{References}
\subsubsection*{Examples}

\begin{sphinxVerbatim}[commandchars=\\\{\}]
\PYG{g+gp}{\PYGZgt{}\PYGZgt{}\PYGZgt{} }\PYG{n}{mean} \PYG{o}{=} \PYG{p}{(}\PYG{l+m+mi}{1}\PYG{p}{,} \PYG{l+m+mi}{2}\PYG{p}{)}
\PYG{g+gp}{\PYGZgt{}\PYGZgt{}\PYGZgt{} }\PYG{n}{cov} \PYG{o}{=} \PYG{p}{[}\PYG{p}{[}\PYG{l+m+mi}{1}\PYG{p}{,} \PYG{l+m+mi}{0}\PYG{p}{]}\PYG{p}{,} \PYG{p}{[}\PYG{l+m+mi}{0}\PYG{p}{,} \PYG{l+m+mi}{1}\PYG{p}{]}\PYG{p}{]}
\PYG{g+gp}{\PYGZgt{}\PYGZgt{}\PYGZgt{} }\PYG{n}{x} \PYG{o}{=} \PYG{n}{np}\PYG{o}{.}\PYG{n}{random}\PYG{o}{.}\PYG{n}{multivariate\PYGZus{}normal}\PYG{p}{(}\PYG{n}{mean}\PYG{p}{,} \PYG{n}{cov}\PYG{p}{,} \PYG{p}{(}\PYG{l+m+mi}{3}\PYG{p}{,} \PYG{l+m+mi}{3}\PYG{p}{)}\PYG{p}{)}
\PYG{g+gp}{\PYGZgt{}\PYGZgt{}\PYGZgt{} }\PYG{n}{x}\PYG{o}{.}\PYG{n}{shape}
\PYG{g+go}{(3, 3, 2)}
\end{sphinxVerbatim}

\sphinxAtStartPar
Here we generate 800 samples from the bivariate normal distribution
with mean {[}0, 0{]} and covariance matrix {[}{[}6, \sphinxhyphen{}3{]}, {[}\sphinxhyphen{}3, 3.5{]}{]}.  The
expected variances of the first and second components of the sample
are 6 and 3.5, respectively, and the expected correlation
coefficient is \sphinxhyphen{}3/sqrt(6*3.5) ≈ \sphinxhyphen{}0.65465.

\begin{sphinxVerbatim}[commandchars=\\\{\}]
\PYG{g+gp}{\PYGZgt{}\PYGZgt{}\PYGZgt{} }\PYG{n}{cov} \PYG{o}{=} \PYG{n}{np}\PYG{o}{.}\PYG{n}{array}\PYG{p}{(}\PYG{p}{[}\PYG{p}{[}\PYG{l+m+mi}{6}\PYG{p}{,} \PYG{o}{\PYGZhy{}}\PYG{l+m+mi}{3}\PYG{p}{]}\PYG{p}{,} \PYG{p}{[}\PYG{o}{\PYGZhy{}}\PYG{l+m+mi}{3}\PYG{p}{,} \PYG{l+m+mf}{3.5}\PYG{p}{]}\PYG{p}{]}\PYG{p}{)}
\PYG{g+gp}{\PYGZgt{}\PYGZgt{}\PYGZgt{} }\PYG{n}{pts} \PYG{o}{=} \PYG{n}{np}\PYG{o}{.}\PYG{n}{random}\PYG{o}{.}\PYG{n}{multivariate\PYGZus{}normal}\PYG{p}{(}\PYG{p}{[}\PYG{l+m+mi}{0}\PYG{p}{,} \PYG{l+m+mi}{0}\PYG{p}{]}\PYG{p}{,} \PYG{n}{cov}\PYG{p}{,} \PYG{n}{size}\PYG{o}{=}\PYG{l+m+mi}{800}\PYG{p}{)}
\end{sphinxVerbatim}

\sphinxAtStartPar
Check that the mean, covariance, and correlation coefficient of the
sample are close to the expected values:

\begin{sphinxVerbatim}[commandchars=\\\{\}]
\PYG{g+gp}{\PYGZgt{}\PYGZgt{}\PYGZgt{} }\PYG{n}{pts}\PYG{o}{.}\PYG{n}{mean}\PYG{p}{(}\PYG{n}{axis}\PYG{o}{=}\PYG{l+m+mi}{0}\PYG{p}{)}
\PYG{g+go}{array([ 0.0326911 , \PYGZhy{}0.01280782])  \PYGZsh{} may vary}
\PYG{g+gp}{\PYGZgt{}\PYGZgt{}\PYGZgt{} }\PYG{n}{np}\PYG{o}{.}\PYG{n}{cov}\PYG{p}{(}\PYG{n}{pts}\PYG{o}{.}\PYG{n}{T}\PYG{p}{)}
\PYG{g+go}{array([[ 5.96202397, \PYGZhy{}2.85602287],}
\PYG{g+go}{       [\PYGZhy{}2.85602287,  3.47613949]])  \PYGZsh{} may vary}
\PYG{g+gp}{\PYGZgt{}\PYGZgt{}\PYGZgt{} }\PYG{n}{np}\PYG{o}{.}\PYG{n}{corrcoef}\PYG{p}{(}\PYG{n}{pts}\PYG{o}{.}\PYG{n}{T}\PYG{p}{)}\PYG{p}{[}\PYG{l+m+mi}{0}\PYG{p}{,} \PYG{l+m+mi}{1}\PYG{p}{]}
\PYG{g+go}{\PYGZhy{}0.6273591314603949  \PYGZsh{} may vary}
\end{sphinxVerbatim}

\sphinxAtStartPar
We can visualize this data with a scatter plot.  The orientation
of the point cloud illustrates the negative correlation of the
components of this sample.

\begin{sphinxVerbatim}[commandchars=\\\{\}]
\PYG{g+gp}{\PYGZgt{}\PYGZgt{}\PYGZgt{} }\PYG{k+kn}{import} \PYG{n+nn}{matplotlib}\PYG{n+nn}{.}\PYG{n+nn}{pyplot} \PYG{k}{as} \PYG{n+nn}{plt}
\PYG{g+gp}{\PYGZgt{}\PYGZgt{}\PYGZgt{} }\PYG{n}{plt}\PYG{o}{.}\PYG{n}{plot}\PYG{p}{(}\PYG{n}{pts}\PYG{p}{[}\PYG{p}{:}\PYG{p}{,} \PYG{l+m+mi}{0}\PYG{p}{]}\PYG{p}{,} \PYG{n}{pts}\PYG{p}{[}\PYG{p}{:}\PYG{p}{,} \PYG{l+m+mi}{1}\PYG{p}{]}\PYG{p}{,} \PYG{l+s+s1}{\PYGZsq{}}\PYG{l+s+s1}{.}\PYG{l+s+s1}{\PYGZsq{}}\PYG{p}{,} \PYG{n}{alpha}\PYG{o}{=}\PYG{l+m+mf}{0.5}\PYG{p}{)}
\PYG{g+gp}{\PYGZgt{}\PYGZgt{}\PYGZgt{} }\PYG{n}{plt}\PYG{o}{.}\PYG{n}{axis}\PYG{p}{(}\PYG{l+s+s1}{\PYGZsq{}}\PYG{l+s+s1}{equal}\PYG{l+s+s1}{\PYGZsq{}}\PYG{p}{)}
\PYG{g+gp}{\PYGZgt{}\PYGZgt{}\PYGZgt{} }\PYG{n}{plt}\PYG{o}{.}\PYG{n}{grid}\PYG{p}{(}\PYG{p}{)}
\PYG{g+gp}{\PYGZgt{}\PYGZgt{}\PYGZgt{} }\PYG{n}{plt}\PYG{o}{.}\PYG{n}{show}\PYG{p}{(}\PYG{p}{)}
\end{sphinxVerbatim}

\end{fulllineitems}

\index{negative\_binomial() (in module metilda.controllers.pitch\_art\_wizard)@\spxentry{negative\_binomial()}\spxextra{in module metilda.controllers.pitch\_art\_wizard}}

\begin{fulllineitems}
\phantomsection\label{\detokenize{metilda.controllers:metilda.controllers.pitch_art_wizard.negative_binomial}}
\pysigstartsignatures
\pysiglinewithargsret{\sphinxcode{\sphinxupquote{metilda.controllers.pitch\_art\_wizard.}}\sphinxbfcode{\sphinxupquote{negative\_binomial}}}{\sphinxparam{\DUrole{n,n}{n}}\sphinxparamcomma \sphinxparam{\DUrole{n,n}{p}}\sphinxparamcomma \sphinxparam{\DUrole{n,n}{size}\DUrole{o,o}{=}\DUrole{default_value}{None}}}{}
\pysigstopsignatures
\sphinxAtStartPar
Draw samples from a negative binomial distribution.

\sphinxAtStartPar
Samples are drawn from a negative binomial distribution with specified
parameters, \sphinxtitleref{n} successes and \sphinxtitleref{p} probability of success where \sphinxtitleref{n}
is \textgreater{} 0 and \sphinxtitleref{p} is in the interval {[}0, 1{]}.

\begin{sphinxadmonition}{note}{Note:}
\sphinxAtStartPar
New code should use the
\sphinxtitleref{\textasciitilde{}numpy.random.Generator.negative\_binomial}
method of a \sphinxtitleref{\textasciitilde{}numpy.random.Generator} instance instead;
please see the \DUrole{xref,std,std-ref}{random\sphinxhyphen{}quick\sphinxhyphen{}start}.
\end{sphinxadmonition}
\begin{quote}\begin{description}
\sphinxlineitem{Parameters}\begin{itemize}
\item {} 
\sphinxAtStartPar
\sphinxstyleliteralstrong{\sphinxupquote{n}} (\sphinxstyleliteralemphasis{\sphinxupquote{float}}\sphinxstyleliteralemphasis{\sphinxupquote{ or }}\sphinxstyleliteralemphasis{\sphinxupquote{array\_like}}\sphinxstyleliteralemphasis{\sphinxupquote{ of }}\sphinxstyleliteralemphasis{\sphinxupquote{floats}}) \textendash{} Parameter of the distribution, \textgreater{} 0.

\item {} 
\sphinxAtStartPar
\sphinxstyleliteralstrong{\sphinxupquote{p}} (\sphinxstyleliteralemphasis{\sphinxupquote{float}}\sphinxstyleliteralemphasis{\sphinxupquote{ or }}\sphinxstyleliteralemphasis{\sphinxupquote{array\_like}}\sphinxstyleliteralemphasis{\sphinxupquote{ of }}\sphinxstyleliteralemphasis{\sphinxupquote{floats}}) \textendash{} Parameter of the distribution, \textgreater{}= 0 and \textless{}=1.

\item {} 
\sphinxAtStartPar
\sphinxstyleliteralstrong{\sphinxupquote{size}} (\sphinxstyleliteralemphasis{\sphinxupquote{int}}\sphinxstyleliteralemphasis{\sphinxupquote{ or }}\sphinxstyleliteralemphasis{\sphinxupquote{tuple}}\sphinxstyleliteralemphasis{\sphinxupquote{ of }}\sphinxstyleliteralemphasis{\sphinxupquote{ints}}\sphinxstyleliteralemphasis{\sphinxupquote{, }}\sphinxstyleliteralemphasis{\sphinxupquote{optional}}) \textendash{} Output shape.  If the given shape is, e.g., \sphinxcode{\sphinxupquote{(m, n, k)}}, then
\sphinxcode{\sphinxupquote{m * n * k}} samples are drawn.  If size is \sphinxcode{\sphinxupquote{None}} (default),
a single value is returned if \sphinxcode{\sphinxupquote{n}} and \sphinxcode{\sphinxupquote{p}} are both scalars.
Otherwise, \sphinxcode{\sphinxupquote{np.broadcast(n, p).size}} samples are drawn.

\end{itemize}

\sphinxlineitem{Returns}
\sphinxAtStartPar
\sphinxstylestrong{out} \textendash{} Drawn samples from the parameterized negative binomial distribution,
where each sample is equal to N, the number of failures that
occurred before a total of n successes was reached.

\sphinxlineitem{Return type}
\sphinxAtStartPar
ndarray or scalar

\end{description}\end{quote}


\begin{sphinxseealso}{See also:}
\begin{description}
\sphinxlineitem{\sphinxcode{\sphinxupquote{random.Generator.negative\_binomial}}}
\sphinxAtStartPar
which should be used for new code.

\end{description}


\end{sphinxseealso}

\subsubsection*{Notes}

\sphinxAtStartPar
The probability mass function of the negative binomial distribution is
\begin{equation*}
\begin{split}P(N;n,p) = \frac{\Gamma(N+n)}{N!\Gamma(n)}p^{n}(1-p)^{N},\end{split}
\end{equation*}
\sphinxAtStartPar
where \(n\) is the number of successes, \(p\) is the
probability of success, \(N+n\) is the number of trials, and
\(\Gamma\) is the gamma function. When \(n\) is an integer,
\(\frac{\Gamma(N+n)}{N!\Gamma(n)} = \binom{N+n-1}{N}\), which is
the more common form of this term in the pmf. The negative
binomial distribution gives the probability of N failures given n
successes, with a success on the last trial.

\sphinxAtStartPar
If one throws a die repeatedly until the third time a “1” appears,
then the probability distribution of the number of non\sphinxhyphen{}“1”s that
appear before the third “1” is a negative binomial distribution.
\subsubsection*{References}
\subsubsection*{Examples}

\sphinxAtStartPar
Draw samples from the distribution:

\sphinxAtStartPar
A real world example. A company drills wild\sphinxhyphen{}cat oil
exploration wells, each with an estimated probability of
success of 0.1.  What is the probability of having one success
for each successive well, that is what is the probability of a
single success after drilling 5 wells, after 6 wells, etc.?

\begin{sphinxVerbatim}[commandchars=\\\{\}]
\PYG{g+gp}{\PYGZgt{}\PYGZgt{}\PYGZgt{} }\PYG{n}{s} \PYG{o}{=} \PYG{n}{np}\PYG{o}{.}\PYG{n}{random}\PYG{o}{.}\PYG{n}{negative\PYGZus{}binomial}\PYG{p}{(}\PYG{l+m+mi}{1}\PYG{p}{,} \PYG{l+m+mf}{0.1}\PYG{p}{,} \PYG{l+m+mi}{100000}\PYG{p}{)}
\PYG{g+gp}{\PYGZgt{}\PYGZgt{}\PYGZgt{} }\PYG{k}{for} \PYG{n}{i} \PYG{o+ow}{in} \PYG{n+nb}{range}\PYG{p}{(}\PYG{l+m+mi}{1}\PYG{p}{,} \PYG{l+m+mi}{11}\PYG{p}{)}\PYG{p}{:} 
\PYG{g+gp}{... }   \PYG{n}{probability} \PYG{o}{=} \PYG{n+nb}{sum}\PYG{p}{(}\PYG{n}{s}\PYG{o}{\PYGZlt{}}\PYG{n}{i}\PYG{p}{)} \PYG{o}{/} \PYG{l+m+mf}{100000.}
\PYG{g+gp}{... }   \PYG{n+nb}{print}\PYG{p}{(}\PYG{n}{i}\PYG{p}{,} \PYG{l+s+s2}{\PYGZdq{}}\PYG{l+s+s2}{wells drilled, probability of one success =}\PYG{l+s+s2}{\PYGZdq{}}\PYG{p}{,} \PYG{n}{probability}\PYG{p}{)}
\end{sphinxVerbatim}

\end{fulllineitems}

\index{noncentral\_chisquare() (in module metilda.controllers.pitch\_art\_wizard)@\spxentry{noncentral\_chisquare()}\spxextra{in module metilda.controllers.pitch\_art\_wizard}}

\begin{fulllineitems}
\phantomsection\label{\detokenize{metilda.controllers:metilda.controllers.pitch_art_wizard.noncentral_chisquare}}
\pysigstartsignatures
\pysiglinewithargsret{\sphinxcode{\sphinxupquote{metilda.controllers.pitch\_art\_wizard.}}\sphinxbfcode{\sphinxupquote{noncentral\_chisquare}}}{\sphinxparam{\DUrole{n,n}{df}}\sphinxparamcomma \sphinxparam{\DUrole{n,n}{nonc}}\sphinxparamcomma \sphinxparam{\DUrole{n,n}{size}\DUrole{o,o}{=}\DUrole{default_value}{None}}}{}
\pysigstopsignatures
\sphinxAtStartPar
Draw samples from a noncentral chi\sphinxhyphen{}square distribution.

\sphinxAtStartPar
The noncentral \(\chi^2\) distribution is a generalization of
the \(\chi^2\) distribution.

\begin{sphinxadmonition}{note}{Note:}
\sphinxAtStartPar
New code should use the
\sphinxtitleref{\textasciitilde{}numpy.random.Generator.noncentral\_chisquare}
method of a \sphinxtitleref{\textasciitilde{}numpy.random.Generator} instance instead;
please see the \DUrole{xref,std,std-ref}{random\sphinxhyphen{}quick\sphinxhyphen{}start}.
\end{sphinxadmonition}
\begin{quote}\begin{description}
\sphinxlineitem{Parameters}\begin{itemize}
\item {} 
\sphinxAtStartPar
\sphinxstyleliteralstrong{\sphinxupquote{df}} (\sphinxstyleliteralemphasis{\sphinxupquote{float}}\sphinxstyleliteralemphasis{\sphinxupquote{ or }}\sphinxstyleliteralemphasis{\sphinxupquote{array\_like}}\sphinxstyleliteralemphasis{\sphinxupquote{ of }}\sphinxstyleliteralemphasis{\sphinxupquote{floats}}) \textendash{} 
\sphinxAtStartPar
Degrees of freedom, must be \textgreater{} 0.

\sphinxAtStartPar
\DUrole{versionmodified,changed}{Changed in version 1.10.0: }Earlier NumPy versions required dfnum \textgreater{} 1.


\item {} 
\sphinxAtStartPar
\sphinxstyleliteralstrong{\sphinxupquote{nonc}} (\sphinxstyleliteralemphasis{\sphinxupquote{float}}\sphinxstyleliteralemphasis{\sphinxupquote{ or }}\sphinxstyleliteralemphasis{\sphinxupquote{array\_like}}\sphinxstyleliteralemphasis{\sphinxupquote{ of }}\sphinxstyleliteralemphasis{\sphinxupquote{floats}}) \textendash{} Non\sphinxhyphen{}centrality, must be non\sphinxhyphen{}negative.

\item {} 
\sphinxAtStartPar
\sphinxstyleliteralstrong{\sphinxupquote{size}} (\sphinxstyleliteralemphasis{\sphinxupquote{int}}\sphinxstyleliteralemphasis{\sphinxupquote{ or }}\sphinxstyleliteralemphasis{\sphinxupquote{tuple}}\sphinxstyleliteralemphasis{\sphinxupquote{ of }}\sphinxstyleliteralemphasis{\sphinxupquote{ints}}\sphinxstyleliteralemphasis{\sphinxupquote{, }}\sphinxstyleliteralemphasis{\sphinxupquote{optional}}) \textendash{} Output shape.  If the given shape is, e.g., \sphinxcode{\sphinxupquote{(m, n, k)}}, then
\sphinxcode{\sphinxupquote{m * n * k}} samples are drawn.  If size is \sphinxcode{\sphinxupquote{None}} (default),
a single value is returned if \sphinxcode{\sphinxupquote{df}} and \sphinxcode{\sphinxupquote{nonc}} are both scalars.
Otherwise, \sphinxcode{\sphinxupquote{np.broadcast(df, nonc).size}} samples are drawn.

\end{itemize}

\sphinxlineitem{Returns}
\sphinxAtStartPar
\sphinxstylestrong{out} \textendash{} Drawn samples from the parameterized noncentral chi\sphinxhyphen{}square distribution.

\sphinxlineitem{Return type}
\sphinxAtStartPar
ndarray or scalar

\end{description}\end{quote}


\begin{sphinxseealso}{See also:}
\begin{description}
\sphinxlineitem{\sphinxcode{\sphinxupquote{random.Generator.noncentral\_chisquare}}}
\sphinxAtStartPar
which should be used for new code.

\end{description}


\end{sphinxseealso}

\subsubsection*{Notes}

\sphinxAtStartPar
The probability density function for the noncentral Chi\sphinxhyphen{}square
distribution is
\begin{equation*}
\begin{split}P(x;df,nonc) = \sum^{\infty}_{i=0}
\frac{e^{-nonc/2}(nonc/2)^{i}}{i!}
P_{Y_{df+2i}}(x),\end{split}
\end{equation*}
\sphinxAtStartPar
where \(Y_{q}\) is the Chi\sphinxhyphen{}square with q degrees of freedom.
\subsubsection*{References}
\subsubsection*{Examples}

\sphinxAtStartPar
Draw values from the distribution and plot the histogram

\begin{sphinxVerbatim}[commandchars=\\\{\}]
\PYG{g+gp}{\PYGZgt{}\PYGZgt{}\PYGZgt{} }\PYG{k+kn}{import} \PYG{n+nn}{matplotlib}\PYG{n+nn}{.}\PYG{n+nn}{pyplot} \PYG{k}{as} \PYG{n+nn}{plt}
\PYG{g+gp}{\PYGZgt{}\PYGZgt{}\PYGZgt{} }\PYG{n}{values} \PYG{o}{=} \PYG{n}{plt}\PYG{o}{.}\PYG{n}{hist}\PYG{p}{(}\PYG{n}{np}\PYG{o}{.}\PYG{n}{random}\PYG{o}{.}\PYG{n}{noncentral\PYGZus{}chisquare}\PYG{p}{(}\PYG{l+m+mi}{3}\PYG{p}{,} \PYG{l+m+mi}{20}\PYG{p}{,} \PYG{l+m+mi}{100000}\PYG{p}{)}\PYG{p}{,}
\PYG{g+gp}{... }                  \PYG{n}{bins}\PYG{o}{=}\PYG{l+m+mi}{200}\PYG{p}{,} \PYG{n}{density}\PYG{o}{=}\PYG{k+kc}{True}\PYG{p}{)}
\PYG{g+gp}{\PYGZgt{}\PYGZgt{}\PYGZgt{} }\PYG{n}{plt}\PYG{o}{.}\PYG{n}{show}\PYG{p}{(}\PYG{p}{)}
\end{sphinxVerbatim}

\sphinxAtStartPar
Draw values from a noncentral chisquare with very small noncentrality,
and compare to a chisquare.

\begin{sphinxVerbatim}[commandchars=\\\{\}]
\PYG{g+gp}{\PYGZgt{}\PYGZgt{}\PYGZgt{} }\PYG{n}{plt}\PYG{o}{.}\PYG{n}{figure}\PYG{p}{(}\PYG{p}{)}
\PYG{g+gp}{\PYGZgt{}\PYGZgt{}\PYGZgt{} }\PYG{n}{values} \PYG{o}{=} \PYG{n}{plt}\PYG{o}{.}\PYG{n}{hist}\PYG{p}{(}\PYG{n}{np}\PYG{o}{.}\PYG{n}{random}\PYG{o}{.}\PYG{n}{noncentral\PYGZus{}chisquare}\PYG{p}{(}\PYG{l+m+mi}{3}\PYG{p}{,} \PYG{l+m+mf}{.0000001}\PYG{p}{,} \PYG{l+m+mi}{100000}\PYG{p}{)}\PYG{p}{,}
\PYG{g+gp}{... }                  \PYG{n}{bins}\PYG{o}{=}\PYG{n}{np}\PYG{o}{.}\PYG{n}{arange}\PYG{p}{(}\PYG{l+m+mf}{0.}\PYG{p}{,} \PYG{l+m+mi}{25}\PYG{p}{,} \PYG{l+m+mf}{.1}\PYG{p}{)}\PYG{p}{,} \PYG{n}{density}\PYG{o}{=}\PYG{k+kc}{True}\PYG{p}{)}
\PYG{g+gp}{\PYGZgt{}\PYGZgt{}\PYGZgt{} }\PYG{n}{values2} \PYG{o}{=} \PYG{n}{plt}\PYG{o}{.}\PYG{n}{hist}\PYG{p}{(}\PYG{n}{np}\PYG{o}{.}\PYG{n}{random}\PYG{o}{.}\PYG{n}{chisquare}\PYG{p}{(}\PYG{l+m+mi}{3}\PYG{p}{,} \PYG{l+m+mi}{100000}\PYG{p}{)}\PYG{p}{,}
\PYG{g+gp}{... }                   \PYG{n}{bins}\PYG{o}{=}\PYG{n}{np}\PYG{o}{.}\PYG{n}{arange}\PYG{p}{(}\PYG{l+m+mf}{0.}\PYG{p}{,} \PYG{l+m+mi}{25}\PYG{p}{,} \PYG{l+m+mf}{.1}\PYG{p}{)}\PYG{p}{,} \PYG{n}{density}\PYG{o}{=}\PYG{k+kc}{True}\PYG{p}{)}
\PYG{g+gp}{\PYGZgt{}\PYGZgt{}\PYGZgt{} }\PYG{n}{plt}\PYG{o}{.}\PYG{n}{plot}\PYG{p}{(}\PYG{n}{values}\PYG{p}{[}\PYG{l+m+mi}{1}\PYG{p}{]}\PYG{p}{[}\PYG{l+m+mi}{0}\PYG{p}{:}\PYG{o}{\PYGZhy{}}\PYG{l+m+mi}{1}\PYG{p}{]}\PYG{p}{,} \PYG{n}{values}\PYG{p}{[}\PYG{l+m+mi}{0}\PYG{p}{]}\PYG{o}{\PYGZhy{}}\PYG{n}{values2}\PYG{p}{[}\PYG{l+m+mi}{0}\PYG{p}{]}\PYG{p}{,} \PYG{l+s+s1}{\PYGZsq{}}\PYG{l+s+s1}{ob}\PYG{l+s+s1}{\PYGZsq{}}\PYG{p}{)}
\PYG{g+gp}{\PYGZgt{}\PYGZgt{}\PYGZgt{} }\PYG{n}{plt}\PYG{o}{.}\PYG{n}{show}\PYG{p}{(}\PYG{p}{)}
\end{sphinxVerbatim}

\sphinxAtStartPar
Demonstrate how large values of non\sphinxhyphen{}centrality lead to a more symmetric
distribution.

\begin{sphinxVerbatim}[commandchars=\\\{\}]
\PYG{g+gp}{\PYGZgt{}\PYGZgt{}\PYGZgt{} }\PYG{n}{plt}\PYG{o}{.}\PYG{n}{figure}\PYG{p}{(}\PYG{p}{)}
\PYG{g+gp}{\PYGZgt{}\PYGZgt{}\PYGZgt{} }\PYG{n}{values} \PYG{o}{=} \PYG{n}{plt}\PYG{o}{.}\PYG{n}{hist}\PYG{p}{(}\PYG{n}{np}\PYG{o}{.}\PYG{n}{random}\PYG{o}{.}\PYG{n}{noncentral\PYGZus{}chisquare}\PYG{p}{(}\PYG{l+m+mi}{3}\PYG{p}{,} \PYG{l+m+mi}{20}\PYG{p}{,} \PYG{l+m+mi}{100000}\PYG{p}{)}\PYG{p}{,}
\PYG{g+gp}{... }                  \PYG{n}{bins}\PYG{o}{=}\PYG{l+m+mi}{200}\PYG{p}{,} \PYG{n}{density}\PYG{o}{=}\PYG{k+kc}{True}\PYG{p}{)}
\PYG{g+gp}{\PYGZgt{}\PYGZgt{}\PYGZgt{} }\PYG{n}{plt}\PYG{o}{.}\PYG{n}{show}\PYG{p}{(}\PYG{p}{)}
\end{sphinxVerbatim}

\end{fulllineitems}

\index{noncentral\_f() (in module metilda.controllers.pitch\_art\_wizard)@\spxentry{noncentral\_f()}\spxextra{in module metilda.controllers.pitch\_art\_wizard}}

\begin{fulllineitems}
\phantomsection\label{\detokenize{metilda.controllers:metilda.controllers.pitch_art_wizard.noncentral_f}}
\pysigstartsignatures
\pysiglinewithargsret{\sphinxcode{\sphinxupquote{metilda.controllers.pitch\_art\_wizard.}}\sphinxbfcode{\sphinxupquote{noncentral\_f}}}{\sphinxparam{\DUrole{n,n}{dfnum}}\sphinxparamcomma \sphinxparam{\DUrole{n,n}{dfden}}\sphinxparamcomma \sphinxparam{\DUrole{n,n}{nonc}}\sphinxparamcomma \sphinxparam{\DUrole{n,n}{size}\DUrole{o,o}{=}\DUrole{default_value}{None}}}{}
\pysigstopsignatures
\sphinxAtStartPar
Draw samples from the noncentral F distribution.

\sphinxAtStartPar
Samples are drawn from an F distribution with specified parameters,
\sphinxtitleref{dfnum} (degrees of freedom in numerator) and \sphinxtitleref{dfden} (degrees of
freedom in denominator), where both parameters \textgreater{} 1.
\sphinxtitleref{nonc} is the non\sphinxhyphen{}centrality parameter.

\begin{sphinxadmonition}{note}{Note:}
\sphinxAtStartPar
New code should use the
\sphinxtitleref{\textasciitilde{}numpy.random.Generator.noncentral\_f}
method of a \sphinxtitleref{\textasciitilde{}numpy.random.Generator} instance instead;
please see the \DUrole{xref,std,std-ref}{random\sphinxhyphen{}quick\sphinxhyphen{}start}.
\end{sphinxadmonition}
\begin{quote}\begin{description}
\sphinxlineitem{Parameters}\begin{itemize}
\item {} 
\sphinxAtStartPar
\sphinxstyleliteralstrong{\sphinxupquote{dfnum}} (\sphinxstyleliteralemphasis{\sphinxupquote{float}}\sphinxstyleliteralemphasis{\sphinxupquote{ or }}\sphinxstyleliteralemphasis{\sphinxupquote{array\_like}}\sphinxstyleliteralemphasis{\sphinxupquote{ of }}\sphinxstyleliteralemphasis{\sphinxupquote{floats}}) \textendash{} 
\sphinxAtStartPar
Numerator degrees of freedom, must be \textgreater{} 0.

\sphinxAtStartPar
\DUrole{versionmodified,changed}{Changed in version 1.14.0: }Earlier NumPy versions required dfnum \textgreater{} 1.


\item {} 
\sphinxAtStartPar
\sphinxstyleliteralstrong{\sphinxupquote{dfden}} (\sphinxstyleliteralemphasis{\sphinxupquote{float}}\sphinxstyleliteralemphasis{\sphinxupquote{ or }}\sphinxstyleliteralemphasis{\sphinxupquote{array\_like}}\sphinxstyleliteralemphasis{\sphinxupquote{ of }}\sphinxstyleliteralemphasis{\sphinxupquote{floats}}) \textendash{} Denominator degrees of freedom, must be \textgreater{} 0.

\item {} 
\sphinxAtStartPar
\sphinxstyleliteralstrong{\sphinxupquote{nonc}} (\sphinxstyleliteralemphasis{\sphinxupquote{float}}\sphinxstyleliteralemphasis{\sphinxupquote{ or }}\sphinxstyleliteralemphasis{\sphinxupquote{array\_like}}\sphinxstyleliteralemphasis{\sphinxupquote{ of }}\sphinxstyleliteralemphasis{\sphinxupquote{floats}}) \textendash{} Non\sphinxhyphen{}centrality parameter, the sum of the squares of the numerator
means, must be \textgreater{}= 0.

\item {} 
\sphinxAtStartPar
\sphinxstyleliteralstrong{\sphinxupquote{size}} (\sphinxstyleliteralemphasis{\sphinxupquote{int}}\sphinxstyleliteralemphasis{\sphinxupquote{ or }}\sphinxstyleliteralemphasis{\sphinxupquote{tuple}}\sphinxstyleliteralemphasis{\sphinxupquote{ of }}\sphinxstyleliteralemphasis{\sphinxupquote{ints}}\sphinxstyleliteralemphasis{\sphinxupquote{, }}\sphinxstyleliteralemphasis{\sphinxupquote{optional}}) \textendash{} Output shape.  If the given shape is, e.g., \sphinxcode{\sphinxupquote{(m, n, k)}}, then
\sphinxcode{\sphinxupquote{m * n * k}} samples are drawn.  If size is \sphinxcode{\sphinxupquote{None}} (default),
a single value is returned if \sphinxcode{\sphinxupquote{dfnum}}, \sphinxcode{\sphinxupquote{dfden}}, and \sphinxcode{\sphinxupquote{nonc}}
are all scalars.  Otherwise, \sphinxcode{\sphinxupquote{np.broadcast(dfnum, dfden, nonc).size}}
samples are drawn.

\end{itemize}

\sphinxlineitem{Returns}
\sphinxAtStartPar
\sphinxstylestrong{out} \textendash{} Drawn samples from the parameterized noncentral Fisher distribution.

\sphinxlineitem{Return type}
\sphinxAtStartPar
ndarray or scalar

\end{description}\end{quote}


\begin{sphinxseealso}{See also:}
\begin{description}
\sphinxlineitem{\sphinxcode{\sphinxupquote{random.Generator.noncentral\_f}}}
\sphinxAtStartPar
which should be used for new code.

\end{description}


\end{sphinxseealso}

\subsubsection*{Notes}

\sphinxAtStartPar
When calculating the power of an experiment (power = probability of
rejecting the null hypothesis when a specific alternative is true) the
non\sphinxhyphen{}central F statistic becomes important.  When the null hypothesis is
true, the F statistic follows a central F distribution. When the null
hypothesis is not true, then it follows a non\sphinxhyphen{}central F statistic.
\subsubsection*{References}
\subsubsection*{Examples}

\sphinxAtStartPar
In a study, testing for a specific alternative to the null hypothesis
requires use of the Noncentral F distribution. We need to calculate the
area in the tail of the distribution that exceeds the value of the F
distribution for the null hypothesis.  We’ll plot the two probability
distributions for comparison.

\begin{sphinxVerbatim}[commandchars=\\\{\}]
\PYG{g+gp}{\PYGZgt{}\PYGZgt{}\PYGZgt{} }\PYG{n}{dfnum} \PYG{o}{=} \PYG{l+m+mi}{3} \PYG{c+c1}{\PYGZsh{} between group deg of freedom}
\PYG{g+gp}{\PYGZgt{}\PYGZgt{}\PYGZgt{} }\PYG{n}{dfden} \PYG{o}{=} \PYG{l+m+mi}{20} \PYG{c+c1}{\PYGZsh{} within groups degrees of freedom}
\PYG{g+gp}{\PYGZgt{}\PYGZgt{}\PYGZgt{} }\PYG{n}{nonc} \PYG{o}{=} \PYG{l+m+mf}{3.0}
\PYG{g+gp}{\PYGZgt{}\PYGZgt{}\PYGZgt{} }\PYG{n}{nc\PYGZus{}vals} \PYG{o}{=} \PYG{n}{np}\PYG{o}{.}\PYG{n}{random}\PYG{o}{.}\PYG{n}{noncentral\PYGZus{}f}\PYG{p}{(}\PYG{n}{dfnum}\PYG{p}{,} \PYG{n}{dfden}\PYG{p}{,} \PYG{n}{nonc}\PYG{p}{,} \PYG{l+m+mi}{1000000}\PYG{p}{)}
\PYG{g+gp}{\PYGZgt{}\PYGZgt{}\PYGZgt{} }\PYG{n}{NF} \PYG{o}{=} \PYG{n}{np}\PYG{o}{.}\PYG{n}{histogram}\PYG{p}{(}\PYG{n}{nc\PYGZus{}vals}\PYG{p}{,} \PYG{n}{bins}\PYG{o}{=}\PYG{l+m+mi}{50}\PYG{p}{,} \PYG{n}{density}\PYG{o}{=}\PYG{k+kc}{True}\PYG{p}{)}
\PYG{g+gp}{\PYGZgt{}\PYGZgt{}\PYGZgt{} }\PYG{n}{c\PYGZus{}vals} \PYG{o}{=} \PYG{n}{np}\PYG{o}{.}\PYG{n}{random}\PYG{o}{.}\PYG{n}{f}\PYG{p}{(}\PYG{n}{dfnum}\PYG{p}{,} \PYG{n}{dfden}\PYG{p}{,} \PYG{l+m+mi}{1000000}\PYG{p}{)}
\PYG{g+gp}{\PYGZgt{}\PYGZgt{}\PYGZgt{} }\PYG{n}{F} \PYG{o}{=} \PYG{n}{np}\PYG{o}{.}\PYG{n}{histogram}\PYG{p}{(}\PYG{n}{c\PYGZus{}vals}\PYG{p}{,} \PYG{n}{bins}\PYG{o}{=}\PYG{l+m+mi}{50}\PYG{p}{,} \PYG{n}{density}\PYG{o}{=}\PYG{k+kc}{True}\PYG{p}{)}
\PYG{g+gp}{\PYGZgt{}\PYGZgt{}\PYGZgt{} }\PYG{k+kn}{import} \PYG{n+nn}{matplotlib}\PYG{n+nn}{.}\PYG{n+nn}{pyplot} \PYG{k}{as} \PYG{n+nn}{plt}
\PYG{g+gp}{\PYGZgt{}\PYGZgt{}\PYGZgt{} }\PYG{n}{plt}\PYG{o}{.}\PYG{n}{plot}\PYG{p}{(}\PYG{n}{F}\PYG{p}{[}\PYG{l+m+mi}{1}\PYG{p}{]}\PYG{p}{[}\PYG{l+m+mi}{1}\PYG{p}{:}\PYG{p}{]}\PYG{p}{,} \PYG{n}{F}\PYG{p}{[}\PYG{l+m+mi}{0}\PYG{p}{]}\PYG{p}{)}
\PYG{g+gp}{\PYGZgt{}\PYGZgt{}\PYGZgt{} }\PYG{n}{plt}\PYG{o}{.}\PYG{n}{plot}\PYG{p}{(}\PYG{n}{NF}\PYG{p}{[}\PYG{l+m+mi}{1}\PYG{p}{]}\PYG{p}{[}\PYG{l+m+mi}{1}\PYG{p}{:}\PYG{p}{]}\PYG{p}{,} \PYG{n}{NF}\PYG{p}{[}\PYG{l+m+mi}{0}\PYG{p}{]}\PYG{p}{)}
\PYG{g+gp}{\PYGZgt{}\PYGZgt{}\PYGZgt{} }\PYG{n}{plt}\PYG{o}{.}\PYG{n}{show}\PYG{p}{(}\PYG{p}{)}
\end{sphinxVerbatim}

\end{fulllineitems}

\index{normal() (in module metilda.controllers.pitch\_art\_wizard)@\spxentry{normal()}\spxextra{in module metilda.controllers.pitch\_art\_wizard}}

\begin{fulllineitems}
\phantomsection\label{\detokenize{metilda.controllers:metilda.controllers.pitch_art_wizard.normal}}
\pysigstartsignatures
\pysiglinewithargsret{\sphinxcode{\sphinxupquote{metilda.controllers.pitch\_art\_wizard.}}\sphinxbfcode{\sphinxupquote{normal}}}{\sphinxparam{\DUrole{n,n}{loc}\DUrole{o,o}{=}\DUrole{default_value}{0.0}}\sphinxparamcomma \sphinxparam{\DUrole{n,n}{scale}\DUrole{o,o}{=}\DUrole{default_value}{1.0}}\sphinxparamcomma \sphinxparam{\DUrole{n,n}{size}\DUrole{o,o}{=}\DUrole{default_value}{None}}}{}
\pysigstopsignatures
\sphinxAtStartPar
Draw random samples from a normal (Gaussian) distribution.

\sphinxAtStartPar
The probability density function of the normal distribution, first
derived by De Moivre and 200 years later by both Gauss and Laplace
independently {\color{red}\bfseries{}{[}2{]}\_}, is often called the bell curve because of
its characteristic shape (see the example below).

\sphinxAtStartPar
The normal distributions occurs often in nature.  For example, it
describes the commonly occurring distribution of samples influenced
by a large number of tiny, random disturbances, each with its own
unique distribution {\color{red}\bfseries{}{[}2{]}\_}.

\begin{sphinxadmonition}{note}{Note:}
\sphinxAtStartPar
New code should use the \sphinxtitleref{\textasciitilde{}numpy.random.Generator.normal}
method of a \sphinxtitleref{\textasciitilde{}numpy.random.Generator} instance instead;
please see the \DUrole{xref,std,std-ref}{random\sphinxhyphen{}quick\sphinxhyphen{}start}.
\end{sphinxadmonition}
\begin{quote}\begin{description}
\sphinxlineitem{Parameters}\begin{itemize}
\item {} 
\sphinxAtStartPar
\sphinxstyleliteralstrong{\sphinxupquote{loc}} (\sphinxstyleliteralemphasis{\sphinxupquote{float}}\sphinxstyleliteralemphasis{\sphinxupquote{ or }}\sphinxstyleliteralemphasis{\sphinxupquote{array\_like}}\sphinxstyleliteralemphasis{\sphinxupquote{ of }}\sphinxstyleliteralemphasis{\sphinxupquote{floats}}) \textendash{} Mean (“centre”) of the distribution.

\item {} 
\sphinxAtStartPar
\sphinxstyleliteralstrong{\sphinxupquote{scale}} (\sphinxstyleliteralemphasis{\sphinxupquote{float}}\sphinxstyleliteralemphasis{\sphinxupquote{ or }}\sphinxstyleliteralemphasis{\sphinxupquote{array\_like}}\sphinxstyleliteralemphasis{\sphinxupquote{ of }}\sphinxstyleliteralemphasis{\sphinxupquote{floats}}) \textendash{} Standard deviation (spread or “width”) of the distribution. Must be
non\sphinxhyphen{}negative.

\item {} 
\sphinxAtStartPar
\sphinxstyleliteralstrong{\sphinxupquote{size}} (\sphinxstyleliteralemphasis{\sphinxupquote{int}}\sphinxstyleliteralemphasis{\sphinxupquote{ or }}\sphinxstyleliteralemphasis{\sphinxupquote{tuple}}\sphinxstyleliteralemphasis{\sphinxupquote{ of }}\sphinxstyleliteralemphasis{\sphinxupquote{ints}}\sphinxstyleliteralemphasis{\sphinxupquote{, }}\sphinxstyleliteralemphasis{\sphinxupquote{optional}}) \textendash{} Output shape.  If the given shape is, e.g., \sphinxcode{\sphinxupquote{(m, n, k)}}, then
\sphinxcode{\sphinxupquote{m * n * k}} samples are drawn.  If size is \sphinxcode{\sphinxupquote{None}} (default),
a single value is returned if \sphinxcode{\sphinxupquote{loc}} and \sphinxcode{\sphinxupquote{scale}} are both scalars.
Otherwise, \sphinxcode{\sphinxupquote{np.broadcast(loc, scale).size}} samples are drawn.

\end{itemize}

\sphinxlineitem{Returns}
\sphinxAtStartPar
\sphinxstylestrong{out} \textendash{} Drawn samples from the parameterized normal distribution.

\sphinxlineitem{Return type}
\sphinxAtStartPar
ndarray or scalar

\end{description}\end{quote}


\begin{sphinxseealso}{See also:}
\begin{description}
\sphinxlineitem{\sphinxcode{\sphinxupquote{scipy.stats.norm}}}
\sphinxAtStartPar
probability density function, distribution or cumulative density function, etc.

\sphinxlineitem{\sphinxcode{\sphinxupquote{random.Generator.normal}}}
\sphinxAtStartPar
which should be used for new code.

\end{description}


\end{sphinxseealso}

\subsubsection*{Notes}

\sphinxAtStartPar
The probability density for the Gaussian distribution is
\begin{equation*}
\begin{split}p(x) = \frac{1}{\sqrt{ 2 \pi \sigma^2 }}
e^{ - \frac{ (x - \mu)^2 } {2 \sigma^2} },\end{split}
\end{equation*}
\sphinxAtStartPar
where \(\mu\) is the mean and \(\sigma\) the standard
deviation. The square of the standard deviation, \(\sigma^2\),
is called the variance.

\sphinxAtStartPar
The function has its peak at the mean, and its “spread” increases with
the standard deviation (the function reaches 0.607 times its maximum at
\(x + \sigma\) and \(x - \sigma\) {\color{red}\bfseries{}{[}2{]}\_}).  This implies that
normal is more likely to return samples lying close to the mean, rather
than those far away.
\subsubsection*{References}
\subsubsection*{Examples}

\sphinxAtStartPar
Draw samples from the distribution:

\begin{sphinxVerbatim}[commandchars=\\\{\}]
\PYG{g+gp}{\PYGZgt{}\PYGZgt{}\PYGZgt{} }\PYG{n}{mu}\PYG{p}{,} \PYG{n}{sigma} \PYG{o}{=} \PYG{l+m+mi}{0}\PYG{p}{,} \PYG{l+m+mf}{0.1} \PYG{c+c1}{\PYGZsh{} mean and standard deviation}
\PYG{g+gp}{\PYGZgt{}\PYGZgt{}\PYGZgt{} }\PYG{n}{s} \PYG{o}{=} \PYG{n}{np}\PYG{o}{.}\PYG{n}{random}\PYG{o}{.}\PYG{n}{normal}\PYG{p}{(}\PYG{n}{mu}\PYG{p}{,} \PYG{n}{sigma}\PYG{p}{,} \PYG{l+m+mi}{1000}\PYG{p}{)}
\end{sphinxVerbatim}

\sphinxAtStartPar
Verify the mean and the variance:

\begin{sphinxVerbatim}[commandchars=\\\{\}]
\PYG{g+gp}{\PYGZgt{}\PYGZgt{}\PYGZgt{} }\PYG{n+nb}{abs}\PYG{p}{(}\PYG{n}{mu} \PYG{o}{\PYGZhy{}} \PYG{n}{np}\PYG{o}{.}\PYG{n}{mean}\PYG{p}{(}\PYG{n}{s}\PYG{p}{)}\PYG{p}{)}
\PYG{g+go}{0.0  \PYGZsh{} may vary}
\end{sphinxVerbatim}

\begin{sphinxVerbatim}[commandchars=\\\{\}]
\PYG{g+gp}{\PYGZgt{}\PYGZgt{}\PYGZgt{} }\PYG{n+nb}{abs}\PYG{p}{(}\PYG{n}{sigma} \PYG{o}{\PYGZhy{}} \PYG{n}{np}\PYG{o}{.}\PYG{n}{std}\PYG{p}{(}\PYG{n}{s}\PYG{p}{,} \PYG{n}{ddof}\PYG{o}{=}\PYG{l+m+mi}{1}\PYG{p}{)}\PYG{p}{)}
\PYG{g+go}{0.1  \PYGZsh{} may vary}
\end{sphinxVerbatim}

\sphinxAtStartPar
Display the histogram of the samples, along with
the probability density function:

\begin{sphinxVerbatim}[commandchars=\\\{\}]
\PYG{g+gp}{\PYGZgt{}\PYGZgt{}\PYGZgt{} }\PYG{k+kn}{import} \PYG{n+nn}{matplotlib}\PYG{n+nn}{.}\PYG{n+nn}{pyplot} \PYG{k}{as} \PYG{n+nn}{plt}
\PYG{g+gp}{\PYGZgt{}\PYGZgt{}\PYGZgt{} }\PYG{n}{count}\PYG{p}{,} \PYG{n}{bins}\PYG{p}{,} \PYG{n}{ignored} \PYG{o}{=} \PYG{n}{plt}\PYG{o}{.}\PYG{n}{hist}\PYG{p}{(}\PYG{n}{s}\PYG{p}{,} \PYG{l+m+mi}{30}\PYG{p}{,} \PYG{n}{density}\PYG{o}{=}\PYG{k+kc}{True}\PYG{p}{)}
\PYG{g+gp}{\PYGZgt{}\PYGZgt{}\PYGZgt{} }\PYG{n}{plt}\PYG{o}{.}\PYG{n}{plot}\PYG{p}{(}\PYG{n}{bins}\PYG{p}{,} \PYG{l+m+mi}{1}\PYG{o}{/}\PYG{p}{(}\PYG{n}{sigma} \PYG{o}{*} \PYG{n}{np}\PYG{o}{.}\PYG{n}{sqrt}\PYG{p}{(}\PYG{l+m+mi}{2} \PYG{o}{*} \PYG{n}{np}\PYG{o}{.}\PYG{n}{pi}\PYG{p}{)}\PYG{p}{)} \PYG{o}{*}
\PYG{g+gp}{... }               \PYG{n}{np}\PYG{o}{.}\PYG{n}{exp}\PYG{p}{(} \PYG{o}{\PYGZhy{}} \PYG{p}{(}\PYG{n}{bins} \PYG{o}{\PYGZhy{}} \PYG{n}{mu}\PYG{p}{)}\PYG{o}{*}\PYG{o}{*}\PYG{l+m+mi}{2} \PYG{o}{/} \PYG{p}{(}\PYG{l+m+mi}{2} \PYG{o}{*} \PYG{n}{sigma}\PYG{o}{*}\PYG{o}{*}\PYG{l+m+mi}{2}\PYG{p}{)} \PYG{p}{)}\PYG{p}{,}
\PYG{g+gp}{... }         \PYG{n}{linewidth}\PYG{o}{=}\PYG{l+m+mi}{2}\PYG{p}{,} \PYG{n}{color}\PYG{o}{=}\PYG{l+s+s1}{\PYGZsq{}}\PYG{l+s+s1}{r}\PYG{l+s+s1}{\PYGZsq{}}\PYG{p}{)}
\PYG{g+gp}{\PYGZgt{}\PYGZgt{}\PYGZgt{} }\PYG{n}{plt}\PYG{o}{.}\PYG{n}{show}\PYG{p}{(}\PYG{p}{)}
\end{sphinxVerbatim}

\sphinxAtStartPar
Two\sphinxhyphen{}by\sphinxhyphen{}four array of samples from the normal distribution with
mean 3 and standard deviation 2.5:

\begin{sphinxVerbatim}[commandchars=\\\{\}]
\PYG{g+gp}{\PYGZgt{}\PYGZgt{}\PYGZgt{} }\PYG{n}{np}\PYG{o}{.}\PYG{n}{random}\PYG{o}{.}\PYG{n}{normal}\PYG{p}{(}\PYG{l+m+mi}{3}\PYG{p}{,} \PYG{l+m+mf}{2.5}\PYG{p}{,} \PYG{n}{size}\PYG{o}{=}\PYG{p}{(}\PYG{l+m+mi}{2}\PYG{p}{,} \PYG{l+m+mi}{4}\PYG{p}{)}\PYG{p}{)}
\PYG{g+go}{array([[\PYGZhy{}4.49401501,  4.00950034, \PYGZhy{}1.81814867,  7.29718677],   \PYGZsh{} random}
\PYG{g+go}{       [ 0.39924804,  4.68456316,  4.99394529,  4.84057254]])  \PYGZsh{} random}
\end{sphinxVerbatim}

\end{fulllineitems}

\index{pareto() (in module metilda.controllers.pitch\_art\_wizard)@\spxentry{pareto()}\spxextra{in module metilda.controllers.pitch\_art\_wizard}}

\begin{fulllineitems}
\phantomsection\label{\detokenize{metilda.controllers:metilda.controllers.pitch_art_wizard.pareto}}
\pysigstartsignatures
\pysiglinewithargsret{\sphinxcode{\sphinxupquote{metilda.controllers.pitch\_art\_wizard.}}\sphinxbfcode{\sphinxupquote{pareto}}}{\sphinxparam{\DUrole{n,n}{a}}\sphinxparamcomma \sphinxparam{\DUrole{n,n}{size}\DUrole{o,o}{=}\DUrole{default_value}{None}}}{}
\pysigstopsignatures
\sphinxAtStartPar
Draw samples from a Pareto II or Lomax distribution with
specified shape.

\sphinxAtStartPar
The Lomax or Pareto II distribution is a shifted Pareto
distribution. The classical Pareto distribution can be
obtained from the Lomax distribution by adding 1 and
multiplying by the scale parameter \sphinxcode{\sphinxupquote{m}} (see Notes).  The
smallest value of the Lomax distribution is zero while for the
classical Pareto distribution it is \sphinxcode{\sphinxupquote{mu}}, where the standard
Pareto distribution has location \sphinxcode{\sphinxupquote{mu = 1}}.  Lomax can also
be considered as a simplified version of the Generalized
Pareto distribution (available in SciPy), with the scale set
to one and the location set to zero.

\sphinxAtStartPar
The Pareto distribution must be greater than zero, and is
unbounded above.  It is also known as the “80\sphinxhyphen{}20 rule”.  In
this distribution, 80 percent of the weights are in the lowest
20 percent of the range, while the other 20 percent fill the
remaining 80 percent of the range.

\begin{sphinxadmonition}{note}{Note:}
\sphinxAtStartPar
New code should use the \sphinxtitleref{\textasciitilde{}numpy.random.Generator.pareto}
method of a \sphinxtitleref{\textasciitilde{}numpy.random.Generator} instance instead;
please see the \DUrole{xref,std,std-ref}{random\sphinxhyphen{}quick\sphinxhyphen{}start}.
\end{sphinxadmonition}
\begin{quote}\begin{description}
\sphinxlineitem{Parameters}\begin{itemize}
\item {} 
\sphinxAtStartPar
\sphinxstyleliteralstrong{\sphinxupquote{a}} (\sphinxstyleliteralemphasis{\sphinxupquote{float}}\sphinxstyleliteralemphasis{\sphinxupquote{ or }}\sphinxstyleliteralemphasis{\sphinxupquote{array\_like}}\sphinxstyleliteralemphasis{\sphinxupquote{ of }}\sphinxstyleliteralemphasis{\sphinxupquote{floats}}) \textendash{} Shape of the distribution. Must be positive.

\item {} 
\sphinxAtStartPar
\sphinxstyleliteralstrong{\sphinxupquote{size}} (\sphinxstyleliteralemphasis{\sphinxupquote{int}}\sphinxstyleliteralemphasis{\sphinxupquote{ or }}\sphinxstyleliteralemphasis{\sphinxupquote{tuple}}\sphinxstyleliteralemphasis{\sphinxupquote{ of }}\sphinxstyleliteralemphasis{\sphinxupquote{ints}}\sphinxstyleliteralemphasis{\sphinxupquote{, }}\sphinxstyleliteralemphasis{\sphinxupquote{optional}}) \textendash{} Output shape.  If the given shape is, e.g., \sphinxcode{\sphinxupquote{(m, n, k)}}, then
\sphinxcode{\sphinxupquote{m * n * k}} samples are drawn.  If size is \sphinxcode{\sphinxupquote{None}} (default),
a single value is returned if \sphinxcode{\sphinxupquote{a}} is a scalar.  Otherwise,
\sphinxcode{\sphinxupquote{np.array(a).size}} samples are drawn.

\end{itemize}

\sphinxlineitem{Returns}
\sphinxAtStartPar
\sphinxstylestrong{out} \textendash{} Drawn samples from the parameterized Pareto distribution.

\sphinxlineitem{Return type}
\sphinxAtStartPar
ndarray or scalar

\end{description}\end{quote}


\begin{sphinxseealso}{See also:}
\begin{description}
\sphinxlineitem{\sphinxcode{\sphinxupquote{scipy.stats.lomax}}}
\sphinxAtStartPar
probability density function, distribution or cumulative density function, etc.

\sphinxlineitem{\sphinxcode{\sphinxupquote{scipy.stats.genpareto}}}
\sphinxAtStartPar
probability density function, distribution or cumulative density function, etc.

\sphinxlineitem{\sphinxcode{\sphinxupquote{random.Generator.pareto}}}
\sphinxAtStartPar
which should be used for new code.

\end{description}


\end{sphinxseealso}

\subsubsection*{Notes}

\sphinxAtStartPar
The probability density for the Pareto distribution is
\begin{equation*}
\begin{split}p(x) = \frac{am^a}{x^{a+1}}\end{split}
\end{equation*}
\sphinxAtStartPar
where \(a\) is the shape and \(m\) the scale.

\sphinxAtStartPar
The Pareto distribution, named after the Italian economist
Vilfredo Pareto, is a power law probability distribution
useful in many real world problems.  Outside the field of
economics it is generally referred to as the Bradford
distribution. Pareto developed the distribution to describe
the distribution of wealth in an economy.  It has also found
use in insurance, web page access statistics, oil field sizes,
and many other problems, including the download frequency for
projects in Sourceforge {\color{red}\bfseries{}{[}1{]}\_}.  It is one of the so\sphinxhyphen{}called
“fat\sphinxhyphen{}tailed” distributions.
\subsubsection*{References}
\subsubsection*{Examples}

\sphinxAtStartPar
Draw samples from the distribution:

\begin{sphinxVerbatim}[commandchars=\\\{\}]
\PYG{g+gp}{\PYGZgt{}\PYGZgt{}\PYGZgt{} }\PYG{n}{a}\PYG{p}{,} \PYG{n}{m} \PYG{o}{=} \PYG{l+m+mf}{3.}\PYG{p}{,} \PYG{l+m+mf}{2.}  \PYG{c+c1}{\PYGZsh{} shape and mode}
\PYG{g+gp}{\PYGZgt{}\PYGZgt{}\PYGZgt{} }\PYG{n}{s} \PYG{o}{=} \PYG{p}{(}\PYG{n}{np}\PYG{o}{.}\PYG{n}{random}\PYG{o}{.}\PYG{n}{pareto}\PYG{p}{(}\PYG{n}{a}\PYG{p}{,} \PYG{l+m+mi}{1000}\PYG{p}{)} \PYG{o}{+} \PYG{l+m+mi}{1}\PYG{p}{)} \PYG{o}{*} \PYG{n}{m}
\end{sphinxVerbatim}

\sphinxAtStartPar
Display the histogram of the samples, along with the probability
density function:

\begin{sphinxVerbatim}[commandchars=\\\{\}]
\PYG{g+gp}{\PYGZgt{}\PYGZgt{}\PYGZgt{} }\PYG{k+kn}{import} \PYG{n+nn}{matplotlib}\PYG{n+nn}{.}\PYG{n+nn}{pyplot} \PYG{k}{as} \PYG{n+nn}{plt}
\PYG{g+gp}{\PYGZgt{}\PYGZgt{}\PYGZgt{} }\PYG{n}{count}\PYG{p}{,} \PYG{n}{bins}\PYG{p}{,} \PYG{n}{\PYGZus{}} \PYG{o}{=} \PYG{n}{plt}\PYG{o}{.}\PYG{n}{hist}\PYG{p}{(}\PYG{n}{s}\PYG{p}{,} \PYG{l+m+mi}{100}\PYG{p}{,} \PYG{n}{density}\PYG{o}{=}\PYG{k+kc}{True}\PYG{p}{)}
\PYG{g+gp}{\PYGZgt{}\PYGZgt{}\PYGZgt{} }\PYG{n}{fit} \PYG{o}{=} \PYG{n}{a}\PYG{o}{*}\PYG{n}{m}\PYG{o}{*}\PYG{o}{*}\PYG{n}{a} \PYG{o}{/} \PYG{n}{bins}\PYG{o}{*}\PYG{o}{*}\PYG{p}{(}\PYG{n}{a}\PYG{o}{+}\PYG{l+m+mi}{1}\PYG{p}{)}
\PYG{g+gp}{\PYGZgt{}\PYGZgt{}\PYGZgt{} }\PYG{n}{plt}\PYG{o}{.}\PYG{n}{plot}\PYG{p}{(}\PYG{n}{bins}\PYG{p}{,} \PYG{n+nb}{max}\PYG{p}{(}\PYG{n}{count}\PYG{p}{)}\PYG{o}{*}\PYG{n}{fit}\PYG{o}{/}\PYG{n+nb}{max}\PYG{p}{(}\PYG{n}{fit}\PYG{p}{)}\PYG{p}{,} \PYG{n}{linewidth}\PYG{o}{=}\PYG{l+m+mi}{2}\PYG{p}{,} \PYG{n}{color}\PYG{o}{=}\PYG{l+s+s1}{\PYGZsq{}}\PYG{l+s+s1}{r}\PYG{l+s+s1}{\PYGZsq{}}\PYG{p}{)}
\PYG{g+gp}{\PYGZgt{}\PYGZgt{}\PYGZgt{} }\PYG{n}{plt}\PYG{o}{.}\PYG{n}{show}\PYG{p}{(}\PYG{p}{)}
\end{sphinxVerbatim}

\end{fulllineitems}

\index{permutation() (in module metilda.controllers.pitch\_art\_wizard)@\spxentry{permutation()}\spxextra{in module metilda.controllers.pitch\_art\_wizard}}

\begin{fulllineitems}
\phantomsection\label{\detokenize{metilda.controllers:metilda.controllers.pitch_art_wizard.permutation}}
\pysigstartsignatures
\pysiglinewithargsret{\sphinxcode{\sphinxupquote{metilda.controllers.pitch\_art\_wizard.}}\sphinxbfcode{\sphinxupquote{permutation}}}{\sphinxparam{\DUrole{n,n}{x}}}{}
\pysigstopsignatures
\sphinxAtStartPar
Randomly permute a sequence, or return a permuted range.

\sphinxAtStartPar
If \sphinxtitleref{x} is a multi\sphinxhyphen{}dimensional array, it is only shuffled along its
first index.

\begin{sphinxadmonition}{note}{Note:}
\sphinxAtStartPar
New code should use the
\sphinxtitleref{\textasciitilde{}numpy.random.Generator.permutation}
method of a \sphinxtitleref{\textasciitilde{}numpy.random.Generator} instance instead;
please see the \DUrole{xref,std,std-ref}{random\sphinxhyphen{}quick\sphinxhyphen{}start}.
\end{sphinxadmonition}
\begin{quote}\begin{description}
\sphinxlineitem{Parameters}
\sphinxAtStartPar
\sphinxstyleliteralstrong{\sphinxupquote{x}} (\sphinxstyleliteralemphasis{\sphinxupquote{int}}\sphinxstyleliteralemphasis{\sphinxupquote{ or }}\sphinxstyleliteralemphasis{\sphinxupquote{array\_like}}) \textendash{} If \sphinxtitleref{x} is an integer, randomly permute \sphinxcode{\sphinxupquote{np.arange(x)}}.
If \sphinxtitleref{x} is an array, make a copy and shuffle the elements
randomly.

\sphinxlineitem{Returns}
\sphinxAtStartPar
\sphinxstylestrong{out} \textendash{} Permuted sequence or array range.

\sphinxlineitem{Return type}
\sphinxAtStartPar
ndarray

\end{description}\end{quote}


\begin{sphinxseealso}{See also:}
\begin{description}
\sphinxlineitem{\sphinxcode{\sphinxupquote{random.Generator.permutation}}}
\sphinxAtStartPar
which should be used for new code.

\end{description}


\end{sphinxseealso}

\subsubsection*{Examples}

\begin{sphinxVerbatim}[commandchars=\\\{\}]
\PYG{g+gp}{\PYGZgt{}\PYGZgt{}\PYGZgt{} }\PYG{n}{np}\PYG{o}{.}\PYG{n}{random}\PYG{o}{.}\PYG{n}{permutation}\PYG{p}{(}\PYG{l+m+mi}{10}\PYG{p}{)}
\PYG{g+go}{array([1, 7, 4, 3, 0, 9, 2, 5, 8, 6]) \PYGZsh{} random}
\end{sphinxVerbatim}

\begin{sphinxVerbatim}[commandchars=\\\{\}]
\PYG{g+gp}{\PYGZgt{}\PYGZgt{}\PYGZgt{} }\PYG{n}{np}\PYG{o}{.}\PYG{n}{random}\PYG{o}{.}\PYG{n}{permutation}\PYG{p}{(}\PYG{p}{[}\PYG{l+m+mi}{1}\PYG{p}{,} \PYG{l+m+mi}{4}\PYG{p}{,} \PYG{l+m+mi}{9}\PYG{p}{,} \PYG{l+m+mi}{12}\PYG{p}{,} \PYG{l+m+mi}{15}\PYG{p}{]}\PYG{p}{)}
\PYG{g+go}{array([15,  1,  9,  4, 12]) \PYGZsh{} random}
\end{sphinxVerbatim}

\begin{sphinxVerbatim}[commandchars=\\\{\}]
\PYG{g+gp}{\PYGZgt{}\PYGZgt{}\PYGZgt{} }\PYG{n}{arr} \PYG{o}{=} \PYG{n}{np}\PYG{o}{.}\PYG{n}{arange}\PYG{p}{(}\PYG{l+m+mi}{9}\PYG{p}{)}\PYG{o}{.}\PYG{n}{reshape}\PYG{p}{(}\PYG{p}{(}\PYG{l+m+mi}{3}\PYG{p}{,} \PYG{l+m+mi}{3}\PYG{p}{)}\PYG{p}{)}
\PYG{g+gp}{\PYGZgt{}\PYGZgt{}\PYGZgt{} }\PYG{n}{np}\PYG{o}{.}\PYG{n}{random}\PYG{o}{.}\PYG{n}{permutation}\PYG{p}{(}\PYG{n}{arr}\PYG{p}{)}
\PYG{g+go}{array([[6, 7, 8], \PYGZsh{} random}
\PYG{g+go}{       [0, 1, 2],}
\PYG{g+go}{       [3, 4, 5]])}
\end{sphinxVerbatim}

\end{fulllineitems}

\index{pitchValueAtTime() (in module metilda.controllers.pitch\_art\_wizard)@\spxentry{pitchValueAtTime()}\spxextra{in module metilda.controllers.pitch\_art\_wizard}}

\begin{fulllineitems}
\phantomsection\label{\detokenize{metilda.controllers:metilda.controllers.pitch_art_wizard.pitchValueAtTime}}
\pysigstartsignatures
\pysiglinewithargsret{\sphinxcode{\sphinxupquote{metilda.controllers.pitch\_art\_wizard.}}\sphinxbfcode{\sphinxupquote{pitchValueAtTime}}}{\sphinxparam{\DUrole{n,n}{sound}}\sphinxparamcomma \sphinxparam{\DUrole{n,n}{time}}}{}
\pysigstopsignatures
\end{fulllineitems}

\index{pitchValueInFrame() (in module metilda.controllers.pitch\_art\_wizard)@\spxentry{pitchValueInFrame()}\spxextra{in module metilda.controllers.pitch\_art\_wizard}}

\begin{fulllineitems}
\phantomsection\label{\detokenize{metilda.controllers:metilda.controllers.pitch_art_wizard.pitchValueInFrame}}
\pysigstartsignatures
\pysiglinewithargsret{\sphinxcode{\sphinxupquote{metilda.controllers.pitch\_art\_wizard.}}\sphinxbfcode{\sphinxupquote{pitchValueInFrame}}}{\sphinxparam{\DUrole{n,n}{sound}}\sphinxparamcomma \sphinxparam{\DUrole{n,n}{frame}}}{}
\pysigstopsignatures
\end{fulllineitems}

\index{pointProcessGetJitter() (in module metilda.controllers.pitch\_art\_wizard)@\spxentry{pointProcessGetJitter()}\spxextra{in module metilda.controllers.pitch\_art\_wizard}}

\begin{fulllineitems}
\phantomsection\label{\detokenize{metilda.controllers:metilda.controllers.pitch_art_wizard.pointProcessGetJitter}}
\pysigstartsignatures
\pysiglinewithargsret{\sphinxcode{\sphinxupquote{metilda.controllers.pitch\_art\_wizard.}}\sphinxbfcode{\sphinxupquote{pointProcessGetJitter}}}{\sphinxparam{\DUrole{n,n}{sound}}\sphinxparamcomma \sphinxparam{\DUrole{n,n}{start}}\sphinxparamcomma \sphinxparam{\DUrole{n,n}{end}}}{}
\pysigstopsignatures
\end{fulllineitems}

\index{pointProcessGetNumPeriods() (in module metilda.controllers.pitch\_art\_wizard)@\spxentry{pointProcessGetNumPeriods()}\spxextra{in module metilda.controllers.pitch\_art\_wizard}}

\begin{fulllineitems}
\phantomsection\label{\detokenize{metilda.controllers:metilda.controllers.pitch_art_wizard.pointProcessGetNumPeriods}}
\pysigstartsignatures
\pysiglinewithargsret{\sphinxcode{\sphinxupquote{metilda.controllers.pitch\_art\_wizard.}}\sphinxbfcode{\sphinxupquote{pointProcessGetNumPeriods}}}{\sphinxparam{\DUrole{n,n}{sound}}\sphinxparamcomma \sphinxparam{\DUrole{n,n}{start}}\sphinxparamcomma \sphinxparam{\DUrole{n,n}{end}}}{}
\pysigstopsignatures
\end{fulllineitems}

\index{pointProcessGetNumPoints() (in module metilda.controllers.pitch\_art\_wizard)@\spxentry{pointProcessGetNumPoints()}\spxextra{in module metilda.controllers.pitch\_art\_wizard}}

\begin{fulllineitems}
\phantomsection\label{\detokenize{metilda.controllers:metilda.controllers.pitch_art_wizard.pointProcessGetNumPoints}}
\pysigstartsignatures
\pysiglinewithargsret{\sphinxcode{\sphinxupquote{metilda.controllers.pitch\_art\_wizard.}}\sphinxbfcode{\sphinxupquote{pointProcessGetNumPoints}}}{\sphinxparam{\DUrole{n,n}{sound}}}{}
\pysigstopsignatures
\end{fulllineitems}

\index{poisson() (in module metilda.controllers.pitch\_art\_wizard)@\spxentry{poisson()}\spxextra{in module metilda.controllers.pitch\_art\_wizard}}

\begin{fulllineitems}
\phantomsection\label{\detokenize{metilda.controllers:metilda.controllers.pitch_art_wizard.poisson}}
\pysigstartsignatures
\pysiglinewithargsret{\sphinxcode{\sphinxupquote{metilda.controllers.pitch\_art\_wizard.}}\sphinxbfcode{\sphinxupquote{poisson}}}{\sphinxparam{\DUrole{n,n}{lam}\DUrole{o,o}{=}\DUrole{default_value}{1.0}}\sphinxparamcomma \sphinxparam{\DUrole{n,n}{size}\DUrole{o,o}{=}\DUrole{default_value}{None}}}{}
\pysigstopsignatures
\sphinxAtStartPar
Draw samples from a Poisson distribution.

\sphinxAtStartPar
The Poisson distribution is the limit of the binomial distribution
for large N.

\begin{sphinxadmonition}{note}{Note:}
\sphinxAtStartPar
New code should use the \sphinxtitleref{\textasciitilde{}numpy.random.Generator.poisson}
method of a \sphinxtitleref{\textasciitilde{}numpy.random.Generator} instance instead;
please see the \DUrole{xref,std,std-ref}{random\sphinxhyphen{}quick\sphinxhyphen{}start}.
\end{sphinxadmonition}
\begin{quote}\begin{description}
\sphinxlineitem{Parameters}\begin{itemize}
\item {} 
\sphinxAtStartPar
\sphinxstyleliteralstrong{\sphinxupquote{lam}} (\sphinxstyleliteralemphasis{\sphinxupquote{float}}\sphinxstyleliteralemphasis{\sphinxupquote{ or }}\sphinxstyleliteralemphasis{\sphinxupquote{array\_like}}\sphinxstyleliteralemphasis{\sphinxupquote{ of }}\sphinxstyleliteralemphasis{\sphinxupquote{floats}}) \textendash{} Expected number of events occurring in a fixed\sphinxhyphen{}time interval,
must be \textgreater{}= 0. A sequence must be broadcastable over the requested
size.

\item {} 
\sphinxAtStartPar
\sphinxstyleliteralstrong{\sphinxupquote{size}} (\sphinxstyleliteralemphasis{\sphinxupquote{int}}\sphinxstyleliteralemphasis{\sphinxupquote{ or }}\sphinxstyleliteralemphasis{\sphinxupquote{tuple}}\sphinxstyleliteralemphasis{\sphinxupquote{ of }}\sphinxstyleliteralemphasis{\sphinxupquote{ints}}\sphinxstyleliteralemphasis{\sphinxupquote{, }}\sphinxstyleliteralemphasis{\sphinxupquote{optional}}) \textendash{} Output shape.  If the given shape is, e.g., \sphinxcode{\sphinxupquote{(m, n, k)}}, then
\sphinxcode{\sphinxupquote{m * n * k}} samples are drawn.  If size is \sphinxcode{\sphinxupquote{None}} (default),
a single value is returned if \sphinxcode{\sphinxupquote{lam}} is a scalar. Otherwise,
\sphinxcode{\sphinxupquote{np.array(lam).size}} samples are drawn.

\end{itemize}

\sphinxlineitem{Returns}
\sphinxAtStartPar
\sphinxstylestrong{out} \textendash{} Drawn samples from the parameterized Poisson distribution.

\sphinxlineitem{Return type}
\sphinxAtStartPar
ndarray or scalar

\end{description}\end{quote}


\begin{sphinxseealso}{See also:}
\begin{description}
\sphinxlineitem{\sphinxcode{\sphinxupquote{random.Generator.poisson}}}
\sphinxAtStartPar
which should be used for new code.

\end{description}


\end{sphinxseealso}

\subsubsection*{Notes}

\sphinxAtStartPar
The Poisson distribution
\begin{equation*}
\begin{split}f(k; \lambda)=\frac{\lambda^k e^{-\lambda}}{k!}\end{split}
\end{equation*}
\sphinxAtStartPar
For events with an expected separation \(\lambda\) the Poisson
distribution \(f(k; \lambda)\) describes the probability of
\(k\) events occurring within the observed
interval \(\lambda\).

\sphinxAtStartPar
Because the output is limited to the range of the C int64 type, a
ValueError is raised when \sphinxtitleref{lam} is within 10 sigma of the maximum
representable value.
\subsubsection*{References}
\subsubsection*{Examples}

\sphinxAtStartPar
Draw samples from the distribution:

\begin{sphinxVerbatim}[commandchars=\\\{\}]
\PYG{g+gp}{\PYGZgt{}\PYGZgt{}\PYGZgt{} }\PYG{k+kn}{import} \PYG{n+nn}{numpy} \PYG{k}{as} \PYG{n+nn}{np}
\PYG{g+gp}{\PYGZgt{}\PYGZgt{}\PYGZgt{} }\PYG{n}{s} \PYG{o}{=} \PYG{n}{np}\PYG{o}{.}\PYG{n}{random}\PYG{o}{.}\PYG{n}{poisson}\PYG{p}{(}\PYG{l+m+mi}{5}\PYG{p}{,} \PYG{l+m+mi}{10000}\PYG{p}{)}
\end{sphinxVerbatim}

\sphinxAtStartPar
Display histogram of the sample:

\begin{sphinxVerbatim}[commandchars=\\\{\}]
\PYG{g+gp}{\PYGZgt{}\PYGZgt{}\PYGZgt{} }\PYG{k+kn}{import} \PYG{n+nn}{matplotlib}\PYG{n+nn}{.}\PYG{n+nn}{pyplot} \PYG{k}{as} \PYG{n+nn}{plt}
\PYG{g+gp}{\PYGZgt{}\PYGZgt{}\PYGZgt{} }\PYG{n}{count}\PYG{p}{,} \PYG{n}{bins}\PYG{p}{,} \PYG{n}{ignored} \PYG{o}{=} \PYG{n}{plt}\PYG{o}{.}\PYG{n}{hist}\PYG{p}{(}\PYG{n}{s}\PYG{p}{,} \PYG{l+m+mi}{14}\PYG{p}{,} \PYG{n}{density}\PYG{o}{=}\PYG{k+kc}{True}\PYG{p}{)}
\PYG{g+gp}{\PYGZgt{}\PYGZgt{}\PYGZgt{} }\PYG{n}{plt}\PYG{o}{.}\PYG{n}{show}\PYG{p}{(}\PYG{p}{)}
\end{sphinxVerbatim}

\sphinxAtStartPar
Draw each 100 values for lambda 100 and 500:

\begin{sphinxVerbatim}[commandchars=\\\{\}]
\PYG{g+gp}{\PYGZgt{}\PYGZgt{}\PYGZgt{} }\PYG{n}{s} \PYG{o}{=} \PYG{n}{np}\PYG{o}{.}\PYG{n}{random}\PYG{o}{.}\PYG{n}{poisson}\PYG{p}{(}\PYG{n}{lam}\PYG{o}{=}\PYG{p}{(}\PYG{l+m+mf}{100.}\PYG{p}{,} \PYG{l+m+mf}{500.}\PYG{p}{)}\PYG{p}{,} \PYG{n}{size}\PYG{o}{=}\PYG{p}{(}\PYG{l+m+mi}{100}\PYG{p}{,} \PYG{l+m+mi}{2}\PYG{p}{)}\PYG{p}{)}
\end{sphinxVerbatim}

\end{fulllineitems}

\index{power() (in module metilda.controllers.pitch\_art\_wizard)@\spxentry{power()}\spxextra{in module metilda.controllers.pitch\_art\_wizard}}

\begin{fulllineitems}
\phantomsection\label{\detokenize{metilda.controllers:metilda.controllers.pitch_art_wizard.power}}
\pysigstartsignatures
\pysiglinewithargsret{\sphinxcode{\sphinxupquote{metilda.controllers.pitch\_art\_wizard.}}\sphinxbfcode{\sphinxupquote{power}}}{\sphinxparam{\DUrole{n,n}{a}}\sphinxparamcomma \sphinxparam{\DUrole{n,n}{size}\DUrole{o,o}{=}\DUrole{default_value}{None}}}{}
\pysigstopsignatures
\sphinxAtStartPar
Draws samples in {[}0, 1{]} from a power distribution with positive
exponent a \sphinxhyphen{} 1.

\sphinxAtStartPar
Also known as the power function distribution.

\begin{sphinxadmonition}{note}{Note:}
\sphinxAtStartPar
New code should use the \sphinxtitleref{\textasciitilde{}numpy.random.Generator.power}
method of a \sphinxtitleref{\textasciitilde{}numpy.random.Generator} instance instead;
please see the \DUrole{xref,std,std-ref}{random\sphinxhyphen{}quick\sphinxhyphen{}start}.
\end{sphinxadmonition}
\begin{quote}\begin{description}
\sphinxlineitem{Parameters}\begin{itemize}
\item {} 
\sphinxAtStartPar
\sphinxstyleliteralstrong{\sphinxupquote{a}} (\sphinxstyleliteralemphasis{\sphinxupquote{float}}\sphinxstyleliteralemphasis{\sphinxupquote{ or }}\sphinxstyleliteralemphasis{\sphinxupquote{array\_like}}\sphinxstyleliteralemphasis{\sphinxupquote{ of }}\sphinxstyleliteralemphasis{\sphinxupquote{floats}}) \textendash{} Parameter of the distribution. Must be non\sphinxhyphen{}negative.

\item {} 
\sphinxAtStartPar
\sphinxstyleliteralstrong{\sphinxupquote{size}} (\sphinxstyleliteralemphasis{\sphinxupquote{int}}\sphinxstyleliteralemphasis{\sphinxupquote{ or }}\sphinxstyleliteralemphasis{\sphinxupquote{tuple}}\sphinxstyleliteralemphasis{\sphinxupquote{ of }}\sphinxstyleliteralemphasis{\sphinxupquote{ints}}\sphinxstyleliteralemphasis{\sphinxupquote{, }}\sphinxstyleliteralemphasis{\sphinxupquote{optional}}) \textendash{} Output shape.  If the given shape is, e.g., \sphinxcode{\sphinxupquote{(m, n, k)}}, then
\sphinxcode{\sphinxupquote{m * n * k}} samples are drawn.  If size is \sphinxcode{\sphinxupquote{None}} (default),
a single value is returned if \sphinxcode{\sphinxupquote{a}} is a scalar.  Otherwise,
\sphinxcode{\sphinxupquote{np.array(a).size}} samples are drawn.

\end{itemize}

\sphinxlineitem{Returns}
\sphinxAtStartPar
\sphinxstylestrong{out} \textendash{} Drawn samples from the parameterized power distribution.

\sphinxlineitem{Return type}
\sphinxAtStartPar
ndarray or scalar

\sphinxlineitem{Raises}
\sphinxAtStartPar
\sphinxstyleliteralstrong{\sphinxupquote{ValueError}} \textendash{} If a \textless{}= 0.

\end{description}\end{quote}


\begin{sphinxseealso}{See also:}
\begin{description}
\sphinxlineitem{\sphinxcode{\sphinxupquote{random.Generator.power}}}
\sphinxAtStartPar
which should be used for new code.

\end{description}


\end{sphinxseealso}

\subsubsection*{Notes}

\sphinxAtStartPar
The probability density function is
\begin{equation*}
\begin{split}P(x; a) = ax^{a-1}, 0 \le x \le 1, a>0.\end{split}
\end{equation*}
\sphinxAtStartPar
The power function distribution is just the inverse of the Pareto
distribution. It may also be seen as a special case of the Beta
distribution.

\sphinxAtStartPar
It is used, for example, in modeling the over\sphinxhyphen{}reporting of insurance
claims.
\subsubsection*{References}
\subsubsection*{Examples}

\sphinxAtStartPar
Draw samples from the distribution:

\begin{sphinxVerbatim}[commandchars=\\\{\}]
\PYG{g+gp}{\PYGZgt{}\PYGZgt{}\PYGZgt{} }\PYG{n}{a} \PYG{o}{=} \PYG{l+m+mf}{5.} \PYG{c+c1}{\PYGZsh{} shape}
\PYG{g+gp}{\PYGZgt{}\PYGZgt{}\PYGZgt{} }\PYG{n}{samples} \PYG{o}{=} \PYG{l+m+mi}{1000}
\PYG{g+gp}{\PYGZgt{}\PYGZgt{}\PYGZgt{} }\PYG{n}{s} \PYG{o}{=} \PYG{n}{np}\PYG{o}{.}\PYG{n}{random}\PYG{o}{.}\PYG{n}{power}\PYG{p}{(}\PYG{n}{a}\PYG{p}{,} \PYG{n}{samples}\PYG{p}{)}
\end{sphinxVerbatim}

\sphinxAtStartPar
Display the histogram of the samples, along with
the probability density function:

\begin{sphinxVerbatim}[commandchars=\\\{\}]
\PYG{g+gp}{\PYGZgt{}\PYGZgt{}\PYGZgt{} }\PYG{k+kn}{import} \PYG{n+nn}{matplotlib}\PYG{n+nn}{.}\PYG{n+nn}{pyplot} \PYG{k}{as} \PYG{n+nn}{plt}
\PYG{g+gp}{\PYGZgt{}\PYGZgt{}\PYGZgt{} }\PYG{n}{count}\PYG{p}{,} \PYG{n}{bins}\PYG{p}{,} \PYG{n}{ignored} \PYG{o}{=} \PYG{n}{plt}\PYG{o}{.}\PYG{n}{hist}\PYG{p}{(}\PYG{n}{s}\PYG{p}{,} \PYG{n}{bins}\PYG{o}{=}\PYG{l+m+mi}{30}\PYG{p}{)}
\PYG{g+gp}{\PYGZgt{}\PYGZgt{}\PYGZgt{} }\PYG{n}{x} \PYG{o}{=} \PYG{n}{np}\PYG{o}{.}\PYG{n}{linspace}\PYG{p}{(}\PYG{l+m+mi}{0}\PYG{p}{,} \PYG{l+m+mi}{1}\PYG{p}{,} \PYG{l+m+mi}{100}\PYG{p}{)}
\PYG{g+gp}{\PYGZgt{}\PYGZgt{}\PYGZgt{} }\PYG{n}{y} \PYG{o}{=} \PYG{n}{a}\PYG{o}{*}\PYG{n}{x}\PYG{o}{*}\PYG{o}{*}\PYG{p}{(}\PYG{n}{a}\PYG{o}{\PYGZhy{}}\PYG{l+m+mf}{1.}\PYG{p}{)}
\PYG{g+gp}{\PYGZgt{}\PYGZgt{}\PYGZgt{} }\PYG{n}{normed\PYGZus{}y} \PYG{o}{=} \PYG{n}{samples}\PYG{o}{*}\PYG{n}{np}\PYG{o}{.}\PYG{n}{diff}\PYG{p}{(}\PYG{n}{bins}\PYG{p}{)}\PYG{p}{[}\PYG{l+m+mi}{0}\PYG{p}{]}\PYG{o}{*}\PYG{n}{y}
\PYG{g+gp}{\PYGZgt{}\PYGZgt{}\PYGZgt{} }\PYG{n}{plt}\PYG{o}{.}\PYG{n}{plot}\PYG{p}{(}\PYG{n}{x}\PYG{p}{,} \PYG{n}{normed\PYGZus{}y}\PYG{p}{)}
\PYG{g+gp}{\PYGZgt{}\PYGZgt{}\PYGZgt{} }\PYG{n}{plt}\PYG{o}{.}\PYG{n}{show}\PYG{p}{(}\PYG{p}{)}
\end{sphinxVerbatim}

\sphinxAtStartPar
Compare the power function distribution to the inverse of the Pareto.

\begin{sphinxVerbatim}[commandchars=\\\{\}]
\PYG{g+gp}{\PYGZgt{}\PYGZgt{}\PYGZgt{} }\PYG{k+kn}{from} \PYG{n+nn}{scipy} \PYG{k+kn}{import} \PYG{n}{stats} 
\PYG{g+gp}{\PYGZgt{}\PYGZgt{}\PYGZgt{} }\PYG{n}{rvs} \PYG{o}{=} \PYG{n}{np}\PYG{o}{.}\PYG{n}{random}\PYG{o}{.}\PYG{n}{power}\PYG{p}{(}\PYG{l+m+mi}{5}\PYG{p}{,} \PYG{l+m+mi}{1000000}\PYG{p}{)}
\PYG{g+gp}{\PYGZgt{}\PYGZgt{}\PYGZgt{} }\PYG{n}{rvsp} \PYG{o}{=} \PYG{n}{np}\PYG{o}{.}\PYG{n}{random}\PYG{o}{.}\PYG{n}{pareto}\PYG{p}{(}\PYG{l+m+mi}{5}\PYG{p}{,} \PYG{l+m+mi}{1000000}\PYG{p}{)}
\PYG{g+gp}{\PYGZgt{}\PYGZgt{}\PYGZgt{} }\PYG{n}{xx} \PYG{o}{=} \PYG{n}{np}\PYG{o}{.}\PYG{n}{linspace}\PYG{p}{(}\PYG{l+m+mi}{0}\PYG{p}{,}\PYG{l+m+mi}{1}\PYG{p}{,}\PYG{l+m+mi}{100}\PYG{p}{)}
\PYG{g+gp}{\PYGZgt{}\PYGZgt{}\PYGZgt{} }\PYG{n}{powpdf} \PYG{o}{=} \PYG{n}{stats}\PYG{o}{.}\PYG{n}{powerlaw}\PYG{o}{.}\PYG{n}{pdf}\PYG{p}{(}\PYG{n}{xx}\PYG{p}{,}\PYG{l+m+mi}{5}\PYG{p}{)}  
\end{sphinxVerbatim}

\begin{sphinxVerbatim}[commandchars=\\\{\}]
\PYG{g+gp}{\PYGZgt{}\PYGZgt{}\PYGZgt{} }\PYG{n}{plt}\PYG{o}{.}\PYG{n}{figure}\PYG{p}{(}\PYG{p}{)}
\PYG{g+gp}{\PYGZgt{}\PYGZgt{}\PYGZgt{} }\PYG{n}{plt}\PYG{o}{.}\PYG{n}{hist}\PYG{p}{(}\PYG{n}{rvs}\PYG{p}{,} \PYG{n}{bins}\PYG{o}{=}\PYG{l+m+mi}{50}\PYG{p}{,} \PYG{n}{density}\PYG{o}{=}\PYG{k+kc}{True}\PYG{p}{)}
\PYG{g+gp}{\PYGZgt{}\PYGZgt{}\PYGZgt{} }\PYG{n}{plt}\PYG{o}{.}\PYG{n}{plot}\PYG{p}{(}\PYG{n}{xx}\PYG{p}{,}\PYG{n}{powpdf}\PYG{p}{,}\PYG{l+s+s1}{\PYGZsq{}}\PYG{l+s+s1}{r\PYGZhy{}}\PYG{l+s+s1}{\PYGZsq{}}\PYG{p}{)}  
\PYG{g+gp}{\PYGZgt{}\PYGZgt{}\PYGZgt{} }\PYG{n}{plt}\PYG{o}{.}\PYG{n}{title}\PYG{p}{(}\PYG{l+s+s1}{\PYGZsq{}}\PYG{l+s+s1}{np.random.power(5)}\PYG{l+s+s1}{\PYGZsq{}}\PYG{p}{)}
\end{sphinxVerbatim}

\begin{sphinxVerbatim}[commandchars=\\\{\}]
\PYG{g+gp}{\PYGZgt{}\PYGZgt{}\PYGZgt{} }\PYG{n}{plt}\PYG{o}{.}\PYG{n}{figure}\PYG{p}{(}\PYG{p}{)}
\PYG{g+gp}{\PYGZgt{}\PYGZgt{}\PYGZgt{} }\PYG{n}{plt}\PYG{o}{.}\PYG{n}{hist}\PYG{p}{(}\PYG{l+m+mf}{1.}\PYG{o}{/}\PYG{p}{(}\PYG{l+m+mf}{1.}\PYG{o}{+}\PYG{n}{rvsp}\PYG{p}{)}\PYG{p}{,} \PYG{n}{bins}\PYG{o}{=}\PYG{l+m+mi}{50}\PYG{p}{,} \PYG{n}{density}\PYG{o}{=}\PYG{k+kc}{True}\PYG{p}{)}
\PYG{g+gp}{\PYGZgt{}\PYGZgt{}\PYGZgt{} }\PYG{n}{plt}\PYG{o}{.}\PYG{n}{plot}\PYG{p}{(}\PYG{n}{xx}\PYG{p}{,}\PYG{n}{powpdf}\PYG{p}{,}\PYG{l+s+s1}{\PYGZsq{}}\PYG{l+s+s1}{r\PYGZhy{}}\PYG{l+s+s1}{\PYGZsq{}}\PYG{p}{)}  
\PYG{g+gp}{\PYGZgt{}\PYGZgt{}\PYGZgt{} }\PYG{n}{plt}\PYG{o}{.}\PYG{n}{title}\PYG{p}{(}\PYG{l+s+s1}{\PYGZsq{}}\PYG{l+s+s1}{inverse of 1 + np.random.pareto(5)}\PYG{l+s+s1}{\PYGZsq{}}\PYG{p}{)}
\end{sphinxVerbatim}

\begin{sphinxVerbatim}[commandchars=\\\{\}]
\PYG{g+gp}{\PYGZgt{}\PYGZgt{}\PYGZgt{} }\PYG{n}{plt}\PYG{o}{.}\PYG{n}{figure}\PYG{p}{(}\PYG{p}{)}
\PYG{g+gp}{\PYGZgt{}\PYGZgt{}\PYGZgt{} }\PYG{n}{plt}\PYG{o}{.}\PYG{n}{hist}\PYG{p}{(}\PYG{l+m+mf}{1.}\PYG{o}{/}\PYG{p}{(}\PYG{l+m+mf}{1.}\PYG{o}{+}\PYG{n}{rvsp}\PYG{p}{)}\PYG{p}{,} \PYG{n}{bins}\PYG{o}{=}\PYG{l+m+mi}{50}\PYG{p}{,} \PYG{n}{density}\PYG{o}{=}\PYG{k+kc}{True}\PYG{p}{)}
\PYG{g+gp}{\PYGZgt{}\PYGZgt{}\PYGZgt{} }\PYG{n}{plt}\PYG{o}{.}\PYG{n}{plot}\PYG{p}{(}\PYG{n}{xx}\PYG{p}{,}\PYG{n}{powpdf}\PYG{p}{,}\PYG{l+s+s1}{\PYGZsq{}}\PYG{l+s+s1}{r\PYGZhy{}}\PYG{l+s+s1}{\PYGZsq{}}\PYG{p}{)}  
\PYG{g+gp}{\PYGZgt{}\PYGZgt{}\PYGZgt{} }\PYG{n}{plt}\PYG{o}{.}\PYG{n}{title}\PYG{p}{(}\PYG{l+s+s1}{\PYGZsq{}}\PYG{l+s+s1}{inverse of stats.pareto(5)}\PYG{l+s+s1}{\PYGZsq{}}\PYG{p}{)}
\end{sphinxVerbatim}

\end{fulllineitems}

\index{rand() (in module metilda.controllers.pitch\_art\_wizard)@\spxentry{rand()}\spxextra{in module metilda.controllers.pitch\_art\_wizard}}

\begin{fulllineitems}
\phantomsection\label{\detokenize{metilda.controllers:metilda.controllers.pitch_art_wizard.rand}}
\pysigstartsignatures
\pysiglinewithargsret{\sphinxcode{\sphinxupquote{metilda.controllers.pitch\_art\_wizard.}}\sphinxbfcode{\sphinxupquote{rand}}}{\sphinxparam{\DUrole{n,n}{d0}}\sphinxparamcomma \sphinxparam{\DUrole{n,n}{d1}}\sphinxparamcomma \sphinxparam{\DUrole{n,n}{...}}\sphinxparamcomma \sphinxparam{\DUrole{n,n}{dn}}}{}
\pysigstopsignatures
\sphinxAtStartPar
Random values in a given shape.

\begin{sphinxadmonition}{note}{Note:}
\sphinxAtStartPar
This is a convenience function for users porting code from Matlab,
and wraps \sphinxtitleref{random\_sample}. That function takes a
tuple to specify the size of the output, which is consistent with
other NumPy functions like \sphinxtitleref{numpy.zeros} and \sphinxtitleref{numpy.ones}.
\end{sphinxadmonition}

\sphinxAtStartPar
Create an array of the given shape and populate it with
random samples from a uniform distribution
over \sphinxcode{\sphinxupquote{{[}0, 1)}}.
\begin{quote}\begin{description}
\sphinxlineitem{Parameters}\begin{itemize}
\item {} 
\sphinxAtStartPar
\sphinxstyleliteralstrong{\sphinxupquote{d0}} (\sphinxstyleliteralemphasis{\sphinxupquote{int}}\sphinxstyleliteralemphasis{\sphinxupquote{, }}\sphinxstyleliteralemphasis{\sphinxupquote{optional}}) \textendash{} The dimensions of the returned array, must be non\sphinxhyphen{}negative.
If no argument is given a single Python float is returned.

\item {} 
\sphinxAtStartPar
\sphinxstyleliteralstrong{\sphinxupquote{d1}} (\sphinxstyleliteralemphasis{\sphinxupquote{int}}\sphinxstyleliteralemphasis{\sphinxupquote{, }}\sphinxstyleliteralemphasis{\sphinxupquote{optional}}) \textendash{} The dimensions of the returned array, must be non\sphinxhyphen{}negative.
If no argument is given a single Python float is returned.

\item {} 
\sphinxAtStartPar
\sphinxstyleliteralstrong{\sphinxupquote{...}} (\sphinxstyleliteralemphasis{\sphinxupquote{int}}\sphinxstyleliteralemphasis{\sphinxupquote{, }}\sphinxstyleliteralemphasis{\sphinxupquote{optional}}) \textendash{} The dimensions of the returned array, must be non\sphinxhyphen{}negative.
If no argument is given a single Python float is returned.

\item {} 
\sphinxAtStartPar
\sphinxstyleliteralstrong{\sphinxupquote{dn}} (\sphinxstyleliteralemphasis{\sphinxupquote{int}}\sphinxstyleliteralemphasis{\sphinxupquote{, }}\sphinxstyleliteralemphasis{\sphinxupquote{optional}}) \textendash{} The dimensions of the returned array, must be non\sphinxhyphen{}negative.
If no argument is given a single Python float is returned.

\end{itemize}

\sphinxlineitem{Returns}
\sphinxAtStartPar
\sphinxstylestrong{out} \textendash{} Random values.

\sphinxlineitem{Return type}
\sphinxAtStartPar
ndarray, shape \sphinxcode{\sphinxupquote{(d0, d1, ..., dn)}}

\end{description}\end{quote}


\begin{sphinxseealso}{See also:}

\sphinxAtStartPar
{\hyperref[\detokenize{metilda.controllers:metilda.controllers.pitch_art_wizard.random}]{\sphinxcrossref{\sphinxcode{\sphinxupquote{random}}}}}


\end{sphinxseealso}

\subsubsection*{Examples}

\begin{sphinxVerbatim}[commandchars=\\\{\}]
\PYG{g+gp}{\PYGZgt{}\PYGZgt{}\PYGZgt{} }\PYG{n}{np}\PYG{o}{.}\PYG{n}{random}\PYG{o}{.}\PYG{n}{rand}\PYG{p}{(}\PYG{l+m+mi}{3}\PYG{p}{,}\PYG{l+m+mi}{2}\PYG{p}{)}
\PYG{g+go}{array([[ 0.14022471,  0.96360618],  \PYGZsh{}random}
\PYG{g+go}{       [ 0.37601032,  0.25528411],  \PYGZsh{}random}
\PYG{g+go}{       [ 0.49313049,  0.94909878]]) \PYGZsh{}random}
\end{sphinxVerbatim}

\end{fulllineitems}

\index{randint() (in module metilda.controllers.pitch\_art\_wizard)@\spxentry{randint()}\spxextra{in module metilda.controllers.pitch\_art\_wizard}}

\begin{fulllineitems}
\phantomsection\label{\detokenize{metilda.controllers:metilda.controllers.pitch_art_wizard.randint}}
\pysigstartsignatures
\pysiglinewithargsret{\sphinxcode{\sphinxupquote{metilda.controllers.pitch\_art\_wizard.}}\sphinxbfcode{\sphinxupquote{randint}}}{\sphinxparam{\DUrole{n,n}{low}}\sphinxparamcomma \sphinxparam{\DUrole{n,n}{high}\DUrole{o,o}{=}\DUrole{default_value}{None}}\sphinxparamcomma \sphinxparam{\DUrole{n,n}{size}\DUrole{o,o}{=}\DUrole{default_value}{None}}\sphinxparamcomma \sphinxparam{\DUrole{n,n}{dtype}\DUrole{o,o}{=}\DUrole{default_value}{int}}}{}
\pysigstopsignatures
\sphinxAtStartPar
Return random integers from \sphinxtitleref{low} (inclusive) to \sphinxtitleref{high} (exclusive).

\sphinxAtStartPar
Return random integers from the “discrete uniform” distribution of
the specified dtype in the “half\sphinxhyphen{}open” interval {[}\sphinxtitleref{low}, \sphinxtitleref{high}). If
\sphinxtitleref{high} is None (the default), then results are from {[}0, \sphinxtitleref{low}).

\begin{sphinxadmonition}{note}{Note:}
\sphinxAtStartPar
New code should use the \sphinxtitleref{\textasciitilde{}numpy.random.Generator.randint}
method of a \sphinxtitleref{\textasciitilde{}numpy.random.Generator} instance instead;
please see the \DUrole{xref,std,std-ref}{random\sphinxhyphen{}quick\sphinxhyphen{}start}.
\end{sphinxadmonition}
\begin{quote}\begin{description}
\sphinxlineitem{Parameters}\begin{itemize}
\item {} 
\sphinxAtStartPar
\sphinxstyleliteralstrong{\sphinxupquote{low}} (\sphinxstyleliteralemphasis{\sphinxupquote{int}}\sphinxstyleliteralemphasis{\sphinxupquote{ or }}\sphinxstyleliteralemphasis{\sphinxupquote{array\sphinxhyphen{}like}}\sphinxstyleliteralemphasis{\sphinxupquote{ of }}\sphinxstyleliteralemphasis{\sphinxupquote{ints}}) \textendash{} Lowest (signed) integers to be drawn from the distribution (unless
\sphinxcode{\sphinxupquote{high=None}}, in which case this parameter is one above the
\sphinxstyleemphasis{highest} such integer).

\item {} 
\sphinxAtStartPar
\sphinxstyleliteralstrong{\sphinxupquote{high}} (\sphinxstyleliteralemphasis{\sphinxupquote{int}}\sphinxstyleliteralemphasis{\sphinxupquote{ or }}\sphinxstyleliteralemphasis{\sphinxupquote{array\sphinxhyphen{}like}}\sphinxstyleliteralemphasis{\sphinxupquote{ of }}\sphinxstyleliteralemphasis{\sphinxupquote{ints}}\sphinxstyleliteralemphasis{\sphinxupquote{, }}\sphinxstyleliteralemphasis{\sphinxupquote{optional}}) \textendash{} If provided, one above the largest (signed) integer to be drawn
from the distribution (see above for behavior if \sphinxcode{\sphinxupquote{high=None}}).
If array\sphinxhyphen{}like, must contain integer values

\item {} 
\sphinxAtStartPar
\sphinxstyleliteralstrong{\sphinxupquote{size}} (\sphinxstyleliteralemphasis{\sphinxupquote{int}}\sphinxstyleliteralemphasis{\sphinxupquote{ or }}\sphinxstyleliteralemphasis{\sphinxupquote{tuple}}\sphinxstyleliteralemphasis{\sphinxupquote{ of }}\sphinxstyleliteralemphasis{\sphinxupquote{ints}}\sphinxstyleliteralemphasis{\sphinxupquote{, }}\sphinxstyleliteralemphasis{\sphinxupquote{optional}}) \textendash{} Output shape.  If the given shape is, e.g., \sphinxcode{\sphinxupquote{(m, n, k)}}, then
\sphinxcode{\sphinxupquote{m * n * k}} samples are drawn.  Default is None, in which case a
single value is returned.

\item {} 
\sphinxAtStartPar
\sphinxstyleliteralstrong{\sphinxupquote{dtype}} (\sphinxstyleliteralemphasis{\sphinxupquote{dtype}}\sphinxstyleliteralemphasis{\sphinxupquote{, }}\sphinxstyleliteralemphasis{\sphinxupquote{optional}}) \textendash{} 
\sphinxAtStartPar
Desired dtype of the result. Byteorder must be native.
The default value is int.

\sphinxAtStartPar
\DUrole{versionmodified,added}{New in version 1.11.0.}


\end{itemize}

\sphinxlineitem{Returns}
\sphinxAtStartPar
\sphinxstylestrong{out} \textendash{} \sphinxtitleref{size}\sphinxhyphen{}shaped array of random integers from the appropriate
distribution, or a single such random int if \sphinxtitleref{size} not provided.

\sphinxlineitem{Return type}
\sphinxAtStartPar
int or ndarray of ints

\end{description}\end{quote}


\begin{sphinxseealso}{See also:}
\begin{description}
\sphinxlineitem{{\hyperref[\detokenize{metilda.controllers:metilda.controllers.pitch_art_wizard.random_integers}]{\sphinxcrossref{\sphinxcode{\sphinxupquote{random\_integers}}}}}}
\sphinxAtStartPar
similar to \sphinxtitleref{randint}, only for the closed interval {[}\sphinxtitleref{low}, \sphinxtitleref{high}{]}, and 1 is the lowest value if \sphinxtitleref{high} is omitted.

\sphinxlineitem{\sphinxcode{\sphinxupquote{random.Generator.integers}}}
\sphinxAtStartPar
which should be used for new code.

\end{description}


\end{sphinxseealso}

\subsubsection*{Examples}

\begin{sphinxVerbatim}[commandchars=\\\{\}]
\PYG{g+gp}{\PYGZgt{}\PYGZgt{}\PYGZgt{} }\PYG{n}{np}\PYG{o}{.}\PYG{n}{random}\PYG{o}{.}\PYG{n}{randint}\PYG{p}{(}\PYG{l+m+mi}{2}\PYG{p}{,} \PYG{n}{size}\PYG{o}{=}\PYG{l+m+mi}{10}\PYG{p}{)}
\PYG{g+go}{array([1, 0, 0, 0, 1, 1, 0, 0, 1, 0]) \PYGZsh{} random}
\PYG{g+gp}{\PYGZgt{}\PYGZgt{}\PYGZgt{} }\PYG{n}{np}\PYG{o}{.}\PYG{n}{random}\PYG{o}{.}\PYG{n}{randint}\PYG{p}{(}\PYG{l+m+mi}{1}\PYG{p}{,} \PYG{n}{size}\PYG{o}{=}\PYG{l+m+mi}{10}\PYG{p}{)}
\PYG{g+go}{array([0, 0, 0, 0, 0, 0, 0, 0, 0, 0])}
\end{sphinxVerbatim}

\sphinxAtStartPar
Generate a 2 x 4 array of ints between 0 and 4, inclusive:

\begin{sphinxVerbatim}[commandchars=\\\{\}]
\PYG{g+gp}{\PYGZgt{}\PYGZgt{}\PYGZgt{} }\PYG{n}{np}\PYG{o}{.}\PYG{n}{random}\PYG{o}{.}\PYG{n}{randint}\PYG{p}{(}\PYG{l+m+mi}{5}\PYG{p}{,} \PYG{n}{size}\PYG{o}{=}\PYG{p}{(}\PYG{l+m+mi}{2}\PYG{p}{,} \PYG{l+m+mi}{4}\PYG{p}{)}\PYG{p}{)}
\PYG{g+go}{array([[4, 0, 2, 1], \PYGZsh{} random}
\PYG{g+go}{       [3, 2, 2, 0]])}
\end{sphinxVerbatim}

\sphinxAtStartPar
Generate a 1 x 3 array with 3 different upper bounds

\begin{sphinxVerbatim}[commandchars=\\\{\}]
\PYG{g+gp}{\PYGZgt{}\PYGZgt{}\PYGZgt{} }\PYG{n}{np}\PYG{o}{.}\PYG{n}{random}\PYG{o}{.}\PYG{n}{randint}\PYG{p}{(}\PYG{l+m+mi}{1}\PYG{p}{,} \PYG{p}{[}\PYG{l+m+mi}{3}\PYG{p}{,} \PYG{l+m+mi}{5}\PYG{p}{,} \PYG{l+m+mi}{10}\PYG{p}{]}\PYG{p}{)}
\PYG{g+go}{array([2, 2, 9]) \PYGZsh{} random}
\end{sphinxVerbatim}

\sphinxAtStartPar
Generate a 1 by 3 array with 3 different lower bounds

\begin{sphinxVerbatim}[commandchars=\\\{\}]
\PYG{g+gp}{\PYGZgt{}\PYGZgt{}\PYGZgt{} }\PYG{n}{np}\PYG{o}{.}\PYG{n}{random}\PYG{o}{.}\PYG{n}{randint}\PYG{p}{(}\PYG{p}{[}\PYG{l+m+mi}{1}\PYG{p}{,} \PYG{l+m+mi}{5}\PYG{p}{,} \PYG{l+m+mi}{7}\PYG{p}{]}\PYG{p}{,} \PYG{l+m+mi}{10}\PYG{p}{)}
\PYG{g+go}{array([9, 8, 7]) \PYGZsh{} random}
\end{sphinxVerbatim}

\sphinxAtStartPar
Generate a 2 by 4 array using broadcasting with dtype of uint8

\begin{sphinxVerbatim}[commandchars=\\\{\}]
\PYG{g+gp}{\PYGZgt{}\PYGZgt{}\PYGZgt{} }\PYG{n}{np}\PYG{o}{.}\PYG{n}{random}\PYG{o}{.}\PYG{n}{randint}\PYG{p}{(}\PYG{p}{[}\PYG{l+m+mi}{1}\PYG{p}{,} \PYG{l+m+mi}{3}\PYG{p}{,} \PYG{l+m+mi}{5}\PYG{p}{,} \PYG{l+m+mi}{7}\PYG{p}{]}\PYG{p}{,} \PYG{p}{[}\PYG{p}{[}\PYG{l+m+mi}{10}\PYG{p}{]}\PYG{p}{,} \PYG{p}{[}\PYG{l+m+mi}{20}\PYG{p}{]}\PYG{p}{]}\PYG{p}{,} \PYG{n}{dtype}\PYG{o}{=}\PYG{n}{np}\PYG{o}{.}\PYG{n}{uint8}\PYG{p}{)}
\PYG{g+go}{array([[ 8,  6,  9,  7], \PYGZsh{} random}
\PYG{g+go}{       [ 1, 16,  9, 12]], dtype=uint8)}
\end{sphinxVerbatim}

\end{fulllineitems}

\index{randn() (in module metilda.controllers.pitch\_art\_wizard)@\spxentry{randn()}\spxextra{in module metilda.controllers.pitch\_art\_wizard}}

\begin{fulllineitems}
\phantomsection\label{\detokenize{metilda.controllers:metilda.controllers.pitch_art_wizard.randn}}
\pysigstartsignatures
\pysiglinewithargsret{\sphinxcode{\sphinxupquote{metilda.controllers.pitch\_art\_wizard.}}\sphinxbfcode{\sphinxupquote{randn}}}{\sphinxparam{\DUrole{n,n}{d0}}\sphinxparamcomma \sphinxparam{\DUrole{n,n}{d1}}\sphinxparamcomma \sphinxparam{\DUrole{n,n}{...}}\sphinxparamcomma \sphinxparam{\DUrole{n,n}{dn}}}{}
\pysigstopsignatures
\sphinxAtStartPar
Return a sample (or samples) from the “standard normal” distribution.

\begin{sphinxadmonition}{note}{Note:}
\sphinxAtStartPar
This is a convenience function for users porting code from Matlab,
and wraps \sphinxtitleref{standard\_normal}. That function takes a
tuple to specify the size of the output, which is consistent with
other NumPy functions like \sphinxtitleref{numpy.zeros} and \sphinxtitleref{numpy.ones}.
\end{sphinxadmonition}

\begin{sphinxadmonition}{note}{Note:}
\sphinxAtStartPar
New code should use the
\sphinxtitleref{\textasciitilde{}numpy.random.Generator.standard\_normal}
method of a \sphinxtitleref{\textasciitilde{}numpy.random.Generator} instance instead;
please see the \DUrole{xref,std,std-ref}{random\sphinxhyphen{}quick\sphinxhyphen{}start}.
\end{sphinxadmonition}

\sphinxAtStartPar
If positive int\_like arguments are provided, \sphinxtitleref{randn} generates an array
of shape \sphinxcode{\sphinxupquote{(d0, d1, ..., dn)}}, filled
with random floats sampled from a univariate “normal” (Gaussian)
distribution of mean 0 and variance 1. A single float randomly sampled
from the distribution is returned if no argument is provided.
\begin{quote}\begin{description}
\sphinxlineitem{Parameters}\begin{itemize}
\item {} 
\sphinxAtStartPar
\sphinxstyleliteralstrong{\sphinxupquote{d0}} (\sphinxstyleliteralemphasis{\sphinxupquote{int}}\sphinxstyleliteralemphasis{\sphinxupquote{, }}\sphinxstyleliteralemphasis{\sphinxupquote{optional}}) \textendash{} The dimensions of the returned array, must be non\sphinxhyphen{}negative.
If no argument is given a single Python float is returned.

\item {} 
\sphinxAtStartPar
\sphinxstyleliteralstrong{\sphinxupquote{d1}} (\sphinxstyleliteralemphasis{\sphinxupquote{int}}\sphinxstyleliteralemphasis{\sphinxupquote{, }}\sphinxstyleliteralemphasis{\sphinxupquote{optional}}) \textendash{} The dimensions of the returned array, must be non\sphinxhyphen{}negative.
If no argument is given a single Python float is returned.

\item {} 
\sphinxAtStartPar
\sphinxstyleliteralstrong{\sphinxupquote{...}} (\sphinxstyleliteralemphasis{\sphinxupquote{int}}\sphinxstyleliteralemphasis{\sphinxupquote{, }}\sphinxstyleliteralemphasis{\sphinxupquote{optional}}) \textendash{} The dimensions of the returned array, must be non\sphinxhyphen{}negative.
If no argument is given a single Python float is returned.

\item {} 
\sphinxAtStartPar
\sphinxstyleliteralstrong{\sphinxupquote{dn}} (\sphinxstyleliteralemphasis{\sphinxupquote{int}}\sphinxstyleliteralemphasis{\sphinxupquote{, }}\sphinxstyleliteralemphasis{\sphinxupquote{optional}}) \textendash{} The dimensions of the returned array, must be non\sphinxhyphen{}negative.
If no argument is given a single Python float is returned.

\end{itemize}

\sphinxlineitem{Returns}
\sphinxAtStartPar
\sphinxstylestrong{Z} \textendash{} A \sphinxcode{\sphinxupquote{(d0, d1, ..., dn)}}\sphinxhyphen{}shaped array of floating\sphinxhyphen{}point samples from
the standard normal distribution, or a single such float if
no parameters were supplied.

\sphinxlineitem{Return type}
\sphinxAtStartPar
ndarray or float

\end{description}\end{quote}


\begin{sphinxseealso}{See also:}
\begin{description}
\sphinxlineitem{{\hyperref[\detokenize{metilda.controllers:metilda.controllers.pitch_art_wizard.standard_normal}]{\sphinxcrossref{\sphinxcode{\sphinxupquote{standard\_normal}}}}}}
\sphinxAtStartPar
Similar, but takes a tuple as its argument.

\sphinxlineitem{{\hyperref[\detokenize{metilda.controllers:metilda.controllers.pitch_art_wizard.normal}]{\sphinxcrossref{\sphinxcode{\sphinxupquote{normal}}}}}}
\sphinxAtStartPar
Also accepts mu and sigma arguments.

\sphinxlineitem{\sphinxcode{\sphinxupquote{random.Generator.standard\_normal}}}
\sphinxAtStartPar
which should be used for new code.

\end{description}


\end{sphinxseealso}

\subsubsection*{Notes}

\sphinxAtStartPar
For random samples from the normal distribution with mean \sphinxcode{\sphinxupquote{mu}} and
standard deviation \sphinxcode{\sphinxupquote{sigma}}, use:

\begin{sphinxVerbatim}[commandchars=\\\{\}]
\PYG{n}{sigma} \PYG{o}{*} \PYG{n}{np}\PYG{o}{.}\PYG{n}{random}\PYG{o}{.}\PYG{n}{randn}\PYG{p}{(}\PYG{o}{.}\PYG{o}{.}\PYG{o}{.}\PYG{p}{)} \PYG{o}{+} \PYG{n}{mu}
\end{sphinxVerbatim}
\subsubsection*{Examples}

\begin{sphinxVerbatim}[commandchars=\\\{\}]
\PYG{g+gp}{\PYGZgt{}\PYGZgt{}\PYGZgt{} }\PYG{n}{np}\PYG{o}{.}\PYG{n}{random}\PYG{o}{.}\PYG{n}{randn}\PYG{p}{(}\PYG{p}{)}
\PYG{g+go}{2.1923875335537315  \PYGZsh{} random}
\end{sphinxVerbatim}

\sphinxAtStartPar
Two\sphinxhyphen{}by\sphinxhyphen{}four array of samples from the normal distribution with
mean 3 and standard deviation 2.5:

\begin{sphinxVerbatim}[commandchars=\\\{\}]
\PYG{g+gp}{\PYGZgt{}\PYGZgt{}\PYGZgt{} }\PYG{l+m+mi}{3} \PYG{o}{+} \PYG{l+m+mf}{2.5} \PYG{o}{*} \PYG{n}{np}\PYG{o}{.}\PYG{n}{random}\PYG{o}{.}\PYG{n}{randn}\PYG{p}{(}\PYG{l+m+mi}{2}\PYG{p}{,} \PYG{l+m+mi}{4}\PYG{p}{)}
\PYG{g+go}{array([[\PYGZhy{}4.49401501,  4.00950034, \PYGZhy{}1.81814867,  7.29718677],   \PYGZsh{} random}
\PYG{g+go}{       [ 0.39924804,  4.68456316,  4.99394529,  4.84057254]])  \PYGZsh{} random}
\end{sphinxVerbatim}

\end{fulllineitems}

\index{random() (in module metilda.controllers.pitch\_art\_wizard)@\spxentry{random()}\spxextra{in module metilda.controllers.pitch\_art\_wizard}}

\begin{fulllineitems}
\phantomsection\label{\detokenize{metilda.controllers:metilda.controllers.pitch_art_wizard.random}}
\pysigstartsignatures
\pysiglinewithargsret{\sphinxcode{\sphinxupquote{metilda.controllers.pitch\_art\_wizard.}}\sphinxbfcode{\sphinxupquote{random}}}{\sphinxparam{\DUrole{n,n}{size}\DUrole{o,o}{=}\DUrole{default_value}{None}}}{}
\pysigstopsignatures
\sphinxAtStartPar
Return random floats in the half\sphinxhyphen{}open interval {[}0.0, 1.0). Alias for
\sphinxtitleref{random\_sample} to ease forward\sphinxhyphen{}porting to the new random API.

\end{fulllineitems}

\index{random\_integers() (in module metilda.controllers.pitch\_art\_wizard)@\spxentry{random\_integers()}\spxextra{in module metilda.controllers.pitch\_art\_wizard}}

\begin{fulllineitems}
\phantomsection\label{\detokenize{metilda.controllers:metilda.controllers.pitch_art_wizard.random_integers}}
\pysigstartsignatures
\pysiglinewithargsret{\sphinxcode{\sphinxupquote{metilda.controllers.pitch\_art\_wizard.}}\sphinxbfcode{\sphinxupquote{random\_integers}}}{\sphinxparam{\DUrole{n,n}{low}}\sphinxparamcomma \sphinxparam{\DUrole{n,n}{high}\DUrole{o,o}{=}\DUrole{default_value}{None}}\sphinxparamcomma \sphinxparam{\DUrole{n,n}{size}\DUrole{o,o}{=}\DUrole{default_value}{None}}}{}
\pysigstopsignatures
\sphinxAtStartPar
Random integers of type \sphinxtitleref{np.int\_} between \sphinxtitleref{low} and \sphinxtitleref{high}, inclusive.

\sphinxAtStartPar
Return random integers of type \sphinxtitleref{np.int\_} from the “discrete uniform”
distribution in the closed interval {[}\sphinxtitleref{low}, \sphinxtitleref{high}{]}.  If \sphinxtitleref{high} is
None (the default), then results are from {[}1, \sphinxtitleref{low}{]}. The \sphinxtitleref{np.int\_}
type translates to the C long integer type and its precision
is platform dependent.

\sphinxAtStartPar
This function has been deprecated. Use randint instead.

\sphinxAtStartPar
\DUrole{versionmodified,deprecated}{Deprecated since version 1.11.0.}
\begin{quote}\begin{description}
\sphinxlineitem{Parameters}\begin{itemize}
\item {} 
\sphinxAtStartPar
\sphinxstyleliteralstrong{\sphinxupquote{low}} (\sphinxstyleliteralemphasis{\sphinxupquote{int}}) \textendash{} Lowest (signed) integer to be drawn from the distribution (unless
\sphinxcode{\sphinxupquote{high=None}}, in which case this parameter is the \sphinxstyleemphasis{highest} such
integer).

\item {} 
\sphinxAtStartPar
\sphinxstyleliteralstrong{\sphinxupquote{high}} (\sphinxstyleliteralemphasis{\sphinxupquote{int}}\sphinxstyleliteralemphasis{\sphinxupquote{, }}\sphinxstyleliteralemphasis{\sphinxupquote{optional}}) \textendash{} If provided, the largest (signed) integer to be drawn from the
distribution (see above for behavior if \sphinxcode{\sphinxupquote{high=None}}).

\item {} 
\sphinxAtStartPar
\sphinxstyleliteralstrong{\sphinxupquote{size}} (\sphinxstyleliteralemphasis{\sphinxupquote{int}}\sphinxstyleliteralemphasis{\sphinxupquote{ or }}\sphinxstyleliteralemphasis{\sphinxupquote{tuple}}\sphinxstyleliteralemphasis{\sphinxupquote{ of }}\sphinxstyleliteralemphasis{\sphinxupquote{ints}}\sphinxstyleliteralemphasis{\sphinxupquote{, }}\sphinxstyleliteralemphasis{\sphinxupquote{optional}}) \textendash{} Output shape.  If the given shape is, e.g., \sphinxcode{\sphinxupquote{(m, n, k)}}, then
\sphinxcode{\sphinxupquote{m * n * k}} samples are drawn.  Default is None, in which case a
single value is returned.

\end{itemize}

\sphinxlineitem{Returns}
\sphinxAtStartPar
\sphinxstylestrong{out} \textendash{} \sphinxtitleref{size}\sphinxhyphen{}shaped array of random integers from the appropriate
distribution, or a single such random int if \sphinxtitleref{size} not provided.

\sphinxlineitem{Return type}
\sphinxAtStartPar
int or ndarray of ints

\end{description}\end{quote}


\begin{sphinxseealso}{See also:}
\begin{description}
\sphinxlineitem{{\hyperref[\detokenize{metilda.controllers:metilda.controllers.pitch_art_wizard.randint}]{\sphinxcrossref{\sphinxcode{\sphinxupquote{randint}}}}}}
\sphinxAtStartPar
Similar to \sphinxtitleref{random\_integers}, only for the half\sphinxhyphen{}open interval {[}\sphinxtitleref{low}, \sphinxtitleref{high}), and 0 is the lowest value if \sphinxtitleref{high} is omitted.

\end{description}


\end{sphinxseealso}

\subsubsection*{Notes}

\sphinxAtStartPar
To sample from N evenly spaced floating\sphinxhyphen{}point numbers between a and b,
use:

\begin{sphinxVerbatim}[commandchars=\\\{\}]
\PYG{n}{a} \PYG{o}{+} \PYG{p}{(}\PYG{n}{b} \PYG{o}{\PYGZhy{}} \PYG{n}{a}\PYG{p}{)} \PYG{o}{*} \PYG{p}{(}\PYG{n}{np}\PYG{o}{.}\PYG{n}{random}\PYG{o}{.}\PYG{n}{random\PYGZus{}integers}\PYG{p}{(}\PYG{n}{N}\PYG{p}{)} \PYG{o}{\PYGZhy{}} \PYG{l+m+mi}{1}\PYG{p}{)} \PYG{o}{/} \PYG{p}{(}\PYG{n}{N} \PYG{o}{\PYGZhy{}} \PYG{l+m+mf}{1.}\PYG{p}{)}
\end{sphinxVerbatim}
\subsubsection*{Examples}

\begin{sphinxVerbatim}[commandchars=\\\{\}]
\PYG{g+gp}{\PYGZgt{}\PYGZgt{}\PYGZgt{} }\PYG{n}{np}\PYG{o}{.}\PYG{n}{random}\PYG{o}{.}\PYG{n}{random\PYGZus{}integers}\PYG{p}{(}\PYG{l+m+mi}{5}\PYG{p}{)}
\PYG{g+go}{4 \PYGZsh{} random}
\PYG{g+gp}{\PYGZgt{}\PYGZgt{}\PYGZgt{} }\PYG{n+nb}{type}\PYG{p}{(}\PYG{n}{np}\PYG{o}{.}\PYG{n}{random}\PYG{o}{.}\PYG{n}{random\PYGZus{}integers}\PYG{p}{(}\PYG{l+m+mi}{5}\PYG{p}{)}\PYG{p}{)}
\PYG{g+go}{\PYGZlt{}class \PYGZsq{}numpy.int64\PYGZsq{}\PYGZgt{}}
\PYG{g+gp}{\PYGZgt{}\PYGZgt{}\PYGZgt{} }\PYG{n}{np}\PYG{o}{.}\PYG{n}{random}\PYG{o}{.}\PYG{n}{random\PYGZus{}integers}\PYG{p}{(}\PYG{l+m+mi}{5}\PYG{p}{,} \PYG{n}{size}\PYG{o}{=}\PYG{p}{(}\PYG{l+m+mi}{3}\PYG{p}{,}\PYG{l+m+mi}{2}\PYG{p}{)}\PYG{p}{)}
\PYG{g+go}{array([[5, 4], \PYGZsh{} random}
\PYG{g+go}{       [3, 3],}
\PYG{g+go}{       [4, 5]])}
\end{sphinxVerbatim}

\sphinxAtStartPar
Choose five random numbers from the set of five evenly\sphinxhyphen{}spaced
numbers between 0 and 2.5, inclusive (\sphinxstyleemphasis{i.e.}, from the set
\({0, 5/8, 10/8, 15/8, 20/8}\)):

\begin{sphinxVerbatim}[commandchars=\\\{\}]
\PYG{g+gp}{\PYGZgt{}\PYGZgt{}\PYGZgt{} }\PYG{l+m+mf}{2.5} \PYG{o}{*} \PYG{p}{(}\PYG{n}{np}\PYG{o}{.}\PYG{n}{random}\PYG{o}{.}\PYG{n}{random\PYGZus{}integers}\PYG{p}{(}\PYG{l+m+mi}{5}\PYG{p}{,} \PYG{n}{size}\PYG{o}{=}\PYG{p}{(}\PYG{l+m+mi}{5}\PYG{p}{,}\PYG{p}{)}\PYG{p}{)} \PYG{o}{\PYGZhy{}} \PYG{l+m+mi}{1}\PYG{p}{)} \PYG{o}{/} \PYG{l+m+mf}{4.}
\PYG{g+go}{array([ 0.625,  1.25 ,  0.625,  0.625,  2.5  ]) \PYGZsh{} random}
\end{sphinxVerbatim}

\sphinxAtStartPar
Roll two six sided dice 1000 times and sum the results:

\begin{sphinxVerbatim}[commandchars=\\\{\}]
\PYG{g+gp}{\PYGZgt{}\PYGZgt{}\PYGZgt{} }\PYG{n}{d1} \PYG{o}{=} \PYG{n}{np}\PYG{o}{.}\PYG{n}{random}\PYG{o}{.}\PYG{n}{random\PYGZus{}integers}\PYG{p}{(}\PYG{l+m+mi}{1}\PYG{p}{,} \PYG{l+m+mi}{6}\PYG{p}{,} \PYG{l+m+mi}{1000}\PYG{p}{)}
\PYG{g+gp}{\PYGZgt{}\PYGZgt{}\PYGZgt{} }\PYG{n}{d2} \PYG{o}{=} \PYG{n}{np}\PYG{o}{.}\PYG{n}{random}\PYG{o}{.}\PYG{n}{random\PYGZus{}integers}\PYG{p}{(}\PYG{l+m+mi}{1}\PYG{p}{,} \PYG{l+m+mi}{6}\PYG{p}{,} \PYG{l+m+mi}{1000}\PYG{p}{)}
\PYG{g+gp}{\PYGZgt{}\PYGZgt{}\PYGZgt{} }\PYG{n}{dsums} \PYG{o}{=} \PYG{n}{d1} \PYG{o}{+} \PYG{n}{d2}
\end{sphinxVerbatim}

\sphinxAtStartPar
Display results as a histogram:

\begin{sphinxVerbatim}[commandchars=\\\{\}]
\PYG{g+gp}{\PYGZgt{}\PYGZgt{}\PYGZgt{} }\PYG{k+kn}{import} \PYG{n+nn}{matplotlib}\PYG{n+nn}{.}\PYG{n+nn}{pyplot} \PYG{k}{as} \PYG{n+nn}{plt}
\PYG{g+gp}{\PYGZgt{}\PYGZgt{}\PYGZgt{} }\PYG{n}{count}\PYG{p}{,} \PYG{n}{bins}\PYG{p}{,} \PYG{n}{ignored} \PYG{o}{=} \PYG{n}{plt}\PYG{o}{.}\PYG{n}{hist}\PYG{p}{(}\PYG{n}{dsums}\PYG{p}{,} \PYG{l+m+mi}{11}\PYG{p}{,} \PYG{n}{density}\PYG{o}{=}\PYG{k+kc}{True}\PYG{p}{)}
\PYG{g+gp}{\PYGZgt{}\PYGZgt{}\PYGZgt{} }\PYG{n}{plt}\PYG{o}{.}\PYG{n}{show}\PYG{p}{(}\PYG{p}{)}
\end{sphinxVerbatim}

\end{fulllineitems}

\index{random\_sample() (in module metilda.controllers.pitch\_art\_wizard)@\spxentry{random\_sample()}\spxextra{in module metilda.controllers.pitch\_art\_wizard}}

\begin{fulllineitems}
\phantomsection\label{\detokenize{metilda.controllers:metilda.controllers.pitch_art_wizard.random_sample}}
\pysigstartsignatures
\pysiglinewithargsret{\sphinxcode{\sphinxupquote{metilda.controllers.pitch\_art\_wizard.}}\sphinxbfcode{\sphinxupquote{random\_sample}}}{\sphinxparam{\DUrole{n,n}{size}\DUrole{o,o}{=}\DUrole{default_value}{None}}}{}
\pysigstopsignatures
\sphinxAtStartPar
Return random floats in the half\sphinxhyphen{}open interval {[}0.0, 1.0).

\sphinxAtStartPar
Results are from the “continuous uniform” distribution over the
stated interval.  To sample \(Unif[a, b), b > a\) multiply
the output of \sphinxtitleref{random\_sample} by \sphinxtitleref{(b\sphinxhyphen{}a)} and add \sphinxtitleref{a}:

\begin{sphinxVerbatim}[commandchars=\\\{\}]
\PYG{p}{(}\PYG{n}{b} \PYG{o}{\PYGZhy{}} \PYG{n}{a}\PYG{p}{)} \PYG{o}{*} \PYG{n}{random\PYGZus{}sample}\PYG{p}{(}\PYG{p}{)} \PYG{o}{+} \PYG{n}{a}
\end{sphinxVerbatim}

\begin{sphinxadmonition}{note}{Note:}
\sphinxAtStartPar
New code should use the \sphinxtitleref{\textasciitilde{}numpy.random.Generator.random}
method of a \sphinxtitleref{\textasciitilde{}numpy.random.Generator} instance instead;
please see the \DUrole{xref,std,std-ref}{random\sphinxhyphen{}quick\sphinxhyphen{}start}.
\end{sphinxadmonition}
\begin{quote}\begin{description}
\sphinxlineitem{Parameters}
\sphinxAtStartPar
\sphinxstyleliteralstrong{\sphinxupquote{size}} (\sphinxstyleliteralemphasis{\sphinxupquote{int}}\sphinxstyleliteralemphasis{\sphinxupquote{ or }}\sphinxstyleliteralemphasis{\sphinxupquote{tuple}}\sphinxstyleliteralemphasis{\sphinxupquote{ of }}\sphinxstyleliteralemphasis{\sphinxupquote{ints}}\sphinxstyleliteralemphasis{\sphinxupquote{, }}\sphinxstyleliteralemphasis{\sphinxupquote{optional}}) \textendash{} Output shape.  If the given shape is, e.g., \sphinxcode{\sphinxupquote{(m, n, k)}}, then
\sphinxcode{\sphinxupquote{m * n * k}} samples are drawn.  Default is None, in which case a
single value is returned.

\sphinxlineitem{Returns}
\sphinxAtStartPar
\sphinxstylestrong{out} \textendash{} Array of random floats of shape \sphinxtitleref{size} (unless \sphinxcode{\sphinxupquote{size=None}}, in which
case a single float is returned).

\sphinxlineitem{Return type}
\sphinxAtStartPar
float or ndarray of floats

\end{description}\end{quote}


\begin{sphinxseealso}{See also:}
\begin{description}
\sphinxlineitem{\sphinxcode{\sphinxupquote{random.Generator.random}}}
\sphinxAtStartPar
which should be used for new code.

\end{description}


\end{sphinxseealso}

\subsubsection*{Examples}

\begin{sphinxVerbatim}[commandchars=\\\{\}]
\PYG{g+gp}{\PYGZgt{}\PYGZgt{}\PYGZgt{} }\PYG{n}{np}\PYG{o}{.}\PYG{n}{random}\PYG{o}{.}\PYG{n}{random\PYGZus{}sample}\PYG{p}{(}\PYG{p}{)}
\PYG{g+go}{0.47108547995356098 \PYGZsh{} random}
\PYG{g+gp}{\PYGZgt{}\PYGZgt{}\PYGZgt{} }\PYG{n+nb}{type}\PYG{p}{(}\PYG{n}{np}\PYG{o}{.}\PYG{n}{random}\PYG{o}{.}\PYG{n}{random\PYGZus{}sample}\PYG{p}{(}\PYG{p}{)}\PYG{p}{)}
\PYG{g+go}{\PYGZlt{}class \PYGZsq{}float\PYGZsq{}\PYGZgt{}}
\PYG{g+gp}{\PYGZgt{}\PYGZgt{}\PYGZgt{} }\PYG{n}{np}\PYG{o}{.}\PYG{n}{random}\PYG{o}{.}\PYG{n}{random\PYGZus{}sample}\PYG{p}{(}\PYG{p}{(}\PYG{l+m+mi}{5}\PYG{p}{,}\PYG{p}{)}\PYG{p}{)}
\PYG{g+go}{array([ 0.30220482,  0.86820401,  0.1654503 ,  0.11659149,  0.54323428]) \PYGZsh{} random}
\end{sphinxVerbatim}

\sphinxAtStartPar
Three\sphinxhyphen{}by\sphinxhyphen{}two array of random numbers from {[}\sphinxhyphen{}5, 0):

\begin{sphinxVerbatim}[commandchars=\\\{\}]
\PYG{g+gp}{\PYGZgt{}\PYGZgt{}\PYGZgt{} }\PYG{l+m+mi}{5} \PYG{o}{*} \PYG{n}{np}\PYG{o}{.}\PYG{n}{random}\PYG{o}{.}\PYG{n}{random\PYGZus{}sample}\PYG{p}{(}\PYG{p}{(}\PYG{l+m+mi}{3}\PYG{p}{,} \PYG{l+m+mi}{2}\PYG{p}{)}\PYG{p}{)} \PYG{o}{\PYGZhy{}} \PYG{l+m+mi}{5}
\PYG{g+go}{array([[\PYGZhy{}3.99149989, \PYGZhy{}0.52338984], \PYGZsh{} random}
\PYG{g+go}{       [\PYGZhy{}2.99091858, \PYGZhy{}0.79479508],}
\PYG{g+go}{       [\PYGZhy{}1.23204345, \PYGZhy{}1.75224494]])}
\end{sphinxVerbatim}

\end{fulllineitems}

\index{rayleigh() (in module metilda.controllers.pitch\_art\_wizard)@\spxentry{rayleigh()}\spxextra{in module metilda.controllers.pitch\_art\_wizard}}

\begin{fulllineitems}
\phantomsection\label{\detokenize{metilda.controllers:metilda.controllers.pitch_art_wizard.rayleigh}}
\pysigstartsignatures
\pysiglinewithargsret{\sphinxcode{\sphinxupquote{metilda.controllers.pitch\_art\_wizard.}}\sphinxbfcode{\sphinxupquote{rayleigh}}}{\sphinxparam{\DUrole{n,n}{scale}\DUrole{o,o}{=}\DUrole{default_value}{1.0}}\sphinxparamcomma \sphinxparam{\DUrole{n,n}{size}\DUrole{o,o}{=}\DUrole{default_value}{None}}}{}
\pysigstopsignatures
\sphinxAtStartPar
Draw samples from a Rayleigh distribution.

\sphinxAtStartPar
The \(\chi\) and Weibull distributions are generalizations of the
Rayleigh.

\begin{sphinxadmonition}{note}{Note:}
\sphinxAtStartPar
New code should use the \sphinxtitleref{\textasciitilde{}numpy.random.Generator.rayleigh}
method of a \sphinxtitleref{\textasciitilde{}numpy.random.Generator} instance instead;
please see the \DUrole{xref,std,std-ref}{random\sphinxhyphen{}quick\sphinxhyphen{}start}.
\end{sphinxadmonition}
\begin{quote}\begin{description}
\sphinxlineitem{Parameters}\begin{itemize}
\item {} 
\sphinxAtStartPar
\sphinxstyleliteralstrong{\sphinxupquote{scale}} (\sphinxstyleliteralemphasis{\sphinxupquote{float}}\sphinxstyleliteralemphasis{\sphinxupquote{ or }}\sphinxstyleliteralemphasis{\sphinxupquote{array\_like}}\sphinxstyleliteralemphasis{\sphinxupquote{ of }}\sphinxstyleliteralemphasis{\sphinxupquote{floats}}\sphinxstyleliteralemphasis{\sphinxupquote{, }}\sphinxstyleliteralemphasis{\sphinxupquote{optional}}) \textendash{} Scale, also equals the mode. Must be non\sphinxhyphen{}negative. Default is 1.

\item {} 
\sphinxAtStartPar
\sphinxstyleliteralstrong{\sphinxupquote{size}} (\sphinxstyleliteralemphasis{\sphinxupquote{int}}\sphinxstyleliteralemphasis{\sphinxupquote{ or }}\sphinxstyleliteralemphasis{\sphinxupquote{tuple}}\sphinxstyleliteralemphasis{\sphinxupquote{ of }}\sphinxstyleliteralemphasis{\sphinxupquote{ints}}\sphinxstyleliteralemphasis{\sphinxupquote{, }}\sphinxstyleliteralemphasis{\sphinxupquote{optional}}) \textendash{} Output shape.  If the given shape is, e.g., \sphinxcode{\sphinxupquote{(m, n, k)}}, then
\sphinxcode{\sphinxupquote{m * n * k}} samples are drawn.  If size is \sphinxcode{\sphinxupquote{None}} (default),
a single value is returned if \sphinxcode{\sphinxupquote{scale}} is a scalar.  Otherwise,
\sphinxcode{\sphinxupquote{np.array(scale).size}} samples are drawn.

\end{itemize}

\sphinxlineitem{Returns}
\sphinxAtStartPar
\sphinxstylestrong{out} \textendash{} Drawn samples from the parameterized Rayleigh distribution.

\sphinxlineitem{Return type}
\sphinxAtStartPar
ndarray or scalar

\end{description}\end{quote}


\begin{sphinxseealso}{See also:}
\begin{description}
\sphinxlineitem{\sphinxcode{\sphinxupquote{random.Generator.rayleigh}}}
\sphinxAtStartPar
which should be used for new code.

\end{description}


\end{sphinxseealso}

\subsubsection*{Notes}

\sphinxAtStartPar
The probability density function for the Rayleigh distribution is
\begin{equation*}
\begin{split}P(x;scale) = \frac{x}{scale^2}e^{\frac{-x^2}{2 \cdotp scale^2}}\end{split}
\end{equation*}
\sphinxAtStartPar
The Rayleigh distribution would arise, for example, if the East
and North components of the wind velocity had identical zero\sphinxhyphen{}mean
Gaussian distributions.  Then the wind speed would have a Rayleigh
distribution.
\subsubsection*{References}
\subsubsection*{Examples}

\sphinxAtStartPar
Draw values from the distribution and plot the histogram

\begin{sphinxVerbatim}[commandchars=\\\{\}]
\PYG{g+gp}{\PYGZgt{}\PYGZgt{}\PYGZgt{} }\PYG{k+kn}{from} \PYG{n+nn}{matplotlib}\PYG{n+nn}{.}\PYG{n+nn}{pyplot} \PYG{k+kn}{import} \PYG{n}{hist}
\PYG{g+gp}{\PYGZgt{}\PYGZgt{}\PYGZgt{} }\PYG{n}{values} \PYG{o}{=} \PYG{n}{hist}\PYG{p}{(}\PYG{n}{np}\PYG{o}{.}\PYG{n}{random}\PYG{o}{.}\PYG{n}{rayleigh}\PYG{p}{(}\PYG{l+m+mi}{3}\PYG{p}{,} \PYG{l+m+mi}{100000}\PYG{p}{)}\PYG{p}{,} \PYG{n}{bins}\PYG{o}{=}\PYG{l+m+mi}{200}\PYG{p}{,} \PYG{n}{density}\PYG{o}{=}\PYG{k+kc}{True}\PYG{p}{)}
\end{sphinxVerbatim}

\sphinxAtStartPar
Wave heights tend to follow a Rayleigh distribution. If the mean wave
height is 1 meter, what fraction of waves are likely to be larger than 3
meters?

\begin{sphinxVerbatim}[commandchars=\\\{\}]
\PYG{g+gp}{\PYGZgt{}\PYGZgt{}\PYGZgt{} }\PYG{n}{meanvalue} \PYG{o}{=} \PYG{l+m+mi}{1}
\PYG{g+gp}{\PYGZgt{}\PYGZgt{}\PYGZgt{} }\PYG{n}{modevalue} \PYG{o}{=} \PYG{n}{np}\PYG{o}{.}\PYG{n}{sqrt}\PYG{p}{(}\PYG{l+m+mi}{2} \PYG{o}{/} \PYG{n}{np}\PYG{o}{.}\PYG{n}{pi}\PYG{p}{)} \PYG{o}{*} \PYG{n}{meanvalue}
\PYG{g+gp}{\PYGZgt{}\PYGZgt{}\PYGZgt{} }\PYG{n}{s} \PYG{o}{=} \PYG{n}{np}\PYG{o}{.}\PYG{n}{random}\PYG{o}{.}\PYG{n}{rayleigh}\PYG{p}{(}\PYG{n}{modevalue}\PYG{p}{,} \PYG{l+m+mi}{1000000}\PYG{p}{)}
\end{sphinxVerbatim}

\sphinxAtStartPar
The percentage of waves larger than 3 meters is:

\begin{sphinxVerbatim}[commandchars=\\\{\}]
\PYG{g+gp}{\PYGZgt{}\PYGZgt{}\PYGZgt{} }\PYG{l+m+mf}{100.}\PYG{o}{*}\PYG{n+nb}{sum}\PYG{p}{(}\PYG{n}{s}\PYG{o}{\PYGZgt{}}\PYG{l+m+mi}{3}\PYG{p}{)}\PYG{o}{/}\PYG{l+m+mf}{1000000.}
\PYG{g+go}{0.087300000000000003 \PYGZsh{} random}
\end{sphinxVerbatim}

\end{fulllineitems}

\index{set\_state() (in module metilda.controllers.pitch\_art\_wizard)@\spxentry{set\_state()}\spxextra{in module metilda.controllers.pitch\_art\_wizard}}

\begin{fulllineitems}
\phantomsection\label{\detokenize{metilda.controllers:metilda.controllers.pitch_art_wizard.set_state}}
\pysigstartsignatures
\pysiglinewithargsret{\sphinxcode{\sphinxupquote{metilda.controllers.pitch\_art\_wizard.}}\sphinxbfcode{\sphinxupquote{set\_state}}}{\sphinxparam{\DUrole{n,n}{state}}}{}
\pysigstopsignatures
\sphinxAtStartPar
Set the internal state of the generator from a tuple.

\sphinxAtStartPar
For use if one has reason to manually (re\sphinxhyphen{})set the internal state of
the bit generator used by the RandomState instance. By default,
RandomState uses the “Mersenne Twister”{\color{red}\bfseries{}{[}1{]}\_} pseudo\sphinxhyphen{}random number
generating algorithm.
\begin{quote}\begin{description}
\sphinxlineitem{Parameters}
\sphinxAtStartPar
\sphinxstyleliteralstrong{\sphinxupquote{state}} (\sphinxstyleliteralemphasis{\sphinxupquote{\{tuple}}\sphinxstyleliteralemphasis{\sphinxupquote{(}}\sphinxstyleliteralemphasis{\sphinxupquote{str}}\sphinxstyleliteralemphasis{\sphinxupquote{, }}\sphinxstyleliteralemphasis{\sphinxupquote{ndarray}}\sphinxstyleliteralemphasis{\sphinxupquote{ of }}\sphinxstyleliteralemphasis{\sphinxupquote{624 uints}}\sphinxstyleliteralemphasis{\sphinxupquote{, }}\sphinxstyleliteralemphasis{\sphinxupquote{int}}\sphinxstyleliteralemphasis{\sphinxupquote{, }}\sphinxstyleliteralemphasis{\sphinxupquote{int}}\sphinxstyleliteralemphasis{\sphinxupquote{, }}\sphinxstyleliteralemphasis{\sphinxupquote{float}}\sphinxstyleliteralemphasis{\sphinxupquote{)}}\sphinxstyleliteralemphasis{\sphinxupquote{, }}\sphinxstyleliteralemphasis{\sphinxupquote{dict\}}}) \textendash{} 
\sphinxAtStartPar
The \sphinxtitleref{state} tuple has the following items:
\begin{enumerate}
\sphinxsetlistlabels{\arabic}{enumi}{enumii}{}{.}%
\item {} 
\sphinxAtStartPar
the string ‘MT19937’, specifying the Mersenne Twister algorithm.

\item {} 
\sphinxAtStartPar
a 1\sphinxhyphen{}D array of 624 unsigned integers \sphinxcode{\sphinxupquote{keys}}.

\item {} 
\sphinxAtStartPar
an integer \sphinxcode{\sphinxupquote{pos}}.

\item {} 
\sphinxAtStartPar
an integer \sphinxcode{\sphinxupquote{has\_gauss}}.

\item {} 
\sphinxAtStartPar
a float \sphinxcode{\sphinxupquote{cached\_gaussian}}.

\end{enumerate}

\sphinxAtStartPar
If state is a dictionary, it is directly set using the BitGenerators
\sphinxtitleref{state} property.


\sphinxlineitem{Returns}
\sphinxAtStartPar
\sphinxstylestrong{out} \textendash{} Returns ‘None’ on success.

\sphinxlineitem{Return type}
\sphinxAtStartPar
None

\end{description}\end{quote}


\begin{sphinxseealso}{See also:}

\sphinxAtStartPar
{\hyperref[\detokenize{metilda.controllers:metilda.controllers.pitch_art_wizard.get_state}]{\sphinxcrossref{\sphinxcode{\sphinxupquote{get\_state}}}}}


\end{sphinxseealso}

\subsubsection*{Notes}

\sphinxAtStartPar
\sphinxtitleref{set\_state} and \sphinxtitleref{get\_state} are not needed to work with any of the
random distributions in NumPy. If the internal state is manually altered,
the user should know exactly what he/she is doing.

\sphinxAtStartPar
For backwards compatibility, the form (str, array of 624 uints, int) is
also accepted although it is missing some information about the cached
Gaussian value: \sphinxcode{\sphinxupquote{state = (\textquotesingle{}MT19937\textquotesingle{}, keys, pos)}}.
\subsubsection*{References}

\end{fulllineitems}

\index{share\_file() (in module metilda.controllers.pitch\_art\_wizard)@\spxentry{share\_file()}\spxextra{in module metilda.controllers.pitch\_art\_wizard}}

\begin{fulllineitems}
\phantomsection\label{\detokenize{metilda.controllers:metilda.controllers.pitch_art_wizard.share_file}}
\pysigstartsignatures
\pysiglinewithargsret{\sphinxcode{\sphinxupquote{metilda.controllers.pitch\_art\_wizard.}}\sphinxbfcode{\sphinxupquote{share\_file}}}{}{}
\pysigstopsignatures
\end{fulllineitems}

\index{shuffle() (in module metilda.controllers.pitch\_art\_wizard)@\spxentry{shuffle()}\spxextra{in module metilda.controllers.pitch\_art\_wizard}}

\begin{fulllineitems}
\phantomsection\label{\detokenize{metilda.controllers:metilda.controllers.pitch_art_wizard.shuffle}}
\pysigstartsignatures
\pysiglinewithargsret{\sphinxcode{\sphinxupquote{metilda.controllers.pitch\_art\_wizard.}}\sphinxbfcode{\sphinxupquote{shuffle}}}{\sphinxparam{\DUrole{n,n}{x}}}{}
\pysigstopsignatures
\sphinxAtStartPar
Modify a sequence in\sphinxhyphen{}place by shuffling its contents.

\sphinxAtStartPar
This function only shuffles the array along the first axis of a
multi\sphinxhyphen{}dimensional array. The order of sub\sphinxhyphen{}arrays is changed but
their contents remains the same.

\begin{sphinxadmonition}{note}{Note:}
\sphinxAtStartPar
New code should use the \sphinxtitleref{\textasciitilde{}numpy.random.Generator.shuffle}
method of a \sphinxtitleref{\textasciitilde{}numpy.random.Generator} instance instead;
please see the \DUrole{xref,std,std-ref}{random\sphinxhyphen{}quick\sphinxhyphen{}start}.
\end{sphinxadmonition}
\begin{quote}\begin{description}
\sphinxlineitem{Parameters}
\sphinxAtStartPar
\sphinxstyleliteralstrong{\sphinxupquote{x}} (\sphinxstyleliteralemphasis{\sphinxupquote{ndarray}}\sphinxstyleliteralemphasis{\sphinxupquote{ or }}\sphinxstyleliteralemphasis{\sphinxupquote{MutableSequence}}) \textendash{} The array, list or mutable sequence to be shuffled.

\sphinxlineitem{Return type}
\sphinxAtStartPar
None

\end{description}\end{quote}


\begin{sphinxseealso}{See also:}
\begin{description}
\sphinxlineitem{\sphinxcode{\sphinxupquote{random.Generator.shuffle}}}
\sphinxAtStartPar
which should be used for new code.

\end{description}


\end{sphinxseealso}

\subsubsection*{Examples}

\begin{sphinxVerbatim}[commandchars=\\\{\}]
\PYG{g+gp}{\PYGZgt{}\PYGZgt{}\PYGZgt{} }\PYG{n}{arr} \PYG{o}{=} \PYG{n}{np}\PYG{o}{.}\PYG{n}{arange}\PYG{p}{(}\PYG{l+m+mi}{10}\PYG{p}{)}
\PYG{g+gp}{\PYGZgt{}\PYGZgt{}\PYGZgt{} }\PYG{n}{np}\PYG{o}{.}\PYG{n}{random}\PYG{o}{.}\PYG{n}{shuffle}\PYG{p}{(}\PYG{n}{arr}\PYG{p}{)}
\PYG{g+gp}{\PYGZgt{}\PYGZgt{}\PYGZgt{} }\PYG{n}{arr}
\PYG{g+go}{[1 7 5 2 9 4 3 6 0 8] \PYGZsh{} random}
\end{sphinxVerbatim}

\sphinxAtStartPar
Multi\sphinxhyphen{}dimensional arrays are only shuffled along the first axis:

\begin{sphinxVerbatim}[commandchars=\\\{\}]
\PYG{g+gp}{\PYGZgt{}\PYGZgt{}\PYGZgt{} }\PYG{n}{arr} \PYG{o}{=} \PYG{n}{np}\PYG{o}{.}\PYG{n}{arange}\PYG{p}{(}\PYG{l+m+mi}{9}\PYG{p}{)}\PYG{o}{.}\PYG{n}{reshape}\PYG{p}{(}\PYG{p}{(}\PYG{l+m+mi}{3}\PYG{p}{,} \PYG{l+m+mi}{3}\PYG{p}{)}\PYG{p}{)}
\PYG{g+gp}{\PYGZgt{}\PYGZgt{}\PYGZgt{} }\PYG{n}{np}\PYG{o}{.}\PYG{n}{random}\PYG{o}{.}\PYG{n}{shuffle}\PYG{p}{(}\PYG{n}{arr}\PYG{p}{)}
\PYG{g+gp}{\PYGZgt{}\PYGZgt{}\PYGZgt{} }\PYG{n}{arr}
\PYG{g+go}{array([[3, 4, 5], \PYGZsh{} random}
\PYG{g+go}{       [6, 7, 8],}
\PYG{g+go}{       [0, 1, 2]])}
\end{sphinxVerbatim}

\end{fulllineitems}

\index{sound\_length() (in module metilda.controllers.pitch\_art\_wizard)@\spxentry{sound\_length()}\spxextra{in module metilda.controllers.pitch\_art\_wizard}}

\begin{fulllineitems}
\phantomsection\label{\detokenize{metilda.controllers:metilda.controllers.pitch_art_wizard.sound_length}}
\pysigstartsignatures
\pysiglinewithargsret{\sphinxcode{\sphinxupquote{metilda.controllers.pitch\_art\_wizard.}}\sphinxbfcode{\sphinxupquote{sound\_length}}}{\sphinxparam{\DUrole{n,n}{upload\_id}}}{}
\pysigstopsignatures
\end{fulllineitems}

\index{spectrumFrequencyBounds() (in module metilda.controllers.pitch\_art\_wizard)@\spxentry{spectrumFrequencyBounds()}\spxextra{in module metilda.controllers.pitch\_art\_wizard}}

\begin{fulllineitems}
\phantomsection\label{\detokenize{metilda.controllers:metilda.controllers.pitch_art_wizard.spectrumFrequencyBounds}}
\pysigstartsignatures
\pysiglinewithargsret{\sphinxcode{\sphinxupquote{metilda.controllers.pitch\_art\_wizard.}}\sphinxbfcode{\sphinxupquote{spectrumFrequencyBounds}}}{\sphinxparam{\DUrole{n,n}{sound}}}{}
\pysigstopsignatures
\end{fulllineitems}

\index{standard\_cauchy() (in module metilda.controllers.pitch\_art\_wizard)@\spxentry{standard\_cauchy()}\spxextra{in module metilda.controllers.pitch\_art\_wizard}}

\begin{fulllineitems}
\phantomsection\label{\detokenize{metilda.controllers:metilda.controllers.pitch_art_wizard.standard_cauchy}}
\pysigstartsignatures
\pysiglinewithargsret{\sphinxcode{\sphinxupquote{metilda.controllers.pitch\_art\_wizard.}}\sphinxbfcode{\sphinxupquote{standard\_cauchy}}}{\sphinxparam{\DUrole{n,n}{size}\DUrole{o,o}{=}\DUrole{default_value}{None}}}{}
\pysigstopsignatures
\sphinxAtStartPar
Draw samples from a standard Cauchy distribution with mode = 0.

\sphinxAtStartPar
Also known as the Lorentz distribution.

\begin{sphinxadmonition}{note}{Note:}
\sphinxAtStartPar
New code should use the
\sphinxtitleref{\textasciitilde{}numpy.random.Generator.standard\_cauchy}
method of a \sphinxtitleref{\textasciitilde{}numpy.random.Generator} instance instead;
please see the \DUrole{xref,std,std-ref}{random\sphinxhyphen{}quick\sphinxhyphen{}start}.
\end{sphinxadmonition}
\begin{quote}\begin{description}
\sphinxlineitem{Parameters}
\sphinxAtStartPar
\sphinxstyleliteralstrong{\sphinxupquote{size}} (\sphinxstyleliteralemphasis{\sphinxupquote{int}}\sphinxstyleliteralemphasis{\sphinxupquote{ or }}\sphinxstyleliteralemphasis{\sphinxupquote{tuple}}\sphinxstyleliteralemphasis{\sphinxupquote{ of }}\sphinxstyleliteralemphasis{\sphinxupquote{ints}}\sphinxstyleliteralemphasis{\sphinxupquote{, }}\sphinxstyleliteralemphasis{\sphinxupquote{optional}}) \textendash{} Output shape.  If the given shape is, e.g., \sphinxcode{\sphinxupquote{(m, n, k)}}, then
\sphinxcode{\sphinxupquote{m * n * k}} samples are drawn.  Default is None, in which case a
single value is returned.

\sphinxlineitem{Returns}
\sphinxAtStartPar
\sphinxstylestrong{samples} \textendash{} The drawn samples.

\sphinxlineitem{Return type}
\sphinxAtStartPar
ndarray or scalar

\end{description}\end{quote}


\begin{sphinxseealso}{See also:}
\begin{description}
\sphinxlineitem{\sphinxcode{\sphinxupquote{random.Generator.standard\_cauchy}}}
\sphinxAtStartPar
which should be used for new code.

\end{description}


\end{sphinxseealso}

\subsubsection*{Notes}

\sphinxAtStartPar
The probability density function for the full Cauchy distribution is
\begin{equation*}
\begin{split}P(x; x_0, \gamma) = \frac{1}{\pi \gamma \bigl[ 1+
(\frac{x-x_0}{\gamma})^2 \bigr] }\end{split}
\end{equation*}
\sphinxAtStartPar
and the Standard Cauchy distribution just sets \(x_0=0\) and
\(\gamma=1\)

\sphinxAtStartPar
The Cauchy distribution arises in the solution to the driven harmonic
oscillator problem, and also describes spectral line broadening. It
also describes the distribution of values at which a line tilted at
a random angle will cut the x axis.

\sphinxAtStartPar
When studying hypothesis tests that assume normality, seeing how the
tests perform on data from a Cauchy distribution is a good indicator of
their sensitivity to a heavy\sphinxhyphen{}tailed distribution, since the Cauchy looks
very much like a Gaussian distribution, but with heavier tails.
\subsubsection*{References}
\subsubsection*{Examples}

\sphinxAtStartPar
Draw samples and plot the distribution:

\begin{sphinxVerbatim}[commandchars=\\\{\}]
\PYG{g+gp}{\PYGZgt{}\PYGZgt{}\PYGZgt{} }\PYG{k+kn}{import} \PYG{n+nn}{matplotlib}\PYG{n+nn}{.}\PYG{n+nn}{pyplot} \PYG{k}{as} \PYG{n+nn}{plt}
\PYG{g+gp}{\PYGZgt{}\PYGZgt{}\PYGZgt{} }\PYG{n}{s} \PYG{o}{=} \PYG{n}{np}\PYG{o}{.}\PYG{n}{random}\PYG{o}{.}\PYG{n}{standard\PYGZus{}cauchy}\PYG{p}{(}\PYG{l+m+mi}{1000000}\PYG{p}{)}
\PYG{g+gp}{\PYGZgt{}\PYGZgt{}\PYGZgt{} }\PYG{n}{s} \PYG{o}{=} \PYG{n}{s}\PYG{p}{[}\PYG{p}{(}\PYG{n}{s}\PYG{o}{\PYGZgt{}}\PYG{o}{\PYGZhy{}}\PYG{l+m+mi}{25}\PYG{p}{)} \PYG{o}{\PYGZam{}} \PYG{p}{(}\PYG{n}{s}\PYG{o}{\PYGZlt{}}\PYG{l+m+mi}{25}\PYG{p}{)}\PYG{p}{]}  \PYG{c+c1}{\PYGZsh{} truncate distribution so it plots well}
\PYG{g+gp}{\PYGZgt{}\PYGZgt{}\PYGZgt{} }\PYG{n}{plt}\PYG{o}{.}\PYG{n}{hist}\PYG{p}{(}\PYG{n}{s}\PYG{p}{,} \PYG{n}{bins}\PYG{o}{=}\PYG{l+m+mi}{100}\PYG{p}{)}
\PYG{g+gp}{\PYGZgt{}\PYGZgt{}\PYGZgt{} }\PYG{n}{plt}\PYG{o}{.}\PYG{n}{show}\PYG{p}{(}\PYG{p}{)}
\end{sphinxVerbatim}

\end{fulllineitems}

\index{standard\_exponential() (in module metilda.controllers.pitch\_art\_wizard)@\spxentry{standard\_exponential()}\spxextra{in module metilda.controllers.pitch\_art\_wizard}}

\begin{fulllineitems}
\phantomsection\label{\detokenize{metilda.controllers:metilda.controllers.pitch_art_wizard.standard_exponential}}
\pysigstartsignatures
\pysiglinewithargsret{\sphinxcode{\sphinxupquote{metilda.controllers.pitch\_art\_wizard.}}\sphinxbfcode{\sphinxupquote{standard\_exponential}}}{\sphinxparam{\DUrole{n,n}{size}\DUrole{o,o}{=}\DUrole{default_value}{None}}}{}
\pysigstopsignatures
\sphinxAtStartPar
Draw samples from the standard exponential distribution.

\sphinxAtStartPar
\sphinxtitleref{standard\_exponential} is identical to the exponential distribution
with a scale parameter of 1.

\begin{sphinxadmonition}{note}{Note:}
\sphinxAtStartPar
New code should use the
\sphinxtitleref{\textasciitilde{}numpy.random.Generator.standard\_exponential}
method of a \sphinxtitleref{\textasciitilde{}numpy.random.Generator} instance instead;
please see the \DUrole{xref,std,std-ref}{random\sphinxhyphen{}quick\sphinxhyphen{}start}.
\end{sphinxadmonition}
\begin{quote}\begin{description}
\sphinxlineitem{Parameters}
\sphinxAtStartPar
\sphinxstyleliteralstrong{\sphinxupquote{size}} (\sphinxstyleliteralemphasis{\sphinxupquote{int}}\sphinxstyleliteralemphasis{\sphinxupquote{ or }}\sphinxstyleliteralemphasis{\sphinxupquote{tuple}}\sphinxstyleliteralemphasis{\sphinxupquote{ of }}\sphinxstyleliteralemphasis{\sphinxupquote{ints}}\sphinxstyleliteralemphasis{\sphinxupquote{, }}\sphinxstyleliteralemphasis{\sphinxupquote{optional}}) \textendash{} Output shape.  If the given shape is, e.g., \sphinxcode{\sphinxupquote{(m, n, k)}}, then
\sphinxcode{\sphinxupquote{m * n * k}} samples are drawn.  Default is None, in which case a
single value is returned.

\sphinxlineitem{Returns}
\sphinxAtStartPar
\sphinxstylestrong{out} \textendash{} Drawn samples.

\sphinxlineitem{Return type}
\sphinxAtStartPar
float or ndarray

\end{description}\end{quote}


\begin{sphinxseealso}{See also:}
\begin{description}
\sphinxlineitem{\sphinxcode{\sphinxupquote{random.Generator.standard\_exponential}}}
\sphinxAtStartPar
which should be used for new code.

\end{description}


\end{sphinxseealso}

\subsubsection*{Examples}

\sphinxAtStartPar
Output a 3x8000 array:

\begin{sphinxVerbatim}[commandchars=\\\{\}]
\PYG{g+gp}{\PYGZgt{}\PYGZgt{}\PYGZgt{} }\PYG{n}{n} \PYG{o}{=} \PYG{n}{np}\PYG{o}{.}\PYG{n}{random}\PYG{o}{.}\PYG{n}{standard\PYGZus{}exponential}\PYG{p}{(}\PYG{p}{(}\PYG{l+m+mi}{3}\PYG{p}{,} \PYG{l+m+mi}{8000}\PYG{p}{)}\PYG{p}{)}
\end{sphinxVerbatim}

\end{fulllineitems}

\index{standard\_gamma() (in module metilda.controllers.pitch\_art\_wizard)@\spxentry{standard\_gamma()}\spxextra{in module metilda.controllers.pitch\_art\_wizard}}

\begin{fulllineitems}
\phantomsection\label{\detokenize{metilda.controllers:metilda.controllers.pitch_art_wizard.standard_gamma}}
\pysigstartsignatures
\pysiglinewithargsret{\sphinxcode{\sphinxupquote{metilda.controllers.pitch\_art\_wizard.}}\sphinxbfcode{\sphinxupquote{standard\_gamma}}}{\sphinxparam{\DUrole{n,n}{shape}}\sphinxparamcomma \sphinxparam{\DUrole{n,n}{size}\DUrole{o,o}{=}\DUrole{default_value}{None}}}{}
\pysigstopsignatures
\sphinxAtStartPar
Draw samples from a standard Gamma distribution.

\sphinxAtStartPar
Samples are drawn from a Gamma distribution with specified parameters,
shape (sometimes designated “k”) and scale=1.

\begin{sphinxadmonition}{note}{Note:}
\sphinxAtStartPar
New code should use the
\sphinxtitleref{\textasciitilde{}numpy.random.Generator.standard\_gamma}
method of a \sphinxtitleref{\textasciitilde{}numpy.random.Generator} instance instead;
please see the \DUrole{xref,std,std-ref}{random\sphinxhyphen{}quick\sphinxhyphen{}start}.
\end{sphinxadmonition}
\begin{quote}\begin{description}
\sphinxlineitem{Parameters}\begin{itemize}
\item {} 
\sphinxAtStartPar
\sphinxstyleliteralstrong{\sphinxupquote{shape}} (\sphinxstyleliteralemphasis{\sphinxupquote{float}}\sphinxstyleliteralemphasis{\sphinxupquote{ or }}\sphinxstyleliteralemphasis{\sphinxupquote{array\_like}}\sphinxstyleliteralemphasis{\sphinxupquote{ of }}\sphinxstyleliteralemphasis{\sphinxupquote{floats}}) \textendash{} Parameter, must be non\sphinxhyphen{}negative.

\item {} 
\sphinxAtStartPar
\sphinxstyleliteralstrong{\sphinxupquote{size}} (\sphinxstyleliteralemphasis{\sphinxupquote{int}}\sphinxstyleliteralemphasis{\sphinxupquote{ or }}\sphinxstyleliteralemphasis{\sphinxupquote{tuple}}\sphinxstyleliteralemphasis{\sphinxupquote{ of }}\sphinxstyleliteralemphasis{\sphinxupquote{ints}}\sphinxstyleliteralemphasis{\sphinxupquote{, }}\sphinxstyleliteralemphasis{\sphinxupquote{optional}}) \textendash{} Output shape.  If the given shape is, e.g., \sphinxcode{\sphinxupquote{(m, n, k)}}, then
\sphinxcode{\sphinxupquote{m * n * k}} samples are drawn.  If size is \sphinxcode{\sphinxupquote{None}} (default),
a single value is returned if \sphinxcode{\sphinxupquote{shape}} is a scalar.  Otherwise,
\sphinxcode{\sphinxupquote{np.array(shape).size}} samples are drawn.

\end{itemize}

\sphinxlineitem{Returns}
\sphinxAtStartPar
\sphinxstylestrong{out} \textendash{} Drawn samples from the parameterized standard gamma distribution.

\sphinxlineitem{Return type}
\sphinxAtStartPar
ndarray or scalar

\end{description}\end{quote}


\begin{sphinxseealso}{See also:}
\begin{description}
\sphinxlineitem{\sphinxcode{\sphinxupquote{scipy.stats.gamma}}}
\sphinxAtStartPar
probability density function, distribution or cumulative density function, etc.

\sphinxlineitem{\sphinxcode{\sphinxupquote{random.Generator.standard\_gamma}}}
\sphinxAtStartPar
which should be used for new code.

\end{description}


\end{sphinxseealso}

\subsubsection*{Notes}

\sphinxAtStartPar
The probability density for the Gamma distribution is
\begin{equation*}
\begin{split}p(x) = x^{k-1}\frac{e^{-x/\theta}}{\theta^k\Gamma(k)},\end{split}
\end{equation*}
\sphinxAtStartPar
where \(k\) is the shape and \(\theta\) the scale,
and \(\Gamma\) is the Gamma function.

\sphinxAtStartPar
The Gamma distribution is often used to model the times to failure of
electronic components, and arises naturally in processes for which the
waiting times between Poisson distributed events are relevant.
\subsubsection*{References}
\subsubsection*{Examples}

\sphinxAtStartPar
Draw samples from the distribution:

\begin{sphinxVerbatim}[commandchars=\\\{\}]
\PYG{g+gp}{\PYGZgt{}\PYGZgt{}\PYGZgt{} }\PYG{n}{shape}\PYG{p}{,} \PYG{n}{scale} \PYG{o}{=} \PYG{l+m+mf}{2.}\PYG{p}{,} \PYG{l+m+mf}{1.} \PYG{c+c1}{\PYGZsh{} mean and width}
\PYG{g+gp}{\PYGZgt{}\PYGZgt{}\PYGZgt{} }\PYG{n}{s} \PYG{o}{=} \PYG{n}{np}\PYG{o}{.}\PYG{n}{random}\PYG{o}{.}\PYG{n}{standard\PYGZus{}gamma}\PYG{p}{(}\PYG{n}{shape}\PYG{p}{,} \PYG{l+m+mi}{1000000}\PYG{p}{)}
\end{sphinxVerbatim}

\sphinxAtStartPar
Display the histogram of the samples, along with
the probability density function:

\begin{sphinxVerbatim}[commandchars=\\\{\}]
\PYG{g+gp}{\PYGZgt{}\PYGZgt{}\PYGZgt{} }\PYG{k+kn}{import} \PYG{n+nn}{matplotlib}\PYG{n+nn}{.}\PYG{n+nn}{pyplot} \PYG{k}{as} \PYG{n+nn}{plt}
\PYG{g+gp}{\PYGZgt{}\PYGZgt{}\PYGZgt{} }\PYG{k+kn}{import} \PYG{n+nn}{scipy}\PYG{n+nn}{.}\PYG{n+nn}{special} \PYG{k}{as} \PYG{n+nn}{sps}  
\PYG{g+gp}{\PYGZgt{}\PYGZgt{}\PYGZgt{} }\PYG{n}{count}\PYG{p}{,} \PYG{n}{bins}\PYG{p}{,} \PYG{n}{ignored} \PYG{o}{=} \PYG{n}{plt}\PYG{o}{.}\PYG{n}{hist}\PYG{p}{(}\PYG{n}{s}\PYG{p}{,} \PYG{l+m+mi}{50}\PYG{p}{,} \PYG{n}{density}\PYG{o}{=}\PYG{k+kc}{True}\PYG{p}{)}
\PYG{g+gp}{\PYGZgt{}\PYGZgt{}\PYGZgt{} }\PYG{n}{y} \PYG{o}{=} \PYG{n}{bins}\PYG{o}{*}\PYG{o}{*}\PYG{p}{(}\PYG{n}{shape}\PYG{o}{\PYGZhy{}}\PYG{l+m+mi}{1}\PYG{p}{)} \PYG{o}{*} \PYG{p}{(}\PYG{p}{(}\PYG{n}{np}\PYG{o}{.}\PYG{n}{exp}\PYG{p}{(}\PYG{o}{\PYGZhy{}}\PYG{n}{bins}\PYG{o}{/}\PYG{n}{scale}\PYG{p}{)}\PYG{p}{)}\PYG{o}{/}  
\PYG{g+gp}{... }                      \PYG{p}{(}\PYG{n}{sps}\PYG{o}{.}\PYG{n}{gamma}\PYG{p}{(}\PYG{n}{shape}\PYG{p}{)} \PYG{o}{*} \PYG{n}{scale}\PYG{o}{*}\PYG{o}{*}\PYG{n}{shape}\PYG{p}{)}\PYG{p}{)}
\PYG{g+gp}{\PYGZgt{}\PYGZgt{}\PYGZgt{} }\PYG{n}{plt}\PYG{o}{.}\PYG{n}{plot}\PYG{p}{(}\PYG{n}{bins}\PYG{p}{,} \PYG{n}{y}\PYG{p}{,} \PYG{n}{linewidth}\PYG{o}{=}\PYG{l+m+mi}{2}\PYG{p}{,} \PYG{n}{color}\PYG{o}{=}\PYG{l+s+s1}{\PYGZsq{}}\PYG{l+s+s1}{r}\PYG{l+s+s1}{\PYGZsq{}}\PYG{p}{)}  
\PYG{g+gp}{\PYGZgt{}\PYGZgt{}\PYGZgt{} }\PYG{n}{plt}\PYG{o}{.}\PYG{n}{show}\PYG{p}{(}\PYG{p}{)}
\end{sphinxVerbatim}

\end{fulllineitems}

\index{standard\_normal() (in module metilda.controllers.pitch\_art\_wizard)@\spxentry{standard\_normal()}\spxextra{in module metilda.controllers.pitch\_art\_wizard}}

\begin{fulllineitems}
\phantomsection\label{\detokenize{metilda.controllers:metilda.controllers.pitch_art_wizard.standard_normal}}
\pysigstartsignatures
\pysiglinewithargsret{\sphinxcode{\sphinxupquote{metilda.controllers.pitch\_art\_wizard.}}\sphinxbfcode{\sphinxupquote{standard\_normal}}}{\sphinxparam{\DUrole{n,n}{size}\DUrole{o,o}{=}\DUrole{default_value}{None}}}{}
\pysigstopsignatures
\sphinxAtStartPar
Draw samples from a standard Normal distribution (mean=0, stdev=1).

\begin{sphinxadmonition}{note}{Note:}
\sphinxAtStartPar
New code should use the
\sphinxtitleref{\textasciitilde{}numpy.random.Generator.standard\_normal}
method of a \sphinxtitleref{\textasciitilde{}numpy.random.Generator} instance instead;
please see the \DUrole{xref,std,std-ref}{random\sphinxhyphen{}quick\sphinxhyphen{}start}.
\end{sphinxadmonition}
\begin{quote}\begin{description}
\sphinxlineitem{Parameters}
\sphinxAtStartPar
\sphinxstyleliteralstrong{\sphinxupquote{size}} (\sphinxstyleliteralemphasis{\sphinxupquote{int}}\sphinxstyleliteralemphasis{\sphinxupquote{ or }}\sphinxstyleliteralemphasis{\sphinxupquote{tuple}}\sphinxstyleliteralemphasis{\sphinxupquote{ of }}\sphinxstyleliteralemphasis{\sphinxupquote{ints}}\sphinxstyleliteralemphasis{\sphinxupquote{, }}\sphinxstyleliteralemphasis{\sphinxupquote{optional}}) \textendash{} Output shape.  If the given shape is, e.g., \sphinxcode{\sphinxupquote{(m, n, k)}}, then
\sphinxcode{\sphinxupquote{m * n * k}} samples are drawn.  Default is None, in which case a
single value is returned.

\sphinxlineitem{Returns}
\sphinxAtStartPar
\sphinxstylestrong{out} \textendash{} A floating\sphinxhyphen{}point array of shape \sphinxcode{\sphinxupquote{size}} of drawn samples, or a
single sample if \sphinxcode{\sphinxupquote{size}} was not specified.

\sphinxlineitem{Return type}
\sphinxAtStartPar
float or ndarray

\end{description}\end{quote}


\begin{sphinxseealso}{See also:}
\begin{description}
\sphinxlineitem{{\hyperref[\detokenize{metilda.controllers:metilda.controllers.pitch_art_wizard.normal}]{\sphinxcrossref{\sphinxcode{\sphinxupquote{normal}}}}}}
\sphinxAtStartPar
Equivalent function with additional \sphinxcode{\sphinxupquote{loc}} and \sphinxcode{\sphinxupquote{scale}} arguments for setting the mean and standard deviation.

\sphinxlineitem{\sphinxcode{\sphinxupquote{random.Generator.standard\_normal}}}
\sphinxAtStartPar
which should be used for new code.

\end{description}


\end{sphinxseealso}

\subsubsection*{Notes}

\sphinxAtStartPar
For random samples from the normal distribution with mean \sphinxcode{\sphinxupquote{mu}} and
standard deviation \sphinxcode{\sphinxupquote{sigma}}, use one of:

\begin{sphinxVerbatim}[commandchars=\\\{\}]
\PYG{n}{mu} \PYG{o}{+} \PYG{n}{sigma} \PYG{o}{*} \PYG{n}{np}\PYG{o}{.}\PYG{n}{random}\PYG{o}{.}\PYG{n}{standard\PYGZus{}normal}\PYG{p}{(}\PYG{n}{size}\PYG{o}{=}\PYG{o}{.}\PYG{o}{.}\PYG{o}{.}\PYG{p}{)}
\PYG{n}{np}\PYG{o}{.}\PYG{n}{random}\PYG{o}{.}\PYG{n}{normal}\PYG{p}{(}\PYG{n}{mu}\PYG{p}{,} \PYG{n}{sigma}\PYG{p}{,} \PYG{n}{size}\PYG{o}{=}\PYG{o}{.}\PYG{o}{.}\PYG{o}{.}\PYG{p}{)}
\end{sphinxVerbatim}
\subsubsection*{Examples}

\begin{sphinxVerbatim}[commandchars=\\\{\}]
\PYG{g+gp}{\PYGZgt{}\PYGZgt{}\PYGZgt{} }\PYG{n}{np}\PYG{o}{.}\PYG{n}{random}\PYG{o}{.}\PYG{n}{standard\PYGZus{}normal}\PYG{p}{(}\PYG{p}{)}
\PYG{g+go}{2.1923875335537315 \PYGZsh{}random}
\end{sphinxVerbatim}

\begin{sphinxVerbatim}[commandchars=\\\{\}]
\PYG{g+gp}{\PYGZgt{}\PYGZgt{}\PYGZgt{} }\PYG{n}{s} \PYG{o}{=} \PYG{n}{np}\PYG{o}{.}\PYG{n}{random}\PYG{o}{.}\PYG{n}{standard\PYGZus{}normal}\PYG{p}{(}\PYG{l+m+mi}{8000}\PYG{p}{)}
\PYG{g+gp}{\PYGZgt{}\PYGZgt{}\PYGZgt{} }\PYG{n}{s}
\PYG{g+go}{array([ 0.6888893 ,  0.78096262, \PYGZhy{}0.89086505, ...,  0.49876311,  \PYGZsh{} random}
\PYG{g+go}{       \PYGZhy{}0.38672696, \PYGZhy{}0.4685006 ])                                \PYGZsh{} random}
\PYG{g+gp}{\PYGZgt{}\PYGZgt{}\PYGZgt{} }\PYG{n}{s}\PYG{o}{.}\PYG{n}{shape}
\PYG{g+go}{(8000,)}
\PYG{g+gp}{\PYGZgt{}\PYGZgt{}\PYGZgt{} }\PYG{n}{s} \PYG{o}{=} \PYG{n}{np}\PYG{o}{.}\PYG{n}{random}\PYG{o}{.}\PYG{n}{standard\PYGZus{}normal}\PYG{p}{(}\PYG{n}{size}\PYG{o}{=}\PYG{p}{(}\PYG{l+m+mi}{3}\PYG{p}{,} \PYG{l+m+mi}{4}\PYG{p}{,} \PYG{l+m+mi}{2}\PYG{p}{)}\PYG{p}{)}
\PYG{g+gp}{\PYGZgt{}\PYGZgt{}\PYGZgt{} }\PYG{n}{s}\PYG{o}{.}\PYG{n}{shape}
\PYG{g+go}{(3, 4, 2)}
\end{sphinxVerbatim}

\sphinxAtStartPar
Two\sphinxhyphen{}by\sphinxhyphen{}four array of samples from the normal distribution with
mean 3 and standard deviation 2.5:

\begin{sphinxVerbatim}[commandchars=\\\{\}]
\PYG{g+gp}{\PYGZgt{}\PYGZgt{}\PYGZgt{} }\PYG{l+m+mi}{3} \PYG{o}{+} \PYG{l+m+mf}{2.5} \PYG{o}{*} \PYG{n}{np}\PYG{o}{.}\PYG{n}{random}\PYG{o}{.}\PYG{n}{standard\PYGZus{}normal}\PYG{p}{(}\PYG{n}{size}\PYG{o}{=}\PYG{p}{(}\PYG{l+m+mi}{2}\PYG{p}{,} \PYG{l+m+mi}{4}\PYG{p}{)}\PYG{p}{)}
\PYG{g+go}{array([[\PYGZhy{}4.49401501,  4.00950034, \PYGZhy{}1.81814867,  7.29718677],   \PYGZsh{} random}
\PYG{g+go}{       [ 0.39924804,  4.68456316,  4.99394529,  4.84057254]])  \PYGZsh{} random}
\end{sphinxVerbatim}

\end{fulllineitems}

\index{standard\_t() (in module metilda.controllers.pitch\_art\_wizard)@\spxentry{standard\_t()}\spxextra{in module metilda.controllers.pitch\_art\_wizard}}

\begin{fulllineitems}
\phantomsection\label{\detokenize{metilda.controllers:metilda.controllers.pitch_art_wizard.standard_t}}
\pysigstartsignatures
\pysiglinewithargsret{\sphinxcode{\sphinxupquote{metilda.controllers.pitch\_art\_wizard.}}\sphinxbfcode{\sphinxupquote{standard\_t}}}{\sphinxparam{\DUrole{n,n}{df}}\sphinxparamcomma \sphinxparam{\DUrole{n,n}{size}\DUrole{o,o}{=}\DUrole{default_value}{None}}}{}
\pysigstopsignatures
\sphinxAtStartPar
Draw samples from a standard Student’s t distribution with \sphinxtitleref{df} degrees
of freedom.

\sphinxAtStartPar
A special case of the hyperbolic distribution.  As \sphinxtitleref{df} gets
large, the result resembles that of the standard normal
distribution (\sphinxtitleref{standard\_normal}).

\begin{sphinxadmonition}{note}{Note:}
\sphinxAtStartPar
New code should use the \sphinxtitleref{\textasciitilde{}numpy.random.Generator.standard\_t}
method of a \sphinxtitleref{\textasciitilde{}numpy.random.Generator} instance instead;
please see the \DUrole{xref,std,std-ref}{random\sphinxhyphen{}quick\sphinxhyphen{}start}.
\end{sphinxadmonition}
\begin{quote}\begin{description}
\sphinxlineitem{Parameters}\begin{itemize}
\item {} 
\sphinxAtStartPar
\sphinxstyleliteralstrong{\sphinxupquote{df}} (\sphinxstyleliteralemphasis{\sphinxupquote{float}}\sphinxstyleliteralemphasis{\sphinxupquote{ or }}\sphinxstyleliteralemphasis{\sphinxupquote{array\_like}}\sphinxstyleliteralemphasis{\sphinxupquote{ of }}\sphinxstyleliteralemphasis{\sphinxupquote{floats}}) \textendash{} Degrees of freedom, must be \textgreater{} 0.

\item {} 
\sphinxAtStartPar
\sphinxstyleliteralstrong{\sphinxupquote{size}} (\sphinxstyleliteralemphasis{\sphinxupquote{int}}\sphinxstyleliteralemphasis{\sphinxupquote{ or }}\sphinxstyleliteralemphasis{\sphinxupquote{tuple}}\sphinxstyleliteralemphasis{\sphinxupquote{ of }}\sphinxstyleliteralemphasis{\sphinxupquote{ints}}\sphinxstyleliteralemphasis{\sphinxupquote{, }}\sphinxstyleliteralemphasis{\sphinxupquote{optional}}) \textendash{} Output shape.  If the given shape is, e.g., \sphinxcode{\sphinxupquote{(m, n, k)}}, then
\sphinxcode{\sphinxupquote{m * n * k}} samples are drawn.  If size is \sphinxcode{\sphinxupquote{None}} (default),
a single value is returned if \sphinxcode{\sphinxupquote{df}} is a scalar.  Otherwise,
\sphinxcode{\sphinxupquote{np.array(df).size}} samples are drawn.

\end{itemize}

\sphinxlineitem{Returns}
\sphinxAtStartPar
\sphinxstylestrong{out} \textendash{} Drawn samples from the parameterized standard Student’s t distribution.

\sphinxlineitem{Return type}
\sphinxAtStartPar
ndarray or scalar

\end{description}\end{quote}


\begin{sphinxseealso}{See also:}
\begin{description}
\sphinxlineitem{\sphinxcode{\sphinxupquote{random.Generator.standard\_t}}}
\sphinxAtStartPar
which should be used for new code.

\end{description}


\end{sphinxseealso}

\subsubsection*{Notes}

\sphinxAtStartPar
The probability density function for the t distribution is
\begin{equation*}
\begin{split}P(x, df) = \frac{\Gamma(\frac{df+1}{2})}{\sqrt{\pi df}
\Gamma(\frac{df}{2})}\Bigl( 1+\frac{x^2}{df} \Bigr)^{-(df+1)/2}\end{split}
\end{equation*}
\sphinxAtStartPar
The t test is based on an assumption that the data come from a
Normal distribution. The t test provides a way to test whether
the sample mean (that is the mean calculated from the data) is
a good estimate of the true mean.

\sphinxAtStartPar
The derivation of the t\sphinxhyphen{}distribution was first published in
1908 by William Gosset while working for the Guinness Brewery
in Dublin. Due to proprietary issues, he had to publish under
a pseudonym, and so he used the name Student.
\subsubsection*{References}
\subsubsection*{Examples}

\sphinxAtStartPar
From Dalgaard page 83 {\color{red}\bfseries{}{[}1{]}\_}, suppose the daily energy intake for 11
women in kilojoules (kJ) is:

\begin{sphinxVerbatim}[commandchars=\\\{\}]
\PYG{g+gp}{\PYGZgt{}\PYGZgt{}\PYGZgt{} }\PYG{n}{intake} \PYG{o}{=} \PYG{n}{np}\PYG{o}{.}\PYG{n}{array}\PYG{p}{(}\PYG{p}{[}\PYG{l+m+mf}{5260.}\PYG{p}{,} \PYG{l+m+mi}{5470}\PYG{p}{,} \PYG{l+m+mi}{5640}\PYG{p}{,} \PYG{l+m+mi}{6180}\PYG{p}{,} \PYG{l+m+mi}{6390}\PYG{p}{,} \PYG{l+m+mi}{6515}\PYG{p}{,} \PYG{l+m+mi}{6805}\PYG{p}{,} \PYG{l+m+mi}{7515}\PYG{p}{,} \PYGZbs{}
\PYG{g+gp}{... }                   \PYG{l+m+mi}{7515}\PYG{p}{,} \PYG{l+m+mi}{8230}\PYG{p}{,} \PYG{l+m+mi}{8770}\PYG{p}{]}\PYG{p}{)}
\end{sphinxVerbatim}

\sphinxAtStartPar
Does their energy intake deviate systematically from the recommended
value of 7725 kJ? Our null hypothesis will be the absence of deviation,
and the alternate hypothesis will be the presence of an effect that could be
either positive or negative, hence making our test 2\sphinxhyphen{}tailed.

\sphinxAtStartPar
Because we are estimating the mean and we have N=11 values in our sample,
we have N\sphinxhyphen{}1=10 degrees of freedom. We set our significance level to 95\% and
compute the t statistic using the empirical mean and empirical standard
deviation of our intake. We use a ddof of 1 to base the computation of our
empirical standard deviation on an unbiased estimate of the variance (note:
the final estimate is not unbiased due to the concave nature of the square
root).

\begin{sphinxVerbatim}[commandchars=\\\{\}]
\PYG{g+gp}{\PYGZgt{}\PYGZgt{}\PYGZgt{} }\PYG{n}{np}\PYG{o}{.}\PYG{n}{mean}\PYG{p}{(}\PYG{n}{intake}\PYG{p}{)}
\PYG{g+go}{6753.636363636364}
\PYG{g+gp}{\PYGZgt{}\PYGZgt{}\PYGZgt{} }\PYG{n}{intake}\PYG{o}{.}\PYG{n}{std}\PYG{p}{(}\PYG{n}{ddof}\PYG{o}{=}\PYG{l+m+mi}{1}\PYG{p}{)}
\PYG{g+go}{1142.1232221373727}
\PYG{g+gp}{\PYGZgt{}\PYGZgt{}\PYGZgt{} }\PYG{n}{t} \PYG{o}{=} \PYG{p}{(}\PYG{n}{np}\PYG{o}{.}\PYG{n}{mean}\PYG{p}{(}\PYG{n}{intake}\PYG{p}{)}\PYG{o}{\PYGZhy{}}\PYG{l+m+mi}{7725}\PYG{p}{)}\PYG{o}{/}\PYG{p}{(}\PYG{n}{intake}\PYG{o}{.}\PYG{n}{std}\PYG{p}{(}\PYG{n}{ddof}\PYG{o}{=}\PYG{l+m+mi}{1}\PYG{p}{)}\PYG{o}{/}\PYG{n}{np}\PYG{o}{.}\PYG{n}{sqrt}\PYG{p}{(}\PYG{n+nb}{len}\PYG{p}{(}\PYG{n}{intake}\PYG{p}{)}\PYG{p}{)}\PYG{p}{)}
\PYG{g+gp}{\PYGZgt{}\PYGZgt{}\PYGZgt{} }\PYG{n}{t}
\PYG{g+go}{\PYGZhy{}2.8207540608310198}
\end{sphinxVerbatim}

\sphinxAtStartPar
We draw 1000000 samples from Student’s t distribution with the adequate
degrees of freedom.

\begin{sphinxVerbatim}[commandchars=\\\{\}]
\PYG{g+gp}{\PYGZgt{}\PYGZgt{}\PYGZgt{} }\PYG{k+kn}{import} \PYG{n+nn}{matplotlib}\PYG{n+nn}{.}\PYG{n+nn}{pyplot} \PYG{k}{as} \PYG{n+nn}{plt}
\PYG{g+gp}{\PYGZgt{}\PYGZgt{}\PYGZgt{} }\PYG{n}{s} \PYG{o}{=} \PYG{n}{np}\PYG{o}{.}\PYG{n}{random}\PYG{o}{.}\PYG{n}{standard\PYGZus{}t}\PYG{p}{(}\PYG{l+m+mi}{10}\PYG{p}{,} \PYG{n}{size}\PYG{o}{=}\PYG{l+m+mi}{1000000}\PYG{p}{)}
\PYG{g+gp}{\PYGZgt{}\PYGZgt{}\PYGZgt{} }\PYG{n}{h} \PYG{o}{=} \PYG{n}{plt}\PYG{o}{.}\PYG{n}{hist}\PYG{p}{(}\PYG{n}{s}\PYG{p}{,} \PYG{n}{bins}\PYG{o}{=}\PYG{l+m+mi}{100}\PYG{p}{,} \PYG{n}{density}\PYG{o}{=}\PYG{k+kc}{True}\PYG{p}{)}
\end{sphinxVerbatim}

\sphinxAtStartPar
Does our t statistic land in one of the two critical regions found at
both tails of the distribution?

\begin{sphinxVerbatim}[commandchars=\\\{\}]
\PYG{g+gp}{\PYGZgt{}\PYGZgt{}\PYGZgt{} }\PYG{n}{np}\PYG{o}{.}\PYG{n}{sum}\PYG{p}{(}\PYG{n}{np}\PYG{o}{.}\PYG{n}{abs}\PYG{p}{(}\PYG{n}{t}\PYG{p}{)} \PYG{o}{\PYGZlt{}} \PYG{n}{np}\PYG{o}{.}\PYG{n}{abs}\PYG{p}{(}\PYG{n}{s}\PYG{p}{)}\PYG{p}{)} \PYG{o}{/} \PYG{n+nb}{float}\PYG{p}{(}\PYG{n+nb}{len}\PYG{p}{(}\PYG{n}{s}\PYG{p}{)}\PYG{p}{)}
\PYG{g+go}{0.018318  \PYGZsh{}random \PYGZlt{} 0.05, statistic is in critical region}
\end{sphinxVerbatim}

\sphinxAtStartPar
The probability value for this 2\sphinxhyphen{}tailed test is about 1.83\%, which is
lower than the 5\% pre\sphinxhyphen{}determined significance threshold.

\sphinxAtStartPar
Therefore, the probability of observing values as extreme as our intake
conditionally on the null hypothesis being true is too low, and we reject
the null hypothesis of no deviation.

\end{fulllineitems}

\index{triangular() (in module metilda.controllers.pitch\_art\_wizard)@\spxentry{triangular()}\spxextra{in module metilda.controllers.pitch\_art\_wizard}}

\begin{fulllineitems}
\phantomsection\label{\detokenize{metilda.controllers:metilda.controllers.pitch_art_wizard.triangular}}
\pysigstartsignatures
\pysiglinewithargsret{\sphinxcode{\sphinxupquote{metilda.controllers.pitch\_art\_wizard.}}\sphinxbfcode{\sphinxupquote{triangular}}}{\sphinxparam{\DUrole{n,n}{left}}\sphinxparamcomma \sphinxparam{\DUrole{n,n}{mode}}\sphinxparamcomma \sphinxparam{\DUrole{n,n}{right}}\sphinxparamcomma \sphinxparam{\DUrole{n,n}{size}\DUrole{o,o}{=}\DUrole{default_value}{None}}}{}
\pysigstopsignatures
\sphinxAtStartPar
Draw samples from the triangular distribution over the
interval \sphinxcode{\sphinxupquote{{[}left, right{]}}}.

\sphinxAtStartPar
The triangular distribution is a continuous probability
distribution with lower limit left, peak at mode, and upper
limit right. Unlike the other distributions, these parameters
directly define the shape of the pdf.

\begin{sphinxadmonition}{note}{Note:}
\sphinxAtStartPar
New code should use the \sphinxtitleref{\textasciitilde{}numpy.random.Generator.triangular}
method of a \sphinxtitleref{\textasciitilde{}numpy.random.Generator} instance instead;
please see the \DUrole{xref,std,std-ref}{random\sphinxhyphen{}quick\sphinxhyphen{}start}.
\end{sphinxadmonition}
\begin{quote}\begin{description}
\sphinxlineitem{Parameters}\begin{itemize}
\item {} 
\sphinxAtStartPar
\sphinxstyleliteralstrong{\sphinxupquote{left}} (\sphinxstyleliteralemphasis{\sphinxupquote{float}}\sphinxstyleliteralemphasis{\sphinxupquote{ or }}\sphinxstyleliteralemphasis{\sphinxupquote{array\_like}}\sphinxstyleliteralemphasis{\sphinxupquote{ of }}\sphinxstyleliteralemphasis{\sphinxupquote{floats}}) \textendash{} Lower limit.

\item {} 
\sphinxAtStartPar
\sphinxstyleliteralstrong{\sphinxupquote{mode}} (\sphinxstyleliteralemphasis{\sphinxupquote{float}}\sphinxstyleliteralemphasis{\sphinxupquote{ or }}\sphinxstyleliteralemphasis{\sphinxupquote{array\_like}}\sphinxstyleliteralemphasis{\sphinxupquote{ of }}\sphinxstyleliteralemphasis{\sphinxupquote{floats}}) \textendash{} The value where the peak of the distribution occurs.
The value must fulfill the condition \sphinxcode{\sphinxupquote{left \textless{}= mode \textless{}= right}}.

\item {} 
\sphinxAtStartPar
\sphinxstyleliteralstrong{\sphinxupquote{right}} (\sphinxstyleliteralemphasis{\sphinxupquote{float}}\sphinxstyleliteralemphasis{\sphinxupquote{ or }}\sphinxstyleliteralemphasis{\sphinxupquote{array\_like}}\sphinxstyleliteralemphasis{\sphinxupquote{ of }}\sphinxstyleliteralemphasis{\sphinxupquote{floats}}) \textendash{} Upper limit, must be larger than \sphinxtitleref{left}.

\item {} 
\sphinxAtStartPar
\sphinxstyleliteralstrong{\sphinxupquote{size}} (\sphinxstyleliteralemphasis{\sphinxupquote{int}}\sphinxstyleliteralemphasis{\sphinxupquote{ or }}\sphinxstyleliteralemphasis{\sphinxupquote{tuple}}\sphinxstyleliteralemphasis{\sphinxupquote{ of }}\sphinxstyleliteralemphasis{\sphinxupquote{ints}}\sphinxstyleliteralemphasis{\sphinxupquote{, }}\sphinxstyleliteralemphasis{\sphinxupquote{optional}}) \textendash{} Output shape.  If the given shape is, e.g., \sphinxcode{\sphinxupquote{(m, n, k)}}, then
\sphinxcode{\sphinxupquote{m * n * k}} samples are drawn.  If size is \sphinxcode{\sphinxupquote{None}} (default),
a single value is returned if \sphinxcode{\sphinxupquote{left}}, \sphinxcode{\sphinxupquote{mode}}, and \sphinxcode{\sphinxupquote{right}}
are all scalars.  Otherwise, \sphinxcode{\sphinxupquote{np.broadcast(left, mode, right).size}}
samples are drawn.

\end{itemize}

\sphinxlineitem{Returns}
\sphinxAtStartPar
\sphinxstylestrong{out} \textendash{} Drawn samples from the parameterized triangular distribution.

\sphinxlineitem{Return type}
\sphinxAtStartPar
ndarray or scalar

\end{description}\end{quote}


\begin{sphinxseealso}{See also:}
\begin{description}
\sphinxlineitem{\sphinxcode{\sphinxupquote{random.Generator.triangular}}}
\sphinxAtStartPar
which should be used for new code.

\end{description}


\end{sphinxseealso}

\subsubsection*{Notes}

\sphinxAtStartPar
The probability density function for the triangular distribution is
\begin{equation*}
\begin{split}P(x;l, m, r) = \begin{cases}
\frac{2(x-l)}{(r-l)(m-l)}& \text{for $l \leq x \leq m$},\\
\frac{2(r-x)}{(r-l)(r-m)}& \text{for $m \leq x \leq r$},\\
0& \text{otherwise}.
\end{cases}\end{split}
\end{equation*}
\sphinxAtStartPar
The triangular distribution is often used in ill\sphinxhyphen{}defined
problems where the underlying distribution is not known, but
some knowledge of the limits and mode exists. Often it is used
in simulations.
\subsubsection*{References}
\subsubsection*{Examples}

\sphinxAtStartPar
Draw values from the distribution and plot the histogram:

\begin{sphinxVerbatim}[commandchars=\\\{\}]
\PYG{g+gp}{\PYGZgt{}\PYGZgt{}\PYGZgt{} }\PYG{k+kn}{import} \PYG{n+nn}{matplotlib}\PYG{n+nn}{.}\PYG{n+nn}{pyplot} \PYG{k}{as} \PYG{n+nn}{plt}
\PYG{g+gp}{\PYGZgt{}\PYGZgt{}\PYGZgt{} }\PYG{n}{h} \PYG{o}{=} \PYG{n}{plt}\PYG{o}{.}\PYG{n}{hist}\PYG{p}{(}\PYG{n}{np}\PYG{o}{.}\PYG{n}{random}\PYG{o}{.}\PYG{n}{triangular}\PYG{p}{(}\PYG{o}{\PYGZhy{}}\PYG{l+m+mi}{3}\PYG{p}{,} \PYG{l+m+mi}{0}\PYG{p}{,} \PYG{l+m+mi}{8}\PYG{p}{,} \PYG{l+m+mi}{100000}\PYG{p}{)}\PYG{p}{,} \PYG{n}{bins}\PYG{o}{=}\PYG{l+m+mi}{200}\PYG{p}{,}
\PYG{g+gp}{... }             \PYG{n}{density}\PYG{o}{=}\PYG{k+kc}{True}\PYG{p}{)}
\PYG{g+gp}{\PYGZgt{}\PYGZgt{}\PYGZgt{} }\PYG{n}{plt}\PYG{o}{.}\PYG{n}{show}\PYG{p}{(}\PYG{p}{)}
\end{sphinxVerbatim}

\end{fulllineitems}

\index{uniform() (in module metilda.controllers.pitch\_art\_wizard)@\spxentry{uniform()}\spxextra{in module metilda.controllers.pitch\_art\_wizard}}

\begin{fulllineitems}
\phantomsection\label{\detokenize{metilda.controllers:metilda.controllers.pitch_art_wizard.uniform}}
\pysigstartsignatures
\pysiglinewithargsret{\sphinxcode{\sphinxupquote{metilda.controllers.pitch\_art\_wizard.}}\sphinxbfcode{\sphinxupquote{uniform}}}{\sphinxparam{\DUrole{n,n}{low}\DUrole{o,o}{=}\DUrole{default_value}{0.0}}\sphinxparamcomma \sphinxparam{\DUrole{n,n}{high}\DUrole{o,o}{=}\DUrole{default_value}{1.0}}\sphinxparamcomma \sphinxparam{\DUrole{n,n}{size}\DUrole{o,o}{=}\DUrole{default_value}{None}}}{}
\pysigstopsignatures
\sphinxAtStartPar
Draw samples from a uniform distribution.

\sphinxAtStartPar
Samples are uniformly distributed over the half\sphinxhyphen{}open interval
\sphinxcode{\sphinxupquote{{[}low, high)}} (includes low, but excludes high).  In other words,
any value within the given interval is equally likely to be drawn
by \sphinxtitleref{uniform}.

\begin{sphinxadmonition}{note}{Note:}
\sphinxAtStartPar
New code should use the \sphinxtitleref{\textasciitilde{}numpy.random.Generator.uniform}
method of a \sphinxtitleref{\textasciitilde{}numpy.random.Generator} instance instead;
please see the \DUrole{xref,std,std-ref}{random\sphinxhyphen{}quick\sphinxhyphen{}start}.
\end{sphinxadmonition}
\begin{quote}\begin{description}
\sphinxlineitem{Parameters}\begin{itemize}
\item {} 
\sphinxAtStartPar
\sphinxstyleliteralstrong{\sphinxupquote{low}} (\sphinxstyleliteralemphasis{\sphinxupquote{float}}\sphinxstyleliteralemphasis{\sphinxupquote{ or }}\sphinxstyleliteralemphasis{\sphinxupquote{array\_like}}\sphinxstyleliteralemphasis{\sphinxupquote{ of }}\sphinxstyleliteralemphasis{\sphinxupquote{floats}}\sphinxstyleliteralemphasis{\sphinxupquote{, }}\sphinxstyleliteralemphasis{\sphinxupquote{optional}}) \textendash{} Lower boundary of the output interval.  All values generated will be
greater than or equal to low.  The default value is 0.

\item {} 
\sphinxAtStartPar
\sphinxstyleliteralstrong{\sphinxupquote{high}} (\sphinxstyleliteralemphasis{\sphinxupquote{float}}\sphinxstyleliteralemphasis{\sphinxupquote{ or }}\sphinxstyleliteralemphasis{\sphinxupquote{array\_like}}\sphinxstyleliteralemphasis{\sphinxupquote{ of }}\sphinxstyleliteralemphasis{\sphinxupquote{floats}}) \textendash{} Upper boundary of the output interval.  All values generated will be
less than or equal to high.  The high limit may be included in the
returned array of floats due to floating\sphinxhyphen{}point rounding in the
equation \sphinxcode{\sphinxupquote{low + (high\sphinxhyphen{}low) * random\_sample()}}.  The default value
is 1.0.

\item {} 
\sphinxAtStartPar
\sphinxstyleliteralstrong{\sphinxupquote{size}} (\sphinxstyleliteralemphasis{\sphinxupquote{int}}\sphinxstyleliteralemphasis{\sphinxupquote{ or }}\sphinxstyleliteralemphasis{\sphinxupquote{tuple}}\sphinxstyleliteralemphasis{\sphinxupquote{ of }}\sphinxstyleliteralemphasis{\sphinxupquote{ints}}\sphinxstyleliteralemphasis{\sphinxupquote{, }}\sphinxstyleliteralemphasis{\sphinxupquote{optional}}) \textendash{} Output shape.  If the given shape is, e.g., \sphinxcode{\sphinxupquote{(m, n, k)}}, then
\sphinxcode{\sphinxupquote{m * n * k}} samples are drawn.  If size is \sphinxcode{\sphinxupquote{None}} (default),
a single value is returned if \sphinxcode{\sphinxupquote{low}} and \sphinxcode{\sphinxupquote{high}} are both scalars.
Otherwise, \sphinxcode{\sphinxupquote{np.broadcast(low, high).size}} samples are drawn.

\end{itemize}

\sphinxlineitem{Returns}
\sphinxAtStartPar
\sphinxstylestrong{out} \textendash{} Drawn samples from the parameterized uniform distribution.

\sphinxlineitem{Return type}
\sphinxAtStartPar
ndarray or scalar

\end{description}\end{quote}


\begin{sphinxseealso}{See also:}
\begin{description}
\sphinxlineitem{{\hyperref[\detokenize{metilda.controllers:metilda.controllers.pitch_art_wizard.randint}]{\sphinxcrossref{\sphinxcode{\sphinxupquote{randint}}}}}}
\sphinxAtStartPar
Discrete uniform distribution, yielding integers.

\sphinxlineitem{{\hyperref[\detokenize{metilda.controllers:metilda.controllers.pitch_art_wizard.random_integers}]{\sphinxcrossref{\sphinxcode{\sphinxupquote{random\_integers}}}}}}
\sphinxAtStartPar
Discrete uniform distribution over the closed interval \sphinxcode{\sphinxupquote{{[}low, high{]}}}.

\sphinxlineitem{{\hyperref[\detokenize{metilda.controllers:metilda.controllers.pitch_art_wizard.random_sample}]{\sphinxcrossref{\sphinxcode{\sphinxupquote{random\_sample}}}}}}
\sphinxAtStartPar
Floats uniformly distributed over \sphinxcode{\sphinxupquote{{[}0, 1)}}.

\sphinxlineitem{{\hyperref[\detokenize{metilda.controllers:metilda.controllers.pitch_art_wizard.random}]{\sphinxcrossref{\sphinxcode{\sphinxupquote{random}}}}}}
\sphinxAtStartPar
Alias for \sphinxtitleref{random\_sample}.

\sphinxlineitem{{\hyperref[\detokenize{metilda.controllers:metilda.controllers.pitch_art_wizard.rand}]{\sphinxcrossref{\sphinxcode{\sphinxupquote{rand}}}}}}
\sphinxAtStartPar
Convenience function that accepts dimensions as input, e.g., \sphinxcode{\sphinxupquote{rand(2,2)}} would generate a 2\sphinxhyphen{}by\sphinxhyphen{}2 array of floats, uniformly distributed over \sphinxcode{\sphinxupquote{{[}0, 1)}}.

\sphinxlineitem{\sphinxcode{\sphinxupquote{random.Generator.uniform}}}
\sphinxAtStartPar
which should be used for new code.

\end{description}


\end{sphinxseealso}

\subsubsection*{Notes}

\sphinxAtStartPar
The probability density function of the uniform distribution is
\begin{equation*}
\begin{split}p(x) = \frac{1}{b - a}\end{split}
\end{equation*}
\sphinxAtStartPar
anywhere within the interval \sphinxcode{\sphinxupquote{{[}a, b)}}, and zero elsewhere.

\sphinxAtStartPar
When \sphinxcode{\sphinxupquote{high}} == \sphinxcode{\sphinxupquote{low}}, values of \sphinxcode{\sphinxupquote{low}} will be returned.
If \sphinxcode{\sphinxupquote{high}} \textless{} \sphinxcode{\sphinxupquote{low}}, the results are officially undefined
and may eventually raise an error, i.e. do not rely on this
function to behave when passed arguments satisfying that
inequality condition. The \sphinxcode{\sphinxupquote{high}} limit may be included in the
returned array of floats due to floating\sphinxhyphen{}point rounding in the
equation \sphinxcode{\sphinxupquote{low + (high\sphinxhyphen{}low) * random\_sample()}}. For example:

\begin{sphinxVerbatim}[commandchars=\\\{\}]
\PYG{g+gp}{\PYGZgt{}\PYGZgt{}\PYGZgt{} }\PYG{n}{x} \PYG{o}{=} \PYG{n}{np}\PYG{o}{.}\PYG{n}{float32}\PYG{p}{(}\PYG{l+m+mi}{5}\PYG{o}{*}\PYG{l+m+mf}{0.99999999}\PYG{p}{)}
\PYG{g+gp}{\PYGZgt{}\PYGZgt{}\PYGZgt{} }\PYG{n}{x}
\PYG{g+go}{5.0}
\end{sphinxVerbatim}
\subsubsection*{Examples}

\sphinxAtStartPar
Draw samples from the distribution:

\begin{sphinxVerbatim}[commandchars=\\\{\}]
\PYG{g+gp}{\PYGZgt{}\PYGZgt{}\PYGZgt{} }\PYG{n}{s} \PYG{o}{=} \PYG{n}{np}\PYG{o}{.}\PYG{n}{random}\PYG{o}{.}\PYG{n}{uniform}\PYG{p}{(}\PYG{o}{\PYGZhy{}}\PYG{l+m+mi}{1}\PYG{p}{,}\PYG{l+m+mi}{0}\PYG{p}{,}\PYG{l+m+mi}{1000}\PYG{p}{)}
\end{sphinxVerbatim}

\sphinxAtStartPar
All values are within the given interval:

\begin{sphinxVerbatim}[commandchars=\\\{\}]
\PYG{g+gp}{\PYGZgt{}\PYGZgt{}\PYGZgt{} }\PYG{n}{np}\PYG{o}{.}\PYG{n}{all}\PYG{p}{(}\PYG{n}{s} \PYG{o}{\PYGZgt{}}\PYG{o}{=} \PYG{o}{\PYGZhy{}}\PYG{l+m+mi}{1}\PYG{p}{)}
\PYG{g+go}{True}
\PYG{g+gp}{\PYGZgt{}\PYGZgt{}\PYGZgt{} }\PYG{n}{np}\PYG{o}{.}\PYG{n}{all}\PYG{p}{(}\PYG{n}{s} \PYG{o}{\PYGZlt{}} \PYG{l+m+mi}{0}\PYG{p}{)}
\PYG{g+go}{True}
\end{sphinxVerbatim}

\sphinxAtStartPar
Display the histogram of the samples, along with the
probability density function:

\begin{sphinxVerbatim}[commandchars=\\\{\}]
\PYG{g+gp}{\PYGZgt{}\PYGZgt{}\PYGZgt{} }\PYG{k+kn}{import} \PYG{n+nn}{matplotlib}\PYG{n+nn}{.}\PYG{n+nn}{pyplot} \PYG{k}{as} \PYG{n+nn}{plt}
\PYG{g+gp}{\PYGZgt{}\PYGZgt{}\PYGZgt{} }\PYG{n}{count}\PYG{p}{,} \PYG{n}{bins}\PYG{p}{,} \PYG{n}{ignored} \PYG{o}{=} \PYG{n}{plt}\PYG{o}{.}\PYG{n}{hist}\PYG{p}{(}\PYG{n}{s}\PYG{p}{,} \PYG{l+m+mi}{15}\PYG{p}{,} \PYG{n}{density}\PYG{o}{=}\PYG{k+kc}{True}\PYG{p}{)}
\PYG{g+gp}{\PYGZgt{}\PYGZgt{}\PYGZgt{} }\PYG{n}{plt}\PYG{o}{.}\PYG{n}{plot}\PYG{p}{(}\PYG{n}{bins}\PYG{p}{,} \PYG{n}{np}\PYG{o}{.}\PYG{n}{ones\PYGZus{}like}\PYG{p}{(}\PYG{n}{bins}\PYG{p}{)}\PYG{p}{,} \PYG{n}{linewidth}\PYG{o}{=}\PYG{l+m+mi}{2}\PYG{p}{,} \PYG{n}{color}\PYG{o}{=}\PYG{l+s+s1}{\PYGZsq{}}\PYG{l+s+s1}{r}\PYG{l+s+s1}{\PYGZsq{}}\PYG{p}{)}
\PYG{g+gp}{\PYGZgt{}\PYGZgt{}\PYGZgt{} }\PYG{n}{plt}\PYG{o}{.}\PYG{n}{show}\PYG{p}{(}\PYG{p}{)}
\end{sphinxVerbatim}

\end{fulllineitems}

\index{update\_analysis() (in module metilda.controllers.pitch\_art\_wizard)@\spxentry{update\_analysis()}\spxextra{in module metilda.controllers.pitch\_art\_wizard}}

\begin{fulllineitems}
\phantomsection\label{\detokenize{metilda.controllers:metilda.controllers.pitch_art_wizard.update_analysis}}
\pysigstartsignatures
\pysiglinewithargsret{\sphinxcode{\sphinxupquote{metilda.controllers.pitch\_art\_wizard.}}\sphinxbfcode{\sphinxupquote{update\_analysis}}}{}{}
\pysigstopsignatures
\end{fulllineitems}

\index{update\_db\_user() (in module metilda.controllers.pitch\_art\_wizard)@\spxentry{update\_db\_user()}\spxextra{in module metilda.controllers.pitch\_art\_wizard}}

\begin{fulllineitems}
\phantomsection\label{\detokenize{metilda.controllers:metilda.controllers.pitch_art_wizard.update_db_user}}
\pysigstartsignatures
\pysiglinewithargsret{\sphinxcode{\sphinxupquote{metilda.controllers.pitch\_art\_wizard.}}\sphinxbfcode{\sphinxupquote{update\_db\_user}}}{}{}
\pysigstopsignatures
\end{fulllineitems}

\index{update\_user\_from\_admin() (in module metilda.controllers.pitch\_art\_wizard)@\spxentry{update\_user\_from\_admin()}\spxextra{in module metilda.controllers.pitch\_art\_wizard}}

\begin{fulllineitems}
\phantomsection\label{\detokenize{metilda.controllers:metilda.controllers.pitch_art_wizard.update_user_from_admin}}
\pysigstartsignatures
\pysiglinewithargsret{\sphinxcode{\sphinxupquote{metilda.controllers.pitch\_art\_wizard.}}\sphinxbfcode{\sphinxupquote{update\_user\_from\_admin}}}{}{}
\pysigstopsignatures
\end{fulllineitems}

\index{vonmises() (in module metilda.controllers.pitch\_art\_wizard)@\spxentry{vonmises()}\spxextra{in module metilda.controllers.pitch\_art\_wizard}}

\begin{fulllineitems}
\phantomsection\label{\detokenize{metilda.controllers:metilda.controllers.pitch_art_wizard.vonmises}}
\pysigstartsignatures
\pysiglinewithargsret{\sphinxcode{\sphinxupquote{metilda.controllers.pitch\_art\_wizard.}}\sphinxbfcode{\sphinxupquote{vonmises}}}{\sphinxparam{\DUrole{n,n}{mu}}\sphinxparamcomma \sphinxparam{\DUrole{n,n}{kappa}}\sphinxparamcomma \sphinxparam{\DUrole{n,n}{size}\DUrole{o,o}{=}\DUrole{default_value}{None}}}{}
\pysigstopsignatures
\sphinxAtStartPar
Draw samples from a von Mises distribution.

\sphinxAtStartPar
Samples are drawn from a von Mises distribution with specified mode
(mu) and dispersion (kappa), on the interval {[}\sphinxhyphen{}pi, pi{]}.

\sphinxAtStartPar
The von Mises distribution (also known as the circular normal
distribution) is a continuous probability distribution on the unit
circle.  It may be thought of as the circular analogue of the normal
distribution.

\begin{sphinxadmonition}{note}{Note:}
\sphinxAtStartPar
New code should use the \sphinxtitleref{\textasciitilde{}numpy.random.Generator.vonmises}
method of a \sphinxtitleref{\textasciitilde{}numpy.random.Generator} instance instead;
please see the \DUrole{xref,std,std-ref}{random\sphinxhyphen{}quick\sphinxhyphen{}start}.
\end{sphinxadmonition}
\begin{quote}\begin{description}
\sphinxlineitem{Parameters}\begin{itemize}
\item {} 
\sphinxAtStartPar
\sphinxstyleliteralstrong{\sphinxupquote{mu}} (\sphinxstyleliteralemphasis{\sphinxupquote{float}}\sphinxstyleliteralemphasis{\sphinxupquote{ or }}\sphinxstyleliteralemphasis{\sphinxupquote{array\_like}}\sphinxstyleliteralemphasis{\sphinxupquote{ of }}\sphinxstyleliteralemphasis{\sphinxupquote{floats}}) \textendash{} Mode (“center”) of the distribution.

\item {} 
\sphinxAtStartPar
\sphinxstyleliteralstrong{\sphinxupquote{kappa}} (\sphinxstyleliteralemphasis{\sphinxupquote{float}}\sphinxstyleliteralemphasis{\sphinxupquote{ or }}\sphinxstyleliteralemphasis{\sphinxupquote{array\_like}}\sphinxstyleliteralemphasis{\sphinxupquote{ of }}\sphinxstyleliteralemphasis{\sphinxupquote{floats}}) \textendash{} Dispersion of the distribution, has to be \textgreater{}=0.

\item {} 
\sphinxAtStartPar
\sphinxstyleliteralstrong{\sphinxupquote{size}} (\sphinxstyleliteralemphasis{\sphinxupquote{int}}\sphinxstyleliteralemphasis{\sphinxupquote{ or }}\sphinxstyleliteralemphasis{\sphinxupquote{tuple}}\sphinxstyleliteralemphasis{\sphinxupquote{ of }}\sphinxstyleliteralemphasis{\sphinxupquote{ints}}\sphinxstyleliteralemphasis{\sphinxupquote{, }}\sphinxstyleliteralemphasis{\sphinxupquote{optional}}) \textendash{} Output shape.  If the given shape is, e.g., \sphinxcode{\sphinxupquote{(m, n, k)}}, then
\sphinxcode{\sphinxupquote{m * n * k}} samples are drawn.  If size is \sphinxcode{\sphinxupquote{None}} (default),
a single value is returned if \sphinxcode{\sphinxupquote{mu}} and \sphinxcode{\sphinxupquote{kappa}} are both scalars.
Otherwise, \sphinxcode{\sphinxupquote{np.broadcast(mu, kappa).size}} samples are drawn.

\end{itemize}

\sphinxlineitem{Returns}
\sphinxAtStartPar
\sphinxstylestrong{out} \textendash{} Drawn samples from the parameterized von Mises distribution.

\sphinxlineitem{Return type}
\sphinxAtStartPar
ndarray or scalar

\end{description}\end{quote}


\begin{sphinxseealso}{See also:}
\begin{description}
\sphinxlineitem{\sphinxcode{\sphinxupquote{scipy.stats.vonmises}}}
\sphinxAtStartPar
probability density function, distribution, or cumulative density function, etc.

\sphinxlineitem{\sphinxcode{\sphinxupquote{random.Generator.vonmises}}}
\sphinxAtStartPar
which should be used for new code.

\end{description}


\end{sphinxseealso}

\subsubsection*{Notes}

\sphinxAtStartPar
The probability density for the von Mises distribution is
\begin{equation*}
\begin{split}p(x) = \frac{e^{\kappa cos(x-\mu)}}{2\pi I_0(\kappa)},\end{split}
\end{equation*}
\sphinxAtStartPar
where \(\mu\) is the mode and \(\kappa\) the dispersion,
and \(I_0(\kappa)\) is the modified Bessel function of order 0.

\sphinxAtStartPar
The von Mises is named for Richard Edler von Mises, who was born in
Austria\sphinxhyphen{}Hungary, in what is now the Ukraine.  He fled to the United
States in 1939 and became a professor at Harvard.  He worked in
probability theory, aerodynamics, fluid mechanics, and philosophy of
science.
\subsubsection*{References}
\subsubsection*{Examples}

\sphinxAtStartPar
Draw samples from the distribution:

\begin{sphinxVerbatim}[commandchars=\\\{\}]
\PYG{g+gp}{\PYGZgt{}\PYGZgt{}\PYGZgt{} }\PYG{n}{mu}\PYG{p}{,} \PYG{n}{kappa} \PYG{o}{=} \PYG{l+m+mf}{0.0}\PYG{p}{,} \PYG{l+m+mf}{4.0} \PYG{c+c1}{\PYGZsh{} mean and dispersion}
\PYG{g+gp}{\PYGZgt{}\PYGZgt{}\PYGZgt{} }\PYG{n}{s} \PYG{o}{=} \PYG{n}{np}\PYG{o}{.}\PYG{n}{random}\PYG{o}{.}\PYG{n}{vonmises}\PYG{p}{(}\PYG{n}{mu}\PYG{p}{,} \PYG{n}{kappa}\PYG{p}{,} \PYG{l+m+mi}{1000}\PYG{p}{)}
\end{sphinxVerbatim}

\sphinxAtStartPar
Display the histogram of the samples, along with
the probability density function:

\begin{sphinxVerbatim}[commandchars=\\\{\}]
\PYG{g+gp}{\PYGZgt{}\PYGZgt{}\PYGZgt{} }\PYG{k+kn}{import} \PYG{n+nn}{matplotlib}\PYG{n+nn}{.}\PYG{n+nn}{pyplot} \PYG{k}{as} \PYG{n+nn}{plt}
\PYG{g+gp}{\PYGZgt{}\PYGZgt{}\PYGZgt{} }\PYG{k+kn}{from} \PYG{n+nn}{scipy}\PYG{n+nn}{.}\PYG{n+nn}{special} \PYG{k+kn}{import} \PYG{n}{i0}  
\PYG{g+gp}{\PYGZgt{}\PYGZgt{}\PYGZgt{} }\PYG{n}{plt}\PYG{o}{.}\PYG{n}{hist}\PYG{p}{(}\PYG{n}{s}\PYG{p}{,} \PYG{l+m+mi}{50}\PYG{p}{,} \PYG{n}{density}\PYG{o}{=}\PYG{k+kc}{True}\PYG{p}{)}
\PYG{g+gp}{\PYGZgt{}\PYGZgt{}\PYGZgt{} }\PYG{n}{x} \PYG{o}{=} \PYG{n}{np}\PYG{o}{.}\PYG{n}{linspace}\PYG{p}{(}\PYG{o}{\PYGZhy{}}\PYG{n}{np}\PYG{o}{.}\PYG{n}{pi}\PYG{p}{,} \PYG{n}{np}\PYG{o}{.}\PYG{n}{pi}\PYG{p}{,} \PYG{n}{num}\PYG{o}{=}\PYG{l+m+mi}{51}\PYG{p}{)}
\PYG{g+gp}{\PYGZgt{}\PYGZgt{}\PYGZgt{} }\PYG{n}{y} \PYG{o}{=} \PYG{n}{np}\PYG{o}{.}\PYG{n}{exp}\PYG{p}{(}\PYG{n}{kappa}\PYG{o}{*}\PYG{n}{np}\PYG{o}{.}\PYG{n}{cos}\PYG{p}{(}\PYG{n}{x}\PYG{o}{\PYGZhy{}}\PYG{n}{mu}\PYG{p}{)}\PYG{p}{)}\PYG{o}{/}\PYG{p}{(}\PYG{l+m+mi}{2}\PYG{o}{*}\PYG{n}{np}\PYG{o}{.}\PYG{n}{pi}\PYG{o}{*}\PYG{n}{i0}\PYG{p}{(}\PYG{n}{kappa}\PYG{p}{)}\PYG{p}{)}  
\PYG{g+gp}{\PYGZgt{}\PYGZgt{}\PYGZgt{} }\PYG{n}{plt}\PYG{o}{.}\PYG{n}{plot}\PYG{p}{(}\PYG{n}{x}\PYG{p}{,} \PYG{n}{y}\PYG{p}{,} \PYG{n}{linewidth}\PYG{o}{=}\PYG{l+m+mi}{2}\PYG{p}{,} \PYG{n}{color}\PYG{o}{=}\PYG{l+s+s1}{\PYGZsq{}}\PYG{l+s+s1}{r}\PYG{l+s+s1}{\PYGZsq{}}\PYG{p}{)}  
\PYG{g+gp}{\PYGZgt{}\PYGZgt{}\PYGZgt{} }\PYG{n}{plt}\PYG{o}{.}\PYG{n}{show}\PYG{p}{(}\PYG{p}{)}
\end{sphinxVerbatim}

\end{fulllineitems}

\index{wald() (in module metilda.controllers.pitch\_art\_wizard)@\spxentry{wald()}\spxextra{in module metilda.controllers.pitch\_art\_wizard}}

\begin{fulllineitems}
\phantomsection\label{\detokenize{metilda.controllers:metilda.controllers.pitch_art_wizard.wald}}
\pysigstartsignatures
\pysiglinewithargsret{\sphinxcode{\sphinxupquote{metilda.controllers.pitch\_art\_wizard.}}\sphinxbfcode{\sphinxupquote{wald}}}{\sphinxparam{\DUrole{n,n}{mean}}\sphinxparamcomma \sphinxparam{\DUrole{n,n}{scale}}\sphinxparamcomma \sphinxparam{\DUrole{n,n}{size}\DUrole{o,o}{=}\DUrole{default_value}{None}}}{}
\pysigstopsignatures
\sphinxAtStartPar
Draw samples from a Wald, or inverse Gaussian, distribution.

\sphinxAtStartPar
As the scale approaches infinity, the distribution becomes more like a
Gaussian. Some references claim that the Wald is an inverse Gaussian
with mean equal to 1, but this is by no means universal.

\sphinxAtStartPar
The inverse Gaussian distribution was first studied in relationship to
Brownian motion. In 1956 M.C.K. Tweedie used the name inverse Gaussian
because there is an inverse relationship between the time to cover a
unit distance and distance covered in unit time.

\begin{sphinxadmonition}{note}{Note:}
\sphinxAtStartPar
New code should use the \sphinxtitleref{\textasciitilde{}numpy.random.Generator.wald}
method of a \sphinxtitleref{\textasciitilde{}numpy.random.Generator} instance instead;
please see the \DUrole{xref,std,std-ref}{random\sphinxhyphen{}quick\sphinxhyphen{}start}.
\end{sphinxadmonition}
\begin{quote}\begin{description}
\sphinxlineitem{Parameters}\begin{itemize}
\item {} 
\sphinxAtStartPar
\sphinxstyleliteralstrong{\sphinxupquote{mean}} (\sphinxstyleliteralemphasis{\sphinxupquote{float}}\sphinxstyleliteralemphasis{\sphinxupquote{ or }}\sphinxstyleliteralemphasis{\sphinxupquote{array\_like}}\sphinxstyleliteralemphasis{\sphinxupquote{ of }}\sphinxstyleliteralemphasis{\sphinxupquote{floats}}) \textendash{} Distribution mean, must be \textgreater{} 0.

\item {} 
\sphinxAtStartPar
\sphinxstyleliteralstrong{\sphinxupquote{scale}} (\sphinxstyleliteralemphasis{\sphinxupquote{float}}\sphinxstyleliteralemphasis{\sphinxupquote{ or }}\sphinxstyleliteralemphasis{\sphinxupquote{array\_like}}\sphinxstyleliteralemphasis{\sphinxupquote{ of }}\sphinxstyleliteralemphasis{\sphinxupquote{floats}}) \textendash{} Scale parameter, must be \textgreater{} 0.

\item {} 
\sphinxAtStartPar
\sphinxstyleliteralstrong{\sphinxupquote{size}} (\sphinxstyleliteralemphasis{\sphinxupquote{int}}\sphinxstyleliteralemphasis{\sphinxupquote{ or }}\sphinxstyleliteralemphasis{\sphinxupquote{tuple}}\sphinxstyleliteralemphasis{\sphinxupquote{ of }}\sphinxstyleliteralemphasis{\sphinxupquote{ints}}\sphinxstyleliteralemphasis{\sphinxupquote{, }}\sphinxstyleliteralemphasis{\sphinxupquote{optional}}) \textendash{} Output shape.  If the given shape is, e.g., \sphinxcode{\sphinxupquote{(m, n, k)}}, then
\sphinxcode{\sphinxupquote{m * n * k}} samples are drawn.  If size is \sphinxcode{\sphinxupquote{None}} (default),
a single value is returned if \sphinxcode{\sphinxupquote{mean}} and \sphinxcode{\sphinxupquote{scale}} are both scalars.
Otherwise, \sphinxcode{\sphinxupquote{np.broadcast(mean, scale).size}} samples are drawn.

\end{itemize}

\sphinxlineitem{Returns}
\sphinxAtStartPar
\sphinxstylestrong{out} \textendash{} Drawn samples from the parameterized Wald distribution.

\sphinxlineitem{Return type}
\sphinxAtStartPar
ndarray or scalar

\end{description}\end{quote}


\begin{sphinxseealso}{See also:}
\begin{description}
\sphinxlineitem{\sphinxcode{\sphinxupquote{random.Generator.wald}}}
\sphinxAtStartPar
which should be used for new code.

\end{description}


\end{sphinxseealso}

\subsubsection*{Notes}

\sphinxAtStartPar
The probability density function for the Wald distribution is
\begin{equation*}
\begin{split}P(x;mean,scale) = \sqrt{\frac{scale}{2\pi x^3}}e^
\frac{-scale(x-mean)^2}{2\cdotp mean^2x}\end{split}
\end{equation*}
\sphinxAtStartPar
As noted above the inverse Gaussian distribution first arise
from attempts to model Brownian motion. It is also a
competitor to the Weibull for use in reliability modeling and
modeling stock returns and interest rate processes.
\subsubsection*{References}
\subsubsection*{Examples}

\sphinxAtStartPar
Draw values from the distribution and plot the histogram:

\begin{sphinxVerbatim}[commandchars=\\\{\}]
\PYG{g+gp}{\PYGZgt{}\PYGZgt{}\PYGZgt{} }\PYG{k+kn}{import} \PYG{n+nn}{matplotlib}\PYG{n+nn}{.}\PYG{n+nn}{pyplot} \PYG{k}{as} \PYG{n+nn}{plt}
\PYG{g+gp}{\PYGZgt{}\PYGZgt{}\PYGZgt{} }\PYG{n}{h} \PYG{o}{=} \PYG{n}{plt}\PYG{o}{.}\PYG{n}{hist}\PYG{p}{(}\PYG{n}{np}\PYG{o}{.}\PYG{n}{random}\PYG{o}{.}\PYG{n}{wald}\PYG{p}{(}\PYG{l+m+mi}{3}\PYG{p}{,} \PYG{l+m+mi}{2}\PYG{p}{,} \PYG{l+m+mi}{100000}\PYG{p}{)}\PYG{p}{,} \PYG{n}{bins}\PYG{o}{=}\PYG{l+m+mi}{200}\PYG{p}{,} \PYG{n}{density}\PYG{o}{=}\PYG{k+kc}{True}\PYG{p}{)}
\PYG{g+gp}{\PYGZgt{}\PYGZgt{}\PYGZgt{} }\PYG{n}{plt}\PYG{o}{.}\PYG{n}{show}\PYG{p}{(}\PYG{p}{)}
\end{sphinxVerbatim}

\end{fulllineitems}

\index{weibull() (in module metilda.controllers.pitch\_art\_wizard)@\spxentry{weibull()}\spxextra{in module metilda.controllers.pitch\_art\_wizard}}

\begin{fulllineitems}
\phantomsection\label{\detokenize{metilda.controllers:metilda.controllers.pitch_art_wizard.weibull}}
\pysigstartsignatures
\pysiglinewithargsret{\sphinxcode{\sphinxupquote{metilda.controllers.pitch\_art\_wizard.}}\sphinxbfcode{\sphinxupquote{weibull}}}{\sphinxparam{\DUrole{n,n}{a}}\sphinxparamcomma \sphinxparam{\DUrole{n,n}{size}\DUrole{o,o}{=}\DUrole{default_value}{None}}}{}
\pysigstopsignatures
\sphinxAtStartPar
Draw samples from a Weibull distribution.

\sphinxAtStartPar
Draw samples from a 1\sphinxhyphen{}parameter Weibull distribution with the given
shape parameter \sphinxtitleref{a}.
\begin{equation*}
\begin{split}X = (-ln(U))^{1/a}\end{split}
\end{equation*}
\sphinxAtStartPar
Here, U is drawn from the uniform distribution over (0,1{]}.

\sphinxAtStartPar
The more common 2\sphinxhyphen{}parameter Weibull, including a scale parameter
\(\lambda\) is just \(X = \lambda(-ln(U))^{1/a}\).

\begin{sphinxadmonition}{note}{Note:}
\sphinxAtStartPar
New code should use the \sphinxtitleref{\textasciitilde{}numpy.random.Generator.weibull}
method of a \sphinxtitleref{\textasciitilde{}numpy.random.Generator} instance instead;
please see the \DUrole{xref,std,std-ref}{random\sphinxhyphen{}quick\sphinxhyphen{}start}.
\end{sphinxadmonition}
\begin{quote}\begin{description}
\sphinxlineitem{Parameters}\begin{itemize}
\item {} 
\sphinxAtStartPar
\sphinxstyleliteralstrong{\sphinxupquote{a}} (\sphinxstyleliteralemphasis{\sphinxupquote{float}}\sphinxstyleliteralemphasis{\sphinxupquote{ or }}\sphinxstyleliteralemphasis{\sphinxupquote{array\_like}}\sphinxstyleliteralemphasis{\sphinxupquote{ of }}\sphinxstyleliteralemphasis{\sphinxupquote{floats}}) \textendash{} Shape parameter of the distribution.  Must be nonnegative.

\item {} 
\sphinxAtStartPar
\sphinxstyleliteralstrong{\sphinxupquote{size}} (\sphinxstyleliteralemphasis{\sphinxupquote{int}}\sphinxstyleliteralemphasis{\sphinxupquote{ or }}\sphinxstyleliteralemphasis{\sphinxupquote{tuple}}\sphinxstyleliteralemphasis{\sphinxupquote{ of }}\sphinxstyleliteralemphasis{\sphinxupquote{ints}}\sphinxstyleliteralemphasis{\sphinxupquote{, }}\sphinxstyleliteralemphasis{\sphinxupquote{optional}}) \textendash{} Output shape.  If the given shape is, e.g., \sphinxcode{\sphinxupquote{(m, n, k)}}, then
\sphinxcode{\sphinxupquote{m * n * k}} samples are drawn.  If size is \sphinxcode{\sphinxupquote{None}} (default),
a single value is returned if \sphinxcode{\sphinxupquote{a}} is a scalar.  Otherwise,
\sphinxcode{\sphinxupquote{np.array(a).size}} samples are drawn.

\end{itemize}

\sphinxlineitem{Returns}
\sphinxAtStartPar
\sphinxstylestrong{out} \textendash{} Drawn samples from the parameterized Weibull distribution.

\sphinxlineitem{Return type}
\sphinxAtStartPar
ndarray or scalar

\end{description}\end{quote}


\begin{sphinxseealso}{See also:}

\sphinxAtStartPar
\sphinxcode{\sphinxupquote{scipy.stats.weibull\_max}}, \sphinxcode{\sphinxupquote{scipy.stats.weibull\_min}}, \sphinxcode{\sphinxupquote{scipy.stats.genextreme}}, {\hyperref[\detokenize{metilda.controllers:metilda.controllers.pitch_art_wizard.gumbel}]{\sphinxcrossref{\sphinxcode{\sphinxupquote{gumbel}}}}}
\begin{description}
\sphinxlineitem{\sphinxcode{\sphinxupquote{random.Generator.weibull}}}
\sphinxAtStartPar
which should be used for new code.

\end{description}


\end{sphinxseealso}

\subsubsection*{Notes}

\sphinxAtStartPar
The Weibull (or Type III asymptotic extreme value distribution
for smallest values, SEV Type III, or Rosin\sphinxhyphen{}Rammler
distribution) is one of a class of Generalized Extreme Value
(GEV) distributions used in modeling extreme value problems.
This class includes the Gumbel and Frechet distributions.

\sphinxAtStartPar
The probability density for the Weibull distribution is
\begin{equation*}
\begin{split}p(x) = \frac{a}
{\lambda}(\frac{x}{\lambda})^{a-1}e^{-(x/\lambda)^a},\end{split}
\end{equation*}
\sphinxAtStartPar
where \(a\) is the shape and \(\lambda\) the scale.

\sphinxAtStartPar
The function has its peak (the mode) at
\(\lambda(\frac{a-1}{a})^{1/a}\).

\sphinxAtStartPar
When \sphinxcode{\sphinxupquote{a = 1}}, the Weibull distribution reduces to the exponential
distribution.
\subsubsection*{References}
\subsubsection*{Examples}

\sphinxAtStartPar
Draw samples from the distribution:

\begin{sphinxVerbatim}[commandchars=\\\{\}]
\PYG{g+gp}{\PYGZgt{}\PYGZgt{}\PYGZgt{} }\PYG{n}{a} \PYG{o}{=} \PYG{l+m+mf}{5.} \PYG{c+c1}{\PYGZsh{} shape}
\PYG{g+gp}{\PYGZgt{}\PYGZgt{}\PYGZgt{} }\PYG{n}{s} \PYG{o}{=} \PYG{n}{np}\PYG{o}{.}\PYG{n}{random}\PYG{o}{.}\PYG{n}{weibull}\PYG{p}{(}\PYG{n}{a}\PYG{p}{,} \PYG{l+m+mi}{1000}\PYG{p}{)}
\end{sphinxVerbatim}

\sphinxAtStartPar
Display the histogram of the samples, along with
the probability density function:

\begin{sphinxVerbatim}[commandchars=\\\{\}]
\PYG{g+gp}{\PYGZgt{}\PYGZgt{}\PYGZgt{} }\PYG{k+kn}{import} \PYG{n+nn}{matplotlib}\PYG{n+nn}{.}\PYG{n+nn}{pyplot} \PYG{k}{as} \PYG{n+nn}{plt}
\PYG{g+gp}{\PYGZgt{}\PYGZgt{}\PYGZgt{} }\PYG{n}{x} \PYG{o}{=} \PYG{n}{np}\PYG{o}{.}\PYG{n}{arange}\PYG{p}{(}\PYG{l+m+mi}{1}\PYG{p}{,}\PYG{l+m+mf}{100.}\PYG{p}{)}\PYG{o}{/}\PYG{l+m+mf}{50.}
\PYG{g+gp}{\PYGZgt{}\PYGZgt{}\PYGZgt{} }\PYG{k}{def} \PYG{n+nf}{weib}\PYG{p}{(}\PYG{n}{x}\PYG{p}{,}\PYG{n}{n}\PYG{p}{,}\PYG{n}{a}\PYG{p}{)}\PYG{p}{:}
\PYG{g+gp}{... }    \PYG{k}{return} \PYG{p}{(}\PYG{n}{a} \PYG{o}{/} \PYG{n}{n}\PYG{p}{)} \PYG{o}{*} \PYG{p}{(}\PYG{n}{x} \PYG{o}{/} \PYG{n}{n}\PYG{p}{)}\PYG{o}{*}\PYG{o}{*}\PYG{p}{(}\PYG{n}{a} \PYG{o}{\PYGZhy{}} \PYG{l+m+mi}{1}\PYG{p}{)} \PYG{o}{*} \PYG{n}{np}\PYG{o}{.}\PYG{n}{exp}\PYG{p}{(}\PYG{o}{\PYGZhy{}}\PYG{p}{(}\PYG{n}{x} \PYG{o}{/} \PYG{n}{n}\PYG{p}{)}\PYG{o}{*}\PYG{o}{*}\PYG{n}{a}\PYG{p}{)}
\end{sphinxVerbatim}

\begin{sphinxVerbatim}[commandchars=\\\{\}]
\PYG{g+gp}{\PYGZgt{}\PYGZgt{}\PYGZgt{} }\PYG{n}{count}\PYG{p}{,} \PYG{n}{bins}\PYG{p}{,} \PYG{n}{ignored} \PYG{o}{=} \PYG{n}{plt}\PYG{o}{.}\PYG{n}{hist}\PYG{p}{(}\PYG{n}{np}\PYG{o}{.}\PYG{n}{random}\PYG{o}{.}\PYG{n}{weibull}\PYG{p}{(}\PYG{l+m+mf}{5.}\PYG{p}{,}\PYG{l+m+mi}{1000}\PYG{p}{)}\PYG{p}{)}
\PYG{g+gp}{\PYGZgt{}\PYGZgt{}\PYGZgt{} }\PYG{n}{x} \PYG{o}{=} \PYG{n}{np}\PYG{o}{.}\PYG{n}{arange}\PYG{p}{(}\PYG{l+m+mi}{1}\PYG{p}{,}\PYG{l+m+mf}{100.}\PYG{p}{)}\PYG{o}{/}\PYG{l+m+mf}{50.}
\PYG{g+gp}{\PYGZgt{}\PYGZgt{}\PYGZgt{} }\PYG{n}{scale} \PYG{o}{=} \PYG{n}{count}\PYG{o}{.}\PYG{n}{max}\PYG{p}{(}\PYG{p}{)}\PYG{o}{/}\PYG{n}{weib}\PYG{p}{(}\PYG{n}{x}\PYG{p}{,} \PYG{l+m+mf}{1.}\PYG{p}{,} \PYG{l+m+mf}{5.}\PYG{p}{)}\PYG{o}{.}\PYG{n}{max}\PYG{p}{(}\PYG{p}{)}
\PYG{g+gp}{\PYGZgt{}\PYGZgt{}\PYGZgt{} }\PYG{n}{plt}\PYG{o}{.}\PYG{n}{plot}\PYG{p}{(}\PYG{n}{x}\PYG{p}{,} \PYG{n}{weib}\PYG{p}{(}\PYG{n}{x}\PYG{p}{,} \PYG{l+m+mf}{1.}\PYG{p}{,} \PYG{l+m+mf}{5.}\PYG{p}{)}\PYG{o}{*}\PYG{n}{scale}\PYG{p}{)}
\PYG{g+gp}{\PYGZgt{}\PYGZgt{}\PYGZgt{} }\PYG{n}{plt}\PYG{o}{.}\PYG{n}{show}\PYG{p}{(}\PYG{p}{)}
\end{sphinxVerbatim}

\end{fulllineitems}

\index{zipf() (in module metilda.controllers.pitch\_art\_wizard)@\spxentry{zipf()}\spxextra{in module metilda.controllers.pitch\_art\_wizard}}

\begin{fulllineitems}
\phantomsection\label{\detokenize{metilda.controllers:metilda.controllers.pitch_art_wizard.zipf}}
\pysigstartsignatures
\pysiglinewithargsret{\sphinxcode{\sphinxupquote{metilda.controllers.pitch\_art\_wizard.}}\sphinxbfcode{\sphinxupquote{zipf}}}{\sphinxparam{\DUrole{n,n}{a}}\sphinxparamcomma \sphinxparam{\DUrole{n,n}{size}\DUrole{o,o}{=}\DUrole{default_value}{None}}}{}
\pysigstopsignatures
\sphinxAtStartPar
Draw samples from a Zipf distribution.

\sphinxAtStartPar
Samples are drawn from a Zipf distribution with specified parameter
\sphinxtitleref{a} \textgreater{} 1.

\sphinxAtStartPar
The Zipf distribution (also known as the zeta distribution) is a
discrete probability distribution that satisfies Zipf’s law: the
frequency of an item is inversely proportional to its rank in a
frequency table.

\begin{sphinxadmonition}{note}{Note:}
\sphinxAtStartPar
New code should use the \sphinxtitleref{\textasciitilde{}numpy.random.Generator.zipf}
method of a \sphinxtitleref{\textasciitilde{}numpy.random.Generator} instance instead;
please see the \DUrole{xref,std,std-ref}{random\sphinxhyphen{}quick\sphinxhyphen{}start}.
\end{sphinxadmonition}
\begin{quote}\begin{description}
\sphinxlineitem{Parameters}\begin{itemize}
\item {} 
\sphinxAtStartPar
\sphinxstyleliteralstrong{\sphinxupquote{a}} (\sphinxstyleliteralemphasis{\sphinxupquote{float}}\sphinxstyleliteralemphasis{\sphinxupquote{ or }}\sphinxstyleliteralemphasis{\sphinxupquote{array\_like}}\sphinxstyleliteralemphasis{\sphinxupquote{ of }}\sphinxstyleliteralemphasis{\sphinxupquote{floats}}) \textendash{} Distribution parameter. Must be greater than 1.

\item {} 
\sphinxAtStartPar
\sphinxstyleliteralstrong{\sphinxupquote{size}} (\sphinxstyleliteralemphasis{\sphinxupquote{int}}\sphinxstyleliteralemphasis{\sphinxupquote{ or }}\sphinxstyleliteralemphasis{\sphinxupquote{tuple}}\sphinxstyleliteralemphasis{\sphinxupquote{ of }}\sphinxstyleliteralemphasis{\sphinxupquote{ints}}\sphinxstyleliteralemphasis{\sphinxupquote{, }}\sphinxstyleliteralemphasis{\sphinxupquote{optional}}) \textendash{} Output shape.  If the given shape is, e.g., \sphinxcode{\sphinxupquote{(m, n, k)}}, then
\sphinxcode{\sphinxupquote{m * n * k}} samples are drawn.  If size is \sphinxcode{\sphinxupquote{None}} (default),
a single value is returned if \sphinxcode{\sphinxupquote{a}} is a scalar. Otherwise,
\sphinxcode{\sphinxupquote{np.array(a).size}} samples are drawn.

\end{itemize}

\sphinxlineitem{Returns}
\sphinxAtStartPar
\sphinxstylestrong{out} \textendash{} Drawn samples from the parameterized Zipf distribution.

\sphinxlineitem{Return type}
\sphinxAtStartPar
ndarray or scalar

\end{description}\end{quote}


\begin{sphinxseealso}{See also:}
\begin{description}
\sphinxlineitem{\sphinxcode{\sphinxupquote{scipy.stats.zipf}}}
\sphinxAtStartPar
probability density function, distribution, or cumulative density function, etc.

\sphinxlineitem{\sphinxcode{\sphinxupquote{random.Generator.zipf}}}
\sphinxAtStartPar
which should be used for new code.

\end{description}


\end{sphinxseealso}

\subsubsection*{Notes}

\sphinxAtStartPar
The probability density for the Zipf distribution is
\begin{equation*}
\begin{split}p(k) = \frac{k^{-a}}{\zeta(a)},\end{split}
\end{equation*}
\sphinxAtStartPar
for integers \(k \geq 1\), where \(\zeta\) is the Riemann Zeta
function.

\sphinxAtStartPar
It is named for the American linguist George Kingsley Zipf, who noted
that the frequency of any word in a sample of a language is inversely
proportional to its rank in the frequency table.
\subsubsection*{References}
\subsubsection*{Examples}

\sphinxAtStartPar
Draw samples from the distribution:

\begin{sphinxVerbatim}[commandchars=\\\{\}]
\PYG{g+gp}{\PYGZgt{}\PYGZgt{}\PYGZgt{} }\PYG{n}{a} \PYG{o}{=} \PYG{l+m+mf}{4.0}
\PYG{g+gp}{\PYGZgt{}\PYGZgt{}\PYGZgt{} }\PYG{n}{n} \PYG{o}{=} \PYG{l+m+mi}{20000}
\PYG{g+gp}{\PYGZgt{}\PYGZgt{}\PYGZgt{} }\PYG{n}{s} \PYG{o}{=} \PYG{n}{np}\PYG{o}{.}\PYG{n}{random}\PYG{o}{.}\PYG{n}{zipf}\PYG{p}{(}\PYG{n}{a}\PYG{p}{,} \PYG{n}{n}\PYG{p}{)}
\end{sphinxVerbatim}

\sphinxAtStartPar
Display the histogram of the samples, along with
the expected histogram based on the probability
density function:

\begin{sphinxVerbatim}[commandchars=\\\{\}]
\PYG{g+gp}{\PYGZgt{}\PYGZgt{}\PYGZgt{} }\PYG{k+kn}{import} \PYG{n+nn}{matplotlib}\PYG{n+nn}{.}\PYG{n+nn}{pyplot} \PYG{k}{as} \PYG{n+nn}{plt}
\PYG{g+gp}{\PYGZgt{}\PYGZgt{}\PYGZgt{} }\PYG{k+kn}{from} \PYG{n+nn}{scipy}\PYG{n+nn}{.}\PYG{n+nn}{special} \PYG{k+kn}{import} \PYG{n}{zeta}  
\end{sphinxVerbatim}

\sphinxAtStartPar
\sphinxtitleref{bincount} provides a fast histogram for small integers.

\begin{sphinxVerbatim}[commandchars=\\\{\}]
\PYG{g+gp}{\PYGZgt{}\PYGZgt{}\PYGZgt{} }\PYG{n}{count} \PYG{o}{=} \PYG{n}{np}\PYG{o}{.}\PYG{n}{bincount}\PYG{p}{(}\PYG{n}{s}\PYG{p}{)}
\PYG{g+gp}{\PYGZgt{}\PYGZgt{}\PYGZgt{} }\PYG{n}{k} \PYG{o}{=} \PYG{n}{np}\PYG{o}{.}\PYG{n}{arange}\PYG{p}{(}\PYG{l+m+mi}{1}\PYG{p}{,} \PYG{n}{s}\PYG{o}{.}\PYG{n}{max}\PYG{p}{(}\PYG{p}{)} \PYG{o}{+} \PYG{l+m+mi}{1}\PYG{p}{)}
\end{sphinxVerbatim}

\begin{sphinxVerbatim}[commandchars=\\\{\}]
\PYG{g+gp}{\PYGZgt{}\PYGZgt{}\PYGZgt{} }\PYG{n}{plt}\PYG{o}{.}\PYG{n}{bar}\PYG{p}{(}\PYG{n}{k}\PYG{p}{,} \PYG{n}{count}\PYG{p}{[}\PYG{l+m+mi}{1}\PYG{p}{:}\PYG{p}{]}\PYG{p}{,} \PYG{n}{alpha}\PYG{o}{=}\PYG{l+m+mf}{0.5}\PYG{p}{,} \PYG{n}{label}\PYG{o}{=}\PYG{l+s+s1}{\PYGZsq{}}\PYG{l+s+s1}{sample count}\PYG{l+s+s1}{\PYGZsq{}}\PYG{p}{)}
\PYG{g+gp}{\PYGZgt{}\PYGZgt{}\PYGZgt{} }\PYG{n}{plt}\PYG{o}{.}\PYG{n}{plot}\PYG{p}{(}\PYG{n}{k}\PYG{p}{,} \PYG{n}{n}\PYG{o}{*}\PYG{p}{(}\PYG{n}{k}\PYG{o}{*}\PYG{o}{*}\PYG{o}{\PYGZhy{}}\PYG{n}{a}\PYG{p}{)}\PYG{o}{/}\PYG{n}{zeta}\PYG{p}{(}\PYG{n}{a}\PYG{p}{)}\PYG{p}{,} \PYG{l+s+s1}{\PYGZsq{}}\PYG{l+s+s1}{k.\PYGZhy{}}\PYG{l+s+s1}{\PYGZsq{}}\PYG{p}{,} \PYG{n}{alpha}\PYG{o}{=}\PYG{l+m+mf}{0.5}\PYG{p}{,}
\PYG{g+gp}{... }         \PYG{n}{label}\PYG{o}{=}\PYG{l+s+s1}{\PYGZsq{}}\PYG{l+s+s1}{expected count}\PYG{l+s+s1}{\PYGZsq{}}\PYG{p}{)}   
\PYG{g+gp}{\PYGZgt{}\PYGZgt{}\PYGZgt{} }\PYG{n}{plt}\PYG{o}{.}\PYG{n}{semilogy}\PYG{p}{(}\PYG{p}{)}
\PYG{g+gp}{\PYGZgt{}\PYGZgt{}\PYGZgt{} }\PYG{n}{plt}\PYG{o}{.}\PYG{n}{grid}\PYG{p}{(}\PYG{n}{alpha}\PYG{o}{=}\PYG{l+m+mf}{0.4}\PYG{p}{)}
\PYG{g+gp}{\PYGZgt{}\PYGZgt{}\PYGZgt{} }\PYG{n}{plt}\PYG{o}{.}\PYG{n}{legend}\PYG{p}{(}\PYG{p}{)}
\PYG{g+gp}{\PYGZgt{}\PYGZgt{}\PYGZgt{} }\PYG{n}{plt}\PYG{o}{.}\PYG{n}{title}\PYG{p}{(}\PYG{l+s+sa}{f}\PYG{l+s+s1}{\PYGZsq{}}\PYG{l+s+s1}{Zipf sample, a=}\PYG{l+s+si}{\PYGZob{}}\PYG{n}{a}\PYG{l+s+si}{\PYGZcb{}}\PYG{l+s+s1}{, size=}\PYG{l+s+si}{\PYGZob{}}\PYG{n}{n}\PYG{l+s+si}{\PYGZcb{}}\PYG{l+s+s1}{\PYGZsq{}}\PYG{p}{)}
\PYG{g+gp}{\PYGZgt{}\PYGZgt{}\PYGZgt{} }\PYG{n}{plt}\PYG{o}{.}\PYG{n}{show}\PYG{p}{(}\PYG{p}{)}
\end{sphinxVerbatim}

\end{fulllineitems}



\subsubsection{metilda.controllers.tempCodeRunnerFile module}
\label{\detokenize{metilda.controllers:metilda-controllers-tempcoderunnerfile-module}}

\subsubsection{Module contents}
\label{\detokenize{metilda.controllers:module-metilda.controllers}}\label{\detokenize{metilda.controllers:module-contents}}\index{module@\spxentry{module}!metilda.controllers@\spxentry{metilda.controllers}}\index{metilda.controllers@\spxentry{metilda.controllers}!module@\spxentry{module}}
\sphinxstepscope


\subsection{metilda.services package}
\label{\detokenize{metilda.services:metilda-services-package}}\label{\detokenize{metilda.services::doc}}

\subsubsection{Submodules}
\label{\detokenize{metilda.services:submodules}}

\subsubsection{metilda.services.audio\_analysis module}
\label{\detokenize{metilda.services:module-metilda.services.audio_analysis}}\label{\detokenize{metilda.services:metilda-services-audio-analysis-module}}\index{module@\spxentry{module}!metilda.services.audio\_analysis@\spxentry{metilda.services.audio\_analysis}}\index{metilda.services.audio\_analysis@\spxentry{metilda.services.audio\_analysis}!module@\spxentry{module}}
\sphinxAtStartPar
Contains utilities for audio analysis. Some of the visualization functions are
based on examples from the parselmouth\sphinxhyphen{}praat library:
\sphinxurl{https://github.com/YannickJadoul/Parselmouth}
\index{audio\_analysis\_image() (in module metilda.services.audio\_analysis)@\spxentry{audio\_analysis\_image()}\spxextra{in module metilda.services.audio\_analysis}}

\begin{fulllineitems}
\phantomsection\label{\detokenize{metilda.services:metilda.services.audio_analysis.audio_analysis_image}}
\pysigstartsignatures
\pysiglinewithargsret{\sphinxcode{\sphinxupquote{metilda.services.audio\_analysis.}}\sphinxbfcode{\sphinxupquote{audio\_analysis\_image}}}{\sphinxparam{\DUrole{n,n}{upload\_path}}\sphinxparamcomma \sphinxparam{\DUrole{n,n}{tmin}\DUrole{o,o}{=}\DUrole{default_value}{\sphinxhyphen{}1}}\sphinxparamcomma \sphinxparam{\DUrole{n,n}{tmax}\DUrole{o,o}{=}\DUrole{default_value}{\sphinxhyphen{}1}}\sphinxparamcomma \sphinxparam{\DUrole{n,n}{min\_pitch}\DUrole{o,o}{=}\DUrole{default_value}{75}}\sphinxparamcomma \sphinxparam{\DUrole{n,n}{max\_pitch}\DUrole{o,o}{=}\DUrole{default_value}{500}}\sphinxparamcomma \sphinxparam{\DUrole{n,n}{output\_path}\DUrole{o,o}{=}\DUrole{default_value}{None}}}{}
\pysigstopsignatures
\end{fulllineitems}

\index{draw\_pitch() (in module metilda.services.audio\_analysis)@\spxentry{draw\_pitch()}\spxextra{in module metilda.services.audio\_analysis}}

\begin{fulllineitems}
\phantomsection\label{\detokenize{metilda.services:metilda.services.audio_analysis.draw_pitch}}
\pysigstartsignatures
\pysiglinewithargsret{\sphinxcode{\sphinxupquote{metilda.services.audio\_analysis.}}\sphinxbfcode{\sphinxupquote{draw\_pitch}}}{\sphinxparam{\DUrole{n,n}{ax}}\sphinxparamcomma \sphinxparam{\DUrole{n,n}{pitch}}\sphinxparamcomma \sphinxparam{\DUrole{n,n}{min\_pitch}}\sphinxparamcomma \sphinxparam{\DUrole{n,n}{max\_pitch}}}{}
\pysigstopsignatures
\end{fulllineitems}

\index{draw\_spectrogram() (in module metilda.services.audio\_analysis)@\spxentry{draw\_spectrogram()}\spxextra{in module metilda.services.audio\_analysis}}

\begin{fulllineitems}
\phantomsection\label{\detokenize{metilda.services:metilda.services.audio_analysis.draw_spectrogram}}
\pysigstartsignatures
\pysiglinewithargsret{\sphinxcode{\sphinxupquote{metilda.services.audio\_analysis.}}\sphinxbfcode{\sphinxupquote{draw\_spectrogram}}}{\sphinxparam{\DUrole{n,n}{ax}}\sphinxparamcomma \sphinxparam{\DUrole{n,n}{spectrogram}}\sphinxparamcomma \sphinxparam{\DUrole{n,n}{dynamic\_range}\DUrole{o,o}{=}\DUrole{default_value}{70}}}{}
\pysigstopsignatures
\end{fulllineitems}

\index{get\_all\_pitches() (in module metilda.services.audio\_analysis)@\spxentry{get\_all\_pitches()}\spxextra{in module metilda.services.audio\_analysis}}

\begin{fulllineitems}
\phantomsection\label{\detokenize{metilda.services:metilda.services.audio_analysis.get_all_pitches}}
\pysigstartsignatures
\pysiglinewithargsret{\sphinxcode{\sphinxupquote{metilda.services.audio\_analysis.}}\sphinxbfcode{\sphinxupquote{get\_all\_pitches}}}{\sphinxparam{\DUrole{n,n}{time\_range}}\sphinxparamcomma \sphinxparam{\DUrole{n,n}{upload\_path}}\sphinxparamcomma \sphinxparam{\DUrole{n,n}{min\_pitch}\DUrole{o,o}{=}\DUrole{default_value}{75}}\sphinxparamcomma \sphinxparam{\DUrole{n,n}{max\_pitch}\DUrole{o,o}{=}\DUrole{default_value}{500}}}{}
\pysigstopsignatures
\end{fulllineitems}

\index{get\_audio() (in module metilda.services.audio\_analysis)@\spxentry{get\_audio()}\spxextra{in module metilda.services.audio\_analysis}}

\begin{fulllineitems}
\phantomsection\label{\detokenize{metilda.services:metilda.services.audio_analysis.get_audio}}
\pysigstartsignatures
\pysiglinewithargsret{\sphinxcode{\sphinxupquote{metilda.services.audio\_analysis.}}\sphinxbfcode{\sphinxupquote{get\_audio}}}{\sphinxparam{\DUrole{n,n}{upload\_path}}\sphinxparamcomma \sphinxparam{\DUrole{n,n}{tmin}\DUrole{o,o}{=}\DUrole{default_value}{\sphinxhyphen{}1}}\sphinxparamcomma \sphinxparam{\DUrole{n,n}{tmax}\DUrole{o,o}{=}\DUrole{default_value}{\sphinxhyphen{}1}}}{}
\pysigstopsignatures
\end{fulllineitems}

\index{get\_avg\_pitch() (in module metilda.services.audio\_analysis)@\spxentry{get\_avg\_pitch()}\spxextra{in module metilda.services.audio\_analysis}}

\begin{fulllineitems}
\phantomsection\label{\detokenize{metilda.services:metilda.services.audio_analysis.get_avg_pitch}}
\pysigstartsignatures
\pysiglinewithargsret{\sphinxcode{\sphinxupquote{metilda.services.audio\_analysis.}}\sphinxbfcode{\sphinxupquote{get\_avg\_pitch}}}{\sphinxparam{\DUrole{n,n}{time\_range}}\sphinxparamcomma \sphinxparam{\DUrole{n,n}{upload\_path}}\sphinxparamcomma \sphinxparam{\DUrole{n,n}{min\_pitch}\DUrole{o,o}{=}\DUrole{default_value}{75}}\sphinxparamcomma \sphinxparam{\DUrole{n,n}{max\_pitch}\DUrole{o,o}{=}\DUrole{default_value}{500}}}{}
\pysigstopsignatures
\end{fulllineitems}

\index{get\_pitches\_in\_range() (in module metilda.services.audio\_analysis)@\spxentry{get\_pitches\_in\_range()}\spxextra{in module metilda.services.audio\_analysis}}

\begin{fulllineitems}
\phantomsection\label{\detokenize{metilda.services:metilda.services.audio_analysis.get_pitches_in_range}}
\pysigstartsignatures
\pysiglinewithargsret{\sphinxcode{\sphinxupquote{metilda.services.audio\_analysis.}}\sphinxbfcode{\sphinxupquote{get\_pitches\_in\_range}}}{\sphinxparam{\DUrole{n,n}{tmin}}\sphinxparamcomma \sphinxparam{\DUrole{n,n}{tmax}}\sphinxparamcomma \sphinxparam{\DUrole{n,n}{snd\_pitch}}}{}
\pysigstopsignatures
\end{fulllineitems}

\index{get\_sound\_length() (in module metilda.services.audio\_analysis)@\spxentry{get\_sound\_length()}\spxextra{in module metilda.services.audio\_analysis}}

\begin{fulllineitems}
\phantomsection\label{\detokenize{metilda.services:metilda.services.audio_analysis.get_sound_length}}
\pysigstartsignatures
\pysiglinewithargsret{\sphinxcode{\sphinxupquote{metilda.services.audio\_analysis.}}\sphinxbfcode{\sphinxupquote{get\_sound\_length}}}{\sphinxparam{\DUrole{n,n}{upload\_path}}}{}
\pysigstopsignatures
\end{fulllineitems}

\index{test() (in module metilda.services.audio\_analysis)@\spxentry{test()}\spxextra{in module metilda.services.audio\_analysis}}

\begin{fulllineitems}
\phantomsection\label{\detokenize{metilda.services:metilda.services.audio_analysis.test}}
\pysigstartsignatures
\pysiglinewithargsret{\sphinxcode{\sphinxupquote{metilda.services.audio\_analysis.}}\sphinxbfcode{\sphinxupquote{test}}}{\sphinxparam{\DUrole{n,n}{upload\_path}}}{}
\pysigstopsignatures
\end{fulllineitems}



\subsubsection{metilda.services.file\_io module}
\label{\detokenize{metilda.services:module-metilda.services.file_io}}\label{\detokenize{metilda.services:metilda-services-file-io-module}}\index{module@\spxentry{module}!metilda.services.file\_io@\spxentry{metilda.services.file\_io}}\index{metilda.services.file\_io@\spxentry{metilda.services.file\_io}!module@\spxentry{module}}\index{available\_files() (in module metilda.services.file\_io)@\spxentry{available\_files()}\spxextra{in module metilda.services.file\_io}}

\begin{fulllineitems}
\phantomsection\label{\detokenize{metilda.services:metilda.services.file_io.available_files}}
\pysigstartsignatures
\pysiglinewithargsret{\sphinxcode{\sphinxupquote{metilda.services.file\_io.}}\sphinxbfcode{\sphinxupquote{available\_files}}}{\sphinxparam{\DUrole{n,n}{dir}}}{}
\pysigstopsignatures
\end{fulllineitems}



\subsubsection{metilda.services.praat module}
\label{\detokenize{metilda.services:module-metilda.services.praat}}\label{\detokenize{metilda.services:metilda-services-praat-module}}\index{module@\spxentry{module}!metilda.services.praat@\spxentry{metilda.services.praat}}\index{metilda.services.praat@\spxentry{metilda.services.praat}!module@\spxentry{module}}\index{runScript() (in module metilda.services.praat)@\spxentry{runScript()}\spxextra{in module metilda.services.praat}}

\begin{fulllineitems}
\phantomsection\label{\detokenize{metilda.services:metilda.services.praat.runScript}}
\pysigstartsignatures
\pysiglinewithargsret{\sphinxcode{\sphinxupquote{metilda.services.praat.}}\sphinxbfcode{\sphinxupquote{runScript}}}{\sphinxparam{\DUrole{n,n}{scriptName}}\sphinxparamcomma \sphinxparam{\DUrole{n,n}{args}}}{}
\pysigstopsignatures
\end{fulllineitems}



\subsubsection{metilda.services.utils module}
\label{\detokenize{metilda.services:module-metilda.services.utils}}\label{\detokenize{metilda.services:metilda-services-utils-module}}\index{module@\spxentry{module}!metilda.services.utils@\spxentry{metilda.services.utils}}\index{metilda.services.utils@\spxentry{metilda.services.utils}!module@\spxentry{module}}\index{deleteCachedImages() (in module metilda.services.utils)@\spxentry{deleteCachedImages()}\spxextra{in module metilda.services.utils}}

\begin{fulllineitems}
\phantomsection\label{\detokenize{metilda.services:metilda.services.utils.deleteCachedImages}}
\pysigstartsignatures
\pysiglinewithargsret{\sphinxcode{\sphinxupquote{metilda.services.utils.}}\sphinxbfcode{\sphinxupquote{deleteCachedImages}}}{\sphinxparam{\DUrole{n,n}{directory}}\sphinxparamcomma \sphinxparam{\DUrole{n,n}{prefix}}}{}
\pysigstopsignatures
\sphinxAtStartPar
Delete cached images starting with prefix

\end{fulllineitems}

\index{fileType() (in module metilda.services.utils)@\spxentry{fileType()}\spxextra{in module metilda.services.utils}}

\begin{fulllineitems}
\phantomsection\label{\detokenize{metilda.services:metilda.services.utils.fileType}}
\pysigstartsignatures
\pysiglinewithargsret{\sphinxcode{\sphinxupquote{metilda.services.utils.}}\sphinxbfcode{\sphinxupquote{fileType}}}{\sphinxparam{\DUrole{n,n}{fileName}}}{}
\pysigstopsignatures
\sphinxAtStartPar
Return file extension

\end{fulllineitems}

\index{isSound() (in module metilda.services.utils)@\spxentry{isSound()}\spxextra{in module metilda.services.utils}}

\begin{fulllineitems}
\phantomsection\label{\detokenize{metilda.services:metilda.services.utils.isSound}}
\pysigstartsignatures
\pysiglinewithargsret{\sphinxcode{\sphinxupquote{metilda.services.utils.}}\sphinxbfcode{\sphinxupquote{isSound}}}{\sphinxparam{\DUrole{n,n}{fileName}}}{}
\pysigstopsignatures
\sphinxAtStartPar
Checks if fileName has a valid sound file extension

\end{fulllineitems}

\index{resizeImage() (in module metilda.services.utils)@\spxentry{resizeImage()}\spxextra{in module metilda.services.utils}}

\begin{fulllineitems}
\phantomsection\label{\detokenize{metilda.services:metilda.services.utils.resizeImage}}
\pysigstartsignatures
\pysiglinewithargsret{\sphinxcode{\sphinxupquote{metilda.services.utils.}}\sphinxbfcode{\sphinxupquote{resizeImage}}}{\sphinxparam{\DUrole{n,n}{image}}}{}
\pysigstopsignatures
\sphinxAtStartPar
Down\sphinxhyphen{}scaling the image to 500x500 pixels

\end{fulllineitems}



\subsubsection{Module contents}
\label{\detokenize{metilda.services:module-metilda.services}}\label{\detokenize{metilda.services:module-contents}}\index{module@\spxentry{module}!metilda.services@\spxentry{metilda.services}}\index{metilda.services@\spxentry{metilda.services}!module@\spxentry{module}}

\section{Submodules}
\label{\detokenize{index:submodules}}

\section{metilda.debug module}
\label{\detokenize{index:module-metilda.debug}}\label{\detokenize{index:metilda-debug-module}}\index{module@\spxentry{module}!metilda.debug@\spxentry{metilda.debug}}\index{metilda.debug@\spxentry{metilda.debug}!module@\spxentry{module}}

\section{metilda.default module}
\label{\detokenize{index:module-metilda.default}}\label{\detokenize{index:metilda-default-module}}\index{module@\spxentry{module}!metilda.default@\spxentry{metilda.default}}\index{metilda.default@\spxentry{metilda.default}!module@\spxentry{module}}

\section{metilda.local\_server module}
\label{\detokenize{index:metilda-local-server-module}}

\section{metilda.main module}
\label{\detokenize{index:module-metilda.main}}\label{\detokenize{index:metilda-main-module}}\index{module@\spxentry{module}!metilda.main@\spxentry{metilda.main}}\index{metilda.main@\spxentry{metilda.main}!module@\spxentry{module}}

\section{Module contents}
\label{\detokenize{index:module-metilda}}\label{\detokenize{index:module-contents}}\index{module@\spxentry{module}!metilda@\spxentry{metilda}}\index{metilda@\spxentry{metilda}!module@\spxentry{module}}\index{CustomJSONEncoder (class in metilda)@\spxentry{CustomJSONEncoder}\spxextra{class in metilda}}

\begin{fulllineitems}
\phantomsection\label{\detokenize{index:metilda.CustomJSONEncoder}}
\pysigstartsignatures
\pysiglinewithargsret{\sphinxbfcode{\sphinxupquote{class\DUrole{w,w}{  }}}\sphinxcode{\sphinxupquote{metilda.}}\sphinxbfcode{\sphinxupquote{CustomJSONEncoder}}}{\sphinxparam{\DUrole{o,o}{*}}\sphinxparamcomma \sphinxparam{\DUrole{n,n}{skipkeys}\DUrole{o,o}{=}\DUrole{default_value}{False}}\sphinxparamcomma \sphinxparam{\DUrole{n,n}{ensure\_ascii}\DUrole{o,o}{=}\DUrole{default_value}{True}}\sphinxparamcomma \sphinxparam{\DUrole{n,n}{check\_circular}\DUrole{o,o}{=}\DUrole{default_value}{True}}\sphinxparamcomma \sphinxparam{\DUrole{n,n}{allow\_nan}\DUrole{o,o}{=}\DUrole{default_value}{True}}\sphinxparamcomma \sphinxparam{\DUrole{n,n}{sort\_keys}\DUrole{o,o}{=}\DUrole{default_value}{False}}\sphinxparamcomma \sphinxparam{\DUrole{n,n}{indent}\DUrole{o,o}{=}\DUrole{default_value}{None}}\sphinxparamcomma \sphinxparam{\DUrole{n,n}{separators}\DUrole{o,o}{=}\DUrole{default_value}{None}}\sphinxparamcomma \sphinxparam{\DUrole{n,n}{default}\DUrole{o,o}{=}\DUrole{default_value}{None}}}{}
\pysigstopsignatures
\sphinxAtStartPar
Bases: \sphinxcode{\sphinxupquote{JSONEncoder}}
\index{default() (metilda.CustomJSONEncoder method)@\spxentry{default()}\spxextra{metilda.CustomJSONEncoder method}}

\begin{fulllineitems}
\phantomsection\label{\detokenize{index:metilda.CustomJSONEncoder.default}}
\pysigstartsignatures
\pysiglinewithargsret{\sphinxbfcode{\sphinxupquote{default}}}{\sphinxparam{\DUrole{n,n}{obj}}}{}
\pysigstopsignatures
\sphinxAtStartPar
Convert \sphinxcode{\sphinxupquote{o}} to a JSON serializable type. See
\sphinxcode{\sphinxupquote{json.JSONEncoder.default()}}. Python does not support
overriding how basic types like \sphinxcode{\sphinxupquote{str}} or \sphinxcode{\sphinxupquote{list}} are
serialized, they are handled before this method.

\end{fulllineitems}


\end{fulllineitems}

\index{get\_app() (in module metilda)@\spxentry{get\_app()}\spxextra{in module metilda}}

\begin{fulllineitems}
\phantomsection\label{\detokenize{index:metilda.get_app}}
\pysigstartsignatures
\pysiglinewithargsret{\sphinxcode{\sphinxupquote{metilda.}}\sphinxbfcode{\sphinxupquote{get\_app}}}{}{}
\pysigstopsignatures
\end{fulllineitems}

\index{react\_app() (in module metilda)@\spxentry{react\_app()}\spxextra{in module metilda}}

\begin{fulllineitems}
\phantomsection\label{\detokenize{index:metilda.react_app}}
\pysigstartsignatures
\pysiglinewithargsret{\sphinxcode{\sphinxupquote{metilda.}}\sphinxbfcode{\sphinxupquote{react\_app}}}{\sphinxparam{\DUrole{n,n}{path}\DUrole{o,o}{=}\DUrole{default_value}{None}}}{}
\pysigstopsignatures
\end{fulllineitems}



\renewcommand{\indexname}{Python Module Index}
\begin{sphinxtheindex}
\let\bigletter\sphinxstyleindexlettergroup
\bigletter{m}
\item\relax\sphinxstyleindexentry{metilda.controllers}\sphinxstyleindexpageref{metilda.controllers:\detokenize{module-metilda.controllers}}
\item\relax\sphinxstyleindexentry{metilda.controllers.controller\_firestore}\sphinxstyleindexpageref{metilda.controllers:\detokenize{module-metilda.controllers.controller_firestore}}
\item\relax\sphinxstyleindexentry{metilda.controllers.pitch\_art\_wizard}\sphinxstyleindexpageref{metilda.controllers:\detokenize{module-metilda.controllers.pitch_art_wizard}}
\item\relax\sphinxstyleindexentry{metilda.controllers.Postgres}\sphinxstyleindexpageref{metilda.controllers:\detokenize{module-metilda.controllers.Postgres}}
\item\relax\sphinxstyleindexentry{metilda.services}\sphinxstyleindexpageref{metilda.services:\detokenize{module-metilda.services}}
\item\relax\sphinxstyleindexentry{metilda.services.audio\_analysis}\sphinxstyleindexpageref{metilda.services:\detokenize{module-metilda.services.audio_analysis}}
\item\relax\sphinxstyleindexentry{metilda.services.file\_io}\sphinxstyleindexpageref{metilda.services:\detokenize{module-metilda.services.file_io}}
\item\relax\sphinxstyleindexentry{metilda.services.praat}\sphinxstyleindexpageref{metilda.services:\detokenize{module-metilda.services.praat}}
\item\relax\sphinxstyleindexentry{metilda.services.utils}\sphinxstyleindexpageref{metilda.services:\detokenize{module-metilda.services.utils}}
\end{sphinxtheindex}

\renewcommand{\indexname}{Index}
\printindex
\end{document}